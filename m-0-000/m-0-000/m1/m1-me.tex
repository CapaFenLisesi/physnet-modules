\revhist{12/26/84 mpm, 8/23/85, 7/7/86; 9/13/89, pss; 6/30/91, pss; 3/23/94, pss; 4/3/02, pss}

\Sect{}{}{\SectType{ModelExam}}{

\begin{one-digit-list}
\item [1.] Verify that \m{f(t) = A \cos\omega t + B \sin\omega t} (where \m{A} and
\m{B} are constant and \m{\omega^2 = k/m}) is a solution of
\begin{displaymath}
m \dfrac{d^2 f(t)}{dt^2} + k f(t) = 0
\end{displaymath}
where \m{d^2 f(t)/dt^2} is the second derivative of \m{f(t)} and \m{k} and \m{m} are
constants.
\item [2.] Evaluate the derivative of \m{(A \cos 5y)} at \m{y = \pi/20}.
\item [3.] Evaluate this integral: \m{\int x^{1/2}\,dx}\,.
\item [4.] For each of the equations below, find the maximum and/or minimum
points and distinguish between them.
\begin{one-digit-list}
\item [a.] \m{y(x) = x^2 + x + 10}.
\item [b.] \m{y(x) = x^3 - 3x + 2}.
\item [c.] Determine \m{A}, \m{B} and \m{C} so the function \m{y(x) = Ax^3 + B x^2 + C} will
have a minimum at \m{x = 1/3}.
\end{one-digit-list}
\end{one-digit-list}

\BriefAns

\begin{one-digit-list}
\item [1.] (A verification).

\item [2.] See this module's \textit{Problem Supplement}, problem~17.

\item [3.] See this module's \textit{Problem Supplement}, problem~18.

\item [4.] See this module's \textit{Problem Supplement}, problem~19.
\end{one-digit-list}

}% /Sect
