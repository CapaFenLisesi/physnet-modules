\revhist{12/29/87, pss; 7/2/91, pss; 9/17/93, pss; 9/29/94, pss; 8/16/95, pss;
         10/12/96, pss; 11/13/97, pss; 2/24/2000, pss; 12/11/2000; 1/5/2001, pss;
         6/14/02, pss}

\Sect{}{}{\SectType{ProblemSet}}{

Note: problems~13 (part), 21, and 22 (part) also occur in this module's
\textit{Model Exam}.

\begin{one-digit-list}
\item [1.] \CenteredUnframedFixedFigure{m2gr08}{Three vectors, \vect{A}, \vect{B}, and \vect{C},
all in a plane, are shown sketched above.
By arranging the vectors in head-to-tail fashion, sketch:}

\begin{one-digit-list}
\item [a.] \m{\vect{A} + \vect{B}}, to which then add \vect{C};
\item [b.] \m{\vect{A} + \vect{C}}, to which then add \vect{B};
\item [c.] \m{\vect{B} + \vect{C}}, to which then add \vect{A};
\item [d.] \m{\vect{A} - \vect{B}}.
\end{one-digit-list}

\item [2.] Write down the ordered triad of numbers which represents a vector
of length 6\unit{units} and that lies along the \m{y}-axis of a fixed coordinate
system.

\item [3.] Write down the ordered triad of numbers which represents a vector
of length 6\unit{units} that lies in the \m{xy}-plane, making an angle of {30\degrees}
with the \m{x}-axis.

\item [4.] Express the vectors in problems~2 and 3 above in terms of the
fixed unit vectors along the Cartesian axes.

\item [5.] A particular vector is given as a zero or null vector.
Write down the triad that represents this vector.

\item [6.] \CenteredUnframedFixedFigure{m2gr09}
\m{|\vect{A}| = 4\unit{units}}, \m{|\vect{B}| = 5\unit{units}}, directions as in the sketch.
\begin{one-digit-list}
\item [a.] Graphically add \vect{A} and \vect{B}.
\item [b.] Compute the magnitude of the resultant.
\item [c.] Determine the direction of the resultant, expressed as:
%\end{one-digit-list}}
%
%\begin{one-digit-list}
\begin{itemize}
\item [(i)] The angle between the vector and the positive \m{x}-axis, measured
counter-clockwise from the  positive \m{x}-axis; this is the standard
representation of a vector's direction.
\item [(ii)] The angle between the vector and the positive \m{y}-axis, measured
clockwise from the positive \m{y}-axis; this is a compass bearing if the positive
\m{y}-axis represents due north and the positive \m{x}-axis represents due east.
\end{itemize}
\end{one-digit-list}

\item [7.] \CenteredUnframedFixedFigure{m2gr10}{\m{|\vect{A}| = 4\unit{units}}, \m{|\vect{B}| = 3\unit{units}},
           \m{\theta = 30\degrees}, \m{|\vect{C}| = 5\unit{units}}, \m{\phi = 60\degrees} (see sketch).
\begin{one-digit-list}
\item [a.] Express \vect{A}, \vect{B}, and \vect{C} in component form. \help{15}
\item [b.] Using these components, evaluate \m{\vect{A} + \vect{B} + \vect{C}}.
\item [c.] Similarly, evaluate \m{\vect{A} + \vect{B} - \vect{C}}.
\end{one-digit-list}}

\item [8.] A vector of unit length, \m{\uvec{x}}, is parallel to the \m{x}-axis of a
Cartesian coordinate system while another vector of unit length, \m{\uvec{y}}, is
along the \m{y}-axis.
Evaluate the scalar products \m{\uvec{x} \cdot \uvec{x}}, \m{\uvec{x} \cdot \uvec{y}},
and \m{\uvec{y} \cdot \uvec{y}}.

\item [9.] Expressing two vectors in terms of unit vectors along the coordinate
axes, \m{\vect{A} = A_x\uvec{x} + A_y\uvec{y} + A_z\uvec{z}} and \m{\vect{B} =
B_x\uvec{x} + B_y\uvec{y} + B_z\uvec{z}}, show that the \Index{scalar product, in terms of components}scalar product may be
written as \m{A_x B_x + A_y B_y + A_z B_z} (use results that you got in
problem~8, above.)

\item [10.] Consider two vectors, \m{\vect{A} = 4\uvec{x} + 3\uvec{y}} and
\m{\vect{B} = 5\uvec{x} - 12\uvec{y}}.
Find: (a) the length of \vect{A} starting from the definition of the scalar
product; (b) the length of \vect{B}; and (c) the angle between \vect{A} and \vect{B}.

\item [11.] \CenteredUnframedFixedFigure{m2gr11}{Assuming magnitudes (see sketch) \m{|\vect{A}| =
|\vect{B}| = |\vect{C}| = R}, evaluate the following in terms of \m{R}:
\begin{one-digit-list}
\item [a.] \m{(\vect{A} \times \vect{B}) \times \vect{C}}
\item [b.] \m{(\vect{B} \times \vect{A}) \times \vect{C}}
\item [c.] \m{\vect{A} \times (\vect{B} \times \vect{C})}
\item [d.] \m{\vect{C} \times (\vect{A} \times \vect{B})}
\end{one-digit-list}}

\item [12.] \CenteredUnframedFixedFigure{m2gr12}{Given the vectors \vect{A} and \vect{B}:
\begin{one-digit-list}
\item [a.] Sketch the graphical representation of the vector sum of \vect{A}
and \vect{B} using the head-to-tail  addition method.
\item [b.] Using the geometrical definition of the scalar product, prove the
\Index{law of cosines}Law of Cosines for the triangle formed by the above vector sum,
\end{one-digit-list}}
\begin{one-digit-list}
\item [  ] i.e.;
%
\Eqn{}{C^2 = A^2 + B^2 - 2\,A\,B \cos\gamma\,,}
%
where \m{C} is the magnitude of \vect{C}, the sum
\m{\vect{A} + \vect{B}}, \m{\gamma} is the angle opposite side \m{C} in the triangle
\m{ABC}, and \m{A} and \m{B} are the magnitudes of \vect{A} and \vect{B},
respectively. \help{14}
\item [c.] Use the definition of the magnitude of the vector product to prove
the Law of Sines for the same triangle, i.e.;
%
\Eqn{}{\dfrac{\sin\alpha}{A}=\dfrac{\sin\beta}{B}=\dfrac{\sin\gamma}{C}}
%
where \m{\alpha}, \m{\beta} and \m{\gamma} are the angles of the triangle that are
opposite sides \m{A}, \m{B} and \m{C}.
\end{one-digit-list}

\item [13.] A vector \vect{R} has its tail at the origin and its head at the
point (2,\,5,\,-3).
Another vector \vect{S} has its tail also at the origin, while its head is at
the point (2,\,-5,\,0).
\begin{one-digit-list}
\item [a.] How long is the \m{y}-component of \vect{R}?
\item [b.] How long is the \m{y}-component of \vect{S}?
\item [c.] What are the coordinates of the head of the vector (-\vect{R})?
\item [d.] What are the coordinates of the head of the vector (or, stated more
simply, the coordinates of the vector) \vect{R} + \vect{S}?
\item [e.] What are the coordinates of the vector \vect{R} - \vect{S}?
\item [f.] What is the length of the vector \vect{R} + \vect{S}?
\item [g.] What is the length of the vector \vect{S} - \vect{R}?
\item [h.] What is the length of the vector \vect{R}?
\end{one-digit-list}

\item [14.] Express vector \vect{R} (of problem~13 above) as a single sum of
components along the coordinate axes in  terms of the unit vectors along
those axes.
%
\Footnote{PS1}{Authors use various notations for Cartesian unit vectors.
We use \m{\uvec{x}}, \m{\uvec{y}} and \m{\uvec{z}} for unit vectors along the
\m{x}-, \m{y}-, and \m{z}-axes, which is a commonly used notation.
Some authors have used: \m{\uvec{i}}, \m{\uvec{j}},
\m{\uvec{k}}; \m{\uvec{e}_1}, \m{\uvec{e}_2}, \m{\uvec{e}_3}; and \m{\uvec{u}_x},
\m{\uvec{u}_y}, \m{\uvec{u}_z}.
All such sets are equivalent.}

\item [15.] Similarly express the vector 3\vect{S}.

\item [16.] Do the same for the vector 3\m{\vect{R} - 2\vect{S}}.

\item [17.] What angle does the vector \m{\vect{R} + \vect{S}} make with the
\m{z}-axis?

\item [18.] What angle does the vector \m{\vect{R} - \vect{S}} make with the
\m{x}-axis?

\item [19.] What is the cosine of the angle between the vector \vect{R} and
the \m{y}-axis?

\item [20.] Vector \vect{R} has its tail at the origin of the coordinate
system, while its head is at the point \m{(2,\,5,\,-3)}.
Vector \vect{S}\, likewise has its tail at the origin of the coordinate
 system, while its head is at point \m{(2,\,-5,\,0)}.
\begin{one-digit-list}
\item [a.] What is the value of \m{\vect{R} \cdot \vect{S}}?
\item [b.] Using the result of (a), find the angle between the vectors
\vect{R} and \vect{S}.
\item [c.] Evaluate the magnitude of the vector \m{\vect{R} \times \vect{S}},
using the result of (b).
\end{one-digit-list}

\item [21.] Vector \vect{T} is 4\unit{units} long and directed along the \m{x}-axis.
Vector \vect{V} is 6\unit{units} long and is directed along the \m{z}-axis.
\begin{one-digit-list}
\item [a.] What is the direction of the vector \m{\vect{T} \times \vect{V}}?
\item [b.] What is the magnitude of the vector \m{\vect{T} \times \vect{V}}?
\item [c.] Express \m{\vect{T} \times \vect{V}} in terms of the unit vectors along
the coordinate axes.
\end{one-digit-list}

\item [22.] Evaluate:
\begin{one-digit-list}
\item [a.] \m{\uvec{x} \times \uvec{z}}
\item [b.] \m{\uvec{y} \cdot  \uvec{z}}
\item [c.] \m{\uvec{z} \times \uvec{z}}
\item [d.] \m{\uvec{y} \times \uvec{z}}
\item [e.] \m{\uvec{x} \cdot  \uvec{x}}
\item [f.] \m{\uvec{x} \times \uvec{y}}
\end{one-digit-list}
\end{one-digit-list}

\newpage

\BriefAns

\begin{one-digit-list}
\item [1.] \ \CenteredUnframedFixedFigure{m2gr13}

\item [2.] \m{(0,\,6,\,0)}

\item [3.] \m{(3\sqrt{3},3,0)}

\item [4.] \m{\vect{A} = 6\uvec{y}}, \m{\vect{B} = 3\sqrt{3}\uvec{x} + 3\uvec{y}}

\item [5.] \m{(0,\,0,\,0)}

\pagebreak[4]
\item [6.] \NullItem
\begin{one-digit-list}
\item [a.] \vect{C} = resultant:

           \CenteredUnframedFixedFigure{m2gr14}
\item [b.] \m{|\vect{C}| = 6.4\unit{units}}
\item [c.] (i) {51\degrees} or 0.90\unit{rad}
\item [  ] (ii) {39\degrees} or 0.67\unit{rad}
\end{one-digit-list}

\item [7.] \NullItem
\begin{one-digit-list}
\item [a.] \m{\vect{A} = 4\uvec{x}}, \m{\vect{B} = 2.60\uvec{x} + 1.5\uvec{y})},
\m{\vect{C} = 2.5\uvec{x} + 4.3\uvec{y})}
\item [b.] \m{\vect{S} = \vect{A} + \vect{B} + \vect{C} = 9.1 \uvec{x} + 5.8 \uvec{y}}
\item [c.] \m{\vect{D} = \vect{A} + \vect{B} - \vect{C} = 4.1 \uvec{x} - 2.8 \uvec{y}}
\end{one-digit-list}

\item [8.] \m{\uvec{x} \cdot \uvec{x} = 1},
\m{\uvec{x} \cdot \uvec{y} = 0}, \m{\uvec{y} \cdot \uvec{y} = 1}

\item [10.] \m{|\vect{A}| = 5}, \m{|\vect{B}| = 13}, \m{\theta = 104\degrees}

\item [11.] \NullItem
\begin{one-digit-list}
\item [a.] \m{(\vect{A} \times \vect{B}) \times \vect{C} = R^3 \uvec{x}}
\item [b.] \m{(\vect{B} \times \vect{A}) \times \vect{C} = - R^3 \uvec{x}}
\item [c.] \m{\vect{A} \times (\vect{B} \times \vect{C}) = 0}
\item [d.] \m{\vect{C} \times (\vect{A} \times \vect{B}) = - R^3 \uvec{x}}
\end{one-digit-list}

\item [12.] a. \CenteredUnframedFixedFigure{m2gr15}

\item [13.] \NullItem
\begin{one-digit-list}
\item [a.] 5\unit{units}
\item [b.] 5\unit{units}
\item [c.] \m{(-2,\,-5,\,3)}
\item [d.] \m{(4,\,0\,-3)}
\item [e.] \m{(0,\,10,\,-3)}
\item [f.] 5\unit{units}
\item [g.] \m{\sqrt{109} = 10.44\unit{units}}
\item [h.] \m{\sqrt{38} = 6.16\unit{units}}
\end{one-digit-list}

\item [14.] \m{2\uvec{x} + 5\uvec{y} - 3\uvec{z}}

\item [15.] \m{6\uvec{x} - 15\uvec{y}}

\item [16.] \m{2\uvec{x} + 25\uvec{y} - 9\uvec{z}}

\item [17.] {127\degrees}

\item [18.] \m{\pi/2}

\item [19.] 0.811

\item [20.] \NullItem
\begin{one-digit-list}
\item [a.] \m{-21\unit{units}}
\item [b.] 129.2\degrees
\item [c.] 25.71\unit{units}
\end{one-digit-list}

\item [21.] \NullItem
\begin{one-digit-list}
\item [a.] along negative \m{y}-axis
\item [b.] 24\unit{units}
\item [c.] \m{-24\uvec{y}}
\end{one-digit-list}

\item [22.] \NullItem
\begin{one-digit-list}
\item [a.] \m{-\uvec{y}}
\item [b.] 0
\item [c.] 0
\item [d.] \m{\uvec{x}}
\item [e.] 1
\item [f.] \m{\uvec{z}}
\end{one-digit-list}

\end{one-digit-list}

}% /Sect
