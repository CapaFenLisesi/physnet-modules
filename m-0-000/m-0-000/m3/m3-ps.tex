\revhist{9/13/89, pss; 6/30/91, pss; 7/30/92, pss; 3/23/94, pss; 3/23/94, pss
         9/26/94, pss; 2/4/96, pss; 9/6/96, pss; 3/13/97, pss}
%
\Sect{}{}{\SectType{ProblemSet}}{
%

\noindent Note: Problems~17-20 are used in this module's \textit{Model Exam}.

\begin{two-digit-list}
\item [ 1.] Starting from the definition of the derivative, derive the
expression for the derivative of \m{y = a x^n}, \Eqnref{B1}.
\item [ 2.] Again working directly from the definition, find the expression
which gives the derivative of \m{f(x)/g(x)} in terms of the derivatives of
\m{f(x)} and \m{g(x)}, \Eqnref{B3}.
\item [ 3.] Use the chain rule, \Eqnref{B4}, to determine the derivative of
\m{F(x) = (ax^2+bx+c)^{1/2}}.
\item [ 4.] Show that the derivative with respect to \m{\theta} of \m{\tan\theta}
is \m{\sec^2\theta}, given the derivatives of the sine and cosine.
\item [ 5.] Given the derivative of \m{\ell n\,x}, \Eqnref{B16}, determine the
expression for the derivative of \m{e^x}, simplified \Eqnref{B18}.
\item [ 6.] Evaluate the derivative of \m{(x^2+7x+2)/(x+2)} at \m{x = 1}.
\item [ 7.] Evaluate the derivative of \m{e^{5x}(x^2+5x-7)} at \m{x = 0.2}.
\item [ 8.] Show that the derivative of \m{-\csc(4\theta-1)} is
\m{4[\csc(4\theta-1)][\cot(4\theta-1)]}, given the derivatives of sine
and cosine.
\item [ 9.] Show that the slope of the function \m{y(x) = 2x^3-3x^2+4}
at \m{x = -1} is \m{+12}.
\item [10.] For the function in problem~9, show that the slope at \m{x = 1/3}
is \m{-4/3}.
\item [11.] For the function in problem~9, show that the locations of the
maximum and minimum are, respectively, (0,4) and (1,3).
With the aid of these results, sketch a graph of the function.
\item [12.] Referring to Appendix~B, show that the integral
%
\Eqn{}{\int\left[Y(x)\dfrac{dZ(x)}{dx} + Z(x)\dfrac{dY(x)}{dx}\right]\;dx}
%
is: \m{Y(x)Z(x) + C}.
\item [13.] Evaluate the integral of \m{6 x^2 - 2} and show that, if at \m{x = 2} the
integral is to have the value \m{10}, then the integral's arbitrary constant is
determined and the integral is \m{2(x^3 - x - 1)}.
\item [14.] Evaluate these definite integrals:
\begin{one-digit-list}
\item [a.] \m{\int_{-2}^{+2}\;(5x^2+x)\,dx}
\item [b.] \m{\int_0^\infty \; e^{-5y}\,dy}
\item [c.] \m{\int_0^{\pi /2}\;\cos\theta\;\sin\theta\,d\theta}
\end{one-digit-list}

\item [15.] Evaluate the integral of \m{A\,\ell n\,kx} where \m{A} and \m{k} are
constants.
Hint: Use the substitution \m{kx = e^y} and the integration by parts method of
problem~12.
\item [16.] A function \m{y(x)} has these properties:
\begin{itemize}
\item [(i)] it is zero for all values of \m{x} up to \m{x = 5}, at which point it has
the value \m{y = 10}.
\item [(ii)] from \m{x = 5} to \m{x = 20} the function falls linearly (as a
straight line) to zero, after which it is zero for all higher values of \m{x}.
\end{itemize}
Sketch the function and evaluate its integral in the interval \m{x = 0} to
\m{x = 50}.
(Hint: Make use of the geometrical interpretation of the integral.)
\item [17.] Verify that \m{f(t) = A \cos\omega t + B \sin\omega t} (where \m{A}
and \m{B} are constant and \m{\omega^2 = k/m}) is a solution of
%
\Eqn{}{m \dfrac{d^2 f(t)}{dt^2} + k f(t) = 0\,,}
%
where \m{d^2 f(t)/dt^2} is the second derivative of \m{f(t)} and \m{k} and \m{m} are
constants.
\item [18.] Evaluate the derivative of \m{(A \cos 5y)} at \m{y = \pi/20}.
\item [19.] Evaluate these integrals:
\begin{one-digit-list}
\item [a.] \m{\int A r e^{kr}\,dr} \qquad (\m{A} and \m{k} are constants) \help{4}
\item [b.] \m{\int x^{1/2}\,dx}
\end{one-digit-list}

\item [20.] For each of the equations below, find the maximum and/or minimum
points and distinguish between them.
\begin{one-digit-list}
\item [a.] \m{y(x) = x^2 + x + 10}.
\item [b.] \m{y(x) = x^3 - 3x + 2}.
\item [c.] Determine \m{A}, \m{B} and \m{C} so the function \m{Ax^3 + B x^2 + C} will
have a minimum at \m{x = 1/3}.
\end{one-digit-list}
\end{two-digit-list}

\BriefAns
\begin{two-digit-list}
\item [3.] \m{\dfrac{a x + b/2}{(a x^2 + b x + c)^{1/2}}}

\item [6.] 17/9

\item [7.] -66.33

\item [14.] (a) 80/3;  (b) 1/5 \help{6}; (c) 1/2 \help{7}

\item [15.] \m{A x [\ell n(kx) - 1] + C}

\item [16.] 75

\item [18.] \m{-5 A/\sqrt{2}}

\item [19.] \NullItem
\begin{one-digit-list}
\item [a.] \m{(A/k^2)\,(kr - 1)e^{kr} + C}
\item [b.] \m{(2/3) x^{3/2} + C}
\end{one-digit-list}

\item [20.] \NullItem
\begin{one-digit-list}
\item [a.] \m{\left(-\dfrac{1}{2}, \dfrac{39}{4}\right)}, minimum
\item [b.] \m{(-1,4)}, maximum; \m{(1,0)}, minimum
\item [c.] as long as \m{A= - 2B}, the constants can be anything.  \help{8}
\end{one-digit-list}
\end{two-digit-list}

} %/Sect

