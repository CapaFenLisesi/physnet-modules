\revhist{10/27/81; 10/5/91, pss; 9/15/94, pss; 8/16/95, pss; 11/7/95, pss; 12/20/01, pss;
         4/16/02, pss; 10/3/02, pss}
%
\Sect{1}{Introduction}{\SectType{TextMultiPara}}{
%
\pcap{1}{a}{Simplifying a Complicated Function}
We frequently encounter some complicated mathematical function,
\m{f(x)}, which is valid over a wide range of \m{x}, in a situation where we
would prefer a simpler polynomial even if it were a good approximation
only over a limited region.
For example, the force holding two atoms together in a diatomic
molecule is an exceedingly complicated function of the interatomic distance.
If we attempt to use this function directly, we find it almost impossible to
calculate the vibrational motions of the atoms.
Even if we succeed through the use of heroic computer calculations, we get
little insight.
However, if we replace the complicated function by a simple best-match
polynomial, the approximate atomic motions can be easily obtained.
Such approximate solutions are very important in applied mechanics and in
molecular chemistry and physics.
In this module we show you how to obtain the best-match polynomial for any
given (complicated) function.

\pcap{1}{b}{The Goal of This Module}
Technically, this module explains how to make a power series expansion of a
function using the Taylor series method, and how to check the validity of the expansion.
No attempt is made to give the mathematical basis for the method.
The module also uses such expansions to derive handy approximations
for the derivatives of a function.
}% /Sect
%
\Sect{2}{The Expansion of a Function}{\SectType{TextMultiPara}}{
%
\pcap{2}{a}{The Taylor Series Method}
Suppose we have a complicated function \m{f(x)} and that we know or can find 
its value and the value of each of its derivatives at some point \m{x_0}.
We can put that value and those derivatives into the Taylor Series and use
it to evaluate \m{f(x)} at any other point \m{x} that is in neighborhood of
\m{x_0}:
%
\Eqn{1}{f(x) = f(x_0) + \dfrac{f'(x_0)}{1!}(x-x_0) + \dfrac{f''(x_0)}{2!}(x-x_0)^2 + \ldots}
%
The complete definition of the Taylor Series is:
%
\Eqn{20}{f(x) = \sum_{n=0}^{n=\infty} \dfrac{f^{(n)}(x_0)}{n!}(x-x_0)^n\,,}
%
where \m{f^{(n)}(x_0)} is the \m{n^{th}} derivative of \m{f(x)}, evaluated at \m{x_0}.
The quantity \m{n!} is the factorial (\m{n! = 1 \cdot 2 \cdot 3 \cdots (n-1) \cdot n}).

\pcap{2}{b}{Truncating the Series}
The Taylor series is an infinite series, but if \m{x} is near to \m{x_0},
the series can usually be truncated (cut-off) after a few
terms and this leaves a simple polynomial.

\pcap{2}{c}{Example of a Taylor Series Expansion}
Here is a specific example: Suppose you wish to have a power series
expansion of \m{f(x) = \sin x} about the point \m{x_0 = 0}, where \m{x} is in
radians.
To make the power series we evaluate the terms of the Taylor Series:
%
\FiveEqns{2}{f(x)       = + \sin x & \Longrightarrow f(0)     = 0   }
            {f'(x)      = + \cos x & \Longrightarrow f'(0)    = 1   }
            {f''(x)     = - \sin x & \Longrightarrow f''(0)   = 0   }
            {f'''(x)    = - \cos x & \Longrightarrow f'''(0)  = - 1}
            {f''''(x)   = + \sin x & \Longrightarrow f''''(0) = 0   }
%
Putting these and further derivatives into \Eqnref{1} gives us the power series:
%
\Eqn{3}{\sin x = x - \dfrac{x^3}{6} + \dfrac{x^5}{120} - \ldots}
%
For cases where \m{x} is much smaller than 1, \m{x^3} is much smaller than
\m{x} and hence \m{x^5} is much much smaller than \m{x}.
Then we can truncate the series after the first term:
%
\Eqn{4}{\sin x \simeq x \qquad (x^2 \ll 1)\,.}
%

\pcap{2}{d}{Convergence of the Series}
How rapidly does the series approach the correct answer?
Technically, this question is phrased: \Quote{How rapidly does the series
converge?}
As an example, suppose \m{x = \pi/6 = 30\degrees} so that \m{\sin x = 0.500}.
Then various truncations of the series for the expansion of \m{\sin x}
about the origin gives:

\renewcommand{\arraystretch}{1.25}\begin{center}
\begin{tabular}{|c | l | l | l | l|}\hline
TERMS & APPROXIMATION & SUM & ERROR & \% ERROR \\ \hline
1     &      \m{x}              & 0.523599 & \m{+ 0.023599} & \m{+ 5}      \\ \hline
2     & \m{x - x^3/6}           & 0.499674 & \m{- 0.000326} & \m{- 0.07}   \\ \hline
3     & \m{x - x^3/6 + x^5/120} & 0.500002 & \m{+ 0.000002} & \m{+ 0.0004} \\ \hline
\end{tabular}
\end{center}
%
Since the series rapidly approaches the exact answer, it is said to
\Quote{converge rapidly} for \m{x = \pi/6}.

\pcap{2}{e}{Always Check For Convergence}
A check for convergence should always be made:
the range of \m{\Delta} for which the series converges may be
unexpectedly small.
%
\Footnote{1}{This would be due to a nearby singularity in the complex plane.
See \Quote{Some Simple Functions in the Complex Plane} (MISN-0-59) for some
highly visual examples.}
%
The usual method of checking is to calculate the next term or two and
see whether the additional contribution to the sum is or is not
significant at the desired level of accuracy.
}% /Sect
%
\Sect{3}{Approximations for Derivatives}{\SectType{TextMultiPara}}{
%
\pcap{3}{a}{Rewriting the Expansion}
In order to work with finite difference equations we usually write
Taylor's series in an especially convenient notation.
%
First we substitute \m{x = x_0 + \Delta} into \Eqnref{1}:
%
\Eqn{6}{f(x_0 + \Delta) = f(x_0) + \dfrac{f'(x_0)}{1!}\Delta +
%
                              \dfrac{f''(x_0)}{2!}\Delta^2 + \ldots}
%
A similar expression is written with \m{x = x_0 - \Delta}.
The subscript on \m{x_0} is then dropped and the two equations
are written as one:
%
\Eqn{7}{f(x\pm\Delta) = f(x)\pm\dfrac{f'(x)}{1!}\Delta+
\dfrac{f''(x)}{2!}\Delta^2\pm\dfrac{f'''(x)}{3!}\Delta^3+\ldots}
%
We say \Quote{\m{f} at \m{x} plus or minus delta equals \m{f} at \m{x} plus or minus \m{f}
prime at \m{x} over one factorial times delta plus \ldots}

\pcap{3}{b}{A Finite Difference Approximation for \m{f'(x)}}
An approximation for \m{f'(x)} can be obtained by subtracting
\m{f(x - \Delta)} from \m{f(x + \Delta)}:
%
\Eqn{8}{f(x+\Delta)-f(x-\Delta)=
          2\dfrac{f'(x)}{1!}\Delta+2\dfrac{f'''(x)}{3!}\Delta^3+\ldots}
%
For small \m{\Delta}, the \m{\Delta^3}, \m{\Delta^5}, etc. terms will
be much smaller than the \m{\Delta} term .
Truncating and solving for \m{f'(x)} gives:
%
\Footnote{2}{Note the similarity of this approximate result to the precise
definition of the derivative:
%
\Eqn{}{f'(x) = \lim_{\Delta x \rightarrow 0} \dfrac{f(x+\Delta x) -
            f(x)}{\Delta x}.}}
%
%
\Eqn{9}{f'(x)\simeq\dfrac{f(x+\Delta)-f(x-\Delta)}{2\Delta}\,.}
%

\pcap{3}{c}{Approximations for Higher Derivatives}
An approximation for \m{f''(x)} can be obtained by adding \m{f(x + \Delta)}
to \m{f(x - \Delta)}.
Truncating and solving for \m{f''(x)} gives:
%
\Eqn{10}{f''(x) \simeq \dfrac{f(x+\Delta) - 2 f(x) +
                                     f(x-\Delta)}{\Delta^2}\,.}
%
Approximations for higher derivatives can be developed by similar techniques.
You might wish to verify that: \help{1}
%
\Eqn{11}{f'''(x)\simeq\dfrac{f(x+2\Delta)-2f(x+\Delta)+2f(x-\Delta)-
                     f(x-2\Delta)}{2\Delta^3}}
%

\pcap{3}{d}{A Handy Check of Formal Derivatives}
%
\Equationsref{9}, \Eqnssref{10}, and \Eqnssref{11} offer a quick
%
calculator method for checking your work when formally differentiating
complicated functions.
Merely choose some convenient point \m{x} and a small value for \m{\Delta} and
use them to evaluate the finite difference derivative.
Then evaluate your formal derivative at the same point \m{x} and compare the
two answers.

\pcap{3}{e}{A Calculator Caution}
Truncation and rounding errors may lead to incorrect results if \m{\Delta} is
too small.
%
For example, in \Eqnref{9} the numerator is small through almost exact
cancellation of its two terms.
If \m{\Delta} is too small, the terms will be the same to as many significant
digits as the calculator carries.
The net result for the numerator is then zero!
A slightly larger \m{\Delta} will result in a non-zero numerator but one with
little accuracy.
Watch out for this effect whenever using the finite difference equations.
}% /Sect
%
\Sect{}{Acknowledgments}{\SectType{Acknowledgments}}{\NsfAcknowledgment
Ray G. Van Ausdal made valuable editorial contributions.}% /Sect

