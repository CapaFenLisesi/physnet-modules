\revhist{2/19/85, mpm; 7/23/85, pss; 9/13/88, pss; 10/7/91, pss; 9/16/94, pss;
         11/7/97, lae; 10/9/02, pss}

\Sect{}{}{\SectType{SpecialAssistance}}{

\begin{one-digit-list}
\item [1.] A LOOK BACK AT UNDERSTANDING FORCE
\begin{one-digit-list}
\item [a.] Statement and example \dotfill AS1
\item [b.] Properties \dotfill AS2
\item [c.] Example 1 \dotfill AS2
\item [d.] Example 2 \dotfill AS2
\item [e.] Example 3 \dotfill AS3
\end{one-digit-list}
\item [2.] UNDERSTANDING A FORCE DIAGRAM
\begin{one-digit-list}
\item [a.] Statement and Example \dotfill AS5
\item [b.] Applicability \dotfill AS6
\item [c.] Problem Example \dotfill AS6
\end{one-digit-list}
\item [3.] UNDERSTANDING EQUILIBRIUM
\begin{one-digit-list}
\item [a.] Statement and Example \dotfill AS7
\item [b.] Interpretation \dotfill AS7
\item [c.] Problem Solution \dotfill AS8
\end{one-digit-list}
\item [4.] UNDERSTANDING TORQUE
\begin{one-digit-list}
\item [a.] Statement and Example \dotfill AS8
\item [b.] Properties \dotfill AS9
\item [c.] Interpretation \dotfill AS9
\item [d.] Comparison \dotfill AS10
\item [e.] Problem Example \dotfill AS10
\end{one-digit-list}
\end{one-digit-list}

\AsSect{1}{A Look Back at Understanding Force}{
%
\xpcap{1}{a}{Statement and example}
A force is a push or pull which alone would cause an object to accelerate.
For instance, if when you push on a closed but unfastened, door expect it to
acquire a velocity.
Your push remains a force even if the door is latched and so does not move.

\xpcap{1}{b}{Properties}
Force is a vector quantity and hence is not specified unless we
know both its magnitude and direction.
An arrow over a letter, such as \vect{F}, is the symbol for a vector.
In the same context, the letter standing alone, \m{F}, or the vector symbol
confined in bars \m{|\vect{F}|}, represents the magnitude of the vector \vect{F}.
An arrow is used to represent a force graphically.
Forces may be added vectorially and/or resolved into components.
The resultant of several given forces has components which are exactly equal
to the sum of the components of the original forces.
Common units of force are newtons (N) and pounds (lb).

\xpcap{1}{c}{Example 1}
Determine which of these statements completely specifies a force:
\begin{itemize}
\item [a.] \vect{F} (on a diagram)
\item [b.] \m{F} (on a diagram)
\item [c.] 10 pounds acting vertically upward
\item [d.] 10 pounds acting horizontally
\item [e.] magnitude and direction
\end{itemize}
Answers: a, c and e

\CaptionedFullFramedFigure{S1}{Resolution of the force acting on the crate into components.}{m5gr28}

\xpcap{1}{d}{Example 2}
A crate is being pulled up a ramp which makes an angle of {10.0\degrees} with
the horizontal.
If the force acting is \m{1.00 \times 10^2} pounds directed
{30\degrees} above the ramp, what are the components of this force
parallel to the ramp and perpendicular to the ramp?

Solution: First construct a drawing (see \Figref{S1}), to be sure that
we understand the problem.
To use regular \m{x},\m{y} components, superimpose an \m{x},\m{y}
coordinate system such that the origin is at the point of
application of the force, the \m{x}-axis is directed up the ramp, and
the \m{y}-axis is perpendicular to the ramp.
The component parallel to the ramp is F\m{_x\uvec{x}} and the
component perpendicular to the ramp is F\m{_y\uvec{y}}.

Using the right triangles formed, we can write:
%
\Eqn{}{F_x = (100.\unit{lb}) \cos 30\degrees = 86.6\unit{lb}, \qquad
F_y = (100\unit{lb}) \sin 30\degrees = 50.0\unit{lb},}
%
where \m{F_x} and \m{F_y} are the magnitudes of the components.

The vector components are:

\noindent
\m{F_x \uvec{x} = 86.8 \uvec{x}} pounds (directed up the ramp
in the positive \m{x}-direction);

\noindent
\m{F_y \uvec{y} = 50.0 \uvec{y}} pounds (directed perpendicular to the
ramp in the positive \m{y}-direction).

Three significant figures are kept because the values given in the
problem had three significant figures.

\CaptionedFullFramedFigure{S2}{Force vectors to be added.}{m5gr29}

\xpcap{1}{e}{Example 3} Add the three forces shown in \Figref{S2} by using
vector components.
Express your final answer by giving the magnitude and direction of
the resultant.
Let: \m{F_1 = 1.00 \times 10^2\unit{N}},
     \m{F_2 = 8.0 \times 10^1\unit{N}},
     \m{F_3 = 8.0 \times 10^1\unit{N}}.

First, recall the properties of vector addition using vector
components.
The resultant will be determined by equations
%
\TwoEqns{21}{\sum F & = \left( \sum F_x^2 + \sum F_y^2 \right)^{1/2}\,,}
            {\tan\theta & = \dfrac{\sum F_y}{\sum F_x}\,,}
%
where
%
\TwoEqns{22}{\sum F_x & = F_{1x} + F_{2x} + F_{3x}\,,}
            {\sum F_y & = F_{1y} + F_{2y} + F_{3y}\,.}
%
Therefore, we begin by finding the components of the given forces
and then add these to find the components of the resultant.
%
\ThreeEqns{23}{F_{1x} & = F \cos 30\degrees = (100\unit{N})(0.866) = 86.6\unit{N}}
              {F_{1y} & = F_1 \sin 30\degrees = (100\unit{N})(0.500) = 50.0\unit{N}}
              {F_{2x} & = - F_2 \cos 45\degrees = -(80\unit{N})(0.707) = - 57\unit{N}}
\ThreeEqns{}{F_{2y} & = F_2 \sin 45\degrees = (80\unit{N})(0.707) = 57\unit{N}}
            {F_{3x} & = - F_3 \cos 45\degrees = - (80\unit{N})(0.707) = - 57\unit{N}}
            {F_{3y} & = - F_3 \sin 45\degrees = - (80\unit{N})(0.707) = - 57\unit{N}}
%
The negative sign indicate that these particular components are in
either the negative \m{x} or negative \m{y} directions.

Hence, for example, \m{F_{2x} \uvec{x} = - 57\unit{N} \uvec{x} =
57\unit{N}\,(-\uvec{x})} means a 57\unit{newton} force in the negative
\m{x}-direction.
Now, if we add the \m{x}-components algebraically, we get
%
\Eqn{}{\sum F_x = (86.6 - 57 - 57)\unit{N} = - 27.4\unit{N}\,.}
%
Similarly
%
\Eqn{}{\sum F_y = 50\unit{N}\,.}

Note that the \Eqnsref{22} written in symbolic form have all positive
signs.

It is only when we substitute in numbers that we introduce the
negative signs to indicate the negative direction [\Eqnsref{23}].

\Figureref{S3} shows these components and their resultant,
\m{\sum \vect{F}}.

Using \Eqnsref{21},
%
\TwoEqns{}{\left| \sum \vect{f}\,\right| & = \left[ (-27.4)^2 + (50)^2 \right]^{1/2}\unit{N}}
          {                             & = 57\unit{N}\,,}
%
and
%
\Eqn{}{\tan\theta = \dfrac{50\unit{N}}{27.4\unit{N}}\,;}
%
\Eqn{}{\theta = \tan^{-1}\left( \dfrac{50\unit{N}}{27.4\unit{N}}
\right) = 61\,.}
%
Hence \m{\sum \vect{F}} is 57\,N directed {61\degrees} above the
negative \m{x}-axis.

We have retained only two significant figures in the resultant
because the data had only two significant figures in most cases.

\TwoCaptionedFramedFigures{S3}{The components and resultant of the forces shown in \Figref{S2}.}{m5gr30}%
                          {S4}{Force diagram for a three hinged door.}{m5gr31}
}% /AsSect
%
\AsSect{2}{Understanding a Force Diagram}{

\xpcap{2}{a}{Statement and Example}
A force diagram is a drawing showing an isolated object and all forces
acting on it.
For example, an ordinary door which is hung by three hinges has four forces
acting on it.
The force diagram is shown in \Figref{S4}.
Note that the forces exerted by the hinges are represented by wavy arrows
with round tips to indicate unknown magnitudes and directions.

\TwoCaptionedFramedFigures{S5}{Child on a Swing.}{m5gr32}%
                          {S6}{Force diagram for the child.}{m5gr33}

\xpcap{2}{b}{Applicability}
In studying a force system for the possible determination
of some unknown forces, we begin by constructing a force diagram.
Actual calculations of unknown forces are carried out in Sect.\,3, the next
section.

\xpcap{2}{c}{Problem Example}
A child of known weight is swinging on a swing which
consists of a swing board supported by two ropes.
A side view is shown in \Figref{S5} where the two ropes are treated as one for
consideration of motion in the plane of the swinging.

When the swing is located at {30\degrees}, draw two force diagrams, one for
the child and one for the seat of the swing.

Assume that the ropes are straight and that they do not exert any force
directly on the child.
Under the conditions given, the child experiences only two forces:
weight, \m{\vect{W}_c}, and the force exerted by the seat, \m{\vect{F}_s}.
The weight has a known magnitude and direction while the seat force has
unknown direction and magnitude.
\Figureref{S6} shows the force diagram for the child.\smallskip

For the swing seat there are three forces acting: the force exerted by the
child, \m{\vect{F}_c}, the force exerted by the ropes, \m{F_r}, and the weight of
the seat, \m{\vect{W}_s}.
The forces, \m{\vect{F}_r} and \m{\vect{W}_s} have known directions.
\m{\vect{F}_c} has unknown magnitude and direction.
\Figureref{S7} shows the force diagram for the swing seat.

The forces \m{\vect{F}_s} and \m{\vect{F}_c} discussed above are action-reaction
forces between the child and the seat.
Recall that a single isolated force cannot exist: every force must have a
partner that has the same magnitude but is in the opposite direction.
This application of Newton's third law of motion tells us that:
\m{\vect{F}_s = - \vect{F}_c}.

\CaptionedLeftFramedFigure{S7}{Force diagram for the seat.}{m5gr34}
} /AsSect
%
\AsSect{3}{Understanding Equilibrium}{
%
\xpcap{3}{a}{Statement and Example}
The only condition necessary for the static 
equilibrium of a point acted upon by forces or for the static equilibrium
of an extended, rigid object acted upon by concurrent forces is that the
vector sum of the forces on the object is zero.
(There will be no need to use rotational considerations in the cases
included in this section.)

For example, a picture with a cord attached near each side is hung by
catching the cord over a nail on the wall (see \Figref{S8}).
The nail is in equilibrium.
The picture is an extended object which is in equilibrium under the action
of concurrent forces.
These forces are not acting at a point but have lines of action that
intersect at a common point, the nail.

\xpcap{3}{b}{Interpretation}
\Figureref{S8} shows a view from behind the picture.
There are three concurrent forces acting on the picture: two of the forces
are exerted by the cord and the third by the weight.
(Any reaction of the wall directly on the picture itself has been
neglected.)

The force diagram showing the concurrent forces is presented in \Figref{S9}.
The vector equilibrium equation is:
%
\Eqn{24}{\vect{F}_A + \vect{F}_B + \vect{W} = 0\,.}

\xpcap{3}{c}{Problem Solution}
The algebraic equations which we would use to solve
for the unknown magnitudes of \m{\vect{F}_a} and \m{\vect{F}_b}, given the weight
of the picture, \m{W = 10\unit{pounds}}, are components of \Eqnref{24},
%
\TwoEqns{25}{x\text{-dir:} & \  F_a \cos 50\degrees - F_b \cos 50\degrees = 0,}
            {y\text{-dir:} & \  F_a \sin 50\degrees + F_b \sin 50\degrees - 10\unit{lb} = 0\,.}
%
(Remember that when putting the selected information from the force diagram
into the component equations we use negative signs to indicate negative \m{x-}
or \m{y}-directions.)
From the first of \Eqnsref{25}, \m{F_a = F_b} and using this in the second
equation yields
%
\Eqn{}{F_a \sin 50\degrees + F_a \sin 50\degrees - 10\unit{lb} = 0\,,}
%
or
%
\Eqn{}{F_a = \dfrac{10\unit{lb}}{2\sin 50\degrees} = 6.5\unit{lb}\,.}
%
Since \m{F_a = F_b}, we get: \m{F_b = 6.5\unit{pounds}}.

\TwoCaptionedFramedFigures{S8}{A hanging picture.}{m5gr35}%
                          {S9}{Force diagram for picture.}{m5gr36}
}% /AsSect
%
\AsSect{4}{Understanding Torque}{
%
\xpcap{4}{a}{Statement and Example}
Torque is defined as the cross product of two vectors,
\m{\vect{r} \times \vect{F}}.
As an example consider exerting a 500\unit{newton} force on a flag pole as shown in
\Figref{S10}.
The torque about the base of the pole produced by this force has a magnitude
%
\Eqn{}{\tau = r F \sin\theta\,.}
%
A diagram showing the force \vect{F} and the position vector \vect{r} is
given in \Figref{S11}.
The magnitude of \vect{F} is 500\unit{newtons} and the magnitude of \vect{r} is%
3.0\unit{meters}.
Then the magnitude of the torque is
%
\ThreeEqns{}{\tau & = (3.0\unit{m})(500\unit{N}) \sin 120\degrees}
            {     & = (3.0\unit{m})(500\unit{N})(0.866)}
            {     & = 1.3 \times 10^3\unit{N\,m}\,.}

\CaptionedLeftFramedFigure{S10}{Flagpole with force acting.}{m5gr37}

\xpcap{4}{b}{Properties}
Torque is a vector quantity with units of newton-meter or pound-feet.
The common symbol is the Greek letter \m{\tau}.
The direction of \vect{\tau} is determined by the right-hand rule.
For example, in \Figsref{S10} and \Figssref{S11} the torque vector is directed into the
page but this actually means a tendency to rotate clockwise.
(A torque directed into the page does not mean that the flagpole tends to
turn into the page.)

\xpcap{4}{c}{Interpretation}
The position vector, \vect{r}, runs from the reference
point \m{O} to any point on the line of action of the force (not necessarily to
the point of application of the force).
The magnitude of \vect{\tau} is \m{r F \sin\theta}, where \m{r} is the magnitude
of the position vector, \m{F} is the magnitude of the force, and \m{\theta} is
the angle between the vectors \vect{F} and \vect{r}.
When the directions of \vect{r} and \vect{F} are fixed (\m{\theta} = constant)
the magnitude of the torque is directly proportional to \m{r} and \m{F}.
(These are magnitudes.)

For example if \vect{F} is constant but is moved twice as far from the
reference point \m{O} as it was originally, the torque will double.
Or if \vect{r} is constant and the force is doubled, the torque will double.
The magnitude of the torque can also be written as \m{\tau = h F}, where
\m{h(= r \sin\theta)} is the lever arm, or as \m{\tau = r F_\perp}, where \m{F_\perp}
is the component of \vect{F} perpendicular to \vect{r}.
The lever arm is the perpendicular distance from the reference point \m{O} to
the line of action of the force.
Hence another view of torque is that for a given force the torque is directly
proportional to the lever arm.

\xpcap{4}{d}{Comparison}
Force and torque are both vector quantities.
For a system of forces acting on the body, the resultant force tells if the
body has a translational acceleration while the resultant torque tells if
the body has a rotational acceleration.
%
\Footnote{AS1}{See \Quote{Non-Concurrent Forces; Centers of Force, Gravity, Mass}
MISN-0-6.}

\TwoCaptionedFramedFigures{S11}{Diagram of force and position vector.}{m5gr38}%
                          {S12}{Lower leg, foot, and deLorme boot with weight.}{m5gr39}

\xpcap{4}{e}{Problem Example}
You have just torn the cartilage in your knee as the result of a skiing
accident.
What is in store for you?
You may have surgery to remove the cartilage and then as soon as possible
begin exercises to restore your leg to full strength and normal functioning.
The exercise can be done by using a DeLorme boot and weights.
In \Figref{S12} assume the total weight, \m{W} of the boot assembly to be
40~pounds.
Calculate the torque produced by the weight about your knee (reference point
\m{O}) when
\begin{one-digit-list}
\item [a.] your foot hangs straight down,
\item [b.] your foot is raised to the point where your lower leg makes a
           {45\degrees} angle with the horizontal and 
\item [c.] your lower leg is extended horizontally.
\end{one-digit-list}
We are only concerned with the weight \m{W} and reference point \m{O} located
2.0~feet away.
\Figureref{S13} shows the three cases in question.

\CaptionedFullFramedFigure{S13}{Force diagram, three leg positions.}{m5gr40}

To calculate torque, we begin by looking at the definition of torque,
%
\Eqn{}{\vect{\tau} = \vect{r} \times \vect{F}\,.}
%
\xpcap{}{}{Part a} Foot hangs straight down (\Figref{S13}a):
%
\Eqn{}{\vect{\tau}_1 = \vect{r}_1 \times \vect{W}\,.}
%
The magnitude of \m{\vect{\tau}_1} is \m{\tau_1 = r_1 W \sin\theta}.
But in this case, \m{\vect{r}_1} and \vect{W} have the same direction and hence
the angle between then is zero degrees.
The result is \m{\tau_1 = 0}, because the \m{\sin 0\degrees = 0}.

\xpcap{}{}{Part b} Raised {45\degrees} (\Figref{S13}b):
%
\Eqn{}{\vect{\tau}_2 = \vect{r}_2 \times \vect{W}\,,}
%
with magnitude
%
\Eqn{}{\tau_2 = r_2 W \sin \theta_2 = (2.0\unit{ft})(40\unit{lb}) \sin 45\degrees =
(2.0)(40)(0.707)\unit{ft\,lb} = 57\unit{ft\,lb}\,.}
%
The direction of \m{\vect{\tau}_2} is into the page (use the right-hand rule).

\xpcap{}{}{Part c} Horizontal (\Figref{S13}c):
%
\Eqn{}{\vect{\tau}_3 = \vect{r}_3 \times \vect{W}\,,}
%
with magnitude
%
\Eqn{}{\tau_3 = r_3 W \sin\theta_3 = (2.0\unit{ft})(40\unit{lb}) \sin 90\degrees =
80\unit{ft\,lb}\,.}
%
The direction of \m{\tau_3} is also into the page.

The implications of this for determining muscle strength can be considered
following a discussion of rotational equilibrium.
%
\Footnote{AS2}{See \Quote{Static Equilibrium, Centers of Force, Gravity and Mass}
(MISN-0-6).}
}% /AsSect

\AsItem{1}{TX-2d}
{Analyze the forces exerted on the pulley attached to the bottom of the foot:
 they should add to zero so the foot doesn't accelerate off somewhere
 (that would be quite a surprise to the patient!).

 The tensions in the rope are just \m{W}, the Weight's weight.

 The force exerted by the leg on the pulley is 22~pounds, in some unknown
 direction below the horizontal (we can designate it by some symbol).
 Thus there are forces acting in three directions on the pulley.
 They should add to zero for static equilibrium.
 Note that static equilibrium will occur unless the leg is pulled off the
 patient: what we are solving for is static equilibrium with the angles and
 leg-force desired.
}
}% /Sect