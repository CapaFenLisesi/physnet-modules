\revhist{12/29/87, pss; 3/16/88; 9/13/88, pss; 4/1/91, pss; 2/24/93, pss;
         9/15/94, pss; 12/2/94, pss; 4/14/95, pss; 9/20/95, lae; 2/22/99, pss;
         10/26/01, pss; 10/3/02, pss; 10/9/02, pss; 10/10/02}
%
\defModTitle{\ph{Tools for Static Equilibrium}}
\defCtAuthor{Leonard \inits{M.}Valley, S\inits{t.}John's University}
\defIdAuthor{Leonard M.\,Valley, Physics Dept., St.\,John's Univ.,
Collegeville, MN}
%
\defIdItems{
    \IdVersEval{10/10/2002}{1}
    \IdHours{1}
    \begin{InputSkills}
    \item [1.]  Given two or more forces, find their vector sum \prrqone{0-14}.
    \item [2.]  Resolve any force into perpendicular components \prrqone{0-72}.
    \item [3.]  Evaluate the vector product of two vectors \prrqone{0-2}.
    \item [4.]  State the relationship between torque and angular acceleration
    \prrqone{0-33}.
    \item [5.]  State the relationship between force and acceleration
    specifically including contact forces \prrqone{0-16}.
    \end{InputSkills}
    %
    \begin{KnowledgeSkills}
    \item [K1.] Distinguish a set of concurrent forces from a set of
    nonconcurrent ones.
    \item [K2.] State the equilibrium condition for an object which is acted
    upon by a set of concurrent forces.
    \end{KnowledgeSkills}
    %
    \begin{ProblemSolvingSkills}
    \item [S1.] Given an object in static equilibrium and its environment,
    determine the forces that act on the object and draw a one-body force
    diagram (\Quote{free-body diagram}) indicating these forces.
    \item [S2.] Using the force diagram for an object, apply the conditions for
    static equilibrium to determine the unknown magnitudes and/or directions of
    forces in a set of concurrent forces.
    \end{ProblemSolvingSkills}
    %
    \begin{PostOptions}
    \item [1.]  \Quote{Static Equilibrium; Centers of Force, Gravity, and Mass}
    (MISN-0-6).
    \end{PostOptions}
}