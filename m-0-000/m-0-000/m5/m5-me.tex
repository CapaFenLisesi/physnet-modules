\revhist{1/31/85, mpm; 7/23/85, pss; 10/7/91, pss; 9/16/94, pss}

\Sect{}{}{\SectType{ModelExam}}{

\begin{one-digit-list}
\item [1.] See Output Skills K1-K2 in this module's \textit{ID Sheet}.

\item [2.] \CenteredUnframedFixedFigure{m5gr22}{A \m{1.0 \times 10^2\unit{pound}} crate rests on an inclined
           plane that makes an angle of {20.0\degrees} with the horizontal.
           What are the components of the weight parallel and perpendicular to
           the incline?}

\item [3.] As the Quadriceps tendon is stretched over the patella (knee cap),
           it makes angles of 39 and 79 degrees with the horizontal.
           The tension in the tendon is 250\unit{lb}.
           Find the force exerted by the bones on the patella (\m{F_c}).
           (Data from the Michigan State University Dept. of Biomechanics).
           \CenteredUnframedFixedFigure{m5gr18}

\item [4.] \CenteredUnframedFixedFigure{m5gr23}{The Queen Elizabeth is being maneuvered into her berth in Los
           Angeles.
           Each of the four ocean-going tugs assisting her exerts a force of
           \m{5.0 \times 10^4\unit{lb}}.
           Find the resultant torque about the point \m{O}.
           (Data from C.\,Lyman, Naval Architect, South Bristol, Maine).}
\end{one-digit-list}

\BriefAns

\begin{one-digit-list}
\item [1.] See this module's \textit{text}.

\item [2.] See this module's \textit{Problem Supplement}, problem~9.

\item [3.] See this module's \textit{Problem Supplement}, problem~5.
           
\item [4.] See this module's \textit{Problem Supplement}, problem~10.
\end{one-digit-list}
}% /Sect
