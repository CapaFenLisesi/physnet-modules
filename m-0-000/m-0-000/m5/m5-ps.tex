\revhist{1/31/85, mpm; 10/6/91, pss; 9/16/94, pss; 12/2/94, pss; 9/21/95, pss;
         11/7/97, lae}

\Sect{}{}{\SectType{ProblemSet}}{

\noindent Problems~5, 9 and 10 also occur in this module's \textit{Model Exam}.

\begin{two-digit-list}
\item [1.] You are pulling someone on a sled across a horizontal field.
           If you are exerting a force of 50 pounds directed {30\degrees}
           above the horizontal, what are the horizontal and vertical
           components of this force?

\item [2.] \ItemFigure{Three people pull on ropes attached to a single
           point on a trunk as shown in the figure.
           The ropes are in the same vertical \m{xy}-plane.
           The forces are: \m{F_1 = 10\unit{lb}} at an angle of {30\degrees} with
           the \m{x}-axis; \m{F_2 = 25\unit{lb}}, upward; and \m{F_3 = 30\unit{lb}} at an
           angle of {220\degrees} with the \m{x}-axis.
           Find the \m{x}-component of the resultant force; the \m{y}-component
           of the resultant force; the magnitude and direction of the
           resultant force.}{m5gr14}

\item [3.] \ItemFigure{A Titan rocket booster (with payload) weighs
           250,000\unit{lb}.
           The engines produce a total thrust of 400,000\unit{lb}.
           Due to air resistance there is a force of 10,000\unit{lb} acting at a
           {15\degrees} angle (as shown).
           If the angle of the thrust, from the horizontal (the dotted line
           in the sketch), is \m{\phi = 85\degrees}, find the resultant force
           on the rocket.
           (Data from NASA).}{m5gr15}

\item [4.] \ItemFigure{A patient needs cervical traction to set a jaw
           fracture.
           The desired upward force is 6\unit{lb}.
           What weight is needed to produce this?
           Treat this problem as if all forces act at one point.
           Ignore the net horizontal force of the device.}{m5gr16}

\item [5.] As the Quadriceps tendon is stretched over the patella (knee cap),
           it makes angles of 39 and 79 degrees with the horizontal.
           The tension in the tendon is 250\unit{lb}.
           Find the force exerted by the bones on the patella (\m{F_c}).
           (Data from the Michigan State University Dept. of Biomechanics).
           \CharacterUnframedFigure{m5gr17}
           \CharacterUnframedFigure{m5gr18}

\item [6.] A coupling for a \Quote{semi} truck-trailer is shown
           in the sketch.
           The forces at \m{B} and \m{C} are directed along
           (\m{\cos 45\degrees \uvec{x} + \sin 45\degrees \uvec{z}}) and are
           500\unit{lb\,in} magnitude.
           The force at \m{D} is directed along
           (\m{\cos 30\degrees \uvec{x} + \sin 30\degrees \uvec{y}}) and is
           750\unit{lb}. in magnitude.
           Find the resultant force on the coupling, due to these three
           forces, and the resultant torque about the origin on it.
           \CenteredUnframedFixedFigure{m5gr19}

\item [7.] \ItemFigure{The beam and weight in the figure are supported by
           a cord that makes an angle of {60\degrees} with the wall.
           What torque does the cord produce on the beam about the point \m{O}
           if the tension in the cord is 100\unit{N} and the beam is 1.5\unit{meters}
           long?}{m5gr20}

\item [8.] In the sketch, a uniform horizontal bar 10\unit{ft} long, weighing
           5\unit{lb}, is hinged about a horizontal axis at \m{O} and is acted on
           by three additional forces.
           All forces are in the vertical plane.
           Find the total torque tending to rotate the bar about \m{O}.
           Assume the weight vector of the beam to act at its center.
           \CenteredUnframedFixedFigure{m5gr21}

\item [9.] \ItemFigure{A \m{1.0 \times 10^2\unit{lb}} crate rests on an
           inclined plane which makes an angle of {20.0\degrees} with the
           horizontal.
           What are the components of the weight parallel and perpendicular
           to the incline?}{m5gr22}

\item [10.] \ItemFigure{The Queen Elizabeth is being maneuvered into her
            berth in Los Angeles.
            Each of the four ocean-going tugs assisting her exerts a force
            of \m{5.0 \times 10^4\unit{lb}}.
            Find the resultant torque about the point \m{O}.
            (Data from C.\,Lyman, Naval Architect, South Bristol, Maine).}
            {m5gr23}
\end{two-digit-list}

\BriefAns

\begin{two-digit-list}
\item [1.] \ItemFigure{%
            \m{\text{horizontal component } = (50\unit{lb})(\cos 30\degrees) = 43.3\unit{lb}}\newline
            \m{\text{vertical component } = (50\unit{lb})(\sin 30\degrees) = 25\unit{lb}}.}
           {m5gr24}

\item [2.] \m{\vect{F}_1 = 10\unit{lb} (\cos 30\degrees \uvec{x} + \sin 30\degrees \uvec{y}})\newline
           \m{\vect{F}_2 = 25\unit{lb}(\uvec{y})}\newline
           \m{\vect{F}_3 = 30\unit{lb}(\cos 220\degrees \uvec{x} + \sin 220\degrees \uvec{y})}\newline
           \m{\sum\vect{F} = (-14.3 \uvec{x} + 10.7 \uvec{y})\unit{lb}},
                                   \m{|\vect{F}| = 17.9\unit{lb}}; \m{\theta = 143.2\degrees}

\item [3.] \m{\vect{T} = 400,000\unit{lb} \uvec{y}}\newline
           \m{\vect{W} = 250,000\unit{lb}\,[\cos(180\degrees + \phi) \uvec{x}
                         + \sin(265\degrees) \uvec{y}]}\newline
           \m{\vect{F} = 10,000\unit{lb}\,[\cos (270\degrees +
                 15\degrees) \uvec{x} + \sin(285\degrees) \uvec{y}]}\newline
           \m{\vect{T} + \vect{W} + \vect{F} =
                          (-1.92 \times 10^4\unit{lb}) \uvec{x} +
                          (1.41 \times 10^5\unit{lb}) \uvec{y}}\newline
           \m{|\vect{T} + \vect{W} + \vect{F}| = 1.42 \times 10^5\unit{lb}}\newline
           \CenteredUnframedFixedFigure{m5gr25}

           OR:

           \CenteredUnframedFixedFigure{m5gr26}
           \m{\vect{T} = (400,000\unit{lb})\,[\cos(180\degrees -
                      \phi)\,\uvec{x} + \sin(95\degrees)\,\uvec{y}]}\newline
           \m{\vect{F} = (10,000\unit{lb})\,[\cos(275\degrees +
               15\degrees)\,\uvec{x} + \sin(290\degrees)\,\uvec{y}]}\newline
           \m{\vect{W} = (250,000\unit{lb})(-\uvec{y})}\newline
           \m{\vect{T} + \vect{F} + \vect{W} = (-3.14 \times 10^4 \uvec{x} +
                                       39 \times 10^5 \uvec{y})}\unit{lb}\newline
           \m{|\vect{T} + \vect{F} + \vect{W}| = 1.42 \times 10^5\unit{lb}}

\item [4.] \ItemFigure{\m{6\unit{lb} = T (2 + \cos 45\degrees)}\newline
                       \m{T = 2.22\unit{lb} = W}}{m5gr27}

\item [5.] \m{\vect{T}_1 = (250\unit{lb})\,[\cos(141\degrees) \uvec{x} +
           \sin(141\degrees) \uvec{y}]}

           \m{\vect{T}_2 = (250\unit{lb})\,[\cos(-79\degrees) \uvec{x} +
           \sin(-79\degrees) \uvec{y}]}

           \m{\vect{T}_1 + \vect{T}_2 + \vect{F}_c = 0}

           \m{\vect{F}_c = -(\vect{T}_1 + \vect{T}_2) =
           (146.6 \uvec{x} + 88.1 \uvec{y})\unit{lb}}

\item [6.] \m{\vect{F}_B = (500\unit{lb})\,[\cos(45\degrees) \uvec{x} +
           \sin(45\degrees) \uvec{z}]}

           \m{\vect{F}_C = \vect{F}_B}

           \m{\vect{F}_D = (750\unit{lb})\,[\cos(30\degrees) \uvec{x} +
           \sin(30\degrees)] \uvec{y}}

           \m{\sum \vect{F} = \vect{F}_B + \vect{F}_C + \vect{F}_D =
                  1357\unit{lb} \uvec{x} + 375\unit{lb} \uvec{y} +
                                                707\unit{lb} \uvec{z}}

           \m{\vect{r}_B = 3\unit{in} \uvec{x} + 4\unit{in} \uvec{y} +
           2\unit{in} \uvec{z}}

           \m{\vect{r}_C = 3\unit{in} \uvec{x} + 4\unit{in} \uvec{y} -
           2\unit{in} \uvec{z}}

           \m{\vect{r}_D = 4\unit{in} \uvec{x}}

           \m{\vect{\tau}_B = \vect{r}_B \times \vect{F}_B = \left|
           \begin{array}[c]{c c c}
              \uvec{x} & \uvec{y} & \uvec{z} \\
                3     &    4    &      2    \\
               353.6  &    0    &    353.6
           \end{array} \right| \unit{in\,lb}}

           \m{\phantom{\vect{\tau}_B} = \left[
           (1414) \uvec{x} + (-353.6) \uvec{y} +
           (-1414) \uvec{z} \right]\unit{in\,lb}}

           \m{\vect{\tau}_C = \vect{r}_C \times \vect{F}_C = \left|
           \begin{array}[c]{c c c}
              \uvec{x} & \uvec{y} & \uvec{z} \\
                3     &    4    &    - 2    \\
               353.6  &    0    &    353.6
           \end{array} \right| \unit{in\,lb}}

           \m{\phantom{\vect{\tau}_C} = \left[
           (1414) \uvec{x} + (-1768) \uvec{y} +
           (-1414) \uvec{z} \right]\unit{in\,lb}}

           \m{\vect{\tau}_D = 4\unit{in} \uvec{x} \times (750\unit{lb})
           (\sin 30\degrees \uvec{y}) = (1500\unit{in\,lb}) \uvec{z}}

           \m{\sum\vect{\tau} = (2828\unit{in\,lb}) \uvec{x} +
             (-2121\unit{in\,lb}) \uvec{y} + (-1328\unit{in\,lb}) \uvec{z}}

\item [7.] \m{\vect{\tau} = \vect{r} \times \vect{F}}

           \m{\phantom{\vect{\tau}} =
                (1.5\unit{m}) \uvec{x} \times (100\unit{N})
                [\cos(150\degrees) \uvec{x} + \sin(150\degrees) \uvec{y}]}

           \m{\phantom{\vect{\tau}} = (75\unit{N\,m}) \uvec{z}} (out of page)

\item [8.] \m{\sum \vect{\tau} = r_A F_A \uvec{z} + r_{W_3} W_3 \uvec{z} +
                              r_T T \sin(45\degrees) \uvec{z} +
                      \left( \dfrac{\ell}{2} \right) (W_B) \uvec{z}}

           \m{\phantom{\sum \vect{\tau}} = (-10\unit{ft})(-4\unit{lb}) \uvec{z}
              + (-7.5\unit{ft})(-3\unit{lb})\,\uvec{z}}

           \m{\phantom{\sum \vect{\tau} = } + (- 5\unit{ft})(10\unit{lb})
                                          (\sin45\degrees) \uvec{z} +
                                 (- 5\unit{ft})(- 5\unit{lb}) \uvec{z}}

           \m{\phantom{\sum \vect{\tau}} = 52\unit{ft\,lb} \uvec{z}}

\item [9.] 93.97\unit{lb} perpendicular and 34.20\unit{lb} parallel

\item [10.] \m{ - 1.01 \times 10^7\unit{ft\,lb} \uvec{z}}
\end{two-digit-list}
}% /Sect