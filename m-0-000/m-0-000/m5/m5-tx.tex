\revhist{2/18/85 mpm; 7/12/85, pss; 10/6/91, pss; 9/15/94, pss; 11/7/97, lae; 10/3/02, pss;
         10/10/02, pss}
%
\Sect[\Index{force diagram}\Index{diagram, force}]{1}{Force Diagrams}{\SectType{TextMultiPara}}{
%
\CaptionedFullFramedFigure{1}{Typical examples of static equilibrium:
(a) A weight suspended from cords;
(b) A ladder resting against a wall;
(c) A uniform beam held by a cord and a support.}{m5gr01}

\pcap{1}{a}{The Importance of Force Diagrams}
In simplest terms, a system in \Index{equilibrium, static}\Index{static equilibrium}static equilibrium is one that does not move.
\Figref{1} shows examples of common types of systems studied in this
subject.

We never begin calculations to determine unknown forces in such
systems as those shown in \Figref{1} without first constructing
a one-body force diagram.
This gives us added assurance that we understand the system being analyzed
and that we are using all the forces pertinent to our calculations.
In addition, such diagrams aid us in determining whether the forces acting
are concurrent or non-concurrent.
The forces acting on an object are concurrent if the lines of action of the
forces all meet at a common point (otherwise they are non-concurrent).
In \Figsref{2} and \Figssref{4} dashed lines extended from the arrows indicate the lines of
action.
In \Figref{1}a the forces are concurrent while in \Figref{4} the forces on the
beam are non-concurrent.
The importance of this distinction will be considered in a later section.

\pcap{1}{b}{Example: A Hanging Weight}
\Figref{1}(a) shows a weight suspended by cords connected to point \m{P} and
attached at two points \m{A} and \m{B} on the ceiling.
Our force diagram for the point \m{P} is shown in \Figref{2}.
The three cords are the only force contacts with point \m{P}.
If we know the weight being supported we can represent the tension in the
vertical cord by a force \vect{W} acting downward on \m{P}.
For the other two cords we know the directions of the forces but not the
magnitudes.
They are drawn as wavy arrows to indicate the known directions but unknown
lengths.
Note: The directions are discerned from certain properties of cords.
A flexible cord can only sustain a tension, that is, it can pull on point \m{P}
but not push.
Furthermore, the cord will align itself in the same direction as the force
which it exerts.

\TwoCaptionedFramedFigures{2}{Force diagram for point \m{P} of \Figref{1}(a).}{m5gr02}%
                          {3}{Force diagram of the ladder in \Figref{1}(b).}{m5gr03}

\pcap{1}{c}{Example: A Leaning Ladder}
\Figref{1}(b) shows a ladder of given weight \m{W} resting against a wall.
In this case we choose the ladder as the object of interest.
We assume the weight of the ladder to act at its center \m{C}.
% 
\Footnote{1}{The reasons are examined in \Quote{Static Equilibrium} (MISN-0-6).}
%
The other two force contacts are at the floor and the wall.
\Figref{3} shows our force diagram for the ladder.
\m{\vect{F}_a} is the force exerted on the ladder by the floor and \m{\vect{F}_b} is
the force exerted on the ladder by the wall.
The direction and magnitude of forces \m{\vect{F}_a} and \m{\vect{F}_b} may be unknown.
We use wavy arrows to indicate a lack of knowledge of both direction
and magnitude.

\CaptionedLeftFramedFigure{4}{Force Diagram for the beam in \Figref{1}(c).}{m5gr04}

\pcap{1}{d}{Example: A Balanced Beam}
\Figref{1}(c) shows a uniform beam resting on a support near one end and
suspended by a cord near the other end.
We construct a force diagram for the beam.
Again we assume the weight to act at the center of the beam.
In this case the cord is assumed to be vertical and therefore the forces
exerted by the cord and the support are both vertically upward.
In \Figref{4} these two forces are shown as \m{\vect{F}_a} and \m{\vect{F}_b}.
We use wavy arrows to indicate the unknown magnitudes but known direction.
\Figref{4} shows our force diagram.

\pcap{1}{e}{Action/Reaction Forces}

The concept of \Index{forces, action-reaction}\Index{action-reaction forces}action-reaction forces is useful in determining the forces
that act on different parts of a system.
Experiment and intuition shows us that when one object exerts a force on a
second object, the second object always reacts by exerting a force on the
first.
The forces in this \Quote{pair} always have the same magnitudes but opposite
directions.
For example, looking at \Figsref{1}(b) and \Figssref{3}, \m{\vect{F}_a} is the force that the
floor exerts on the ladder and hence the ladder pushes on the floor with a
force of \m{- \vect{F}_a} (same magnitude, opposite direction).
This was summarized by Newton in his \Index{Newton's third law of motion}third law of motion: for every
action there is always an equal and opposite reaction.

\tryit A lamp hangs from a ceiling, supported by two strings (see \Figref{5}a).
The tension in string \m{AB} is 4.0\unit{lb}, while the tension in string \m{BC} is
5.35\unit{lb}.
Construct the force diagram for point \m{B}.

\tryit A telecommunications tower is erected (see \Figref{5}b.).
If the tension in the left side cable is 7500\unit{lb}, construct the force
diagram for the tower.
(Data from Cook Communications Inc., Lansing, MI)

\tryit For more examples see this module's \textit{Special Assistance
Supplement}, Sect.\,2-3.

\CaptionedFullFramedFigure{5}{(a) A lamp hanging from a ceiling; (b) a telecommunications tower.}{m5gr05}
}% /Sect
%
\Sect[\Index{forces, concurrent}\Index{concurrent forces}]{2}{Concurrent Forces}{\SectType{TextMultiPara}}{
%
\pcap{2}{a}{Static Equilibrium for a Point}
Newton stated his first law as something like this: every body persists in
its state of rest or constant velocity unless it is compelled to change
by forces applied to it.
One interpretation of \Index{Newton's first law of motion}Newton's first law as applied to objects staying at
rest is: the net \Index{force}force (the vector sum of all forces) acting on the body
must be zero.
In \Figsref{1}(a) and \Figssref{2} we show a point \m{P} which is at rest and hence the net force
must be zero.
In other words, the \Index{forces, vector addition of}vector sum of the forces acting must be zero:
%
\Eqn{1}{\vect{F}_a + \vect{F}_b + \vect{W} = 0}
%
Because the forces all act at the same point \m{P}, this is the only condition
necessary for \m{P} to remain at rest.
This is the condition for static equilibrium of a point.

\pcap{2}{b}{Concurrent Forces on an Extended Object}
There is a special case of static equilibrium for extended (non-point)
objects: if all the lines of force intersect at some point \m{P}, the forces act
just as if they all originated at \m{P}.
Even if \m{P} is geometrically not located on the object, as in \Figref{6}, this
statement is still true (if it makes you feel better, you can consider the
object to be part of a much larger rigid structure that includes point \m{P}).
Such forces are said to be \Quote{\Index{forces, concurrent}\Index{concurrent forces}concurrent.}

Based on the above, we now state this equilibrium condition: If the
forces acting on a \Index{rigid body, equilibrium condition for}\Index{equilibrium condition, for rigid body}rigid object are concurrent, then the only condition
necessary for equilibrium is that the vector sum of the forces is zero.
In the case of the object in \Figref{6},
%
\Eqn{2}{\vect{F}_a + \vect{F}_b + \vect{W} = 0\,.}
%
is the equation for equilibrium.

In general, if a rigid object is acted upon by N concurrent forces the
equilibrium condition gives us:
%
\Eqn{3}{\vect{F}_1 + \vect{F}_2 + \ldots +
                   \vect{F}_N \equiv \sum_{i = 1}^N \vect{F}_i = 0\,.}

\CaptionedLeftFramedFigure{6}{Forces can intersect outside the object.}{m5gr06}

\pcap{2}{c}{Cartesian Component Equations}
If \m{\vect{F}_R = 0}, \m{F_{Rx}} and \m{F_{Ry}} must also be zero.
Thus, taking components of \Eqnref{3} gives:
%
\Eqn{}{F_{Rx} = F_{1x} + F_{2x} + \ldots + F_{Nx} =
                                     \sum_{i = 1}^N F_{ix} = 0\,,}
%
\Eqn{4}{F_{Ry} = F_{1y} + F_{2y} + \ldots F_{Ny} =
                              \sum_{i = 1}^N F_{iy} = 0\,.}
%
The equilibrium condition, \Eqnref{1}, in \Index{equilibrium condition, in component form}component form is:
%
\Eqn{}{F_{ax} + F_{bx} + W_x = 0\,,}
%
\Eqn{5}{F_{ay} + F_{by} + W_y = 0\,.}
%
These equations involve components of vectors and hence
there may be positive or negative values.
In this case the values for \m{F_{ax}} and \m{W_y} will be negative while the
others will be positive.
(\m{W_x = 0}, as there is no component of weight in the \m{x}-direction).
Equationsref{5} become:
%
\Eqn{}{- F_a \cos 60\degrees + F_b \cos 30\degrees = 0\,,}
%
\Eqn{6}{ F_a \sin 60\degrees + F_b \sin 30\degrees - W = 0\,.}
%
We thus have two equations and two unknowns, allowing us to solve for \m{F_a}
and \m{F_b} if given \m{W}.

\CaptionedLeftFramedFigure{7}{A traction device for setting broken bones.}{m5gr07}

\pcap{2}{d}{Example: A Broken Leg}

\tryit A Patient with a broken leg is in the Health Center.
To allow his leg to set, a \Quote{Bucks} traction device (nothing to do with the
cost involved) is used, as shown in \Figref{7}.
If the traction required for the bone to set is 22\unit{lb}, find the weight
needed to produce this traction.
(Data from the Michigan State University College of Osteopathic Medicine.)
(Ans. 14\unit{lb}; \help{1} near the end of the \textit{Special Assistance
Supplement}).

\tryit For more examples see this module's \textit{Special Assistance
Supplement}, Sect.\,4.

\CaptionedFullFramedFigure{8}{Illustration of force amplification.}{m5gr08}

\pcap{2}{e}{Example: A Ditched Car}

\tryit A driver runs off the road in a snowstorm (the kind we have
in April) and becomes stranded in a ditch.
She knows she can't use brute strength to pull the car out, so she tries an
alternate method.
First, she ties a length of rope to a huge boulder, and then connects it to
her car.
When a 40~lb force is exerted perpendicular to the rope, the rope is
displaced to an angle of {6\degrees} (see \Figref{8}).
What force is then being exerted on the car?
(Data from Southend Total, Lansing, MI)
(Ans.: 192\unit{lb}.)
}% /Sect
%
\Sect[\Index{non-concurrent forces}]{3}{Non-Concurrent Forces}{\SectType{TextMultiPara}}{
%
\CaptionedLeftFramedFigure{9}{Force, \vect{F}, with given line of
                              action and position vector \vect{r}.}{m5gr09}

\pcap{3}{a}{Torque for Non-Concurrent Forces}
If the forces acting on a \Index{rigid body}body are not concurrent, there is a possibility
that the object is not in rotational equilibrium.
Consider a door you are pushing open.
The forces exerted by the hinges and your hand are not concurrent and the
door is rotating.
In contrast, the beam shown in \Figsref{1} and \Figssref{4} is acted upon by forces which
are not concurrent and it is not rotating.
Thus a rigid body which is acted upon by a set of non-concurrent forces
may or may not rotate.
The statement of the conditions necessary to extend static equilibrium to
include non-rotation uses the concept of torque.
In anticipation of the discussion of rotational equilibrium in another
module,
%
\Footnote{2}{\Quote{Static Equilibrium, Center of Mass} (MISN-0-6).}
%
we here define torque and calculate its value for simple cases.

\pcap{3}{b}{Definition of Torque}
The \Index{torque}torque, sometimes called the \Quote{\Index{moment, of a force}\Index{force, moment of a}moment} of a force, is defined to be
(see \Figref{9}):
%
\Eqn{7}{\vect{\tau} = \vect{r} \times \vect{F}\,,}
%
where \vect{\tau} is the torque, \vect{r} is the position vector from the
\Quote{center of torque} to any point on the \Index{line of action, of force}line of action of the force, and
\vect{F} is the force.
Torque is a vector because it is the vector product (\Quote{cross product}) of
\vect{r} and \vect{F}.
%
\Footnote{3}{See \Quote{Vectors: Sums, Differences, and Products} (MISN-0-2).
Actually, torque is a psuedovector, which means that it behaves like a
vector except under a mirror-like inversion.}
%
From the definition of the vector product, the magnitude of
\m{\vect{r} \times \vect{F}} is \m{r F \sin\theta} and its direction is determined
by the right-hand rule (into the page in this case).
\m{F\,\sin\theta} is just the component of the force \vect{F} perpendicular
to the position vector \vect{r}.
Letting \m{F\,\sin\theta \equiv F_{\perp}}, the magnitude of the torque is
just \m{r\,F_{\perp}}.

\CaptionedFullFramedFigure{10}{A force, \vect{F}, shown with three possible position vectors.}{m5gr10}

\pcap{3}{c}{Torque is Independent of Choice for \vect{r}}
The vector \vect{r} that is used to find a torque is not unique.
\Figref{10} shows only three of an infinite number of \vect{r} vectors which
could be drawn from point \m{O} to the line of action of the force \vect{F}.
Based on this diagram and the definition of torque, \Eqnref{7}, we
get:
%
\Eqn{8}{\vect{\tau} = \vect{r}_1 \times \vect{F} = \vect{r}_2 \times \vect{F} =
\vect{r}_3 \times \vect{F}\,.}
%
This \vect{\tau} is directed into the page and has magnitude:
%
\Eqn{9}{\tau = r_1\,F \sin\theta_1 = r_2\,F \sin\theta_2 =
r_3\,F \sin\theta_3\,.}
%
Using \Figref{10}, we see that:
%
\Eqn{10}{r_1\,\sin\theta_1 = r_2\,\sin\theta_2 = r_3\,\sin\theta_3 =
\ell\,,}
%
and hence from \Eqnref{9}, any one of the three could be used.
Simpler still is the equivalent expression:
%
\Eqn{11}{\tau = \ell\,F\,.}
%
Here \m{\ell}, the perpendicular distance from point \m{O} to the line of action
of the force, is called the force's \Quote{\Index{lever arm}lever arm} or \Quote{\Index{moment arm}\Index{arm, moment}moment arm.}
As a result, the magnitude of the torque is always equal to the product of
the lever arm and the magnitude of the force.

\CaptionedLeftFramedFigure{11}{Two forces producing a resultant torque.}{m5gr11}

\pcap{3}{d}{Torque from Two or More Forces}
Since torque is a vector, if there are two or more torques relative to point
\m{O}, the \Index{resultant torque}\Index{torque, resultant}resultant torque about
point \m{O} is just the vector sum of the individual torques.
From \Figref{11},
%
\TwoEqns{}{\vect{\tau}_1 & = \vect{r}_1 \times \vect{F}_1}
          {\vect{\tau}_2 & = \vect{r}_2 \times \vect{F}_2}
%
yielding a resultant torque:
\m{\vect{\tau} = \vect{\tau}_1 + \vect{\tau}_2}.\newline

\pcap{3}{e}{Example: A Gearbox}

\tryit A reduction gearbox is shown in \Figref{12}.
The torques on it, about the point of intersection of the shafts, a point in
the \m{xz}-plane, are 20\,\uvec{z}\unit{ft\,lb} and 4\,\uvec{x}\unit{ft\,lb}.
Find the total resultant torque on the gearbox.
(Ans. The vector sum.)

\pcap{3}{f}{Example: A Test of Strength}

\tryit At a local count fair, a barker challenges you to the \Quote{game} shown
in \Figref{13}.
You place your wrist in the strap, flex your arm, and try to exert a force
of 200\unit{lb} on the spring scale.
If the distance from your wrist to your elbow is 9\unit{inches}, find the torque
exerted by the strap about your elbow and the force on your muscle.
(Ans. 1800\unit{in\,lb} and 1800\unit{lb}.)

\TwoCaptionedFramedFigures{12}{A gearbox.}{m5gr12}%
                          {13}{A test of strength.}{m5gr13}
}% /Sect
%
\Sect{}{Acknowledgments}{\SectType{Acknowledgments}}{\NsfAcknowledgment}% /Sect

