\revhist{10/12/91, pss; 9/16/94, pss; 11/13/97, pss; 12/11/2000, pss; 6/6/02, pss; 10/21/02, pss}

\Sect{}{}{\SectType{SpecialAssistance}}{

\begin{one-digit-list}
\item [1.] THE EQUILIBRIUM CONDITIONS
\begin{one-digit-list}
\item [a.] Statement \dotfill AS1
\item [b.] Properties \dotfill AS1
\end{one-digit-list}
\item [2.] APPLICATION OF THE CONDITIONS
\begin{one-digit-list}
\item [a.] Strategies for Problem Solving \dotfill AS2
\item [b.] An Example \dotfill AS3
\item [c.] Forces Add to Zero \dotfill AS4
\item [d.] Torques Add to Zero \dotfill AS5
\end{one-digit-list}
\item [3.] CENTER OF GRAVITY OR CENTER OF MASS
\begin{one-digit-list}
\item [a.] Volume Mass Distribution \dotfill AS5
\item [b.] Surface Mass Distribution \dotfill AS7
\item [c.] Constant Surface Density \dotfill AS7
\item [d.] Guessing the CM \dotfill AS8
\item [e.] Objects with Holes \dotfill AS8
\end{one-digit-list}
\end{one-digit-list}

\AsSect{1}{The Equilibrium Conditions}{
%
\xpcap{1}{a}{Statement} Can you understand the two conditions that must exist for an
extended body to be in static equilibrium in a given force environment?

\xpcap{1}{b}{Properties} Remember that force and torque are both vector quantities.
In what direction is the torque associated with a given force and reference
moment arm?
Because of the vector nature of force and torque, the two equilibrium
conditions in vector notation actually become six scalar equations.

If forces are confined to a plane (say the \m{x}-\m{y} plane) then only three
equations are useful, so only three unknowns may be found.
Can you demonstrate this to yourself?
}% /AsSect
%
\AsSect{2}{Application of the Conditions}{
%
\xpcap{2}{a}{Strategies for Problem Solving}
\begin{itemize}
\item [1.] Read the problem carefully so that you understand what is given and
           what is asked for.
\item [2.] Draw a force diagram for the object in question.
           Include a representation of all forces both known and unknown.
           Remember that, in general, it takes three numbers to express both
           the magnitude and direction of a force.
           You may denote the three force components (F\m{_x},F\m{_y},F\m{_z}), or as
           the magnitude of the force and two angles that specify its
           direction, etc.
           Sometimes one force component is obviously zero and need not be
           considered further.
           Don't forget forces, such as gravity, which are imposed without
           contact.
           Note that the force of gravity upon an object can always be
           considered as acting upon the center of gravity (or the center of
           mass) of the object.
           

\item [3.] If you wish a complete solution (values found for all unknowns) you
           need as many independent equations as unknowns.
           Using the equilibrium conditions as expressed in component form,
           write the necessary equations.
           It is usually possible to obtain simple torque equations by
           carefully choosing the point about which the torques are computed.
\item [4.] Solve the set of equations for the desired unknown forces.
\item [5.] Check to see that your results are reasonable, units correct, etc.
\end{itemize}
}% /AsSect
%
\xpcap{2}{b}{An Example}{

\CenteredUnframedFixedFigure{m6gr19}
Let us apply these problem solving strategies to an example.
An L-shaped object consists of two very thin uniform beams joined rigidly
so that they make an angle of {90\degrees} with each other.
One end of one of the beams is pinned to a vertical wall making a {60\degrees}
angle with the wall.

The point where the beams are joined is also tied to the vertical wall by
means of a horizontal cable (see sketch).
The beam of length \m{L_1 = 5\unit{ft}}, weighs 150\unit{pounds}, while the beam of length
\m{L_2 = 3}\unit{ft}, weighs 90\unit{pounds}.
The system is in equilibrium.

The questions we wish to answer are: (a) What is the tension in the cable; and
(b) What force does the vertical wall exert on the lower end of beam \m{L_1}?

Because the system is in static equilibrium, we know that all of the
forces and torques on the object must add to zero, so we use this
fact to find the unknown forces.
(The center-of-mass of each of the thin uniform beams is in the
middle of the beam).

\CenteredUnframedFixedFigure{m6gr20}
As has been emphasized, the first thing we must do is draw a
one-body diagram of the object under consideration (in this case
the L-shaped object) and replace each object (the wall, the cable,
gravity) in contact with it by the force this object exerts on the
L-shaped object.

These are all the forces due to external agents acting upon the
L-shaped object.
Note that because we didn't know the magnitude or the direction of
the force exerted by the wall, we have two unknowns and we've
represented these by an unknown horizontal force, \m{\vect{F}_H}, and
an unknown vertical one, \m{\vect{F}_V}.
The conditions for equilibrium are:
\begin{itemize}
\item [(i)] all the forces must add to zero (and, since no horizontal
            force can cancel a vertical one, the horizontal and vertical
            forces must separately add to zero); and
\item [(ii)] all the torques about any point (we are free to choose
        the point) must add to zero.
            We are free to choose any number of such points because the
            object has no angular acceleration about \textit{any} point.
\end{itemize}

\xpcap{2}{c}{Forces Add to Zero}
For the forces to add to zero, relative to some fixed Cartesian Coordinate
system, all of the components must separately add to zero.
If our coordinate directions are (1) the horizontal, and (2) the vertical
directions in the plane of the page, and (3) the direction perpendicular to
the page; then in the direction perpendicular to the page there are no
forces, so we need to consider only the two components in the plane of the
page.
So, 
%
\Eqn{}{F_H = T \qquad \text{and} \qquad F_V = W_1 + W_2 = 240\unit{lb}\,.}
%
Now we need to find either \m{F_H} or \m{T} and all forces will be known.

\xpcap{2}{d}{Torques Add to Zero}

\CenteredUnframedFixedFigure{m6gr21}{Taking torques about the point where \m{F_V} and \m{F_H} act on our
object, the torque due to these two forces are each zero.
The torque of \m{T} is out of the page, tending to produce clockwise rotation.
So for equilibrium the torque of \m{T} must equal the sum of the torques of \m{W_1}
and \m{W_2}.}

From the condition that the clockwise and counterclockwise torques balance:
%
\Eqn{}{h T = r_1 W_1 + r_2 W_2}
%
\Eqn{}{r_1 = \dfrac{L_1}{2} \cos 30\degrees = \sqrt{3} \dfrac{L_1}{4} = \dfrac{5 \sqrt{3}}{4}\unit{ft}\,.}
%
\Eqn{}{r_2 = L_1 \cos 30\degrees + \dfrac{L_2}{2} \sin 30\degrees =
\left( \dfrac{5 \sqrt{3}}{2} + \dfrac{3}{4} \right)\unit{ft}\,.}
%
\Eqn{}{h = L_1 \cos 60\degrees = \dfrac{L_1}{2} = 2.5\unit{ft}\,.}

This allows us to solve for \m{T},
%
\Eqn{}{2.5\unit{ft}\,T = \dfrac{5 \sqrt{3}}{4}\unit{ft} \times 150\unit{lb} +
       \left( \dfrac{5 \sqrt{3}}{2} + \dfrac{3}{4} \right)\unit{ft} \times 90\unit{lb}.}
%
\Eqn{}{T = 312.8\unit{lb} = F_H}
%
So all the forces on the L-shaped object are determined.
\m{\vect{F}_V} and \m{\vect{F}_H} may be added vectorially to find the magnitude and
direction of the single force the wall exerts on the object.
}% /AsSect
%
\AsSect{3}{Center of Gravity, Center of Mass}{
%
\xpcap{3}{a}{Volume Mass Distribution}
The center of gravity and center of mass of an extended object are defined
in different ways, but are really the same point.

The position of the center of mass of an object is defined by \Eqnref{B4}:
%
\Eqn{}{x_\text{CM} = \dfrac{1}{M} \int x\, dM}
%
\Eqn{}{y_\text{CM} = \dfrac{1}{M} \int y\, dM}
%
\Eqn{}{z_\text{CM} = \dfrac{1}{M} \int z\, dM,}
%
or, in a more compact form,
%
\Eqn{}{\vect{r}_\text{CM} = \dfrac{1}{M} \int \vect{r}\, dM.}

The integrals are taken over the entire mass of the object.
We can make the above equations easier to grasp if we replace the
\Quote{infinitesimal mass element} \m{dM} by the more physical infinitesimal volume
element \m{dV = dx\,dy\,dz}.
If the mass density of the object is \m{\rho (\vect{r}) = \rho(x,y,z)},
%
\Footnote{AS1}{\m{\rho} is a function of position---we allow the mass density to
vary across the object.}
%
in units of mass per unit volume (e.g.,\unit{kg/m\up{3}}) then the (infinitesimal)
amount of mass contained in the (infinitesimal) volume element \m{dV} around the
point \vect{r} is
%
\Eqn{}{dM = \rho\, (\vect{r}) \,dV,}
%
and so our definition of the center of mass becomes
%
\Eqn{}{x_\text{CM} = \dfrac{1}{M} \int \int \int x\, \rho(x,y,z)\,dx\,dy\,dz}
%
\Eqn{}{y_\text{CM} = \dfrac{1}{M} \int \int \int y\, \rho(x,y,z)\,dx\,dy\,dz}
%
\Eqn{}{z_\text{CM} = \dfrac{1}{M} \int \int \int z\, \rho(x,y,z)\,dx\,dy\,dz,}
%
or, in a more compact form,
%
\Eqn{}{\vect{r}_\text{CM} = \dfrac{1}{M}\int \vect{r}\, \rho (\vect{r})\,dV.}

The integrals are now over the volume of the object, i.e., the shape of the
object determines the limits on the integrals.

Since \m{M} is the mass of the whole object, \m{\rho (\vect{r})} is related to \m{M} by
%
\Eqn{}{M = \int \rho (\vect{r})\,dV.}
%
That is, \m{M} is the integral of \m{\rho} over the whole volume of the object.

\xpcap{3}{b}{Surface Mass Distribution}
A good many center-of-mass problems involve two-dimensional objects, which
make the integrations a good deal simpler.
If the surface mass density \m{\sigma (x,y)}, in units of mass per unit area
(e.g. \unit{kg/m\up{2}}) is given, then the position of the center of mass is given
by:
%
\Eqn{}{x_\text{CM} = \dfrac{1}{M} \int \int x\, \sigma (x,y)\,dx\,dy}
%
\Eqn{}{y_\text{CM} = \dfrac{1}{M} \int \int y\, \sigma (x,y)\,dx\,dy,}
%
where the integrals are over the entire area of the object, and 
%
\Eqn{}{M = \int \int \sigma (x,y)\,dx\,dy}
%
relates the total mass \m{M} to the surface mass density \m{\sigma(x,y)}.

\xpcap{3}{c}{Constant Surface Density}
Most such two-dimensional problems involve a further simplification - an
assumption that the surface mass density is a constant, rather than a varying
function of position.
The only tricky part of the problem lies in picking the right limits for the
integrals.
Problem~7 in the Problem Supplement is of just this type.
The surface mass density \m{\sigma} is a constant, so the total mass of the
object is
%
\Eqn{}{M_\text{object} = \sigma \int \int\, dx\,dy = \dfrac{1}{2} b h \sigma,}
%
since \m{b\,h/2} is the area of the object.
If we choose to integrate over \m{y} first, then the limits on the integrals
become \m{y = 0} to \m{y = (h/b)x} for \m{y}, and \m{x = 0} to \m{x = b} for \m{x}, because
the upper edge of the triangle is the line \m{y = (h/b)x}, the right edge is the
line \m{x = b}, and the bottom edge is the line \m{y = 0}.
Therefore, the position of the center of mass is given by:
%
\Eqn{}{x_\text{CM} = \dfrac{1}{b h \sigma/2} \int_0^b \,dx
           \int_0^{(h/b)x} \,dy\, x\, \sigma}
%
and
%
\Eqn{}{y_\text{CM} = \dfrac{1}{b h \sigma /2} \int_0^b \,dx
           \int_0^{(h/b)x} \,dy\, y\, \sigma}
%
The surface mass density \m{\sigma} is a constant and comes outside both
integrals, where it cancels out of the problem entirely.

We work out the integrals:
%
\FiveEqns{}%
{x_\text{CM} & = \dfrac{1}{b h/2} \int_0^b \,dx \int_0^{(h/b)x} \,dy\, \,x}
{           & = \dfrac{2}{b h} \int_0^b \,dx\,x \left| y \right|_0^{(h/b)x}}
{           & = \dfrac{2}{b h} \int_0^b \,dx\,(h/b)x^2}
{           & = \dfrac{2}{b^2} \left| \dfrac{1}{3} x^3 \right|_0^b}
{           & = \dfrac{2}{3}b,}
%
and
%
\FiveEqns{}%
{y_\text{CM} & = \dfrac{1}{b h\sigma/2}\int_0^b\,dx\,\int_0^{(h/b)x}\,dy\,y}
{           & = \dfrac{2}{b h}\int_0^b\,dx\,\left|\dfrac{1}{2}y^2\right|_0^{(h/b)x}}
{           & = \dfrac{1}{b h}\int_0^b\,dx\,(h/b)^2 x^2}
{           & = \dfrac{h}{b^3} \left| \dfrac{1}{3}x^3 \right|_0^b}
{           & = \dfrac{1}{3} h\,,}
%
%
The only conceptual problem was picking the right limits---the rest is
ordinary calculus and algebra.

\xpcap{3}{d}{Guessing the CM}
We can often guess the position of the center of mass of a simple system.
For instance, the center of mass of a sphere with a constant mass density is
just the center of the sphere.
For any object having a constant mass density, the center of mass is located
at the geometric center of the object.

\xpcap{3}{e}{Objects with Holes}
Another common type of center-of-mass problem involves objects with \Quote{holes}
or \Quote{chunks} cut out of them.
Problems~8 and 9 are of this type.
In problem~9, the object is a whole circle with a smaller circular hole cut
out of it.
We can use a simple trick to solve this type of problem.

Consider the fact that we can \Quote{reconstruct} a solid disc by adding together
the object we have in Problem~9 with the circular piece that was cut out of
the disc in order to make the object.
More specifically, the integral of the quantity \m{\vect{r} \sigma \,dx\,dy} over
the solid disc equals the integral of that quantity over the object plus the
integral of the quantity over the circular cutout:
%
\Eqn{}{\int_\text{disc} \,\vect{r}\,\sigma\,dx\,dy =
   \int_\text{object}\,\vect{r}\,\sigma\,dx\,dy +
      \int_\text{cut-out}\,\vect{r}\,\sigma\,dx\,dy.}
%
Furthermore, the centers of mass of each of the three things involved here are:
%
\Eqn{}{\vect{r}_\text{CM,disc} =
   \dfrac{1}{M_{disc}}\; \int_\text{disc} \vect{r} \sigma\,dx\,dy}
%
\Eqn{}{\vect{r}_\text{CM,object} =
   \dfrac{1}{M_\text{object}}\; \int_\text{object} \vect{r} \sigma\,dx\,dy}
%
\Eqn{}{\vect{r}_\text{CM,cut-out} =
   \dfrac{1}{M_\text{cut-out}}\; \int_\text{cut-out} \vect{r} \sigma\,dx\,dy}
%
Combining all of the above equations yields the final result:
%
\Eqn{}{M_\text{disc}\; \vect{r}_\text{CM,disc} =
      M_\text{object}\; \vect{r}_\text{CM,object} +
 M_\text{cut-out}\; \vect{r}_\text{CM,cut-out}.}
%
The masses are related by
%
\Eqn{}{M_\text{disc} = M_\text{object} + M_\text{cut-out}.}
%
In the case of problem~9, the positions of the centers of mass of the solid
disc and the cutout are obvious:
%
\Eqn{}{x_\text{CM,disc} = 1\unit{m};\;y_\text{CM,disc} = 0}
%
\Eqn{}{x_\text{CM,cut-out} = (1 + 1/2)\unit{m} = 3/2\unit{m};
y_\text{CM,cut-out} = 0\,,}
%
which are the positions of the centers of the two circles in question.

Since the object has a constant surface mass density, the masses of the
things involved are
%
\Eqn{}{M_\text{disc} = \pi\,(1\,m)^2\,\sigma = \sigma\,\pi\,m^2\,,}
%
\Eqn{}{M_\text{cut-out} =
          \pi \left( \dfrac{1}{2} m \right)^2 \sigma =
                            \dfrac{1}{4} \sigma \pi m^2\,,}
%
\Eqn{}{M_\text{object} = M_\text{whole\;circle} - M_\text{cut-out} =
                \dfrac{3}{4} \sigma \pi m^2\,.}

Therefore, using the above handy equation:
%
\Eqn{}{(\sigma \pi m^2)(1\unit{m}) =
     \left( \dfrac{3}{4} \sigma \pi m^2 \right) x_\text{CM,object} +
   \left( \dfrac{1}{4} \sigma \pi m^2 \right) \left( \dfrac{3}{2} m \right)}
%
\Eqn{}{(\sigma \pi m^2)(0) =
            \left( \dfrac{3}{4} \sigma \pi m^2 \right) y_\text{CM,object} +
            \left( \dfrac{1}{4} \sigma \pi m^2 \right) (0)\,,}

which gives
%
\Eqn{}{x_\text{CM,object} = \dfrac{\sigma \pi m^3 - \dfrac{3}{8} \sigma \pi m^3}
{\dfrac{3}{4} \sigma \pi m^2} = \dfrac{5}{6} m\,,}
%
and
%
\Eqn{}{y_\text{CM,object} = 0\,.}

One can use the same technique to solve any similar problem.
The equation used in this one particular case can be generalized to cover
any object consisting of a total object (abbreviated \Quote{total}) and a \Quote{hole}
cutout:
%
\Eqn{}{M_\text{total} \vect{r}_\text{CM,total} =
                   M_\text{object} \vect{r}_\text{CM,object} +
                           M_\text{cut-out} \vect{r}_\text{CM,cut-out}}
%
where:
\Eqn{}{M_\text{total} = M_\text{object} + M_\text{cut-out}\,.}

\AsItem{1}{PS-problem~4}
{\begin{itemize}
 \item [1.] \CenteredUnframedFixedFigure{m6gr22}{Sketch a one-body diagram of the forces on the drawbridge, giving
            the force at \m{P} an unknown magnitude \m{F} and an angle \m{\phi} above
            the horizontal.}

 \item [2.] Sum horizontal and vertical forces separately.
 \item [3.] Sum the torques about \m{P} so as to eliminate the unknowns, \m{F} and
            \m{\phi}, from the equation.
            Assign the drawbridge an unknown length \m{\ell}: it drops out.
 \end{itemize}
}

\AsItem{2}{TX-Sect.\,4c}
{The integral for \m{y_{CM}} proceeds similarly to that for \m{x_{CM}}, starting with
 %
 \Eqn{}{y_{CM} = \dfrac{1}{M} \int y\,dM\,,}
 %
 and with \m{dA = x\,dy} as a strip of area all at the same value for \m{y} as in \Figref{8}.
 Then the integral is:
 %
 \Eqn{}{y_{CM} = \dfrac{1}{A} \int_0^h y\,x(y)\,dy\,.}
 %
 We interpret this integral as: \Quote{We weight each element of area, \m{x(y)\,dy}, with its
 value of \m{y}, and integrate over the whole area.}

 \m{x(y) = b - (by/h)} so:
 %
 \Eqn{}{y_{CM} = \dfrac{2}{bh} \int_0^h y\,\left(b-\dfrac{by}{h}\right)\,dy = \dfrac{2}{h^2} \int_0^h\,(hy-y^2)\,dx =
 \dfrac{1}{3}\,h\,.}
}

}% /AsSect
}% /Sect

\endinput

\item [2.] CENTER OF FORCE \dotfill AS2

\AsSect{2}{Center of Force}{
%
A rod of length \m{\ell} has forces acting on it (see sketch).
Here \m{F_1 = F_3 = 2f} and \m{F_2 = f} where \m{f} is not specified.

\CenteredUnframedFixedFigure{m6gr18}\newline
Show that the center of force is at point \m{P}.
Recall that the center of force is a point through which the resultant force
must act in order to produce the same torque on the object as the system of
forces.

First we need to know the resultant force and torques.
Calling upward forces positive we see that \m{\vect{F}_R = -f}.
Taking torques about the point 0 at distance \m{\ell} to the left, the resultant
torque is: \m{\tau_R = F_1 \ell - F_3 (3/2\,\ell) + F_2 (2 \ell)}.
We have let clockwise torques be positive.
Inserting the values for the forces in terms of the quantity \m{f},
\m{\tau_R = f \ell}.
We now ask the question:  Where must the resultant force \m{-f} act to produce a
clockwise torque \m{f \ell}?
Clearly a downward force at \m{P} of magnitude \m{f} will produce this torque
because the moment arm will be \m{\ell}.
Thus, the center of force is located at the point \m{P}.

You may verify this result for yourself by choosing another point (the right
end of the bar, for example) about which to take torques.
}% /AsSect
