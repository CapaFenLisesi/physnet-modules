\revhist{7/23/85, mpm; 10/12/91, pss; 9/16/94, pss; 6/6/02, pss}

\Sect{}{}{\SectType{ModelExam}}{

\begin{one-digit-list}
\item [1.] See Output Skills K1-K3 in this module's \textit{ID Sheet}.

\item [2.] \CenteredUnframedFixedFigure{m6gr16}{A van is loaded so that the load on each pair of
            wheels, front and back, is the same: 2400\unit{pounds}.
            The rear axle is 12\unit{feet} behind the front axle.
            The rear overhang of the van extends two feet behind the rear axle.
            If an additional weight of 600\unit{pounds} is loaded onto the very back
            of the van what is the new load distribution of the front and rear
            tires?}

           (Be sure to draw a correct one-body diagram showing all of the
           forces acting upon the object under consideration.)

\item [3.] \CenteredUnframedFixedFigure{m6gr24}
           Determine the tensions \m{T_1} and \m{T_2} in the ropes if the mass \m{M}
           weighs 10\unit{lb}.

\item [4.] The mass of the earth is \m{5.98 \times 10^{24}\unit{kilograms}} while the
           mass of the moon is \m{7.34 \times 10^{22}\unit{kilograms}}.
           The average earth-moon separation (center-to-center) is
           \m{3.84 \times 10^8\unit{meters}}.
           Find the location of the center-of-mass of the earth-moon system.
\end{one-digit-list}

\BriefAns

\begin{one-digit-list}
\item [1.] See this module's \textit{text}.

\item [2.] See this module's \textit{Problem Supplement}, problem~10.

\item [3.] See this module's \textit{Problem Supplement}, problem~2.

\item [4.] See this module's \textit{Problem Supplement}, problem~11.
\end{one-digit-list}
}% /Sect
