\revhist{10/11/91, pss; 9/16/94, pss; 11/12/97, pss; 1/12/00, pss; 5/13/02, pss; 10/21/02, pss}
%
\Sect{1}{Introduction}{\SectType{TextMultiPara}}{
%
\pcap{1}{a}{Conditions for Motionlessness}
An important question, with widespread application in physics and
engineering, is:  Under what conditions will a motionless system not tend to
start moving in any direction or rotating about any axis; that is, what are
the conditions for \Quote{\Index{equilibrium, static}\Index{static equilibrium}static equilibrium?}
The answer is:  When the resultants of the forces and the torques acting on
the system are zero.

\pcap{1}{b}{Dealing With Extended Objects}
The objects considered in this unit are \Index{object, extendeds}extended objects; that is, they have
finite size in contrast to mathematically idealized point objects.
The study of extended objects, acted upon by gravity and able to rotate,
brings to mind this question:
How do we take into account the weight of each infinitesimable part of the object?
We certainly do not want to draw a vector force diagram for each infinitesimal element and then compute
the torque on each such element.
The answer is to use the rather simple concept of the object's \Quote{Center of Gravity,} which,
for objects small compared to the earth, is just the object's Center of Mass.
We will explore these concepts at some length and develop their use in solving static equilibrium
problems.

\CaptionedLeftFramedFigure{1}{Two equal but opposite forces that tend to rotate an object.}{m6gr01}
%
}% /Sect
%
\Sect{2}{Application to Extended Objects}{\SectType{TextMultiPara}}{
%
\pcap{2}{a}{Zero Resultant Force and Torque}
For static equilibrium to occur it is not enough that the sum of all of the
forces on an object be zero.
This can be seen by referring to the object shown in \Figref{1}.
The object is not in equilibrium in spite of the fact that the net force is zero.
Two forces act on the object.
They are equal in magnitude and oppositely directed so the resultant force is zero.
Because the forces are not along the same line, their effect is to tend to rotate the
object (clockwise for the forces shown in \Figref{1}).

The system in \Figref{1} is not in static equilibrium because
the forces produce a net \Index{torque, on extended object}torque which will be accompanied by
rotational acceleration.
Complete equilibrium will occur only when the net force and the resultant torque,
relative to any point whatever, are both zero.
Hence, with forces \m{\vect{F}_1}, \m{\vect{F}_2}, \m{\ldots}, \m{\vect{F}_n} producing
torques \m{\vect{\tau}_1}, \m{\vect{\tau}_2}, \m{\ldots}, \m{\vect{\tau}_n} relative to
some arbitrary fixed point, the conditions for equilibrium are that the resultant force,
\m{\vect{F}_R}, and the resultant torque, \m{\vect{\tau}_R}, are both zero:
%
\Eqn{1}{\vect{F}_R \equiv \sum^n_{i=1} \vect{F}_i \equiv \vect{F}_1 + \vect{F}_2 + \dots +
         \vect{F}_n = 0\,,}
%
\Eqn{2}{\vect{\tau}_R \equiv \sum^n_{i=1} \vect{\tau}_i =
          \vect{\tau}_1 + \vect{\tau}_2 + \ldots + \vect{\tau}_n = 0\,.}
%

\pcap{2}{b}{Using Vector Components}
In finding solutions to problems, component equations are often found to be more useful
than vector equations such as \Eqnsref{1} and \Eqnssref{2}.
To find the Cartesian \m{x}-component equations equivalent to \Eqnsref{1} and \Eqnssref{2}, we simply
multiply each of those equations by the \m{x}-coordinate unit vector, \uvec{x}, and obtain the
\m{x}-components of the resultant force and torque vectors:
%
\Eqn{3}{F_{Rx} = \vect{F}_R \cdot \uvec{x} = \sum^N_{i=1}(\vect{F}_i \cdot \uvec{x}) = \sum^N_{i=1}F_{ix} = 0\,,}
%
\Eqn{4}{\tau_{Rx} = \vect{\tau}_R \cdot \uvec{x} = \sum^N_{i=1}(\vect{\tau}_i \cdot \uvec{x}) = \sum^N_{i=1}\tau_{ix} = 0\,,}
%
where \m{F_{ix}} indicates the \m{x}-component of the \nth{i} force.
There are similar equations for the \m{y}- and \m{z}-components, obtained
through multiplying by the unit vectors \uvec{y} and \uvec{z} rather than by \uvec{x}.
Note that a force or torque \Emph{component} may be either positive or negative depending upon the
direction of the force in relation to the direction of the corresponding coordinate axis.
}% /Sect
%
\Sect{3}{The Center of Gravity}{\SectType{TextMultiPara}}{
%
\pcap{3}{a}{When Gravity is the Force}
%
The \Index{weight}weight of an object is a force due to the gravitational attraction
between the matter in the object and the matter in the earth.
This gravitational pull should not be thought of as a single force between
the earth and the body as whole.
Each submicroscopic particle that makes up the body experiences a gravitational
force due to the presence of the large amount of matter comprising the earth.
It is the sum of all these nearly parallel forces that make up the resultant
force we call the weight of the object.
The point at which this single force, equal to the resultant force, must act
to have the same effect as all of the forces between the constituent particles
and the earth is called the \Quote{center of gravity} (hereafter \m{CG}).

\pcap{3}{b}{The Center of Gravity}
When no forces other than gravity are acting on an object, we can always find
a point called the object's \Quote{Center of Gravity} where, for purposes of
static equilibrium, the object's mass may be considered to be concentrated.
By this we mean that we can replace the actual object by a weightless rigid structure which
has the same shape as the object but whose weight occurs entirely at the object's
Center of Gravity.
An object's Center of Gravity is a single point whose position can be calculated from the distribution
of weight within the actual object.
Once the position of the Center of Gravity, \vect{r}\dn{CG}, has been determined, the resultant gravitational torque on the object
is given by (see this module's Appendix A):
%
\Eqn{5}{\vect{\tau}_R = \vect{r}_{CG} \times W\uvec{g}\,,}
%
where \uvec{g} is a unit vector pointing downward, toward the center of the earth, and \m{W} is
the weight of the object (the weight of the object is the magnitude of the gravitational
force on the object).
If the object is to be in static equilibrium, the resultant torque must be zero.
Solving a static equilibrium problem for an extended object usually begins with
determining the location of the object's Center of Gravity.
Note that we have not specified any particular origin for computing the vector to the Center of Gravity,
so that origin could be anywhere and the equation would still be valid.
In general, the origin is chosen as a point about which the system would naturally rotate,
such as the point of suspension of the system or a point at which the system is in contact with
the ground or with a road.

\CaptionedFullFramedFigure{2}{Locating the Center of Gravity of an object by drawing vertical lines
from several different points of suspension.}{m6gr28}

\pcap{3}{c}{Locating the Center of Gravity Experimentally}
To experimentally determine the center of gravity of an object we suspend it from any point on
its surface and draw a vertical line through the object from that point downward, as in \Figref{2}a.
We then pick some different point on the surface of the object and repeat the procedure, as
in \Figref{2}b.
The point where the two lines cross is the Center of Gravity.
If we repeat the procedure again, as in \Figref{2}c, all three lines will cross at the Center of Gravity.
Any number of such lines can be drawn and the lines will all cross each other at the Center of Gravity
\Derivation{(derived in Appendix C; Appendices A and B are prerequisites)}{Appendix A}.

\tryit Cut out an oblong piece of cardboard or wood and suspend it from an edge using a thread or
string as appropriate for its weight.
Then: (1) carry out the above-outlined procedure and see for yourself that all of the lines you draw on
the object, as you suspend it from diffent points on its edge, really do cross each other at the
same point; and (2) suspend the object from your newly found center of gravity by poking a needle
or pencil through the point of intersection and show that the object has no preferred rotational
orientation about this point.

\pcap{3}{d}{Locating the Center of Gravity by Calculation}
The vector to the Center of Gravity of a system of \m{N} particles, for cases where the system is small
compared to the size of the earth, can be calculated to high accuracy from this equation for the
system's Center of Mass \Derivation{(derived in Appendix A)}{Appendix A}:
%
\Eqn{6}{\vect{r}_{CG} = \vect{r}_{CM} =
        \dfrac{1}{M}\sum^N_{i=1} m_i\,\vect{r}_i \qquad \text{(system small compared to earth)}\,,}
%
where the \nth{i} particle has mass \m{m_i} and position vector \m{\vect{r}_i}.
For an extended object we have a similar equation \Derivation{(derived in Appendix B; Appendix A is
prerequisite)}{Appendix B}:
%
\Eqn{7}{\vect{r}_{CG} = \vect{r}_{CM} = \dfrac{1}{M}\int \vect{r}\,dm(\vect{r}) \qquad \text{(object small compared to earth)}\,,}
%
where the integral is over the volume of the extended object.
%
}% /Sect

\CaptionedLeftFramedFigure{4}{Determining the equilibrium point of suspension of a
tringular sign.}{m6gr04}

\Sect{4}{A Calculational Example}{\SectType{TextOnePara}}{
The owner of the Triangle Bar wishes to erect a sign (of triangular shape,
naturally) from a single point, as shown in \Figref{4}.
The owner needs to find the distance \m{x_R} at which the sign will balance.
The sign is exceedingly small compared to the size of the earth and it is an extended object so we use
\Eqnref{7}.

\CaptionedLeftFramedFigure{7}{An element of area with constant \m{x} for determining \m{x_{CM}}
for the sign in \Figref{4}.}{m6gr25}
%
\CaptionedLeftFramedFigure{8}{An element of area with constant \m{y} for determining \m{y_{CM}}
for the sign in \Figref{4}.}{m6gr26}

We start by taking the \m{x}-component of \Eqnref{7}:
%
\Eqn{8}{x_{CG} = x_{CM} = \dfrac{1}{M} \int x\,dM\,.}
%
To convert to an integral over space, we define the surface density of the sign, its mass per
unit area, as \m{\sigma} so that \m{M = \sigma\,A} and \m{dM = \sigma\,dA}.
Here \m{A} is the area of the sign.
Then we can rewrite \Eqnref{8} as:
%
\Eqn{9}{x_{CM} = \dfrac{1}{A} \int x\,dA\,.}
%
We now interpret \m{dA} to be all parts of the area that have the same value for \m{x}.
An example is shown as the shaded area in \Figref{7}.
The size of the area is \m{dA = y\,dx} and all parts of it have the same value for \m{x}.
Therefore the integral over all area becomes:
%
\Eqn{10}{x_{CM} = \dfrac{1}{A} \int_0^b x\,y(x)\,dx\,.}
%
We interpret this integral as: \Quote{We weight each element of area, \m{dA = y(x)\,dx}, with its
value of \m{x}, and integrate over the whole area.}
Now \m{y(x) = hx/b} and \m{A = bh/2} so we get:
%
\Eqn{11}{x_{CM} = \dfrac{2}{bh} \int_0^b x\,\dfrac{hx}{b}\,dx = \dfrac{2}{b^2} \int_0^b\,x^2\,dx =
\dfrac{2}{3}\,b\,,}
%
hence:
%
\Eqn{}{x_{CG} = \dfrac{2}{3}\,b\,.}

\tryit Show that the \m{y}-position of the Center of Gravity is at \m{h/3}. \help{2}

}% /Sect
%
\Sect{}{Acknowledgments}{\SectType{Acknowledgments}}{\NsfAcknowledgment
Jules Kovacs made useful suggestions.
}% /Sect
%
\Sect{A}{Centers of Gravity and Mass for a Particle System}{\SectType{AppendixOnePara}}{
%
We treat the case of a system of point particles near the surface of the earth and we assume that
gravity is the only force acting on the particles in the system.
This means that the force on each particle in the system is just the force of gravity, the particle's
weight.
We write the weight of the \nth{i} particle as \vect{w}\dn{i}, whereupon the gravitational
torque on the \nth{i} particle becomes:
%
\Eqn{A1}{\vect{\tau}_i = \vect{r}_i \times \vect{F}_i = \vect{r}_i \times \vect{w}_i\,.}
%
Then \Eqnref{4} for the resultant torque on the system is:
%
\Eqn{A4}{\vect{\tau}_R = \sum^N_{i=1} \vect{\tau}_i =  \sum^N_{i=1} \vect{r}_i \times \vect{w}_i\,.}
%
The weights are all forces in the downward direction, toward the center of the earth.
We denote that direction with the unit vector \uvec{g} so \m{\vect{w}_i = w_i \uvec{g}} and we have:
%
\Eqn{A8}{\vect{\tau}_R = \sum^N_{i=1} \vect{r}_i \times \vect{w}_i =
                        \sum^N_{i=1} \vect{r}_i \times w_i \uvec{g} =
                        \left(\sum^N_{i=1} \vect{r}_i w_i\right) \times \uvec{g}\,,}
%
where we have factored out the cross product with \uvec{g} because it is common to all terms in the sum.
The vector sum in the parenthesis in \Eqnref{A8} is just the sum of the position vectors
to the individual particles but with each position vector weighted by the weight of the particle.
This vector sum, divided by the weight of the system, is the \Quote{Center of Gravity}
of the system:
%
\Eqn{A7}{\vect{r}_{CG} = \dfrac{1}{W}\,\sum^N_{i=1} w_i\,\vect{r}_i\,.}
%
Finally, we rewrite \Eqnref{A7} to emphasize that each particle's position vector is weighted
by the particle's fraction of the system's weight:
%
\Eqn{A9}{\vect{r}_{CG} = \sum^N_{i=1} \dfrac{w_i}{W}\,\vect{r}_i\,.}
%
Using \Eqnref{A9}, \Eqnref{A8} can be written:
%
\Eqn{A10}{\vect{\tau}_R = \vect{r}_{CG} \times W\uvec{g}\,.}
%
This demonstrates that the resultant gravitational torque on a system of particles is exactly
the same as what would be produced by a single point particle weighing \m{W}, located at the
system's Center of Gravity.

We can write the weight of each contributing particle as its mass times the acceleration of gravity,
\m{g}, and the total weight of the system, \m{W}, as \m{g} times the total mass of the system, \m{M}.
Then \Eqnref{A9} becomes:
%
\Eqn{A11}{\vect{r}_{CG} = \dfrac{1}{M} \sum^N_{i=1} m_i\,\vect{r}_i \qquad \text{(system small compared to earth)}\,.}
%
For an ordinary-size system near the surface of the earth, this equation for the Center of Gravity,
\Eqnref{A11}, no longer contains any reference to gravity.
In fact, it is the equation that defines the position vector to the \Quote{Center of Mass} of the system,
defined by giving each particle's position vector a weight equal to its fraction of the mass
of the system of particles:
%
\Footnote{3}{Note that in this sentence the word \Quote{weight} is used in the mathematical or statistical
sense, not in the gravitational sense.}
%
\Eqn{A12}{\vect{r}_{CM} \equiv \dfrac{1}{M} \sum^N_{i=1} m_i\,\vect{r}_i\,.}
%
Then one can write:
%
\Eqn{A13}{\vect{r}_{CG} = \vect{r}_{CM} \qquad \text{(system small compared to earth)}\,,}
%
}% /Sect
%
\Sect{B}{Centers of Gravity and Mass for an Object}{\SectType{AppendixOnePara}}{
Any extended object can be treated as if it is composed of many infinitesimal objects,
each having an infinitesimal mass \m{dm} that contributes to \Eqnref{A9}.
Then the sum in \Eqnref{A9} becomes an integral over the entire object:
%
\Footnote{4}{The conversion of a sum over discrete objects to an integral over infinitesimal
parts of an extended object is developed in \Modref{1}{Simple Differentiation and Integration}.}
%
\Eqn{B1}{\vect{r}_{CG} = \dfrac{1}{M}\int \vect{r}\,dm(\vect{r}) \qquad \text{(object small compared to earth)}\,,}
%
Here the integral is over the mass of each infinitesimal element of the extended object.
The infinitesimal mass of the infinitesimal element at position \vect{r} has been written as
\m{dm(\vect{r})}.
\Equationref{B1} is actually the definition of an extended object's Center of Mass:
%
\Eqn{B4}{\vect{r}_{CM} \equiv \dfrac{1}{M}\int \vect{r}\,dm(\vect{r})\,,}
%
so one can write:
%
\Eqn{B2}{\vect{r}_{CG} = \vect{r}_{CM} \qquad \text{(object small compared to earth)}\,,}

Note that \Eqnref{A10} is the same as \Eqnref{B2} so the same equation is valid for both extended
systems and discrete ones.

One can multiply \Eqnref{B1} by the unit vector in the \m{x}-direction and get the \m{x}-component
of the vector to the Center of Gravity:
%
\Eqn{B3}{x_{CG} = \dfrac{1}{M}\int x\,dm\,. \qquad \text{(object small compared to earth)}\,,}
%
and there are similar equations for the \m{y}- and \m{z}-components, obtained by using \uvec{y} and
\uvec{z} in place of \uvec{x}.
}% /Sect
%
\CaptionedLeftFramedFigure{9}{The object will experience a torque about the suspension point,
due to gravity, unless \m{\theta = 0}.}{m6gr27}
%
\Sect{C}{The Center of Gravity is Below a Suspension Point}{\SectType{AppendixOnePara}}{
The reason why a vertical line drawn downward from any point of suspension point passes through
the suspended object's the Center of Gravity can be easily seen in \Figref{9}.
The figure shows an object suspended from a point on its edge, but rotated so it is unbalanced.
By this we mean that if we let it go it will start swinging downward, rotating about the point of
suspension.
The reason it starts rotating is because there is a torque on the object equal to the vector
product of the Center of Gravity position vector and the weight vector of the object.
%
\Footnote{5}{See Sect.\,3 for the reason this is correct.}
%
In terms of magnitude:
%
\Eqn{}{\tau_R = \left| \vect{r}_{CG} \times W\,\uvec{g} \right| = r_{CG}\,W\,\sin\theta\,.}
%
where \m{\theta} is the angle between the direction of the Center of Gravity position vector and the
downward direction of the force of gravity (see \Figref{9}), \m{W} is the weight of the object,
\m{\vect{r}_{CG}} is the distance from the poi.
There is no torque on the object only if \m{\theta = 0} and that is the case if we just let the
object hang downward without rotation.
In such a case, a vertical line drawn downward from the point of suspension will certainly pass
through the Center of Gravity, no matter what suspension point is used.
}% /Sect
%
