\revhist{7/11/85; mpm; 9/13/88, pss; 12/8/89, pss; 7/5/91, pss; 10/4/94, pss;
         11/12/97, pss; 7/15/99, abs; 12/11/00, pss; 12/17/02, pss}
%
\Sect{1}{Introduction}{\SectType{TextMultiPara}}{
%
\CaptionedLeftFramedFigure{1}{Height of the bottom of a flag, as a function of
time, as it is being raised and then lowered to half mast.}{m7gr01}

\pcap[\Index{kinematics}]{1}{a}{Kinematics}
Kinematics is the study of the motion of \Index{particle}particles in terms of space and
time.
By a particle we mean an identifiable physical object with spatial
dimensions so small so that it can be located at a point in a
coordinate system.

\pcap{1}{b}{The Reason for One Dimension}
The real world consists of three space-dimensions but in this module
we will be dealing only with those \Index{one-dimensional motion}\Index{motion, in one dimension}motions that are one-dimensional,
motions that are along a straight line.
This is because motion in a straight line is the simplest
motion to analyze so its study is a good introduction to motion in general.
Furthermore, when motion does occur in more than one dimension, one often
solves for the Cartesian components of the vector quantities.
The equations for these Cartesian components have much in common with their
one-dimensional counterparts that you will see in this module.

A major reason that it is easier to begin with one-dimensional motion is
that one does not have to have a multitude of vector symbols obscuring the
other concepts that are being introduced.
To get rid of vectors, we always choose a coordinate system in which the
straight-line motion being examined is along a coordinate axis.
Then there is only one common unit vector and it multiplies
all terms in all vector equations, so it can be eliminated (\Quote{canceled})
from the equations.
Although we will thus not use vectors much in one dimension, we suggest that
when interpreting positive and negative values for quantities that have
direction, you think of those values as being multiplied by the appropriate
unit vectors.

\pcap{1}{c}{Why Forces are Not Considered}
Sometimes a person encounters the subject of kinematics for the first time
and asks a question like this:
\Quote{You calculated the acceleration of the object from its speed as a function of
time, but you didn't take into account the forces on the object and they, too, cause
acceleration.}

You do not need force in a kinematics problem because the quantities given
to you in such a problem have already taken you past the point in the calculation where
you would need to put in a force.
Thus there is no symbol representing force in any of the kinematics equations.
There do exist \emph{dynamical} equations with which you can calculate force in a kinematics problem,
but you will not be asked for force in such a problem.

\CaptionedFullFramedFigure{2}{Illustration of \Index{displacement}displacement quantities (see
text).}{m7gr02}
}% /Sect
%
\Sect{2}{Position, Displacement}{\SectType{TextMultiPara}}{
%
\pcap{2}{a}{Introduction}
In straight line motion, \Index{position}position is defined as distance along the
line of motion as measured from some chosen origin.
For example, when a flag is run up a flagpole, the position of the bottom of
the flag can be taken as its distance above the ground.
This position can be shown by a graph of height versus time (see \Figref{1}).
In this diagram the bottom of the flag reaches height \m{h_1} at time \m{t_1}
and \m{h_2} at \m{t_2}; it is then lowered, reaching \m{h_3} at \m{t_3}, after which
it remains at the half-mast position.
Since the selection of the coordinate system and its origin is arbitrary,
position may be negative or positive in value.
The standard SI unit of \Index{SI units, of length}length is the \Index{meter}meter,
where 1\unit{meter} equals 3.28\unit{feet} or 1.09\unit{yards}.

\pcap{2}{b}{Displacement is Change of Position}
Position is a vector quantity; for example, \m{\vect{r} = x \uvec{x}}.
Displacement, written \m{\Delta\vect{r}}, is defined as change in position.
For example,
%
\Eqn{1}{\Delta\vect{r} = \vect{r}_f - \vect{r}_o = (x_f - x_o)\,\uvec{x}  =
                             \uvec{x} \Delta x\,,}
%
where the subscript \m{f} indicates final position and the subscript \m{o} indicates
starting or originating position for the time interval \m{t_f - t_o}, and
\m{\uvec{x}} is a unit vector in the positive \m{x}-direction (see \Figref{2}).
}% /Sect
%
\Sect{3}{Velocity}{\SectType{TextMultiPara}}{
%
\CaptionedLeftFramedFigure{3}{Illustration of quantities used to find
average velocity (see text).}{m7gr03}

\pcap{3}{a}{Overview}
\Index{vlocity}Velocity is the time rate of change of position.
When we change position, we move.
We may move slowly or rapidly.
We may move forward or backward.
Mathematically, velocity is the rate at which one's position changes.
Since the rate at which position changes can itself be continually changing,
velocity can be different at each instant of time (think of a car speedometer
that is continually changing).
When beginning to study physics, it is sometimes quite difficult to imagine
a quantity as being defined for an infinite continuum of instants during a
finite interval of time.
In fact, Newton invented calculus just so he could deal with the real world's
infinite continuum of instants.
To make things a little easier, we will first deal with a finite number
of average quantities, then graduate to the real thing.

\pcap[\Index{velocity, average}\Index{average velocity}]{3}{b}{Average Velocity}
If a particle is at position (\m{x_o\uvec{x}}) at time \m{t_o} and at position
(\m{x_f\uvec{x}}) at a later time \m{t_f}, the average velocity over the time
interval is (see \Figref{3}): (\help{2})
%
\Eqn{2}{\vect{v}_{av}=\dfrac{\Delta \vect{r}}{\Delta t} =
              \dfrac{\left(x_f - x_o\right)\uvec{x}}{t_f - t_o} =
                                \uvec{x} \dfrac{\Delta x}{\Delta t}\,.}

\pcap{3}{c}{Instantaneous Velocity and Speed from \m{x(t)}}
The \Index{velocity, instantaneous}\Index{instantaneous velocity}instantaneous velocity, called simply \Quote{the velocity,} is the limit of the
average velocity as the length of the time interval over which one is
averaging approaches zero; that is, as \m{t_f} approaches \m{t_o}.
Dropping the unit vectors in \Eqnref{2} and taking the limit, we get:
%
\Eqn{3}{v=\lim_{\Delta t\rightarrow 0} v_{av} =
      \lim_{\Delta t\rightarrow 0} \dfrac{\Delta x}{\Delta t} \equiv
                                                       \dfrac{dx}{dt}\,.}
%
The always-positive magnitude of \m{v}, written \m{|v|}, is the \Index{speed, instantaneous}\Index{instantaneous speed}instantaneous speed,
or simply \Quote{the \Index{speed}speed.}
It is the quantity a car's speedometer is designed to display, in miles per
hour and/or kilometers per hour.
The international standard \Index{SI units, of speed}(SI) unit of speed is meters per second.

\CaptionedFullFramedFigure{4}{The small triangles show how to measure
      \m{\Delta x} and \m{\Delta t} to determine instantaneous velocities
                \m{\Delta x/\Delta t} at times \m{t_1}, \m{t_2}, \m{t_3}, \m{t_4}.}{m7gr04}

\pcap{3}{d}{Instantaneous Velocity From Position Graph}
If a graph of position versus time is constructed from a data table, or
drawn by a recording instrument, the velocity at any time can be
found graphically.
The slope of the tangent to the curve at any particular point is \m{dx/dt} at
that point and this is the instantaneous velocity at that time.

This tangent to the curve can be called the \Index{slope, physical}\Index{physical slope}physical slope to distinguish it
from a geometrical slope measured in degrees or radians.
Unlike a geometrical slope, a physical slope has units determined by
the scale of the graph, those of the ordinate divided by those of the
abscissa.
These slopes can be determined by drawing tangents to the curve at points
on the curve, and subsequently using the tangents as the hypotenuses of
right triangles that can be drawn and measured (see \Figref{4}).

\pcap{3}{e}{Units}
The standard \Index{SI units, of speed}\Index{SI units, of velocity}SI unit for speed and velocity is one meter per second,
which is approximately equal to \m{3.28\unit{feet/second}} or \m{2.24\unit{mph}}---a brisk
walking speed.
To run a four minute mile, a track star must average \m{22\unit{ft/s}} or \m{15\unit{mph}} (the
maximum speed posted for many school zones).
In SI units this is \m{6.70\unit{m/s}}.
Tropical storms are called hurricanes as soon as their winds reach \m{33\unit{SI\;units}},
\m{33\unit{m/s}}, equivalent to \m{64\unit{knots}} or \m{74\unit{mph}}.
The speed of sound is approximately \m{330\unit{SI\;units}}, \m{330\unit{m/s}}.

\TwoCaptionedFramedFigures{5}{Table graph.}{m7gr05}%
                          {6}{Getting \m{v_\text{av}}.}{m7gr06}

\pcap{3}{f}{Example}
The motion of a particle traveling along a straight line can be described
roughly by giving its position at a number of times.
Here is an example:
%
\begin{center}\begin{tabular}{|c c c c c c c|}\hline
 \m{t}(\unit{s}) & 0.10 & 0.20 & 0.30 & 0.40 & 0.50 & 0.60 \\
 \m{x}(\unit{m}) & 0.15 & 0.55 & 0.60 & 0.40 & 0.35 & 0.50 \\ \hline
\end{tabular}\end{center}
%
This information can also be shown by plotting a graph, as in \Figref{5}.
Since we believe such a particle travels smoothly, we would normally
connect the points by a smooth line as indicated.
In any case, if we collected more and more data on the particle, we could
plot more and more points until the graph took on a smooth appearance as
in \Figref{6}.

Now suppose we need to find the average velocity over the interval from
\m{t = 0.10\unit{s}} to 0.20\unit{s}.
We can use data table\,to find:
%
\TwoEqns{}{v_{av} & = \dfrac{\Delta x}{\Delta t} =
                        \dfrac{0.55\unit{m} - 0.15\unit{m}}{0.20\unit{s} -
                        0.10\unit{s}}}
          {       & = 4.0\unit{m/s}\,.}
%
Or we can measure on our (carefully constructed) graph (\Figref{6}) to discover
that:
%
\Eqn{}{v_{av}=\dfrac{\Delta x}{\Delta t} =
            \dfrac{0.40\unit{m}}{0.10\unit{s}} = 4.0\unit{m/s}\,.}
%
This is the slope of the dashed line connecting the end points of the
interval in \Figref{6}.

On the other hand, if we want the instantaneous velocity at \m{t = 0.10\unit{s}}, we
let the \m{\Delta t} in \Figref{6} shrink toward zero:
%
\Eqn{}{v(0.10\unit{s}) = \lim_{\Delta t\rightarrow 0}
        \left.\dfrac{\Delta x}{\Delta t}\right|_{0.10\unit{s}} 
            = \left.\dfrac{\Delta x}{\Delta t}\right|_{0.10\unit{s}}\,.}
%
which is just the slope of the first dashed line in \Figref{7}.
That is, the (instantaneous) velocity at any given time is the slope of the
graph, the time derivative of the function, at that time.

We can immediately see from \Figref{7} that \m{v} is positive throughout the
interval from \m{t = 0.10\unit{s}} to \m{0.20\unit{s}} (for example), because \m{x} is always
increasing with \m{t} throughout this interval.

\CaptionedLeftFramedFigure{7}{Getting \m{v(t)}.}{m7gr07}
}% /Sect
%
\Sect{4}{Acceleration}{\SectType{TextMultiPara}}{
%
\pcap{4}{a}{Overview}
The word \Quote{\Index{acceleration}acceleration} implies a change in velocity.
Thus we must associate acceleration with change in velocity over some
interval of time; we must not associate it with any one particular
\Index{acceleration, instantaneous}\Index{instantaneous acceleration}instantaneous velocity.
Both direction and magnitude of velocity change are important.
For example, a ball thrown upward into the air slows down, momentarily
stops, then picks up downward velocity, all because of the constant downward
acceleration due to gravity.

\pcap[\Index{acceleration, average}\Index{average acceleration}]{4}{b}{Average Acceleration}
If a particle has a velocity \m{v_0\uvec{x}} at time \m{t_0}, and a velocity
\m{v_f\uvec{x}} at a later time \m{t_f}, the average acceleration over that time
interval is:
%
\Eqn{4}{\vect{a}_{av}=\dfrac{\Delta \vect{v}}{\Delta t} =
         \dfrac{\left(v_f - v_o\right)\uvec{x}}{t_f-t_0} =
                           \uvec{x}\dfrac{\Delta v}{\Delta t}\,.}

\pcap[\Index{acceleration, instantaneou}\Index{instantaneous acceleration}]{4}{c}{Instantaneous Acceleration}
The instantaneous acceleration, called simply \Quote{the \Index{acceleration}acceleration,} is the
limit of the average acceleration as \m{t_f \rightarrow t_0}.
Dropping the unit vectors in \Eqnref{4} and going to the limit,
%
\Eqn{5}{a = \lim_{\Delta t\rightarrow 0}\dfrac{\Delta v}{\Delta t} \equiv
                  \dfrac{dv}{dt}=\dfrac{d^2x}{dt^2}\,.}
%
This is \Quote{the acceleration of a particle at the time \m{t_0}.}
The acceleration is in the direction of the \m{x}-axis and has the dimensions
\m{\unit{length/time\up{2}}}.
When its value is non-zero, its direction may be to the right (positive
value) or to the left (negative value).

\pcap[\Index{instantaneous acceleration, from velocity}]{4}{d}{Instantaneous Acceleration From Velocity Graph}
A curve of velocity versus time, whether the velocities are obtained from
graphs or tables, can be quite useful.
Not only does the slope give instantaneous acceleration but, as we shall see
later, the area between the velocity curve and the time axis gives the
displacement.
The slope, \m{dv/dt} (which is also \m{a}), can be determined by drawing tangents and triangles
at desired times (see \Figref{8}).
Here we drew the same shape for \m{v(t)} as we did for \m{x(t)} in \Figref{4} so as to emphasize that
acceleration relates to velocity in somewhat the same manner as velocity relates
to position.

\CaptionedFullFramedFigure{8}{The small triangles show how to
determine instantaneous acceleration \m{\Delta v/\Delta t} at times
\m{t_1}, \m{t_2}, \m{t_3}, \m{t_4}.
This is not the velocity corresponding to the displacement in \Figref{4}.}{m7gr08}

\pcap[\Index{instantaneous acceleration, from position}]{4}{e}{Instantaneous Acceleration From Position Graph}
Since the slope of the velocity curve, \m{dv/dt}, is the time rate of change of
velocity, it is \m{d^2x/dt^2} which is called the \Quote{bending function} of the
position/time curve.
It is instructive to draw separate position, velocity and acceleration
curves, one above the other, using a common time scale (see \Figref{9}).

Geometrically, \m{a(t)} is the rate of change of the slope of \m{x(t)}; it is the
rate at which that function \Quote{bends.}
For instance, in \Figref{7} the slope is positive at \m{t = 0.10\unit{s}} but negative at
\m{t = 0.30\unit{s}}.
In fact, the slope decreases continuously from \m{t = 0.10\unit{s}} to \m{t = 0.30\unit{s}}
as the curve continues to bend negatively.
Therefore the acceleration is negative throughout this interval.

In general, the acceleration \m{a} is positive where the graph of \m{x} as a
function of \m{t} bends upward (positively), like an outstretched palm, as one
proceeds to the right.
Of course \m{a = 0} where the graph is a straight line; \m{a} is negative when the
curve is bending negatively downward.

Suppose, for example, we wish to examine the motion of a photophobic bug that
continually moves in order to stay in the (noonday) shadow of a swinging
pendulum.
The bug's motion, which is technically called \Quote{simple harmonic motion}
(students may question the word \Quote{simple}), can be described by the equation:
%
\Eqn{}{x = A \sin\omega t.}
%
Here A is the farthest the bug gets from the center of its \Quote{back and forth}
travels and \m{\omega} (\Quote{omega}) is 2\m{\pi} times the bug's number of complete
circuits per unit time.

The velocity of the bug is the first derivative of position:
%
\Footnote{1}{See \Quote{Review of Mathematical Skills - Calculus: Differentiation and
Integration} (MISN-0-1).}
%
\Eqn{}{v=\dfrac{dx}{dt}=\omega A\cos\omega t\,.}
%
Its acceleration is the next derivative:
%
\Eqn{}{a=\dfrac{dv}{dt}=\dfrac{d^2x}{dt^2}=-\omega^2 A\sin\omega t\,.}
%
This can be written:
%
\Eqn{}{a = - \omega^2 x\,.}
%
\Figref{9} shows the bug's position, velocity, and acceleration as functions
of time.
You should check to see if each of the lower two curves is the slope of the
one above it, and that the third is the bending function of the first.
\Figref{10} illustrates what happens when there is constant position,
velocity, and/or acceleration.
This position curve is composed of several distinct segments, as can be seen
more easily in the velocity and acceleration curves.
%
\CaptionedFullFramedFigure{9}{Plots of bug position, velocity and
acceleration on the same time scale (see text).}{m7gr09}
%
Where the position curve is bending downward as time increases, note that
the velocity is decreasing and the acceleration is negative.
Where the position curve is bending upward as time increases, note that
the velocity is increasing and the acceleration is positive.
The acceleration is zero at the point of inflection, the point where
the bending changes from downward to upward and the acceleration from
negative to positive.
The acceleration is also zero wherever a straight line segment of the
position curve shows that the velocity is constant.

Note the difference in appearance between the curves of \Figref{9}
and the three successive parabolas on the right hand side of the
position curve in \Figref{10}.
%
\Footnote{2}{The dashed lines show where a quantity is undefined (ambiguous).
Where the velocity \Quote{is} a vertical line, the acceleration would be
infinite.
Such a situation cannot occur in real life, so such an x(t) is said to be
\Quote{unphysical.}
Nevertheless, such x(t) curves are often close enough to real-life curves so
they can be used as approximations: they are often easy to deal with
mathematically.}
%
Although the displacement curves are rather similar, the graphs of
velocity and acceleration are not, as can be easily seen by evaluating the
derivatives.
This illustrates the difficulty in accurately determining
position, velocity, and acceleration relationships from graphs.

\pcap[\Index{derivatives, higher order}\Index{higher order derivatives}]{4}{f}{Higher Order Derivatives}
Derivatives of position beyond the second can be taken and in general they
will be non-zero.
For example, the first derivative of acceleration, which is the third
derivative of position, is called the \Quote{\Index{jiggle}jiggle} or \Quote{\Index{jerk}jerk,} and it is used in
studying vibrations.
In general, one or more of the higher derivatives is of interest only when
it is directly related to some other quantity involved in the motion.

\CaptionedFullFramedFigure{10}{Concurrent plots of position,
 velocity and acceleration when one or more remains constant with time.}{m7gr10}

\pcap{4}{g}{Units}
One of the most common accelerations is that \Index{gravity, acceleration due to}\Index{acceleration, due to gravity}due to gravity near the surface
of the earth.
Generally called \Quote{\Index{g}\m{g},} this is \m{9.8\unit{m/s\up{2}}}, or \m{32\unit{ft/s\up{2}}}.
One \Index{SI units of acceleration}\Index{acceleration, SI units of}SI unit of acceleration, therefore, is about one tenth the acceleration
of gravity near the surface of the earth.
When dropped from rest near the surface of the earth, a particle undergoes
an increase in velocity of about \m{1\unit{m/s}} every tenth of a second.
Half way to the moon (a distance of \m{30\unit{earth radii}}, or \m{2\times10^7\unit{m}}), the
acceleration of gravity is about one SI unit, \m{1\unit{m/s\up{2}}}.
A particle in that vicinity and in free fall would find its velocity
increasing toward the earth at the rate of 1\unit{m/s} every second.

\pcap{4}{h}{Example} A Problem: Given that a particle moves along the \m{x}-axis
with acceleration \m{a(t) = A + B t^2}, starting from rest at \m{x = 5.0\unit{m}} at
\m{t = 0}.
Find its position at all instants of time, \m{x(t)}.
\newline\noindent{\bf Solution}: Since a = dv/dt, write:\FnRef{1}
%
\TwoEqns{}{v & = \int\,dv = \int a\,dt = \int[A + Bt^2]\,dt =
                   A \int dt + B \int t^2\,dt}
          {  & = At + \dfrac{1}{3}Bt^3 + C\,,}
%
where \m{C} is a constant that can be determined from the given initial
condition that \m{v = 0} when \m{t = 0}; \m{v(0) = 0}.
To do so, we can set \m{t = 0} in the equation above to obtain:
%
\Eqn{}{0 = 0 + 0 + C\,,}
%
so
%
\Eqn{}{v(t) = A t + B t^3/3\,.}
%
Next use \m{v = dx/dt} to obtain:
%
\Eqn{}{x = \int dx = \int v\,dt = \int[At+\dfrac{1}{3}Bt^3]\,dt =
             \dfrac{1}{2} A t ^2 + \dfrac{1}{12} B t^4 + D\,,}
%
and applying the initial conditions on \m{x} we get:
%
\Eqn{}{x(t) = \dfrac{1}{2} A t^2 + \dfrac{1}{12} B t^4 + 5.0\unit{m}\,.}
%
}% /Sect
%
\Sect{5}{\m{a(t) \rightarrow v(t) \rightarrow x(t)} Using Integration}{\SectType{TextMultiPara}}{
%
\pcap{5}{a}{Start With Acceleration}
In dynamics it is common to analyze the motion of an object by examining
its acceleration.
This is because acceleration can often be deduced from known forces, but
also because instruments that measure acceleration (\Quote{accelerometers})
are used on ships, submarines, aircraft, and rockets for \Quote{inertial
navigation.}
Accelerometers are used because they need not be in contact with the
earth.
Assuming the acceleration has been obtained as a function of time during
a journey, either by instrument or from known forces, the velocity and
position of the traveler can be obtained provided they are known for some
one time in the journey (for example, at the beginning point).

\pcap[\Index{velocity, from acceleration graph}]{5}{b}{Change in Velocity From Acceleration Graph}
The area between an acceleration curve and the time axis is the
integral \m{\int a(t)\,dt}, so this gives the change in velocity over the period
of time being used.
The sign of the area gives the sign of the acceleration, hence determines
the acceleration's direction and this can be either positive or negative.
Therefore the total or net change in velocity over any period of time
is equal to the net area that is bounded by the beginning and ending times
(see \Figref{11}).
The \Index{acceleration, average}\Index{average acceleration}average acceleration for the interval
is the change in velocity during the time interval, the net area, divided by the length of
the time interval.

\pcap[\Index{velocity, as integral of acceleration}]{5}{c}{Velocity as an Integral}
Starting with the defining equation for acceleration, \m{a(t) =  dv(t)/dt},
we change the symbol for time from \m{t} to \m{t'} and then integrate both sides
of the equation with respect to \m{t'}:
%
\Eqn{}{\int_{t_0}^t\,\left(\dfrac{dv}{dt'}\right)\,dt' =
                                      \int_{t_0}^t\,a(t')\,dt'\,.}
%
But:
%
\Eqn{}{\int_{t_0}^t\,\left(\dfrac{dv}{dt'}\right)\,dt' =
                  \int_{t_0}^t\,dv = v(t)-v(t_0) \equiv v-v_0\,.}
%
Then:
%
\Eqn{}{v - v_0 = \int_{t_0}^t\,a(t')\,dt'\,.}
%
Rearranging,
%
\Eqn{6}{v = v_0 + \int_{t_0}^t a(t')\,dt'\,.}
%
We can think of \Quote{\m{a(t')\,dt'}} as representing the change in velocity over
the small  time increment \m{dt'}.
Then we can think of summing over all such small changes in velocity made during
each of many small time increments in our interval from \m{t_0} to \m{t}.
The integral is then the limit as the size of each time increment approaches
zero so the number of such increments in our time interval goes to infinity.

\CaptionedFullFramedFigure{11}{Graph of a hypothetical \m{a(t)}.
The net area between the curve and the time axis gives the object's
change in velocity from time \m{t_0} to time \m{t_f}.}{m7gr11}

\pcap[\Index{displacement, from velocity graph}]{5}{d}{Displacement From Velocity Graph}
The net area between the \m{v(t)} curve and the time-axis is the
integral \m{\int v(t)\,dt}, and this is the displacement, the change in
position during the period concerned (see \Figref{12}).

The \Index{velocity, average}\Index{average velocity}average velocity for the interval
is the change in displacement, the net area, divided by the length of the time interval.

\CaptionedFullFramedFigure{12}{Graph of a hypothetical \m{v(t)}.
The net area between the curve and the time axis gives the
displacement from \m{t_0} to \m{t_f}.
The curve is not the \m{v(t)} corresponding to the \m{a(t)} of \Figref{11}.}{m7gr12}

\pcap[\Index{position, as integral of velocity}\Index{displacement, as integral of velocity}]{5}{e}{Position as an Integral}
Writing \m{v(t) = dx(t)/dt} in the form \m{dx(t') = v(t')\,dt'} and integrating,
we get:
%
\Eqn{}{\int_{x_0}^x\,dx' = \int_{t_0}^t\,v(t')\,dt'\,.}
%
Integrating the left hand side, we get:
%
\Eqn{7}{x(t) = x_0 + \int_{t_0}^t\,v(t')\,dt'\,,}
%
where \m{v(t')dt'} can be thought of as the small displacement of the particle
in the small increment of time \m{dt'}(see \Figref{12}).
We can think of the integral as the sum of many small changes in
displacement.
}% /Sect
%
\Sect{6}{Constant Acceleration}{\SectType{TextOnePara}}{
%
In this section we will particularize the equations of motion to the
restricted case of objects undergoing constant acceleration.
Such constant acceleration occurs when the net force acting on an object
is itself constant in time.
A number of real-life motions are close enough to this situation so that the
constant acceleration equations we develop can be used as good
approximations.
The chief merit in using constant-acceleration equations is their
mathematical simplicity.

Starting with \Eqnref{6} and with \m{a(t') = a}, a constant, we get:
%
\Eqn{8}{v = v_0 + a t\,.}
%
Note that we have chosen \m{t_0 = 0}.
Substituting that result into \Eqnref{7} we get:
%
\ThreeEqns{9}{x(t) & = x_0 + \int_0^t\,(v_0 + a t')\,dt'}
          {        & = x_0 + v_0 \int_0^t\,dt' + a \int_0^t\,t'\,dt'}
          {        & = x_0 + v_0 t + \dfrac{1}{2} a t^2\,.}
%
If \m{v_0} is not given in a constant-acceleration problem, you can eliminate
it between Eqs.\,(8) and (9).
Try it now and make sure you get: \help{1}
%
\Eqn{10}{x = x_0 + v t - \dfrac{1}{2} a t^2\,.}
%
Do not memorize that equation: just make sure you can derive it when you
need it.

Similarly, if \m{t} is not given you can eliminate it between Eqs.\,(8) and (9).
Try it now and make sure you get: \help{1}
%
\Eqn{11}{v^2 - v_0^2 = 2\,a\,(x - x_0)\,.}
%
Remember, whenever you see \m{a}, rather than a(t), as in the equations of this
section, it means that the equations you are looking at are valid only for
problems involving constant acceleration.
If the acceleration is not constant, do not use them: instead, use
equations involving \m{a(t)}.
}% /Sect
%
\Sect{}{Acknowledgments}{\SectType{Acknowledgments}}{
This module is based, in part, on modules prepared by D.\,W.\,Joseph,
J.\,S.\,Kovacs, and P.\,S.\,Signell.
\NsfAcknowledgment}% /Sect
%
\Sect{A}{Communicating Word-Problem Solutions}{\SectType{AppendixMultiPara}}{
%
In order for you to communicate the fact that you have solved a word-problem
and have understood your solution, we have found from experience
that the most effective lay-out is the one which is commonly used for
communication in the professional scientific and engineering journals.
We introduce you to a slight variation here as we give one more example.
%
\SubSubSect{}{Example:}{\medskip\newline
\noindent Given:  \m{x(1.0\unit{s}) = 7.0\unit{m}};\newline
\phantom{Given:}  \m{v(t) = \alpha t^2 + \beta t + \gamma; \alpha =
9.0\unit{m/s\up{3}}};\newline
\phantom{Given:}  \m{\beta = 4.0\unit{m/s\up{2}}; \gamma = - 8.0\unit{m/s}}.\smallskip\newline
\noindent Find:   \m{a(t)} for \m{t = 2.0\unit{s}} and \m{t = 4.0\unit{s}}, and \m{x(t)}.
\begin{one-digit-list}
\item [a.] \m{a(t) = \dfrac{dv(t)}{dt} = 2\alpha t + \beta}
\item [] \m{a(2.0\unit{s}) = (2)(9.0\unit{m/s\up{3}})(2.0\unit{s}) + (4.0\unit{m/s\up{2}})}
\item [] \m{\phantom{a(2.0\unit{s})} = 36.0\unit{m/s\up{2}} + 4.0\unit{m/s\up{2}}}
\item [] \m{\phantom{a(2.0\unit{s})} = 40.0\unit{m/s\up{2}}}
\item [b.] \m{a(4.0\unit{s}) = (2)(9.0\unit{m/s\up{3}})(4.0\unit{s}) + (4.0\unit{m/s\up{2}}) =
76.0\unit{m/s\up{2}}}
\item [c.] \m{x(t) = \int v(t)\,dt = \dfrac{\alpha t^3}{3} +
\dfrac{\beta t^2}{2}+\gamma t + C}.\smallskip\newline
This can be written in a more interesting manner by noting that the position
at \m{t = 0} is \m{x(0) = C}:\smallskip\newline
\m{x(t) = x(0) + \dfrac{\alpha t^3}{3} + \dfrac{\beta t^2}{2} +
\gamma t}.
\end{one-digit-list}
\begin{one-digit-list}
\item [d.] \m{x(t) = x(0) + \dfrac{\alpha t^3}{3} + \dfrac{\beta t^2}{2} +
\gamma t}\newline
7.0\unit{m} = \m{x(0) + \dfrac{(9.0\unit{m/s\up{3}})(1.0\unit{s\up{3}})}{3} +
         \dfrac{(4.0\unit{m/s\up{2}})(1.0\unit{s\up{2}})}{2} +
(-8.0\unit{m/s})(1.0\unit{s})}\smallskip\newline
\m{x(0) = 7.0\unit{m} - 3.0\unit{m} - 2.0\unit{m} + 8.0\unit{m} =
10.0\unit{m}}.\newline
\m{x(t) = 10\unit{m} + \dfrac{\alpha t^3}{3} + \dfrac{\beta t^2}{2} + \gamma t}
\end{one-digit-list}
}% /SubSubSect
%
\SubSubSect{}{Notice that:}{
\begin{itemize}
\item [1.] There is a vertical alignment  of equality signs (=) as much
as possible;
\item [2.] units, such as meters and seconds, are written in
explicitly and their appropriate powers are computed algebraically;
\item [3.] symbolic answers are obtained first and are boxed, then numerical
answers are obtained and boxed (the substitution of numbers for symbols being
clearly shown); and
\item [4.] there is no extraneous material.
\end{itemize}
}% /SubSubSect
%
How did the above example shown above come to look so neat?
The solution was first written out on scratch paper with false starts,
erasures, crossed out parts, and other extraneous material.
The pertinent parts were then arranged on this sheet in the form shown.
}% /Sect