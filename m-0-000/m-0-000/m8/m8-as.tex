\revhist{8/4/91, pss; 3/12/93, pss; 6/8/94, pss}

\Sect{}{}{\SectType{SpecialAssistance}}{

\AsItem{1}{PS-1a}{Find \m{x(t)}, \m{y(t)} for the given \m{\vect{r}(t)}. \help{19}}

\AsItem{2}{PS-1b}{Find the time when \m{x = 2\unit{m}}.
This is the same time as when \m{y = 3\unit{m}}.}

\AsItem{3}{PS-2}{\m{\vect{a} = d^2\vect{r}/dt^2}. \help{20}}

\AsItem{4}{PS-3}{\m{\vect{v} = d\vect{r}/dt}.}

\AsItem{5}{PS-4}{The object is at rest if the coordinates describing
its position do not change in time. \help{21}}

\AsItem{6}{PS-5a}{Partial information about each component is given.
Solve for the motion of each component vector separately. \help{22}}

\AsItem{7}{PS-5b}{Using vector rotation, add the components of part (a).}

\AsItem{8}{PS-6}{The trajectory is determined by the curve of \m{y(x)} or
(equivalently) \m{x(y)}. \help{23}}

\AsItem{9}{PS-7a}{Average velocity is defined as:
\m{\int_{t_1}^{t_2} \vect{v}(t')\,dt'/(t_2 - t_1) = \Delta \vect{r}/\Delta t}. \help{24}}

\AsItem{10}{PS-7b}{See problem~1(b). \help{25}}

\AsItem{11}{PS-7c}{See problem~1(b). \help{26}}

\AsItem{12}{PS-8}{\m{a_x = 0} and \m{a_y = 0}. \help{27}}

\AsItem{13}{PS-9}{This is a constant acceleration with \m{x}- and
\m{y}-components. \help{28}}

\AsItem{14}{PS-head}{For example, in \m{(t/\unit{s})} the \Quote{\m{t}} is the
variable \Quote{time} and the \Quote{s} is the unit \Quote{seconds.}
For example: if \m{t = 3\unit{s}}, then \m{(t/\unit{s}) = (3\unit{s})/(\unit{s}) = 3}.}

\AsItem{15}{PS-11a}{The path has zero slope at \m{t = \pm 1\unit{s}}.}

\AsItem{16}{PS-11b}{\m{\vect{v} =
\left[\uvec{x} + \left(4 \dfrac{t^2}{\unit{s\up{2}}} -
4\right) \uvec{y} \right]\unit{m/s}}. \help{29}}

\AsItem{17}{PS-11c}{See method of part (b).}

\AsItem{18}{PS-11d}{\m{\vect{a} = 8 t \uvec{y}\unit{m/s\up{3}}}.}

\AsItem{19}{[S-1]}{Solve for \m{t(y)} and substitute into \m{x(t)}.}

\AsItem{20}{[S-3]}{\m{\vect{a} = d\vect{v}/dt = (d/dt)(d\vect{r}/dt)}.
If these give trouble, review how to differentiate a vector.}

\AsItem{21}{[S-5]}{To meet the above requirement, \m{\vect{v} = 0} and
\m{\vect{a} = 0} for all times.}

\AsItem{22}{[S-6]}{The motion of each vector is one dimensional. \help{30}}

\AsItem{23}{[S-8]}{\m{x} and \m{y} are components of \vect{r}. \help{31}}

\AsItem{24}{[S-9]}{\m{\int_{t_1}^{t_2} v_x\,dt' =} area under the \m{v_x(t)} vs. \m{t}
curve from \m{t_1} to \m{t_2}. \help{32}}

\AsItem{25}{[S-10]}{\m{\vect{a} = d\vect{v}/dt}. \help{33}}

\AsItem{26}{[S-11]}{\m{\vect{a}_{av} = \left[\vect{v}(60\,s) -
\vect{v}(20\unit{s})\right]/(60\unit{s} - 20\unit{s})},\newline
\m{\vect{a}_{av} \neq \left[\vect{a}(60\unit{s}) + \vect{a}(20\unit{s})\right]/2}.
\help{34}}

\AsItem{27}{[S-12]}{Both \m{v_x} and \m{v_y} are constant. \help{35}}

\AsItem{28}{[S-13]}{\m{v_x = A t}. \help{36}}

\AsItem{29}{[S-16]}{\m{\vect{v}(1\unit{s}) = \uvec{x}\unit{m/s}}.}

\AsItem{30}{[S-22]}{Some equations for motion in one dimension with constant
acceleration are:  \m{x = x_0 + v_0 t + a t^2/2}, \m{v = v_0 + a t},
\m{v^2 = v_0^2 + 2 a (x - x_0)}.}

\AsItem{31}{[S-23]}{\m{\vect{r}(t) =
\int_0^t\,dt'\,\int_0^{t'} \vect{a}(t'')\,dt'' + \vect{r}_0}. \help{37}}

\AsItem{32}{[S-24]}{The area of a triangle is \m{(1/2)b h}. \help{38}}

\AsItem{33}{[S-25]}%
{At \m{t = 20\unit{s}}, \m{dv_x/dt = - (20/40)\unit{m/s\up{2}} = - (1/2)\unit{m/s\up{2}}}.
This problem requires a derivative to be taken from a curve, which is the
slope of the line at the time given.}

\AsItem{34}{[S-26]}{\m{\vect{v}(60\unit{s}) =
(10 \uvec{x} - 20 \uvec{y})\unit{m/s}}.
In this problem, \m{\vect{v}(t)} must be found from the graphical representations
of the components.
If you had trouble with parts~(b) and (c), review problem~1(b).}

\AsItem{35}{[S-27]}{The parachutist will reach the ground in 37.5\unit{s}.}

\AsItem{36}{[S-28]}{\m{v_{x,av} = \dfrac{1}{t} \int_0^t A\,t'\,dt'}.}

\AsItem{37}{[S-31]}{For the particle to be at rest at \m{t = 0}, \m{\vect{v} = 0}.
\help{39}}

\AsItem{38}{[S-32]}{If \m{v_y < 0}, area has a negative sign, indicating
a displacement in the negative direction.}

\AsItem{39}{[S-37]}{For simplicity, put \m{\vect{r}_0 = 0}. \help{40}}

\AsItem{40}{[S-39]}{\m{\vect{v} = (3\unit{m/s\up{2}}) t \uvec{x} -
(2\unit{m/s\up{2}}) t \uvec{y}} and \m{\vect{r}(t) = \int_0^t v(t')\,dt'}.
If this problem gives trouble, work on how to make transformations of the type
\m{x(t) \leftrightarrow t(x)}.}

\AsItem{41}{PS-10}{Draw a graph with a single vector extending out from the
origin at an angle of {240\degrees} CCW from the positive \m{x}-axis.
Label this vector \vect{g}.
Then:
\m{\vect{g} = g \cos240\degrees \,\uvec{x} + g \sin240\degrees \,\uvec{y}}.
Alternatively, simply look at the diagram and write:
\m{\vect{g} = -g \sin30\degrees \,\uvec{x} - g \cos30\degrees \,\uvec{y}}.
}

\AsItem{42}{PS-15}{We let \m{b \equiv t/\unit{s}} so the \m{y}-equation
 above becomes:
 %
 \Eqn{}{4.9 b^2 - 14.4 b + 1.0 = 0\,.}
 %
 Then solving this quadratic equation (see any high school or college
 algebra book):
 %
 \Eqn{}{b = \dfrac{14.4 \pm \sqrt{(-14.4)^2 -4(4.9)(1.0)}}
                  {(2)(4.9)}\,.}
 %
}

}% /Sect
