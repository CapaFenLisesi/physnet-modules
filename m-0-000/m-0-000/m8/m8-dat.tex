\revhist{9/13/88, pss; 12/8/89, pss; 4/1/91, pss; 8/5/92,mcr; 3/12/93, pss;
         6/8/94, pss; 10/12/94, pss; 3/3/95, pss; 3/31/95, pss; 7/10/95, ???;
         8/27/95, pss; 10/3/95, pss; 5/1/96, pss; 9/18/96, pss; 3/21/97, pss;
         3/28/97, pss; 11/13/97, pss; 2/22/99, pss; 12/15/99, pss; 3/2/00, pss;
         12/18/00, pss; 2/8/02, pss; 9/25/02, pss; 10/3/02, pss; 2/21/03, pss}
%
\defModTitle{\ph{Two-Dimensional Motion}}
\defCtAuthor{\inits{H.}\inits{T.}Hudson and Ray \inits{G.}Van Ausdal}
\defIdAuthor{H.\,T.\,Hudson, University of Houston and Ray G.\,Van Ausdal,
Univ. of Pittsburgh at Johnstown}
%
\defIdItems{
    \IdVersEval{2/21/2003}{0}
    \IdHours{1}
    \begin{InputSkills}
    \item [1.]  Vocabulary: displacement, velocity and acceleration vectors
    \prrqone{0-7}.
    \item [2.]  Differentiate an expression which includes unit vectors \m{\uvec{x}}
    and \m{\uvec{y}}.
    \prrqone{0-2}.
    \item [3.]  Find velocity, given the time-dependent position of an object
    \prrqone{0-7}.
    \item [4.]  Find acceleration, given the time-dependent velocity.
    \prrqone{0-7}.
    \item [5.]  Find position given initial conditions and either velocity
    or acceleration in one dimension \prrqone{0-7}.
    \item [6.]  Write the specific equations describing motion in one
    dimension with constant acceleration \prrqone{0-7}.
    \end{InputSkills}
    %
    \begin{KnowledgeSkills}
    \item [K1.] Write the vector equations relating position, velocity and
    acceleration in component form.
    \item [K2.] Explain how the motion of component vectors can be used to
    describe the motion of an object.
    \end{KnowledgeSkills}
    %
    \begin{ProblemSolvingSkills}
    \item [S1.] Given (either graphically or analytically) one of the functions
    \m{\vect{r}(t)}, \m{\vect{v}(t)}, or \m{\vect{a}(t)}, plus initial conditions, find the
    two other functions.
    \item [S2.] Given \m{\vect{r}(t)}, derive the equation of the trajectory for an
    object.
    \item [S3.] Given a special case of constant velocity or constant
    acceleration for one component, write the appropriate equations
    of motion by using pre-derived one dimensional relationships.
    \item [S4.] Determine the range, maximum height and equation of the trajectory
    for an object in ballistic motion.
    \end{ProblemSolvingSkills}
}