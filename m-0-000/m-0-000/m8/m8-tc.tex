\tcsc {1. Introduction}{}
\tcpc {a. Why We Study Motion in Two Dimensions}{1}
\tcpc {b. The Job at Hand}{1}
\tcpc {c. The Fundamental Relationship}{1}
\tcsc {2. Analysis by Components}{}
\tcpc {a. Vector Equations Reduce to Component Equations}{2}
\tcpc {b. Component Descriptors Relate to Actual Motion}{2}
\tcpc {c. Motion of the $x$- and $y$-Component Vectors}{2}
\tcpc {d. Describing the Component Motion}{2}
\tcpc {e. Dual Roles of Component Descriptors}{2}
\tcsc {3. Problem-Solving Techniques}{}
\tcpc {a. Methods of Specifying Information}{3}
\tcpc {b. Analytical Method: General Approach}{3}
\tcpc {c. Two Special Cases}{3}
\tcpc {d. Graphical Method}{3}
\tcsc {4. Examples and Cautions}{}
\tcpc {a. Sample Problem: Analytical Method}{4}
\tcpc {b. Sample Problem: Graphical Method}{5}
\tcpc {c. The Artificial Nature of the Examples}{5}
\tcpc {d. Choice of Coordinates}{6}
\tcsc {5. Ballisitic Motion}{}
\tcpc {a. Falling and Free Falling}{6}
\tcpc {b. Ballistic Motion Example}{6}
\tcpc {c. The Example in Cartesian Coordinates}{7}
\tcpc {d. Equation of the Path: the Trajectory}{7}
\tcpc {e. The Range}{8}
\tcpc {f. Maximum Height of a Projectile}{8}
\tcsc {Acknowledgments}{8}
\tcsc {}{}
\tcsc {}{}
\tcsc {}{}
\defmodlength {32}
