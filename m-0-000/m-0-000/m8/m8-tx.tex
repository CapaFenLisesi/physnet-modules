\revhist{7/19/85, pss; 12/8/89, pss; 7/31/91, pss; 3/12/93, pss; 6/8/94, pss;
         5/1/96, pss; 8/21/94, abs; 3/2/2000, pss; 9/25/02, pss; 10/3/02, pss; 2/21/03, pss}
%
\Sect{1}{Introduction}{\SectType{TextMultiPara}}{
%
\pcap[\Index{motion, in two dimensions}]{1}{a}{Why We Study Motion in Two Dimensions}
The real world is three-dimensional, so why do we bother with
two-dimensional motion?
First, two-dimensional motion is easier to describe, easier to deal with
mathematically, and easier to sketch on a piece of flat paper.
This makes two-dimensional motion a good place for introducing concepts that
are peculiar to motion in more than one dimension.
Second, many objects actually do exhibit motion in a plane, motion that
needs only two dimensions for its complete description.
Any motion under constant acceleration can always be described in terms of
just two dimensions.
Even if the acceleration is not constant, many objects still move in a plane
(e.g., a tractor on a level field, a rider on a ferris wheel).

\pcap{1}{b}{The Job at Hand}
Our basic kinematical problem is to give quantitative information about the
time-dependent positions, velocities, and accelerations of objects.
This information is to be specified either formally (\Quote{analytically}),
in words, or graphically.

\pcap{1}{c}{The Fundamental Relationship}
Here are the equations that summarize the fundamental relationships
used in this module:
%
\Footnote{1}{See also \Quote{\Index{kinematics}Kinematics in Three Dimensions} (MISN-0-37).}
%
\Eqn{1}{\text{position } = \vect{r}}
%
\Eqn{2}{\text{displacement } = \Delta \vect{r} = \vect{r}_2 - \vect{r}_1}
%
\Eqn{3}{\Index{velocity, average}\text{average velocity } =  \vect{v}_{av} = \Delta \vect{r}/\Delta t}
%
\Eqn{4}{\Index{velocity, instantaneous}\text{inst. vel. } = \vect{v} = d\vect{r}/dt}
%
\Eqn{5}{\Index{acceleration, average}\text{av. accel. } = \vect{a}_{av} = \Delta \vect{v}/\Delta t}
%
\Eqn{6}{\Index{acceleration, instantaneous}\text{inst. accel. } =  \vect{a}   = d\vect{v}/dt = d^2\vect{r}/dt^2}
%
\Equationref{4} can be inverted, giving:
%
\Eqn{7}{\vect{r}(t) = \int_0^t \vect{v}(t')\,dt' + \vect{r}(0).}
%
\Equationref{6} can be inverted, giving:
%
\Eqn{8}{\vect{v}(t) = \int_0^t \vect{a}(t')\,dt' + \vect{v}(0).}
}% /Sect
%
\Sect[\Index{acceleration, in terms of components}]{2}{Analysis by Components}{\SectType{TextMultiPara}}{
%
\pcap[\Index{position vector}\Index{component vectors}]{2}{a}{Vector Equations Reduce to Component Equations}
Equationstoref{1}{8} are vector equations.
Each equation could be rewritten as two \m{x} and \m{y} component equations,
so that the two dimensional motion of the object could also be
treated as two simultaneous one-dimensional problems.
For instance, \Eqnref{3} is equivalent to the two one-dimensional
equations \m{v_{x,av} = \Delta x/\Delta t} and \m{v_{y,av} = \Delta y/\Delta t}.
Equation (4) is equivalent to the two one-dimensional equations
\m{v_x = dx/dt} and \m{v_y = dy/dt}.

\pcap[\Index{velocity, in terms of components}]{2}{b}{Component Descriptors Relate to Actual Motion}
The \m{x} and \m{y} \Index{components of the position}position, in terms of components, \Index{displacement}displacement, velocity
and acceleration vectors can be related more graphically to the
actual motion of the object.
The component description is more than a mere exercise in
mathematical symbolism.

\pcap{2}{c}{Motion of the \m{x}- and \m{y}-Component Vectors}
As a particle moves along a complicated path, as in \Figref{1},
its position vector \vect{r} and the component vectors \vect{x} and
\vect{y} also move.
Envision in your mind how each component vector tip moves as the
particle moves from \m{A} to \m{B}.

\pcap{2}{d}{Describing the Component Motion}
The one-dimensional motion of the tip of the \vect{x} vector can be
described
%
\Footnote{2}{See \Quote{Kinematics in One Dimension} (MISN-0-7).}
%
by its position \m{x}, velocity \m{v_x}, and acceleration \m{a_x}.
Similarly, the motion of the \vect{y} vector can be described by \m{y}, \m{v_y}
and \m{a_y}.
You can thus use two one-dimensional motions to completely
describe one two-dimensional motion.

\CaptionedFullFramedFigure{1}{A particle moving on a two dimensional path
has \m{x} and \m{y} component vectors that each move in one dimension.}{m8gr01}

\pcap{2}{e}{Dual Roles of Component Descriptors}
The quantities \m{x}, \m{y}, \m{v_x}, \m{v_y}, \m{a_x}, and \m{a_y} can be used in two
different ways.
They can either describe the motion of the tips of the \vect{x} and
\vect{y} vectors, or they can describe the components of the actual
displacement, velocity and acceleration vectors of the moving object.
Problem-solving techniques can take advantage of this dual meaning.
}% /Sect
%
\Sect{3}{Problem-Solving Techniques}{\SectType{TextMultiPara}}{
%
\pcap{3}{a}{Methods of Specifying Information}
Typically, information about the motion of an object is specified
in either analytical or graphical form.
The position, velocity or acceleration, or some combination of
their components, might be given.
We can detail some differences between the handling of analytical
vs. graphical data.

\pcap{3}{b}{Analytical Method: General Approach}
\Equationstoref{1}{8} analytically relate the variables \vect{a}, \vect{v}, and
\vect{r}.
If an analytical form for \m{\vect{a}(t)}, \m{\vect{v}(t)} or \m{\vect{r}(t)} can be
found, the derivatives or integrals can be performed so that \vect{a},
\vect{v} and \vect{r} will all be known.
The \Index{path}path or \Quote{trajectory} of the object can be found in the form
of an equation for \m{y(x)} by eliminating \m{t} between the equations for \m{x(t)}
and \m{y(t)}.
The trajectory could also be found by plotting \m{(x,y)} points for appropriate
time values and then drawing a smooth curve through those points.

\pcap{3}{c}{Two Special Cases}
Often, an object moves in such a way that the \m{x}- and/or \m{y}-component vector
tip moves with either constant velocity or constant acceleration.
In these cases \m{x}, \m{v_x}, \m{a_x} (and/or \m{y}, \m{v_y}, \m{a_y}) are related by the
previously derived equations for an object with one dimensional constant
acceleration.
The need for differentiation and integration is bypassed.

\pcap{3}{d}{Graphical Method}
Sometimes the time dependence for the components of one of the quantities
\m{\vect{a}(t)}, \m{\vect{v}(t)} or \m{\vect{r}(t)} is given in graphical form.
The interpretation of the derivative as the (physical) slope
of a curve and integral as the (physical) area under a
curve could then be used to find \vect{a}, \vect{v} and \vect{r}.
}% /Sect
%
\Sect{4}{Examples and Cautions}{\SectType{TextMultiPara}}{
%
\pcap{4}{a}{Sample Problem: Analytical Method}
A sample problem will illustrate the techniques of the
analytical method.
Suppose that you are given that the velocity of an object
is a constant 2\unit{m/s} in the x-direction, and that it increases
linearly with \m{t} in the y-direction: \m{v_y = (3\unit{m/s\up{2}})\,t}.
Then:
%
\Eqn{}{\vect{v}(t) = (2\unit{m/s}) \uvec{x} + (3\unit{m/s\up{2}})\,t \uvec{y}.}
%
The acceleration is:
%
\Eqn{}{\vect{a} = \dfrac{d\vect{v}}{dt} = 0 \uvec{x} + 3\unit{m/s\up{2}}\,\uvec{y}.}
%
That is, the object has a constant acceleration in the
\m{y}-direction.
The position of the particle is:
%
\ThreeEqns{}{\vect{r}(t) & = \int_0^t \vect{v}(t')\,dt' + \vect{r}(0)}
            {           & = \int_0^t (2\unit{m/s}\,\uvec{x} + 3\unit{m/s\up{2}}\,t'\,\uvec{y})\,dt' + \vect{r}(0)}
            {           & = (2\unit{m/s})\,t\,\uvec{x} + \dfrac{1}{2}(3\unit{m/s\up{2}})t^2 \uvec{y} + \vect{r}(0).}
%
If the problem further stated \m{\vect{r}(0)}; for example, as \Quote{initially the
object is at \m{x = 4\unit{m}}, \m{y = 5\unit{m}},} we could write:
%
\Eqn{}{\vect{r}(t) = (2\unit{m/s}\,t + 4\unit{m})\uvec{x} +
\left(\dfrac{3}{2}\unit{m/s\up{2}}\,t^2 + 5\unit{m}\right)\,\uvec{y}.}
%
\CaptionedFullFramedFigure{2}{The relations \m{v_x(t)} and \m{v_y(t)} specified
graphically.}{m8gr02}
%
The path can now be found.
The above vector equation gives the component equations:
\m{x(t) = (2\unit{m/s})\,t + 4\unit{m}}; \m{y(t) = [(3/2)\unit{m/s\up{2}}]\,t^2 + 5\unit{m}}.
Solving \m{x(t)} for \m{t} and substituting that into \m{y(t)} gives \m{y(x)}:
%
\Eqn{}{y(x) = \dfrac{3}{2}\unit{m/s\up{2}}\,
\left(\dfrac{x - 4\unit{m}}{2\unit{m/s}}\right)^2 + 5\unit{m},}
%
which is the equation of a parabola: the object moves in a \Index{path, parabolic}\Index{parabolic path}parabolic path.

\xpcap{}{}{Note for those interested}
The integration could have been bypassed by noting that the \m{x}-component
motion is at constant velocity and the \m{y}-component motion is at constant
acceleration.
Thus \m{x(t)} and \m{y(t)} fit the general form of the one-dimensional constant
acceleration equation: \m{x = x_0 + v_0t + a t^2/2}.
For example, for the \m{y}-direction:
\m{a = 3\unit{m/s\up{2}}}, \m{v_0 = 0}, \m{x_0 \equiv y_0 = 5\unit{m}},
so: \m{y = 5\unit{m} + 0 + (3\unit{m/s\up{2}})\,t^2/2}.

\pcap{4}{b}{Sample Problem: Graphical Method}
The previous problem could have specified the velocity components
graphically, as in \Figref{2}.
Now \m{a_x} is the slope of the tangent line to the \m{v_x}(t) curve:
in this case, \Figref{2}, the slope is always zero.
Similarly, \m{a_y} is the slope of the \m{v_y}(t) curve, which in this case is
always 3\unit{m/s\up{2}}.

The \m{x} and \m{y} coordinates can be found using the area under the curve.
For example, to calculate \m{y}(1\unit{sec}), \Eqnref{7} gives:
%
\Eqn{}{y(1\unit{s}) = \int_0^{1\unit{s}} v_y(t')\,dt' + y(0).}
%
The integral is given by the shaded area
%
\Footnote{3}{This area must be calculated as the \Quote{physical} area, not the
geometric area.
See \Quote{The Counting Squares Technique for Numerical Integration}
(MISN-0-250, Appendix A).}
%
in \Figref{2}, so that:
%
\Eqn{}{y(1\unit{s}) = \dfrac{3}{2}\unit{m} + 5\unit{m} = \dfrac{13}{2}\unit{m},}
%
which can be verified from the previous analytical solution for \m{y(t)}.

\pcap{4}{c}{The Artificial Nature of the Examples}
The real world does not usually present motion problems so neatly
specified as the previous examples.
These examples have presented information as given that in
actuality must have been derived from other information.
For example, knowledge of applied forces gives information about
the acceleration.
%
\Footnote{4}{See \Quote{Particle Dynamics} (MISN-0-14).}
%
Also, coordinate systems and initial times have been implicitly chosen.

\pcap{4}{d}{Choice of Coordinates}
If the coordinate system is unspecified in a problem, you may
choose to use any system you desire.
The motion of the object will not depend on the coordinate system
that you use to describe the motion.
Be prepared to try different coordinate systems; the \Quote{best}
choice will ease the mathematical manipulation in the problem.
}% /Sect
%
\Sect{5}{Ballisitic Motion}{\SectType{TextMultiPara}}{
%
\pcap[\Index{free-fall}]{5}{a}{Falling and Free Falling}
The acceleration of an object falling above the Earth
depends upon its distance from the Earth's surface and upon
air resistance.
You are familiar with this motion, for example, when you
observe a baseball in flight.
If the \Index{speed}speed of the object is sufficiently low, the effects of the air
resistance are negligible.
%
\Footnote{5}{How low is \Quote{sufficiently low}?
The answer depends upon how precisely you wish to describe the motion,
and the relative magnitudes of the force of gravity and the force of the
air.}
%
If the object's path does not vary significantly in altitude,
the effects of gravity are constant.
Under these special conditions, called
\Index{motion, ballistic}\Index{ballistic motion}ballistic motion, the
object is \Quote{free falling} and will have a constant acceleration of
\m{g = 9.8\unit{m/s\up{2}}} vertically downward, and will therefore move
in a plane.

\pcap{5}{b}{Ballistic Motion Example}
A \Index{projectile}projectile is fired with initial velocity \m{\vect{v}_0}
making an angle \m{\theta} with the horizontal (see \Figref{3}).
Ignore the height of the end of the barrel.

\CaptionedFullFramedFigure{3}{A cannon fires a projectile at an angle
\m{\theta} above the horizontal.}{m8gr00}

We choose a coordinate system such that the horizontal coordinate
is \m{x}, the origin is at the cannon (so \m{\vect{r}_0 \equiv \vect{r}_0 = 0}), and
the vertical coordinate \m{y} is positive upward (with result
\m{\vect{a} = - g \uvec{y}}).  We choose time zero to be when the cannon was fired.
Then at time zero we have:
%
\Eqn{14}{\vect{a}(0) = - g \uvec{y}; \quad \vect{v}(0) = \vect{v}_0; \quad \vect{r}(0) = 0 \,.}
%
We can get these quantities as a function of time by integrating the acceleration to
get the velocity and by integrating the velocity to get the position (see \Eqnref{8}).
The result for the velocity is:
%
\Eqn{15}{\vect{v} = \vect{v}_0 + \int_0^t \vect{a(t'})\,dt' =
                   \vect{v}_0 - \int_0^t g\,dt'\,\uvec{y} =
                   \vect{v}_0 - g t \uvec{y}\,.}
%
In the following three exercises, illustrate by using or changing \Figref{4}.

\tryit Suppose gravity is turned off.  Show that the object would follow a
straight-line trajectory at constant speed and that its distance from the origin
would increase linearly with time.

\tryit suppose that gravity is increased until it is so large that its terms in
\Eqnsref{15} overwhelm the \m{\vect{v}_0} terms.  Show that under those circumstances
the object would appear to simply fall to the ground.

\tryit Describe a real projectile's path as being between the trajectories
that would result from zero gravity and from very large gravity.

\pcap{5}{c}{The Example in Cartesian Coordinates}
\Equationsref{14} and \Eqnssref{15} for ballistic motion can be written in terms
of the projectile's non-accelerated \m{x}-components and constantly-accelerated
\m{y}-components:
%
\ThreeEqns{}{\vect{a}(t) & = - g \uvec{y}\,,}
            {\vect{v}(t) & = (v_0\,\cos\theta_0)\, \uvec{x} + (- g t + v_0 \sin\theta_0)\,\uvec{y}\,,}
            {\vect{r}(t) & = v_0 t \cos\theta_0 \uvec{x} + (- g t^2/2 + v_0 t \sin\theta_0)\,\uvec{y}.}
%
Notice that each component can be integrated separately because the Cartesian unit vectors
are independent of time (they stay fixed as time progresses).

\pcap{5}{d}{Equation of the Path: the Trajectory}
The Cartesian trajectory equation, \m{y(x)}, can be found by eliminating \m{t} in the two equations
\m{x(t)} and \m{y(t)}.
From the above we get:
%
\Eqn{}{x(t) = v_0 t \cos\theta_0\,,}
%
and
%
\Eqn{}{y(t) = - g t^2/2 + v_0 t \sin\theta_0\,.}
%
Solving the first for \m{t} and substituting that into the second, gives
%
\Eqn{}{y = \left[- g / (2 v_0^2 \cos^2\theta_0)\right] x^2 + [\tan\theta_0]\,x\,.}
%
This is the equation of a parabola, which is indeed the path of a
\Index{projectile path}\Index{path, of projectile}projectile undergoing idealized ballistic motion
(see \Figref{4}).

\CaptionedFullFramedFigure{4}{An object in ballistic motion follows
a \Index{parabolic motion}\Index{motion, parabolic}parabolic trajectory.
The acceleration and velocity are indicated at three different places.
Noticer how the object \Quote{falls} as it moves from left to right.}{m8gr04}

\pcap[\Index{range, of projectile}\Index{projectile, range of}]{5}{e}{The Range}
The trajectory equation, \m{y(x)}, is useful in answering questions that relate
to position coordinates only.
Will the object clear a wall?
How far will it go?
How high will it go?
For example, the range \m{R} of a ballistic missile is the distance travelled
in \m{x} before the projectile strikes the ground.
In \Figref{4}, \m{x} is \m{R} when \m{y(x) = 0}.
Thus you can set \m{y(R) = 0} in the quadratic expression,
%
\Eqn{}{y(x) = \left[ - g / (2 v_0^2 \cos^2\theta)\right]\,x^2 + [\tan\theta]\,x\,,}
%
and, using the identity \m{\sin 2\theta = 2 \sin\theta \cos\theta}, find:
%
\Eqn{}{R = \dfrac{2 v_0^2 \sin\theta \cos\theta}{g} = \dfrac{v_0^2 \sin 2\theta}{g}.}
%
\BlackTriangle This equation gives \m{R(v_0,\theta)}.
Show that for \textit{any} fixed \m{v_0} the maximum range occurs for
\m{\theta = 45\degrees}.

\pcap[\Index{projectile, maximum height of}\Index{maximum height, of projectile}]{5}{f}{Maximum Height of a Projectile}
The equation of the path can also be used to develop an
equation relating the maximum height to the initial velocity
of the projectile.
At the point of maximum height, \m{dy/dx = 0}.
Differentiating the equation for the trajectory \m{y(x)}
gives \m{dy/dx = 0} at \m{x = (v_0^2/g)(\sin\theta \cos\theta)}.

\noindent
\BlackTriangle Show that substituting this into the equation for \m{y} gives
the maximum height as \m{y_\text{max} = (v_0^2 \sin^2\theta)/(2g)}.
}% /Sect
%
\Sect{}{Acknowledgments}{\SectType{Acknowledgments}}{We wish to thank Kristen Schalm and Prof. James Linnemann for pointing out typos.  \NsfAcknowledgment}% /Sect

