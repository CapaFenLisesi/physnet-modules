\revhist{12/23/87, pss; 12/16/90, pss; 3/11/93, pss; 9/15/94, pss; 9/15/95, pss;
         8/25/96, pss; 10/15/96, pss; 4/22/98, pss; 1/29/03, pss}

\Sect{}{}{\SectType{SpecialAssistance}}{

\AsItem{1}{TX-Appendix Exercise 1}%
{\begin{one-digit-list}
 \item [a.]  \m{\uvec{r} \cdot \uvec{r} = \left| \dfrac{\vect{r}}{r}\right|
 \left| \dfrac{\vect{r}}{r}\right|
 (\cos{\theta}) = \dfrac{\left| \vect{r} \right|^2}{r^2}\,(1) = 1}, so
 \m{\uvec{r}} is a unit vector.
 \item [b.]  \m{\uvec{r}} and \vect{r} are in the same direction so
 \m{\theta} = 0 so \m{\uvec{r} \cdot \vect{r} = r}.
 \end{one-digit-list}
}

\AsItem{2}{TX-Appendix Exercise 2}%
{\begin{one-digit-list}
 \item [a.] See if \m{\uvec{\theta} \cdot \uvec{\theta} = 1}
 as for \m{\uvec{r}} in [S-1].
 \item [b.] \m{\uvec{\theta} \cdot \uvec{r} = 0}, so either:
 (1) \m{ | \uvec{\theta} | = 0}; or (2) \m{ | \uvec{r} | = 0};
 or (3) they are perpendicular.
 Choices (1) and (2) are obviously false so (3) must be true.
 \end{one-digit-list}
}

\AsItem{3}{PS-Exercise 3}%
{\begin{one-digit-list}
 \item [a.] \m{ | ds/dt |} and ds/dt could differ in sign.
 What is the sign of the magnitude of a vector?
 \item [b.] The derivative is the slope of the graph.
 \end{one-digit-list}
}

\AsItem{4}{TX-Appendix, Sect.~1}%
{Use the chain rule:
 %
 \Eqn{}{\dfrac{d\uvec{r}}{dt}=\dfrac{d\uvec{r}}{ds}\dfrac{ds}{dt} \qquad \text{and} \qquad
 \dfrac{d\uvec{\theta}}{dt}=\dfrac{d\uvec{\theta}}{ds}\dfrac{ds}{dt}}
}

\AsItem{5}{TX-Appendix Exercise 3}%
{You are given \vect{v} and are asked about \m{\vect{v} + \Delta \vect{v}}
 a short time \m{\Delta t} later.  \m{\Delta \vect{v}} is \m{\vect{a}_t \Delta t}.
 Draw vector addition diagrams to decide if \m{\vect{v} + \Delta \vect{v}}
 is longer than, shorter than, or the same length as \vect{v}.
}

\AsItem{6}{PS-Exercise 1}%
{Use \m{a_r = v^2/r}\, and \m{v = 2 \pi r / T}.}

\AsItem{7}{PS-Exercise 2}%
{\begin{one-digit-list}
 \item [a.] Use \m{v = \omega\,r} and \m{\nu = v/2 \pi r}.
 \item [b.] Use \m{a = v^2/r} and \m{v = \omega \,r}.
 \end{one-digit-list}
}

\AsItem{8}{TX-4f}%
{The direction of \m{\vect{\omega} \times \vect{r}}, by definition of the
 vector product, must be perpendicular to \vect{\omega}.
 Therefore this vector product lies in the plane of motion.
 It must also be perpendicular to \vect{r}; therefore it lies along the
 tangent.
 Using the right hand rule, the product points in the direction of the
 tangential velocity.
 Since \vect{\omega} and \vect{r} are perpendicular, the magnitude of
 \m{\vect{\omega} \times \vect{r}} is just \m{\omega\,r}.
 This, in turn, is just the magnitude of the tangential velocity, \m{v}.
}

\AsItem{9}{TX-2b}%
{Here is one answer: \m{\theta/180\degrees = s/(\pi r)}.
 Here is another equivalent one: \m{\theta/360\degrees = s/(2 \pi r)}.}

\AsItem{10}{PS-Problem~2}%
{The Earth makes one complete rotation per day: that's what a day is!
 Therefore the Earth's period of rotation \m{T} is \ldots (you finish it).
 Draw a diagram and put the vectors on it.
}

\AsItem{11}{PS-Problem~3}%
{The car's speed is 50\unit{mph} only at the quoted instant of time.
 Slightly earlier in time its speed was less than that, and slightly later
 its speed will be greater.
 Therefore it has tangential acceleration.
 Because it is in circular motion, it also has radial acceleration.
 \help{13}
}

\AsItem{12}{TX-2a}%
{For help on whether you should use radians or degrees, read the text of
 this module.
}

\AsItem{13}{[S-11]}%
{The tangential acceleration is 5.0\unit{miles per hour per second}, which is
 7.33\unit{ft/s\up{2}} (you must check that conversion from miles to feet and from
 hour to seconds to make sure you understand what is going on).
 \help{14}
}

\AsItem{14}{[S-13]}%
{You must compute \m{\vect{a}_r} and \m{\vect{a}_t} and add them as vectors.
 Note that they are mutually perpendicular vectors.
}

\AsItem{16}{TX-4e}%
{Note that \m{r} is constant for circular motion and use the chain rule for
 differentiation of products (see MISN-0-1).
}

\AsItem{17}{TX-4d}%
{The \Quote{second} hand makes one complete circuit of the dial in one minute,
 which is 60\unit{seconds}, so its period is: \m{T = 60\unit{sec}}.
 Now convert this to frequency.

 The \Quote{hour} hand makes one complete circuit of the dial in 12\unit{hours}, which
 is \m{12 \times 3600\unit{sec}}, so its period is: \m{T = 43,200\unit{sec}}.
 Now convert this to frequency.
}

\AsItem{18}{TX-4e}%
{See Table of Contents, \Quote{Angular Speed.}}

\AsItem{19}{PS-Problem~7b}%
{\m{\vec{v} = \dfrac{d\vec{r}}{dt}}; \m{\dfrac{d\uvec{r}(\theta)}{dt} = \dfrac{d\theta}{dt}\;\uvec{\theta}(\theta)}; \m{\dfrac{dr}{dt} = 0}
}

\AsItem{20}{PS-Problem~7c}%
{\m{\vec{a} = \dfrac{d\vec{v}}{dt}}; \m{\dfrac{d\uvec{\theta}(\theta)}{dt} = -\dfrac{d\theta}{dt}\;\uvec{r}(\theta)}
}

\AsItem{22}{PS-Problem~7e}%
{To get the direction of motion (CW or CCW), use the values of the components of the radius and velocity
vectors to either draw the velocity vector in your head or to sketch it on a piece of paper (see the
velocity vectors sketched throughout the text of this module).  To get the direction of the acceleration,
do the same with the radius and acceleration vectors.  In each case the radius vector is said to be needed
only in order to get the correct angle for the unit vectors.
}

\AsItem{24}{PS-Problem~7g}%
{See the \textit{Table of Contents}, \Quote{Velocity and Speed,} and
 \Quote{Acceleration: Tangential and Radial.}
}

\AsItem{25}{PS-Problem~1}%
{Do all the Exercises (above the Problems) successfully before tackling this
 problem.
}

}% /Sect
