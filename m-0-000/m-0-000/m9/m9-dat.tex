\revhist{4/29/92, pss; 12/8/92, pss; 2/12/93, pss; 3/11/93, pss; 9/2/94, pss;
         9/14/94, pss; 10/14/94, pss; 6/21/95 lae; 8/27/95, pss; 9/15/95, pss;
         8/25/96, pss; 10/15/96, pss; 4/22/98, pss; 2/22/99, pss; 11/1/99, pss;
         3/3/2000, pss; 11/5/01, pss; 12/19/02, pss; 1/21/03, pss; 1/29/03, pss}
%
\defModTitle{\ph{Circular Motion: Kinematics}}
\defCtAuthor{James \inits{M.}Tanner, Georgia Institute of Technology}
\defIdAuthor{James M. Tanner, School of Physics, Georgia Inst. of Tech.,
Atlanta, GA 30332}
%
\defIdItems{
    \IdVersEval{1/29/2003}{0}
    \IdHours{1}
    \begin{InputSkills}
    %
    \item [1.] Define kinematically: velocity, speed, and acceleration
    \prrqone{0-7}, \prrqone{0-8}.
    %
    \item [2.] Use the Cartesian unit vectors \m{\uvec{x}}, \m{\uvec{y}}, \m{\uvec{z}}
    to write a vector in terms of its Cartesian components \prrqone{0-1},
    \prrqone{0-2}.
    %
    \item [3.] Use the chain rule to differentiate sine and cosine \prrqone{0-1},
    \prrqone{0-2}.
    %
    \item [4.] Given the position, velocity, and acceleration vectors for a
    particle executing circular motion, determine: (1) if the motion is
    clockwise or counterclockwise; and (2) if the particle's speed is
    increasing, decreasing or unchanging \prrqone{0-72}.
    %
    \item [5.] Given a particle's position on a circle, its direction
    of motion as clockwise or counterclockwise, and whether its speed is
    increasing, decreasing, or unchanging, draw plausible position, velocity,
    and acceleration vectors with correct spatial orientations \prrqone{0-72}.
    \end{InputSkills}
    %
    \begin{KnowledgeSkills}
    \item [K1.] Define angular velocity in terms of angular displacement and
    explain how to determine the direction of the angular velocity vector.
    \end{KnowledgeSkills}
    %
    \begin{RuleApplicationSkills}
    \item [R1.] A particle is moving along a circular path.  Given the radius of the
    circle and the particle's position as measured along the arc of its path as a function
    of time, formally calculate its position, velocity, and acceleration vectors at any instant.
    %
    \item [R2.] A particle in uniform circular motion has these four
    descriptors: radius, speed, radial acceleration (magnitude), plus either of
    period or frequency. Given any two of these quantities, determine the other
    two.
    \end{RuleApplicationSkills}
}