\revhist{12/14/90, pss; 12/8/92, pss; 2/12/93, pss; 3/11/93, pss; 9/2/94, pss;
         9/15/94, pss; 10/14/94, pss; 11/7/97, lae; 1/1/29/03, pss}

\Sect{}{}{\SectType{ProblemSet}}{

\ProbSet{Exercises}{
\begin{one-digit-list}
\item [1.] Show that, for uniform circular motion:
           \m{a_r = 4 \pi^2 r / T^2 = 4 \pi^2 \nu^2 r}.  \help{6}
\item [2.] Show that, for uniform circular motion:
\begin{one-digit-list}
\item [a.] \m{\omega = 2 \pi \nu}
\item [b.] \m{a_r = \omega^2 r}   \help{7}
\end{one-digit-list}
\item [3.] \NullItem
\begin{one-digit-list}
\item [a.] Explain why \m{v = | ds/dt |}, \m{v \neq ds/dt}.
\item [b.] If \m{s} is graphed versus time, how can \m{v} be determined from this graph?   \help{3}
\end{one-digit-list}
\end{one-digit-list}
}% /ProbSet
%
\ProbSet{Problems}{
\noindent Note:  Remember to set your calculator to the proper angular
measure, radians or degrees, for the equations you are using! \help{12}
\begin{one-digit-list}
\item[1.] The moon orbits the earth in an approximately circular path (mean
radius\,\m{ = 3.84 \times 10^5\unit{km}}) every 27.3\unit{days}.  Calculate the moon's
speed and radial acceleration in this orbit. \help{25}
\item [2.] Calculate the speed and acceleration of a person on the equator 
resulting from the earth's spinning about its axis through the poles.
The earth's mean radius is \m{6.37 \times 10^6\unit{m}}. \help{10}
\item [3.] At a particular instant of time a car on a circular track
(radius\,\m{ = 8.0 \times 10^2\unit{ft}}) is traveling at 50.0\unit{miles per hour}
and is increasing its speed at a rate of 5.0\unit{miles per hour per second}.
Determine the magnitude of its acceleration, at this particular instant of
time, in \unit{ft/s\up{2}}. \help{11}
\item [4.] Calculate the acceleration of a pilot at the bottom of a dive in
a plane  traveling at a constant \m{5.0 \times 10^2\unit{miles per hour}} along a
circular path of radius 2.0\unit{miles}.

\item [5.] \TextAndFigure{The particle at point \m{A} is traveling
clockwise with an increasing speed.
Draw its velocity and acceleration vectors.}{m9gr09}

\item [6.] \TextAndFigure{The position, velocity, and acceleration
vectors for a particle executing circular motion are shown.
Is the particle traveling CW (clockwise) or CCW (counterclockwise)?
Is its speed increasing, decreasing, or constant?}{m9gr10}

\item [7.] \TextAndFigure{\textit{Note: If you have trouble with any part of this problem:
(1) make sure you have thoroughly understood each and every paragraph of this
module's text; (2) work all of the other problems successfully before tackling
this one; (3) if necessary, review the prerequisite material on derivatives
in MISN-0-1 and unit vectors in MISN-0-2; and (4) use the references,
labeled \help{\ }, that refer to this module's Special Assistance Supplement.}}{m9gr11}
%
An object moves around the circle shown such in the sketch, with:
%
\Eqn{}{s(t) = (8.0\unit{m/s}) t - (1.00\unit{m/s\up{2}}) t^2\,.}
%
The problem is to determine the object's kinematic variables at several
different times.

\begin{one-digit-list}
\item [a.] To start out with, write: \m{s(t) = b t + c t^2}.
\item [  ] Determine \vect{r} in terms of \m{r}, \m{\theta}, \m{\uvec{r}}, \m{\uvec{\theta}}, \m{b}, \m{c}, and \m{t}.
\item [b.] Determine \vect{v} in terms of \m{r}, \m{\theta}, \m{\uvec{r}}, \m{\uvec{\theta}}, \m{b}, \m{c}, and \m{t}.
\help{19}
\item [c.] Determine \vect{a} in terms of \m{r}, \m{\theta}, \m{\uvec{r}}, \m{\uvec{\theta}}, \m{b}, \m{c}, and \m{t}.
\help{20}
\item [d.] Substitute the numerical values for \m{r}, \m{b}, and \m{c} and determine the object's position
vectors, \m{\vect{r}(t)}, at \m{t = 2.0\unit{sec}}, \m{4.0\unit{sec}}, and \m{6.0\unit{sec}}.
\item [e.] Same as in part (d) but for the object's velocity.
\item [f.] Same as in part (d) but for the object's acceleration.
\item [g.] Determine the speed and direction of travel (CW or CCW) at \m{t = 2.0\unit{sec}},
           \m{4.0\unit{sec}}, and \m{6.0\unit{sec}}. \help{22}
\item [h.] Determine the magnitude and direction of the acceleration (CW/CCW and inward/outward) at \m{t = 2.0\unit{sec}},
           \m{4.0\unit{sec}}, and \m{6.0\unit{sec}}. \help{22}
\item [j.] Sketch \vect{r}, \vect{v}, and \vect{a} for each of the times
           of part (d).
\item [k.] Determine whether the speed is increasing, decreasing, or not
           changing at each of the times of part (d).
\item [l.] What is happening to the direction of motion at \m{t = 4\unit{s}}?
\end{one-digit-list}
\end{one-digit-list}

\BriefAns

\begin{one-digit-list}
\item [1.] \m{1.02 \times 10^3\unit{m/s}}, \m{2.72 \times 10^{-3}\unit{m/s\up{2}}} inward.
\item [2.] 463\unit{m/s}, 0.0337\unit{m/s\up{2}} (toward the center of the earth).
\item [3.] 9.9\unit{ft/s\up{2}}.
\item [4.] 51\unit{ft/s\up{2}} (toward the center of the circle).
\item [5.] \CharacterUnframedFigure{m9gr12}
\item [6.] CCW, decreasing.
\item [7.] \NullItem
\begin{one-digit-list}
\item [a.] \m{\vect{r} = r \, \uvec{r}(\theta)}, where \m{\theta = \dfrac{s}{r} = \dfrac{bt + ct^2}{r}}\,.
\item [b.] \m{\vect{v} = (b + 2ct)\,\uvec{\theta}(\theta)}\,.
\item [c.] \m{\vect{a} = 2c\,\uvec{\theta}(\theta) - \dfrac{(b + 2ct)^2}{r}\,\uvec{r}(\theta)}\,.
\item [d.] \m{\vect{r}(2.0\unit{s}) = 4.0\unit{m}\,\uvec{r}(\theta_2)}; \m{\theta_2 = 3.00\unit{rad} = 171.9\degrees \,\rightarrow\, 1.7 \times 10^2\unit{degrees}}.
\item [  ] \m{\vect{r}(4.0\unit{s}) = 4.0\unit{m}\,\uvec{r}(\theta_4)}; \m{\theta_4 = 4.00\unit{rad} = 229.2\degrees \,\rightarrow\, 2.3 \times 10^2\unit{degrees}}.
\item [  ] \m{\vect{r}(6.0\unit{s}) = 4.0\unit{m}\,\uvec{r}(\theta_6)}; \m{\theta_6 = 3.00\unit{rad} = 171.9\degrees \,\rightarrow\, 1.7 \times 10^2\unit{degrees}}.
\item [e.] \m{\vect{v}(2.0\unit{s}) = 4.0\unit{m/s}\;\uvec{\theta}(\theta_2)}.
\item [  ] \m{\vect{v}(4.0\unit{s}) = 0.0}.
\item [  ] \m{\vect{v}(6.0\unit{s}) = -4.0\unit{m/s}\;\uvec{\theta}(\theta_6)}.
\item [f.] \m{\vect{a}(2.0\unit{s}) = [-2.0\;\uvec{\theta}(\theta_2)-4.0\,\uvec{r}(\theta_2)]\unit{m/s\up{2}}},
\item [  ] \m{\vect{a}(4.0\unit{s}) = [-2.0\;\uvec{\theta}(\theta_4)]\unit{m/s\up{2}}},
\item [  ] \m{\vect{a}(6.0\unit{s}) = [-2.0\;\uvec{\theta}(\theta_6)-4.0\,\uvec{r}(\theta_6)]\unit{m/s\up{2}}},
\item [g.] \m{v(2.0\unit{s}) = 4.0\unit{m/s}}, CCW.
\item [  ] \m{v(4.0\unit{s}) = 0.0\unit{m/s}}, (stationary).
\item [  ] \m{v(6.0\unit{s}) = 4.0\unit{m/s}}, CW.
\item [h.] \m{a(2.0\unit{s}) = 4.5\unit{m/s\up{2}}}, CW and inward.
\item [  ] \m{a(4.0\unit{s}) = 2.0\unit{m/s\up{2}}}, CW.
\item [  ] \m{a(6.0\unit{s}) = 4.5\unit{m/s\up{2}}}, CW and inward.
\item [i.] \TextAndFigure{The sketch is for \m{t = 2} sec.
For the other times, check that the vectors agree with the values you
computed above and with common sense.}{m9gr13}
\item [j.] For 2, 4, 6\unit{sec}: decreasing, not changing, increasing.
\item [k.] It is reversing.
\end{one-digit-list}
\end{one-digit-list}
}% /ProbSet

}% /Sect