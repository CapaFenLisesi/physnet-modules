\revhist{12/17/90, pss; 6/3/91, pss; 9/8/94, pss; 4/26/95, pss; 11/8/95, pss;
         5/9/98, pss; 11/3/98, pss; 11/23/98, pss; 10/25/99, pss; 3/3/2000, pss; 12/11/2000, pss;
         4/13/02, pss}

\Sect{}{}{\SectType{SpecialAssistance}}{

\AsItem{1}{TX-4a}{If you have trouble with sine, cosine, and/or tangent
functions, study the trigonometry section in \Quote{Math Review,} MISN-0-401
or study the relevant section in any high school trigonometry book.
}

\AsItem{2}{TX-5b}{A rope can be treated just like any other object.
If it has mass, then the difference between the magnitudes of the forces
at its two ends is just (\m{m\,a}) for the rope.  If the mass of the rope
is negligible, then the difference between the magnitudes of the forces at
its two ends is negligible.}

\AsItem{3}{TX-5b}
{\m{F_\text{incline force} = \sqrt{(1.4\unit{lb})^2 + (12.8\unit{lb})^2} = 13\unit{lb}}

 \m{\phi_\text{incline force} = \tan^{-1}\left(\dfrac{12.8\unit{lb}}{-1.4\unit{lb}}\right) = - 84\degrees} or {96\degrees}
 (all arctangents have a 2-quadrant ambiguity: use other information to deduce which is the right answer). 
}

\AsItem{4}{PS-Problem~1}
{\m{F_\text{gravity,x} = -29.49\unit{N}}; \m{F_\text{gravity,y} = -39.1\unit{N}}\newline
 \m{F_{P,x} = 39.93\unit{N}}; \m{F_{P,y} = 30.09\unit{N}}\newline
 \m{F_{R,x} = m\,a_x = 15\unit{N}}; \m{F_{R,y} = m\,a_y = 0}
}

\AsItem{5}{PS-Problem~2}
{\m{F_\text{gravity,x} = 0}; \m{F_\text{gravity,y} = -48\unit{lb}}\newline
 \m{F_{P,x} = 4.0\times10^1\unit{lb}}; \m{F_{P,y} = -3.0\times10^1\unit{lb}}\newline
 \m{F_{R,x} = m\,a_x = 3.0\unit{lb}}; \m{F_{R,y} = m\,a_y = 0}
}

\AsItem{6}{PS-Problem~3}
{\m{F_\text{gravity,x} = 0}; \m{F_\text{gravity,y} = -9.8\unit{N}}\newline
 \m{F_{R,x} = m\,a_x = 3.6\unit{N}}; \m{F_{R,y} = m\,a_y = 0}\newline
 Also, note how the angle \m{\theta} is defined in this problem.
}

\AsItem{7}{PS-Problem~4}
{\m{F_\text{gravity,x} = -62.5\unit{lb}}; \m{F_\text{gravity,y} = -147.3\unit{lb}}\newline
 \m{F_\text{rope,x} = 100\unit{lb}}; \m{F_{ropeP,y} = 0}\newline
 \m{F_{R,x} = m\,a_x = 25\unit{lb}}; \m{F_{R,y} = m\,a_y = 0}
}

\AsItem{8}{PS-Problem~5}
{\m{F_\text{gravity,x} = -127.8\unit{lb}}; \m{F_\text{gravity,y} = -96.3\unit{lb}}\newline
 \m{F_{R,x} = m\,a_x = 120\unit{lb}}; \m{F_{R,y} = m\,a_y = 0}
}

\AsItem{9}{PS-Problem~5}
{If you have worked your way through all of the text in the module and have worked
all of the previous problems in the Problem Supplement and still have trouble with
this one, go back to the text of this module and review all parts of it.
Then review your worked-out solutions to all of the prior problems.
After doing those two things in their entirety, attempt this one again.}

\AsItem{10}{TX-4a}{If you got \m{-73\degrees} then you have run into the
\Quote{ambiguous quadrant} problem and you must do further work.
Calculators generally give the arctangent of a positive number as being
between {0\degrees} and {90\degrees}, in the first quadrant, although
the actual angle may be in either the first or the third quadrant, and they
give the arctangent of a negative number as being between {0\degrees} and
\m{-90\degrees}, in the fourth quadrant, although the actual angle may be in
either the second or the fourth quadrant.
To resolve either of these \Quote{quadrant ambiguities,} you must also examine
the signs of the \m{x}- and \m{y}-coordinates of a point at the angle in question.
}

}% /Sect
