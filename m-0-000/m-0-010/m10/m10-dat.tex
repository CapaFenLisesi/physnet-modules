\revhist{8/1/91, pss; 10/22/91, pss; 4/24/92, pss; 2/17/93, pss; 5/8/94, pss;
         5/9/94, pss; 10/21/94, pss; 10/4/95, pss; 11/8/96, pss; 2/11/97, pss;
         8/23/97, pss; 10/20/97, pss; 6/4/98, pss; 9/23/98, pss; 11/23/98, pss;
         12/1/98, pss; 12/3/98, pss; 2/22/99, pss, 5/3/99, pss; 10/14/99, pss;
         10/25/99, pss; 2/24/00, pss; 3/3/00, pss; 12/18/00, pss; 10/26/01, pss;
         4/13/02, pss; 4/24/02, pss; 9/12/02, pss}
%
\defModTitle{\ph{One-Body Diagrams} \ph{and Contact Forces}}
\defCtAuthor{Harry \inits{G.} Dulaney, Georgia Institute of Technology,
           and Peter Signell, Michigan State University}
\defIdAuthor{Harry G.\,Dulaney, Georgia Inst.\,of Tech., Atlanta, GA, and Peter Signell, Mich.\,State Univ., East Lansing, MI}
%
\defIdItems{
    \IdVersEval{9/12/2002}{1}
    \IdHours{1}
    \begin{InputSkills}
    \item [1.]  State Newton's laws of motion and apply them to problems
    involving constant forces acting on one particle traveling in one
    dimension \prrqone{0-14}, \prrqone{0-15}.
    \end{InputSkills}
    %
    \begin{KnowledgeSkills}
    \item [K1.] Vocabulary: contact force, non-contact force, surface force,
                resultant force.
    \end{KnowledgeSkills}
    %
    \begin{RuleApplicationSkills}
    \item [R1.] Given a non-rotating object on a linear trajectory, or two
    such objects connected by a string or rope, possibly over a pulley,
    and given the information on the acceleration of the object and all
    forces except a surface force, determine the magnitude and direction
    of the surface force:
    \begin{one-digit-list}
    \item [a.] Draw a one-body diagram for each object showing the given
    forces acting on the object, with each force clearly labeled.
    Resolve those forces into components with one axis along the direction
    of acceleration and draw those components on another one-body force
    diagram.
    \item [b.] Write Newton's second law in component form for the known
    forces and the resultant force.
    \item [c.] Solve for the unknown surface force and add it the the
    original one-body force diagram.
    \end{one-digit-list}
    \end{RuleApplicationSkills}
}