\revhist{9/7/89, pss; 12/8/89, pss; 8/1/91, pss; 4/24/92, pss; 5/9/94, pss;
         10/21/94, pss; 9/11/95, pss; 11/7/97, lae; 5/6/98, pss; 9/23/98, pss;
         11/23/98, pss; 12/3/98, pss; 6/21/99, pss; 7/15/99, abs; 12/11/00, pss;
         3/29/01, kag; 10/26/01, pss; 4/13/02, pss}
%
\Sect{1}{Introduction}{\SectType{TextMultiPara}}{
%
\pcap{1}{a}{Overview}
In this module we help you develop understandings and techniques that
are useful in almost any problem in science and engineering where
forces are involved, as well as being useful for understanding every-day
phenomena.
To keep the new understandings and techniques in focus, we will illustrate
them with a single type of problem:
%
\begin{itemize}
\item Given the mass and acceleration of an object, and all but one
of the forces acting on it, determine that unknown force.
\end{itemize}
For this purpose, one must use Newton's second law.%
%
\Footnote{1}{See \Quote{Particle Dynamics; The Laws of Motion} (MISN-0-14).}
%

\pcap{1}{b}{Mass, Acceleration, Resultant Force}
Given an object's mass \m{m} and acceleration \vect{a}, Newton's second law gives
the \textit{resultant force} on the object:
%
\Eqn{}{m \; \vect{a} = \vect{F}_R\,,}
%
where the resultant force on an object, \m{\vect{F}_R}, is defined as the sum
of all forces \Index{force| resultant}which are acting on that object.
Mathematically,
%
\Eqn{}{\vect{F}_R = \sum \vect{F}_i\,,}
%
where the sum is over all of the forces \m{\vect{F}_i}, \m{i = 1, 2, 3, ...} acting on the object.

\pcap{1}{c}{Solving Unknown Force Problems}
To solve the problem of determining an unknown force that is acting on an
object we concentrate our attention on that object, call it \Quote{object \m{A}.}
Our first task is to determine all of the other objects that are
exerting forces on object \m{A} and record the characteristics of each of
those forces.
We do this by making a stylized sketch of object \m{A} and adding a labeled
vector to the sketch for each force, as we identify it, showing its
approximate direction and magnitude.
In this manner we draw the vectors for all of the forces acting on object
\m{A} except for the vector representing the unknown force.
Then we merely see what force vector we need to draw so that all of the
forces acting on \m{A}, including the unknown one, add to the right
resultant.
The resultant is the force required by Newton's second law (remember
that the object's mass and acceleration are \textit{given} in the type of
problem we are considering).
We then add the necessary force vector to the diagram and it is the
approximate answer.
To find the precise direction and magnitude of the unknown force,
we use the sketch as a guide and solve the component versions of Newton's
second law.
Finally, we check the precise answer against the approximate answer that we
drew on the diagram.

\pcap{1}{d}{Straight-line Trajectories, No Explicit Friction}
In this module we restrict ourselves to treating objects that are
following straight-line trajectories and that are not rotating.
In addition, we do not consider frictional forces explicitly but simply
consider them as included in the forces exerted on objects
through surfaces with which the objects are in contact.
Frictional forces are separated out from non-frictional forces elsewhere.
%
\Footnote{2}{See \Quote{Friction in Applications of Newton's Second Law} (MISN-0-16).}
%
Despite the restrictions imposed in this module, the understandings and
techniques we develop here have wide applicability.
}% /Sect
%
\Sect{2}{Identifying, Representing Forces}{\SectType{TextMultiPara}}{
%
\pcap{2}{a}{Identifying Forces}
The forces acting on an object (which we will call object \m{A}) should be
systematically determined by separately identifying two kinds of forces
that act on the object: (1) \Index{force| non-contact}non-contact (\Quote{action at a distance}) forces;
and (2) \Index{force| contact}contact forces.

The only non-contact force \Index{non-contact force}that we deal with in this module is
the gravitational force of the earth on our object.
The strength of this force is what is commonly called the \Quote{weight} of
the object and is commonly denoted by the symbol \m{W}.
This force on the object is always directed \Quote{downward,} toward the
center of the earth.
%
\Footnote{3}{Electric and magnetic forces are examples of other non-contact
forces but these are dealt with elsewhere.  See \Quote{Introduction to
Electricity and Magnetism,} MISN-0-121.}
%

Anything that touches the object, such as ropes, flat surfaces, and springs,
can exert a \Quote{contact force} \Index{contact force}on the object.
To find these contact forces you must determine which external objects
\Index{external force}touch the object.
%
\ThreeCaptionedFramedFigures{1}{}{m10gr01}{2}{}{m10gr02}{3}{}{m10gr03}%
%
\pcap{2}{b}{Avoid Common Errors} Here are some rules.\newline
%
\icap{Rule 1}: While identifying the forces acting on a object, include
only forces for which you can identify the other objects that exert those
forces\Index{external force}.
For each force that you include, you must be able to answer the question,
\Quote{What other object exerts that force on the object we are examining?}
%
\SubSubSect{}{Example:}{
A satellite moves in a circular orbit about the earth (\Figref{1}).
How many forces act on the satellite (assuming the atmosphere is so
thin at this altitude that it can be neglected)?
There is certainly one non-contact \Quote{action at a distance} force: the
gravitational force of the earth on the satellite.
Since nothing is in contact with the satellite, there are no contact forces.
The final answer: only one force acting on the satellite; the non-contact
gravitational force.
}% /SubSubSect
%
\SubSubSect{}{Example:}{
A skier skis down a steep incline (\Figref{2}).
How many forces act on the skier, neglecting air resistance?
First, there is the non-contact gravitational force of the earth on the
skier (the skier's \Quote{weight}) and the contact force of the incline on the
skier.
The incline is the only thing touching the skier so it exerts the only
contact force present.
The final answer: two forces, the incline contact force and the
gravitational non-contact force.
}% /SubSubSect
%
\icap{Rule 2}: Include only external forces \Index{force| external}acting on the object.
Omit internal forces\Index{force| internal}, those that are acting only between different parts of
the object itself.
%
\SubSubSect{}{Example:}{
An automobile accelerates along a level road (\Figref{3}).
Neglecting air resistance, how many forces act on the entire auto?
Two: the non-contact gravitational force of the earth on the car (its
\Quote{weight}) and the contact force of the road on the car (this force is
divided among the four wheels).
Do not include the force of the engine on the rest of the car.
This is an internal force so it is not to be included in Newton's second law
for the car.
Only external forces exerted on the car by other objects in its environment
can be included when solving for the motion of the entire car (how the
engine's force results in acceleration of cars will be apparent only
after introducing road friction into the problem).
}% /SubSubSect

\icap{Rule 3}: Include only forces exerted on the object.
Omit forces exerted by it on other objects.

\ThreeCaptionedFramedFigures{4}{}{m10gr04}{5}{}{m10gr05}{6}{}{m10gr06}
%
\SubSubSect{}{Example:}{
A block is suspended by a rope from the ceiling of an elevator that
is accelerating vertically upward (\Figref{4}).
How many forces act on the block?
Two: the non-contact gravitational force of the earth on the block and the
contact force of the rope on the block.
Do not include the force of the block on the rope or the force of the
elevator on the rope.
These forces do not act directly on the block.
In order to apply Newton's second law to a object, you must mentally isolate
the object and consider only those forces acting directly on the object.
}% /SubSubSect

\icap{Rule 4}: Use common sense in reaching conclusions about the nature of
particular forces.
For instance, your knowledge of a rope tells you that the force exerted on
an object by a rope will always act along the direction of the rope and be
directed away from the point of contact with the object.

\icap{Rule 5}: Be wary of fixing the magnitude or direction of unknown
forces according to intuition.
%
\SubSubSect{}{Example:}{
A person stands on a large box which is on the floor
of an elevator that is accelerating vertically downward (\Figref{5}).
Three forces act on the box: the non-contact gravitational force of the
earth on the box, the contact force of the elevator floor on the box, and
the contact force of the person on the box.

Intuition might tell you that the force of the person on the box has a
magnitude just equal to the weight of the person.
If so, your intuition has led you astray.
If that were true, then from Newton's third law the force of the box on the
person would also be just equal to the weight of the person.
Hence, the person would have zero resultant force on her or him and zero
acceleration.
This cannot be true, since the person has the same downward acceleration as
the elevator.
In this problem the one-body technique should be applied to each object
separately, with unknowns simply left as symbols.

Avoid the temptation to use intuition to guess answers.
Instead, work through the entire problem systematically, letting numerical
values for unknowns come from the simultaneous solution of sets of equations.
Then seek intuitive support for the answer.
}% /SubSubSect
%

\pcap{2}{c}{Example: A Pulled Sled}
A sled is pulled at constant speed along a horizontal surface by means of
a rope, as shown in \Figref{6}.
Determine the forces that act on the sled and draw a one-body diagram
showing those forces.

As usual, the \textit{non-contact} gravitational force of the earth acts
on the sled and is directed downward (toward the center of the earth).
In addition, two objects (the rope and the snow surface) touch the sled
and hence exert \textit{contact} forces on the sled.
Thus there are three forces on the sled; one non-contact force and two
contact forces.
Our knowledge of ropes tells us that the force of the rope on the sled is
along the direction of the rope.
The acceleration of the sled is zero (\Quote{constant speed}) so the resultant
force is zero.
Then the force of the snow surface on the sled must be such that all
three forces add to zero.
}% /Sect
%
\Sect{3}{Drawing a One-Body Force Diagram}{\SectType{TextMultiPara}}{
%
\pcap{3}{a}{Overview}
When we start working on a problem involving forces acting on an
object, we almost always draw a \Quote{one-body force diagram,}
%
\Footnote{4}{In current physics this is universally called a \Quote{free-body
force diagram\Index{free body diagram}.}
We do not use that term because in current physics the term \Quote{free
body} universally means a body on which no forces are acting.}
%
so called because the diagram shows all of the forces acting on one body.
If more than one object is involved, we draw a separate force diagram
for each object.
The diagrams we draw are somewhat stylized, in that we almost always
draw the object as a rectangle.
In addition, we draw the forces as though they were all exerted at
the same point on the object, usually taken as the \Quote{center of mass}
of the object
%
\Footnote{5}{For a rigorous discussion of the \Quote{center of mass} concept,
see \Quote{Tools for Static Equilibrium}, MISN-0-5.}
%
(see \Figref{7}, a stylized representation of the sled in \Figref{6} and of
the forces acting on the sled).

\pcap{3}{b}{The Component Diagram}
Once the one-body force diagram has been drawn, the forces on it must be
broken down into component forces.
This is because Newton's second law is a vector equation, so it can be
expressed as a separate equation for each of the components.  That is
generally the form one needs for solution of the problem.
The orientation of the coordinate axes, usually called the \m{x}- and
\m{y}-axes, is free to be chosen by the problem solver and it is usually
chosen on the basis of what would make solution easier.
We urge you to draw the force components on a second one-body force
diagram as in \Figref{8}.

\pcap{3}{c}{Orienting the Coordinate System}
You should orient your coordinate system so as to simplify the component
form of Newton's second law.
In many problems, due to the constraints present, such as ropes or
other surfaces, it is known that the acceleration component in a particular
direction is zero.
In this case, coordinates parallel and perpendicular to this direction
should be chosen.
For example, the skier in \Figref{2} is known to have zero component of
acceleration perpendicular to the surface of the snow because that
component of the skier's \textit{velocity} is constant (it has the
constant value \Quote{zero}).
To make the problem easier to solve, choose one axis of the coordinate
system in the direction of zero acceleration.
That puts other axes in the plane of the snowy surface.

\pcap{3}{d}{The Diagram Action Point}
On a one-body force diagram we draw the forces as all acting at the \Quote{center
of mass} of the object, even though this may be far from where the forces
actually act.
This shifting of the forces' points of action is immaterial as long
as the object does not start to rotate and as
long as we are not interested in how close the object is to rotational
instability.
%
\TwoCaptionedFramedFigures{7}{Forces acting in \Figref{6}.}{m10gr07}%
                          {8}{Components of the forces in \Figref{7}.}{m10gr08}
}% /Sect
%
\Sect{4}{Complete Example Solutions}{\SectType{TextMultiPara}}{
%
\pcap{4}{a}{A Pulled Sled}
Suppose the sled described in Sect.\,2c weighs 16\unit{lb}
and has a 5.0\unit{lb} force exerted on it by the attached rope which is
at an angle of {37\degrees} to the horizontal (see \Figref{6}).
Given this data, determine the force of the snow surface \Index{force| normal}on the sled.

First, we sketch the one-body force diagram, putting in the forces we
identified in Sect.\,2c.
We draw the gravity and rope forces roughly to scale and draw the
snow-surface force such that the sum of all three forces roughly equals
\m{m\,\vect{a}} (zero in this case).

Next, we orient the coordinate system so as to make solution as easy
as possible.  In this case the force of gravity is a vertical force
so we pick \Quote{up} as the positive \m{y}-direction, partly because of
tradition and partly so the gravitational force will have only
one component that is non-zero.
We pick the \m{x}-direction as being \Quote{to the right,} parallel to the
snow surface, out of tradition.

Next, we resolve the known forces acting on the sled into components
along the coordinate axes we have just picked.
We use \m{\vect{F}_{rope}}, \m{\vect{F}_\text{snow}}, and \m{\vect{F}_\text{gravity}} to
indicate the three forces and symbols without the vector signs to
indicate the magnitudes of the forces (except for the magnitude of the
gravitational force which is traditionally indicated by the symbol
\m{W} for \Quote{Weight.}).
We use \m{\theta} and \m{\phi} to indicate the angles that the snow force
and the rope force make, respectively, with the positive \m{x}-axis:
\help{1}
%
\begin{eqnarray*}
%
F_\text{rope,x} & = & F_\text{rope} \cos{\theta} = 5.0\unit{lb} \cos{37\degrees} = 4.0\unit{lb}\,,\\
%
F_\text{rope,y} & = & F_\text{rope} \sin{\theta} = 5.0\unit{lb} \sin{37\degrees} = 3.0\unit{lb}\,,\\
%
F_\text{gravity,x} & = & 0 \,,\\
%
F_\text{gravity,y} & = & -W = -16\unit{lb}\,.
%
\end{eqnarray*}
%
By Newton's second law\Index{Newton's second law, application of} both the \m{x}- and \m{y}-components of the
resultant force must be zero (for this zero-acceleration case) so we
add in the right amount of snow force to make those resultant components zero:
%
\begin{eqnarray*}
%
F_\text{snow,x} & = & -4.0\unit{lb}\,,\\
%
F_\text{snow,y} & = & 13\unit{lb}\,.
%
\end{eqnarray*}
%
The angle and magnitude of the snow force can now be found from
these two components.
First, the ratio \m{F_y/F_x} gives the tangent of the angle, \m{\tan{\phi}},
so:
%
\newsavebox{\hlp}
\sbox{\hlp}{\,\help{10}}
\Eqn{}{\phi =
  \tan^{-1}\left(\dfrac{13\unit{lb}}{-4.0\unit{lb}}\right) = 107\degrees\,\usebox{\hlp}.}
%
Next, the magnitude of the snow force is given by the square root
of the sum of the squares of its components:
%
\Eqn{}{F_\text{snow} = \sqrt{(-4.0\unit{lb})^2 + (13\unit{lb})^2} =
                               13.6\unit{lb} \approx 14\unit{lb}\,.}
%
This angle and magnitude checks against the approximate value drawn
earlier in the one-body force diagram (\Figref{7}).
%A formal solution is given in the Appendix.
%
\TwoCaptionedFramedFigures{9}{As in \Figref{7}, but with acceleration.}{m10gr09}
                          {10}{Force components for \Figref{9}.}{m10gr10}

\pcap{4}{b}{The Sled with Acceleration}
Suppose we are asked to solve the sled problem of Sect.\,4a when the
sled is accelerating at, say, \m{a = 8.0\unit{ft/s\up{2}}\,\uvec{x}} where
the coordinate axes are as in \Figref{8}.
The mass of the sled is: \m{m = W/g = 16\unit{lb}/(32\unit{ft/s\up{2}}) =
0.50\unit{lb}\unit{s\up{2}}/\unit{ft}}.  Then the resultant force must be:
\m{\vect{F_R} = m \vect{a} = 4.0\unit{lb}\,\uvec{x}}.
Looking at \Figref{7} where the resultant force was zero, we must add
the new \vect{F_R} to the snow force \Index{normal force}so that the result of adding
all the forces, the new resultant, will be the new non-zero \vect{F_R}.
The result is shown in \Figref{9} where one can easily see that the sum
of the forces points approximately in the \m{\uvec{x}} direction.
Breaking the forces into components gives \Figref{10}: the snow
force is now at an angle of {90\degrees} with a magnitude of
13\unit{lb}.
Mentally add the components and prove to yourself that the resultant is
indeed equal to \m{m \vect{a}}.

\pcap{4}{c}{Block on Incline}
A block weighing \m{98\unit{N}} is at rest on a {37\degrees} incline.
Our job is to determine the force of the incline on the block.

The forces acting on the block are two: the non-contact force of gravity
and the contact force exerted on the block by the surface of the incline.
The gravity force is \m{F_\text{gravity} = - 98\unit{N}\,\uvec{y}} and the
resultant is zero (the block is \Quote{at rest}) so the force exerted on the
block by the incline surface must be \m{F_\text{incline} = + 98\unit{N}\,\uvec{y}}.
This is shown in \Figref{11}.
%
\TwoCaptionedFramedFigures{11}{A block on an inclined plane.}{m10gr11}
                          {12}{Two blocks connected by a string over a pulley.}{m10gr16}
%
}% /Sect
%
\Sect{5}{Connected Bodies}{\SectType{TextMultiPara}}{
%
\pcap{5}{a}{Introduction}
A type of problem frequently encountered which deserves special
consideration involves two or more bodies that are coupled together by
contact or by a rope or by a pole\Index{connected bodies}.
Such bodies are constrained to move together and will have equal magnitudes
of acceleration.
If the connection is a rope, we can often neglect the weight of the
rope since that force is small compared to the forces exerted at the ends
of the rope.
Now the forces acting on the ends of the rope are in opposite
directions so their (vector) sum is just the difference of their
magnitudes.
By Newton's second law \Index{Newton's second law, application of}that
difference is just the rope's mass times its acceleration.
If we consider the rope's mass to be too small to be important, then
the rope will always have equal magnitudes of force on its two ends.
That is what we almost always assume in determining approximate solutions.

\pcap{5}{b}{Connected Blocks; One on Incline, One Hanging}
Consider the system shown in \Figref{12}.
A block weighing 16\unit{lb} is on a {37\degrees} incline and is attached to an
inextensible massless rope which passes over a frictionless pulley to a
second block that weighs 20\unit{lb}.
The hanging block is accelerating downward at 8\unit{ft/s\up{2}}.
Our problem is to determine, for the block that is on the incline, the force exerted
on it by the surface of the incline.

Since the rope is massless, it will have equal forces on its two ends
\help{2}.
This means we can solve for the rope force on the hanging block and that
will be the rope force on the incline block.
We draw one-body force diagrams for each block separately and these are
shown in \Figref{13}, where the force exerted by each end of the rope is
labeled \Quote{\texttt{rope}}.
%
\CaptionedFullFramedFigure{13}{The forces for \Figref{12}: (a) Block 1; (b) Block 2.}{m10gr17}
%
\CaptionedFullFramedFigure{14}{The force components for \Figref{13}.}{m10gr18}
%
From the one-body force diagram for the hanging block we get that the
resultant force (\m{\vect{F}_R = m\vect{a}}) is 5\unit{lb} downward and the
gravity force is 20\unit{lb} (given) downward.
Then the rope force must be 15\unit{lb} upward.
Check that the sum of the two quoted forces on the block is indeed equal
to the required resultant force on the block.

Since the incline block is accelerating up the incline, we take our
coordinate axes along the incline, as shown in \Figref{14}.
We break the gravity force up into components along the axes shown
and we get for the forces on the incline block:
%
\begin{eqnarray*}
F_\text{gravity,x} & = & W \sin{37\degrees} = -9.6\unit{lb}\\
F_\text{gravity,y} & = & W \cos{37\degrees} = -12.8\unit{lb}\\
F_\text{rope,x} & = & 15\unit{lb} \\
F_\text{rope,y} & = & 0 \\
F_{R,x} & = & m a_x = 4.0\unit{lb}\\
F_{R,y} & = & m a_y = 0
\end{eqnarray*}
%
Now gravity plus rope plus incline must equal resultant so:
%
\begin{eqnarray*}
F_\text{incline,x} & = & -1.4\unit{lb}\\
F_\text{incline,y} & = & 12.8\unit{lb}
\end{eqnarray*}

Then the force of the incline on the incline block is 13\unit{lb} at an angle of
{96\degrees} from the positive \m{x}-axis. \help{3}
}% /Sect
%
\Sect{}{Acknowledgments}{\SectType{Acknowledgments}}{\NsfAcknowledgment}% /Sect

