\revhist{8/6/84, wcl; 9/7/91, pss; 9/29/94, pss}

\Sect{}{}{\SectType{SpecialAssistance}}{

\AsItem{1}{PS-problem~3g}
{Write down the vector relation between \m{\vect{r}_{QO}}, \m{\vect{r}_{QO'}},
 and \m{\vect{r}_{O'O}} and  draw a vector addition diagram illustrating this
 relation.
 Now use either the law of cosines or the properties of the
 dot product to find the angle between \m{\vect{r}_{QO}} and \m{\vect{r}_{QO'}}.
}

\AsItem{2}{PS-problem~3h}
{Looking along the \m{z}-axis, one does not see the \m{z}-component of a vector.
 Then \m{r^2_{QO} = x^2_{QO} + y^2_{QO}}, etc.
}

\AsItem{3}{TX-2e}
{This is a 1-dimensional problem so the vector notation can be dropped:
 %
 \Eqn{}{v_{BA} = v_B - v_A\,.}
 %
 You are being observed by the trooper, relative to the trooper, so the
 equation reads:
 %
 \Eqn{}{v_\text{you relative to trooper} = v_\text{you} - v_\text{trooper}\,,}
 and so:
 \Eqn{}{v_\text{you relative to trooper} = 45\unit{mph} - 65\unit{mph} =
 - 20\unit{mph}\,.}
}

\AsItem{4}{TX-2e}
{Write the equations in [S-3] as vector equations.
 Align one of the coordinate axes along one of the roads and take components
 of the equations.
 There will be sines and cosines in the component equations.
 After solving the equations, sketch the vectors approximately to scale and
 see if your answer is approximately right.
}

\AsItem{5}{PS, problem~3c}
{Apply the right equations from the text to get:
 \m{\vect{r}_{O'O} = -\vect{r}_{QO'} + \vect{r}_{QO}}.
}

}% /Sect
