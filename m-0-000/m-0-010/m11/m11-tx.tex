\revhist{8/6/84 wcl; 9/6/91, pss; 9/29/94, pss; 4/7/97, lae}
%
\Sect{1}{Applications of Relative Motion}{\SectType{TextMultiPara}}{
%
\pcap{1}{a}{Navigation}
An application of relative motion which immediately comes to mind is that of
navigation through currents of water and air.
Consider two observers and an object, where each observer sees both the
other observer and the object as moving.
Given the equation of motion of the object as seen by one observer, what
equation of motion will the other observer see for it?
Here the equation of motion set by the pilot, with respect to the fluid
through which he moves, must be such as to produce the correct equation of
motion with respect to the land.
Can you tell which are the two \Quote{observers} and which is the \Quote{object?}
Can you transform the desired land-based equation into the fluid-based one
needed for navigation?

\pcap{1}{b}{Solar System Dynamics}
For another application, consider the apparent motions of the planets across
the night sky.
Since very ancient times, the positions of the planets at various times had
been carefully recorded.
In about 140 A.D., from this vast set of numbers, the planetary equations of
motion were finally deduced by Ptolemy.
These equations were so excruciatingly complicated, however, that the forces
which would produce them must have seemed incomprehensible.
Then in 1543, in what must have been one of the most thrilling discoveries
in history, Copernicus transformed the equations of motion to the way they
would be observed from the sun and thereupon found them to be trivially
simple.
This eventually spurred Newton to discover the elegantly simple universal
laws of motion and of gravitation, and to invent calculus.
From that day to this, a fundamental assumption of physicists has been that
all of the forces of nature will be found to be elegantly simple when
properly expressed.
}% /Sect
%
\Sect{2}{Double Subscript Notation}{\SectType{TextOnePara}}{
%
\pcap{2}{a}{Relative Position Vectors}
Double subscript notation is a simple and consistent method of dealing with
the relative positions of two objects.
If \m{\vect{r}_A} is the position vector of object \m{A} and \m{\vect{r}_B} is the
position vector of object \m{B} in a given coordinate system, then the
position vector of \m{B} relative to \m{A} is defined by
%
\Eqn{1}{\vect{r}_{BA} = \vect{r}_B - \vect{r}_A\,.}
%
Notice that \m{\vect{r}_{BA}} is the vector which, when added to \m{\vect{r}_A},
gives \m{\vect{r}_B}:
%
\Eqn{2}{\vect{r}_A + \vect{r}_{BA} = \vect{r}_B\,.}
%
That is, if you start at the position of object \m{A}, and move along
\m{\vect{r}_{BA}}, you end up at the position of object \m{B} (see \Figref{1}).

\CaptionedLeftFramedFigure{1}{Vector interpretation of double subscript
notation.}{m11gr01}

\pcap{2}{b}{The Order of the Subscripts}
The order of the subscripts of a relative position vector determines the
direction of the vector.
The vector \m{\vect{r}_{BA}} is directed from object \m{A} to object \m{B}.
The vector \m{\vect{r}_{AB}}, from \m{B} to \m{A}, would obviously point in the
opposite direction, and have exactly the same length as \m{\vect{r}_{BA}}, so
%
\Eqn{3}{\vect{r}_{AB} = - \vect{r}_{BA}\,.}

\pcap{2}{c}{Velocity and Acceleration from Relative Position}
Once the relative position has been determined, it is simple to get the
velocity and acceleration of \m{B} relative to \m{A}, just by differentiating
\Eqnref{1}:
%
\TwoEqns{}{\vect{v}_{BA} & = \dfrac{d}{dt}(\vect{r}_{BA}) =
  \dfrac{d}{dt}(\vect{r}_B - \vect{r}_A) =
  \dfrac{d}{dt}\vect{r}_B - \dfrac{d}{dt}\vect{r}_A = \vect{v}_B - \vect{v}_A}
          {\vect{a}_{BA} & = \dfrac{d}{dt}\vect{v}_{BA} =
  \dfrac{d}{dt}(\vect{v}_B - \vect{v}_A) =
  \dfrac{d}{dt}\vect{v}_B - \dfrac{d}{dt}\vect{v}_A = \vect{a}_B - \vect{a}_A\,,}
so
%
\Eqn{4}{\vect{v}_{BA} = \vect{v}_B - \vect{v}_A\,,}
%
\Eqn{5}{\vect{a}_{BA} = \vect{a}_B - \vect{a}_A\,.}
%
Notice that the mathematical relationships between \m{\vect{v}_{BA}},
\m{\vect{v}_A}, and \m{\vect{v}_B}, and between \m{\vect{a}_{BA}}, \m{\vect{a}_A}, and
\m{\vect{a}_B} are identical to those between \m{\vect{r}_{BA}}, \m{\vect{r}_A},
\m{\vect{r}_B}.
For example, if you start out having velocity \m{\vect{v}_A} (or acceleration
\m{\vect{a}_A}), give yourself the additional velocity \m{\vect{v}_{BA}} (or
acceleration \m{\vect{a}_{BA}}), you end up having velocity \m{\vect{v}_B} (or
acceleration \m{\vect{a}_B}).

\pcap{2}{d}{Subscript Reversal on \vect{v} and \vect{a}}
It is also easy to see that the velocity of \m{A} relative to \m{B},
\m{\vect{v}_{AB}}, and the acceleration of \m{A} relative to \m{B}, \m{\vect{a}_{AB}},
are related to \m{\vect{v}_{BA}} and \m{\vect{a}_{BA}} in just the
same way as \m{\vect{r}_AB} is related to \m{\vect{r}_{BA}}:
%
\Eqn{6}{\vect{v}_{AB} = - \vect{v}_{BA}\,,}
%
\Eqn{7}{\vect{a}_{AB} = - \vect{a}_{BA}\,.}

\pcap{2}{e}{An Example}
As an example of the velocity addition law, \Eqnref{4}, suppose
you are driving down a highway at 45\unit{mph} when a state trooper approaches you
from the rear at 65\unit{mph}.
If the trooper has a radar gun trained on you, what will it read?
Using \Eqnref{4}, we find the answer to be 20\unit{mph}. \help{3}
Perhaps the answer was obvious to you without using \Eqnref{4}.

Now suppose the trooper is traveling along a road that makes an angle with
the road on which you are traveling.
What will the radar gun read then?
Perhaps it is obvious that \Eqnref{4} is needed in order to
determine the result in this case. \help{4}
}% /Sect
%
\Sect{3}{Frames of Reference}{\SectType{TextOnePara}}{
%
\CaptionedFullFramedFigure{2}{Relative position vectors defined for
various observer labels where \m{\vect{r}_{BA}} is (a) the difference between
two vectors, or (b) the sum of two vectors.}{m11gr02}

\pcap{3}{a}{\Quote{Observer} Labels}
Sometimes it may be easier to deal solely with relative quantities like
\m{\vect{r}_{AB}} instead of \Quote{absolute} ones like \m{\vect{r}_A}.
To do this, we can simply place an observer \m{O} at the origin of the
coordinate system and measure all the vector quantities relative to that
observer.
The coordinate system is called the observer's \Quote{frame of reference.}
Thus \m{\vect{r}_A} becomes \m{\vect{r}_{AO}}, \m{\vect{v}_A} becomes \m{\vect{v}_{AO}},
and so on (see \Figref{2}a).
\Equationref{1} then becomes
%
\Eqn{8}{\vect{r}_{BA} = \vect{r}_{BO} - \vect{r}_{AO}\,,}
%
or, by applying \Eqnref{3} to get \m{\vect{r}_{AO} =
- \vect{r}_{OA}} (see \Figref{2}b), we have:
%
\Eqn{9}{\vect{r}_{BA} = \vect{r}_{OA} + \vect{r}_{BO}\,.}

\pcap{3}{b}{Checking an Equation's Subscript Sequence}
The addition equation, \Eqnref{9}, may be read \Quote{\m{A} to \m{B} equals \m{A}
to \m{O} plus \m{O} to \m{B}.}
Note that one reads the subscripts from right to left.
This provides a powerful check on the correctness of relative-vector
equations, since it is usually quite easy to rearrange terms into the
addition form.

\pcap{3}{c}{Converting From Other Notations}
Many scientists use a \Quote{prime-unprimed} notation when describing vectors
relative to two frames of reference.
Vectors in that notation may be converted to double subscript notation by
placing observers (say, \m{O} and \m{O'}) at the origins of the reference frames
and expressing the vector quantities relative to these observers.
For example, consider the \Quote{prime-unprimed} velocity addition equation
%
\Eqn{10}{\vect{v} = \vect{v}' + \vect{u}\,,}
%
where \vect{v} is the velocity of an object in the unprimed reference frame,
\m{\vect{v}'} is the velocity of the object in the primed reference frame, and
\vect{u} is the velocity of the primed frame relative to unprimed frame.
\Equationref{10} can be rewritten as
%
\Eqn{11}{\vect{v}_{AO} = \vect{v}_{AO'} + \vect{v}_{O'O}\,,}
%
by making the identifications
%
\ThreeEqns{}{\vect{v}  & = \vect{v}_{AO}  = \text{velocity of } A \text{ relative to } O}
            {\vect{v}' & = \vect{v}_{AO'} = \text{velocity of } A \text{ relative to } O'}
            {\vect{u}  & = \vect{v}_{O'O} = \text{velocity of } O'\text{ relative to } O\,.}
%
The double subscript form is easy to check visually.
}% /Sect
%
\Sect{}{Acknowledgments}{\SectType{Acknowledgments}}{
\NsfAcknowledgment
}% /Sect
%
\Sect{}{Glossary}{\SectType{Glossary}}{
\GlossaryItem{frame of reference} a spatial coordinate system that is used to determine
the position or describe the motion of some object of interest.
\GlossaryItem{observer} a person making measurements, descriptions, or determinations
of some event.
}% /Sect
