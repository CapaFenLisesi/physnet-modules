\revhist{10/15/91, pss; 12/8/92, pss; 1/8/93, pss; 4/28/93, pss; 9/29/94, pss;9/15/95, lae;
         1/31/97, pss; 4/7/97, pss; 8/23/97, pss; 2/22/99, pss}
%
\defModTitle{\ph{Special Relativity:} \ph{The Lorentz Transformation,} \ph{The Velocity Addition Law}}
\defCtAuthor{\inits{P.}Signell, \inits{J.}Borysowicz, and \inits{M.}Brandl}
\defIdAuthor{P.\,Signell, J.\,Borysowicz, M.\,Brandl, Dept.\,of Physics, Mich.\,State
Univ}
%
\defIdItems{
    \IdVersEval{2/1/2000}{1}
    \IdHours{1}
    \begin{InputSkills}
    \item [1.]  Vocabulary: invariance, observer, relative linear motion, velocity addition law \prrqone{0-11}.
    \item [2.]  Given an object's motion relative to one observer, and that observer's motion relative to a second observer, describe the object's motion relative to the second observer \prrqone{0-11}.
    \end{InputSkills}
    %
    \begin{KnowledgeSkills}
    \item [K1.] Vocabulary: Galilean transformation, Lorentz transformation, relativistic velocity addition law.
    \item [K2.] Derive the Galilean transformation from the
    non-relativistic velocity addition law and show that it is invariant under interchange of the observer labels.
    Explain why one intuitively believes that this should be so.
    \item [K3.] Explain clearly how the observed constancy of the speed of light with respect to all observers shows that the Galilean transformation must be erroneous.
    \item [K4.] Given the Lorentz transformation, derive the relativistic velocity addition law from it.
    \item [K5.] Show that the relativistic velocity addition law is in agreement
    with the fact that the speed of light is the same for all observers.
    \item [K6.] Show that, for low enough relative speeds, the difference between the relativistic and non-relativistic velocity addition laws becomes unobservable.
    \end{KnowledgeSkills}
}