\revhist{10/10/83 ljw; 10/14/91, pss; 12/8/92, pss; 4/28/93, pss; 9/29/94, pss;
         23/8/97, pss}
%
\Sect{1}{The Galilean Transformation}{\SectType{TextMultiPara}}{
%
\pcap{1}{a}{Applicable Situations}
The Galilean transformation applies to the physics of \Quote{everyday} life, where
objects move at \Quote{normal} speeds.
It follows from the \Quote{ordinary} velocity addition law.

\pcap{1}{b}{Velocity Addition Law}
Consider an object \m{O} with (one-dimensional) velocity \m{v_\text{OA}(t)} as
determined by (\Quote{relative to}) observer \m{A} and velocity \m{v_\text{OB}(t)},
relative to observer \m{B}.
The ordinary velocity addition law states that:
%
\Eqn{1}{v_\text{OB}(t) = v_\text{OA}(t) - v_\text{BA}\,,}
%
where \m{v_\text{BA}} is the velocity of observer \m{B} relative to observer \m{A}.
%
\Footnote{1}{In one-dimensional problems all vectors can be arranged to point
along a single axis.
Then in vector equations the same unit vector is common to all terms so it
can be factored out and dropped.
That leaves the vector components along that single axis, for which the axis
subscript can also be dropped.
These quantities, such as \m{v_\text{OB}}, \m{v_\text{OA}}, and \m{v_\text{BA}} in
\Eqnref{1} are not vector \textit{magnitudes}, which would always be
positive; instead they are vector \textit{components} which can have negative
as well as positive numerical values.

The double subscript notation is developed in more detail in \Quote{Relative
Linear Motion, Frames of Reference} (MISN-0-11).}
%
Velocity \m{v_\text{BA}} is restricted to being independent of time for the
purposes of this unit.

\pcap{1}{c}{Transformation Derivation}
In order to obtain the Galilean transformation, we must get \m{x}'s into the
equation.
This is easy since
%
\Eqn{}{v_\text{OB}(t_\text{B}) \equiv dx_\text{OB}(t_\text{B})/dt_B\,,}
%
and
%
\Eqn{2}{v_\text{OA}(t_\text{A}) \equiv dx_\text{OA}(t_\text{A})/dt_\text{A}\,,}
%
where \m{t_A} is the time as measured on \m{A}'s clock, while \m{t_\text{B}} is the time
as measured on \m{B}'s clock.
Then \m{v_\text{OB}(t_\text{B})} is the velocity of \m{O} as seen by \m{B} at the time \m{t_\text{B}}.

In words, \Eqnref{2} shows that the velocity of \m{O} as observed by
\m{A} is the time rate of change of the position of the object, as observed by
\m{A} with \m{A}'s measuring devices (clocks and rulers).
Substituting \Eqnref{2} into \Eqnref{1}, and integrating with
respect to \m{t_\text{A}}:
%
\Eqn{}{\int_{t_0}^{t} \dfrac{dx_\text{OB}(t_\text{B})}{dt_\text{B}} dt_\text{A} =
                  \int_{t_0}^{t} \dfrac{dx_\text{OA} (t_\text{A})}{dt_\text{A}}dt_\text{A}-
                                           \int_{t_0}^{t} v_\text{BA} dt_\text{A}\,,}
%
where the left-hand-side (LHS) is not integrable until we put in the Galilean
time transformation, \m{t_\text{B} = t_\text{A}}.
This is meant to be obvious, in the sense that \Quote{everyone knows} that \m{A}'s
clocks can all be synchronized with \m{B}'s.
For convenience, we replace \m{t_\text{B}} and \m{t_\text{A}} by the single symbol
\m{t}.
Integrating, we get
%
\Eqn{}{x_\text{OB}(t) - x_\text{OB} (t_0) = x_\text{OA}(t) - x_\text{OA}(t_0) -
                                              v_\text{BA}(t - t_0)\,.}
%
Again in the interest of simplicity we set \m{t_0 = 0} and set up \m{A}'s and \m{B}'s
\m{x}-axis origins at such places that they coincide at time zero.
Then the two observers' measured positions of the object will agree at that
time and so:
%
\Eqn{3}{x_\text{OB}(t) = x_\text{OA}(t) - v_\text{BA}t\,.}
%
This equation, along with \m{t_\text{B} = t_\text{A} \equiv t} is the Galilean
transformation.

\pcap{1}{d}{Invariance Under Label Interchange}
To show that \Eqnref{3} is invariant under interchange of observer
labels, simply replace \m{A} by \m{B} and \m{B} by \m{A} everywhere in the equation
and show that by a little manipulation you can get \Eqnref{3} back
again.
One intuitively feels that the laws of nature should be independent of which
label one gives to which observer.

\pcap{1}{e}{Predicted Light Speed Varies}
The velocity addition law associated with the Galilean transformation is, of
course, the same one we started with: \m{v_\text{OB} = v_\text{OA} - v_\text{BA}}.
We now apply this addition law to the case where the object \m{O} is the front
end of a beam of light (L).
Hence \m{v_\text{OB} = v_\text{LB}} is the velocity of light measured by observer
\m{B}, \m{v_\text{OA} = v_\text{LA}} is the velocity measured by observer \m{A} and the
two velocities are related by:
%
\Eqn{}{v_\text{LB} = v_\text{LA} - v_\text{BA}\,.}
%
Notice that, if \m{A} and \m{B} are moving relative to each other
(\m{v_\text{BA} \neq 0}), then each sees a different speed of light.
This is, of course, the \Quote{common sense} result.
}% /Sect
%
\Sect{2}{The Lorentz Transformation}{\SectType{TextMultiPara}}{
%
\pcap{2}{a}{Observed Constancy of Light Speed}
Toward the end of the nineteenth century, this aspect of the Galilean
transformation---the non-constancy of the speed of light---came into conflict
with Maxwell's formulation of the theory of electricity and magnetism and
with the experimental tests of that formulation.
The theoretical and experimental results pointed toward one conclusion - that
the velocity of light must be the same for all observers, regardless of their
motion relative to each other, that is,
\m{v_\text{LB} = v_\text{LA} = c}, a constant.

The Galilean transformation cannot give this result; the one that does is
the Lorentz transformation.

\pcap{2}{b}{The Transformation}
The Lorentz transformation is given by:
%
\Eqn{4}{x_\text{OB} = k \left( x_\text{OA} - v_\text{BA} t_\text{A} \right)\,,}
%
\Eqn{5}{t_\text{B} = k \left( t_\text{A} - v_\text{BA} x_\text{OA}/c^2 \right)\,,}
%
\Eqn{6}{k \equiv \left( 1 - v_\text{BA}^2/c^2 \right)^{-1/2}\,.}
%
The Lorentz transformation has the peculiar features that the length of an
object such as a meter stick is contracted and the rate of a clock is slowed
when observed from a moving reference frame.
%
\Footnote{2}{See \Quote{The Length Contraction and Time Dilation Effects of Special
Relativity} (MISN-0-13).
This unit culminates in an examination of the \Quote{twin paradox}: The Lorentz
transformation predicts that, if twin \m{A} is moving, A will age more slowly
than will stationary twin \m{B}.
However, twin \m{A} could equally well say that it is twin \m{B} who is moving,
albeit in the opposite direction (\m{v_\text{BA} = - v_{AB}}), and therefore
it is \m{B} who should age more slowly.}
%

Note, that, in the limit that \m{v_\text{BA} << c}, the Lorentz transformation
reduces to the Galilean transformation.
}% /Sect
%
\Sect{3}{The Velocity Addition Law}{\SectType{TextMultiPara}}{
%
\pcap{3}{a}{Derivation}
The relativistic velocity addition law can be easily derived from the
Lorentz transformation.
First,
%
\Eqn{7}{v_\text{OB} = \dfrac{dx_\text{OB}}{dt_B} = \dfrac{dx_\text{OB}}{dt_A} \cdot
                  \dfrac{dt_A}{dt_B} = \dfrac{dx_\text{OB}}{dt_A} \cdot
                             \left( \dfrac{dt_B}{dt_A} \right)^{-1}\,.}
%
Differentiating \Eqnref{4} with respect to t\m{_\text{OA}} gives:
%
\Eqn{}{\dfrac{dx_\text{OB}}{dt_A} = k \left( \dfrac{dx_\text{OA}}{dt_A} -
             v_\text{BA}\dfrac{dt_A}{dt_A} \right) = k(v_\text{OA} - v_\text{BA})\,.}
%
And differentiating \Eqnref{5} with respect to \m{t_A} gives:
%
\Eqn{}{\dfrac{dt_B}{dt_A} = k \left( \dfrac{dt_A}{dt_A} -
 v_\text{BA}\dfrac{dx_\text{OA}}{dt_A}/c^2 \right) = k (1 - v_\text{BA} v_\text{OA}/c^2)\,.}
%
So substituting these into \Eqnref{7} gives:
%
\Eqn{}{ v_\text{OB} = k (v_\text{OA} - v_\text{BA}) \left[k (1 - v{BA}
                                     v{OA}/c^2) \right]^{-1}\,,}
%
so:
%
\Eqn{8}{v_\text{OB} = \dfrac{v_\text{OA} - v_\text{BA}}{1 - v_\text{BA} v_\text{OA}/c^2}\,,}
%
which is the relativistic velocity addition law.

\pcap{3}{b}{Velocity Much Less Than the Speed of Light}
Note that here, too, in the limit that \m{v_\text{BA} << c}, our result reduces
to the \Quote{ordinary} velocity addition law.

\pcap{3}{c}{Predicting Constant Light Speed}
We can verify that the speed of light is the same for all observers.
If we put \m{v_\text{OA} = v_\text{LA} = c} into \Eqnref{8} then
%
\Eqn{}{v_\text{LB} = \dfrac{c - v_\text{BA}}{1 - v_\text{BA}c/c^2} =
                         \dfrac{c - v_\text{BA}}{1 - v_\text{BA}/c} = c\,.}
%
So, both \m{A} and \m{B} see light as having the same speed.

\pcap{3}{d}{Reduction to Galilean Law}
We have pointed out that, in the limit that \m{v_\text{BA}} is much less than the
speed of light, the relativistic velocity addition law reduces to the ordinary
velocity addition law.
This can be demonstrated by putting \Quote{everyday} speeds into the velocity
addition laws and comparing the results.

\pcap{3}{e}{A Numerical Example}
\begin{center}
\linefill{0.50in} \raisebox{-2pt}{\m{\stackrel{A}{.}}} \linefill{0.75in}
                  \raisebox{-2pt}{\m{\stackrel{B}{.}}} \linefill{0.75in}
                  \raisebox{-2pt}{\m{\stackrel{O}{.}}} \linefill{0.75in}
\end{center}
%
Consider, for instance, the following situation.
Observer \m{A} sees object \m{O} moving in the \m{-x}-direction, toward him/her,
at a speed of 1000\unit{m/s} (2237\unit{mi/hr}), and \m{A} also sees observer \m{B}
moving in the \m{+x}-direction, away from her/him, at 1000\unit{m/s}.
According to the ordinary velocity addition law, observer \m{B} will therefore
see object \m{O} as having velocity:
%
\Eqn{}{v_\text{OB} = v_\text{OA} - v_\text{BA} =
(-1000\unit{m/s}) - 1000\unit{m/s} = - 2000\unit{m/s}\,.}
%
That is, \m{O} is seen by \m{B} as approaching him/her at 2000\unit{m/s}.
Using the relativistic velocity addition law, \m{B} sees \m{O} as having
velocity
%
\jot=10pt
\FourEqns{}{v_\text{OB} & = \dfrac{v_\text{OA} - v_\text{BA}}{1 - v_\text{BA} v_\text{OA}/c^2}}
           {       & = \dfrac{(-1000\unit{m/s}) - 1000\unit{m/s}}{1-(1000\unit{m/s})(-1000\unit{m/s})/
                                      (3 \times 10^8\unit{m/s})^2}}
           {       & = \dfrac{-2000\unit{m/s}}{1 + \dfrac{10^6\unit{m\up{2}/s\up{2}}}{9 \times 10^{16}\unit{m\up{2}/s\up{2}}}}
                                    \simeq \dfrac{-2000\unit{m/s}}{1 + 1.1 \times 10^{-11}}}
           {       & = \dfrac{-2000\unit{m/s}}{1.000000000011} = - 1999.9999978\unit{m/s}\,.}
%
The difference between the two results is of the order of one part in a
billion.
So, for this case, the difference between the results of the ordinary velocity
addition law and the relativistic velocity addition law are too small to be
measurable in the everyday world.

\tryit A spaceship is approaching the earth at a speed of 0.9000\unit{c}.
A cyclotron mounted on the spaceship sends out a beam of protons with speed
0.9000\unit{c} relative to the spaceship.
Show that an observer on the earth sees the beam of protons approaching
her or him with a speed of 0.9945\unit{c}.
}% /Sect
%
\Sect{}{Acknowledgments}{\SectType{Acknowledgments}}{\NsfAcknowledgment}% /Sect
%
\Sect{}{Glossary}{\SectType{Glossary}}{
\GlossaryItem{Galilean transformation} the transformation between frames of
         reference that applies when the relative linear velocity of one
         reference frame with respect to the other is much less than the speed
         of light.

\GlossaryItem{Lorentz transformation} the transformation between frames of reference
         that applies when the relative linear velocity of one reference frame
         with respect to the other has any value whatsoever.

\GlossaryItem{relativistic velocity addition law} a statement of the way in which
velocities transform from one constant-speed frame of reference to another.
}% /Sect

