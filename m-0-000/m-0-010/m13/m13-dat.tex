\revhist{3/17/87, pss; 4/1/91, pss; 10/14/91, pss; 4/10/92, pss; 2/3/94, pss;
         9/29/94, pss;9/15/95, lae; 11/14/96, pss; 4/7/97, pss; 2/22/99, pss;
         10/19/99, pss; 10/3/02, pss}
%
\defModTitle{\ph{Length Contraction} \ph{and Time Dilation}}
\defCtAuthor{\inits{P.}Signell, \inits{J.}Borysowicz, and \inits{M.}Brandl, Michigan State University}
\defIdAuthor{P.\,Signell and J.\,Borysowicz and M.\,Brandl, Department of
Physics, Michigan State University}
%
\defIdItems{
    \IdVersEval{10/3/2002}{1}
    \IdHours{1}
    \begin{InputSkills}
    \item [1.]  Vocabulary: frame of reference \prrqone{0-11}, Lorentz transformation
    \prrqone{0-12}.
    \item [2.]  Transform single-event time and position measurements using
    the Lorentz transformation \prrqone{0-12}.
    \end{InputSkills}
    %
    \begin{KnowledgeSkills}
    \item [K1.] Vocabulary: laboratory frame, rest frame, length contraction,
    time dilation, twin paradox.
    \item [K2.] Given the Lorentz transformation, derive the relativistic length
    contraction and time dilation factors.
    \item [K3.] Show that the Lorentz transformation is independent of which
    frame of reference is considered to be moving and explain how
    this leads to the twin paradox.
    \end{KnowledgeSkills}
    %
    \begin{ProblemSolvingSkills}
    \item [S1.] Given length and time intervals measured in one frame of reference,
    find the corresponding length and time intervals measured
    in a different frame of reference.
    \end{ProblemSolvingSkills}
}