\revhist{5/31/85 mpm; 10/13/91, pss; 9/29/94, pss; 4/7/97, lae; 10/3/02, pss}
%
\Sect{1}{Overview}{\SectType{TextOnePara}}{
%
The measured length of a moving object is always found to be shortened and
the rate of a moving clock is found to be decreased.
These effects are derived from the Lorentz transformation and are applied
to physical systems.
The symmetry of the Lorentz transformation is used to show the reasoning
behind the simple form of the \Quote{twin paradox.}
}% /Sect
%
\Sect{2}{Length Contraction \& Time Dilation}{\SectType{TextMultiPara}}{
%
\CaptionedLeftFramedFigure{1}{Defining an object's \Quote{length in the
direction of travel.}}{m13gr01}

\pcap{2}{a}{The Length-Measurement Problem}
The measurement of the length of an object which is
traveling past you at high speed is not trivial.
% 
\Footnote{1}{For fifty years, physicists believed and taught that an object
traveling past at high speed would appear physically distorted
(\Quote{Lorentz contracted}).
Then Terrell showed how it would really look.
See \Quote{Appearances at Relativistic Speeds} (MISN-0-44).}
%
You should put a stationary meter stick ahead of the object, parallel to its
future path of travel.
Then, when the object is alongside the meter stick, you must note the
positions of the two ends simultaneously (just think of what would result
otherwise).
Subtracting the two end-position readings would give you the length
(see \Figref{1}).

\pcap{2}{b}{One Event in Two Frames}
If an observer in one reference frame, \m{A}, and an observer in another
reference frame, \m{B}, each measures the position and time of an event \m{O},
the position and time measurements in the two frames are related by the
Lorentz transformation:
%
\Eqn{1}{x_{OB}(t_{OB}) = k[x_{OA}(t_{OA}) - v_{BA}\,t_{OA}]}
%
\Eqn{2}{t_{OB} = k[t_{OA} - v_{BA}\,x_{OA}(t_{OA})/c^2]}
%
where
%
\Eqn{}{k = [1 - v_{BA}^2/c^2]^{-1/2}\,.}
%
In our current case, \m{x_{OA}(t_{OA})} might refer to the location of
one end of our
%
\Footnote{2}{The notation conventions are given in \Quote{Special Relativity: The
Lorentz Transformation and the Velocity Addition Law} (MISN-0-12).}
%
object as measured in the Lab frame at a particular Lab time.
We say that the object is moving with respect to a Laboratory frame of
reference \m{L}, and our task is to measure the moving object's length in its
direction of motion, first in the Lab frame and then in a frame of reference
which is moving along with the object.
Since the object is at rest in the latter frame, that frame is called the
object's Rest frame \m{R}.

\pcap{2}{c}{Derivation of the Length Contraction Factor}
We label the leading edge of the object \#1 and the trailing \#2, as in
the diagram.
Then the Lab frame end-measurements are related to the Rest frame
end-measurements by:
%
\Eqn{}{x_{1R}(t_{1R}) =
k_{RL}[x_{1L}(t_{1L}) - v_{RL}\,t_{1L}]}
%
\Eqn{}{x_{2R}(t_{2R}) =
k_{RL}[x_{2L}(t_{2L}) - v_{RL}\,t_{2L}]\,.}
%
As discussed in Sect.\,2a, the Lab frame end-measurements must be
simultaneous,
%
\Eqn{}{t_{1L} = t_{2L}\,,}
%
because the object is moving.
Subtracting the second equation from the first and using the simultaneity of
measurement in the Lab frame,
%
\Eqn{}{x_{1R}(t_{1R}) - x_{2R}(t_{2R}) =
k_{RL}[x_{1L}(t_L) - x_{2L}\,(t_L)]\,,}
%
where \m{t_{L}} is the single time of measurement in the Lab.

Now in the Rest frame the object is not moving so measurements of the
positions of the ends are independent of time, enabling us to write:
%
\Eqn{}{x_{1R} - x_{2R} = k_{RL}[x_{1L} - x_{2L}]\,.}
%
Each of the above differences is the length as measured in the indicated
frame and so we finally obtain:
%
\Eqn{}{\ell_{OR} = k_{RL}\ell_{OL}\,.}
%
This is usually written the other way around:
%
\Eqn{3}{\ell_{OL} = (1 - v_{RL}^2/c^2)^{1/2}\,\ell_{OR}\,.}
%
Thus a moving object will always be measured to be shorter in the direction
of travel than it will be when at rest and this length contraction is called
the \Quote{Lorentz contraction.}

\pcap{2}{d}{Derivation of the Time Dilation Factor}
Now suppose we measure a time interval between two events.
For concreteness we will follow a cosmic ray pion (\Quote{p\={i}\m{'}-on}),
denoted \m{\pi}, from its time of creation at time \m{t_1} in the upper
atmosphere to the end of its lifetime at time \m{t_2} near the earth's
surface.
Our two frames of reference are the Rest frame of the \m{\pi} and the Lab frame
which is stationary on the surface of the earth.
We wish to compare the lifetimes of the \m{\pi} as observed from the two
frames.
%
From \Eqnref{2},
%
\Eqn{4}{t_{1L} =
     k_{LR}[t_{1R} - v_{LR}\,x_{1R}(t_{1R})/c^2]}
%
\Eqn{5}{t_{2L} =
      k_{LR}[t_{2R} - v_{LR}\,x_{2R}(t_{2R})/c^2]\,.}
%
To an observer in the Rest frame of the \m{\pi}, and \m{\pi} is not moving so its
creation and annihilation are at the same space point:
%
\Eqn{6}{x_{1R}(t_{1R}) = x_{2R}(t_{2R})\,.}
%
Subtracting (4) from (5) and using (6) gives:
%
\Eqn{}{t_{2L} - t_{1L} = k_{LR}(t_{2R} - t_{1R})\,.}
%
Since the time difference \m{t_2 - t_1} is the lifetime \m{\tau} of the pion, we
finally get,
%
\Eqn{}{\tau_{L} =
         k \tau_{R} = (1 - v^2/c^2)^{-1/2}\,\tau_{R}\,.}
%
If there is a frame of reference \m{R} in which two events occur at the same
space point, that will be the frame in which the measured time interval
between the events will be the smallest.
Measured values by other observers will be larger by the factor \m{k}, and
this is called the \Quote{time dilation} effect.
Thus all slowly moving pions appear to have about the same lifetime while
those moving at speeds near that of light appear to live longer, exactly
in accordance with the factor \m{k}.

One can say that the fast moving pion appears to age more slowly!
}% /Sect
%
\Sect{3}{The \Quote{Twin Paradox}}{\SectType{TextMultiPara}}{
%
\pcap{3}{a}{The Paradox}
The \Quote{twin paradox} refers to the differential aging of a set of twins, one of
whom goes away on a space trip and eventually returns to earth.
The twin who stayed at home is seen to be much older than the one who took
the journey.
You may or may not feel that this is truly a \Quote{paradox} but that is what it
is called.

\pcap{3}{b}{\m{A}'s Space-time Point}
The Lorentz transformation is independent of which frame is considered
to be the moving one.

From twin \m{A}'s point of view, twin \m{B} is moving with velocity \m{v_{BA}}
so that:
%
\Eqn{7}{x_{OB}(t_{OB}) =
                 k[x_{OA}(t_{OA}) - v_{BA} t_{OA}]}
%
\Eqn{8}{t_{OB} = k[t_{OA} - v_{BA}x_{OA}(t_{OA})/c^2]\,.}
%
However, from \m{B}'s viewpoint, \m{A} is moving and has velocity \m{v_{AB} =
-v_{BA}}.
Hence we have:
%
\Eqn{9}{x_{OA}(t_{OA}) =
                  k[x_{OB}(t_{OB}) + v_{BA}\,t_{OB}]}
%
\Eqn{10}{t_{OA} =
            k[t_{OB} + v_{BA}\,x_{OB}(t_{OB})/c^2]\,.}
%
\Equationsref{9} and \Eqnssref{10} are exactly equivalent to \Eqnsref{7} and \Eqnssref{8}.
We show the equivalence by deriving \Eqnsref{7} and \Eqnssref{8} from
\Eqnsref{9} and \Eqnssref{10}.

\pcap{3}{c}{\m{B}'s Space Point}
%
Solving \Eqnref{10} for \m{t_{OB}} gives, 
%
\Eqn{11}{t_{OB} = k^{-1} t_{OA} - v_{BA} x_{OB}(t_{OB})/c^2.}
%
Substituting \Eqnref{11} into \Eqnref{9} gives,
%
\ThreeEqns{}{x_{OA}(t_{OA}) & = k[x_{OB}(t_{OB}) + v_{BA}(k^{-1}\,t_{OA} - v_{BA}x_{OB}(t_{OB})/c^2]}
            {               & = k [k^{-1} v_{BA}t_{OA} + x_{OB}(t_{OB})(1 - v_{BA}^2/c^2)]}
            {               & = k[k^{-1}\,v_{BA}t_{OA} + k^{-2} x_{OB}(t_{OB})]\,.}
%
\Eqn{12}{x_{OA}(t_{OA}) = v_{BA}t_{OA} + k^{-1} x_{OB}(t_{OB})}
%
Solving \Eqnref{12} for x\m{_{OB}} gives \Eqnref{7}:
%
\Eqn{}{x_{OB}(t_{OB}) = k[x_{OA}(t_{OA}) - v_{BA}t_{OA}]\,.}

\pcap{3}{d}{\m{B}'s Event Time}
%
We can similarly obtain \Eqnref{8} from \Eqnsref{9} and
\Eqnssref{10}.
We first solve \Eqnref{9} for \m{x_{OB}(t_{OB})}:
%
\Eqn{13}{x_{OB}(t_{OB}) = k^{-1} x_{OA}(t_{OA}) - v_{BA}t_{OB}.}
%
Then we use \Eqnref{13} to eliminate \m{x_{OB}(t_{OB})} from
\Eqnref{10}, and obtain:
%
\ThreeEqns{}{t_{OA} & = k\{t_{OB} + v_{BA}[k^{-1} x_{OA}(t_{OA}) - v_{BA}t_{OB}]/c^2\}}
            {       & = k[t_{OB}(1 - v_{BA}^2/c^2) + k^{-1} v_{BA}x_{OA}(t_{OA})/c^2]}
            {       & = k[k^{-2} t_{OB} + k^{-1} v_{BA}x_{OA}(t_{OA})/c^2]}
%
\Eqn{14}{t_{OA} = k^{-1} t_{OB} + v_{BA}x_{OA}(t_{OA})/c^2}
%
Solving \Eqnref{14} for \m{t_{OB}} gives \Eqnref{8}:
%
\Eqn{}{t_{OB} = k[t_{OA} - v_{BA}x_{OA}(t_{OA})/c^2].}
%

\pcap{3}{e}{Time Interval Symmetry}
Since the Lorentz transformation is symmetric for \m{A} \m{\leftrightarrow} \m{B}
(with \m{v_{BA} \leftrightarrow -v_{BA}}), all general results have that
same symmetry.
Thus if \m{A} sees \m{B}'s clock running slower, \m{B} must see \m{A}'s clock running
slower.
If \m{A} and \m{B} are otherwise the same (\Quote{twins}) each sees the other age less
rapidly.
Thus twin \m{A} should see \m{B} as the younger and twin \m{B} should see \m{A} as the
younger.
Hence the paradox: how can each see the other as younger?
Note, however, that in our derivation the twins could never come back
together to compare their ages in the same frame of reference; we had no
\Quote{turning around} or \Quote{stopping.}
The question of what would happen if one twin turned around, came back, and
stopped by the other one, is not answered by Special Relativity: it has
nothing to say about what happens during accelerations.
For that one must go to the much more complex General theory of Relativity.
%
\Footnote{3}{See \Quote{The Equivalence Principle: An Introduction to Relativistic
Gravitation} (MISN-0-110).}
%
}% /Sect
%
\Sect{}{Acknowledgments}{\SectType{Acknowledgments}}{\NsfAcknowledgment}% /Sect

