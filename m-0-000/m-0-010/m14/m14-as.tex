\revhist{7/16/85, mpm; 8/8/91, pss; 10/3/94, pss; 8/20/97, abs; 11/13/97, pss; 3/3/00, pss;
         4/13/02, pss}

\Sect{}{}{\SectType{SpecialAssistance}}{

\xpcap{}{}{PURPOSE}
This part of the module provides you with additional assistance in mastering
the concepts presented in the text and in applying them to the problems.
You need not study this material if you feel you have mastered the material
in the text and if you can do the problems.\smallskip

\xpcap{}{}{CONTENTS}\vspace*{-7pt}
\begin{itemize}
\item [1.] Short-answer drill on the second law of motion and
some additional comments on the application of the third law of motion.
\item [2.] Detailed solutions to the exercises.
\item [3.] Detailed solutions to selected practice problems.
\end{itemize}

\xpcap{}{}{PART~1: The Second Law of Motion (Text, Section 3)}
The following drill is intended to help sharpen your understanding of the
second law of motion.
To get the most benefit from it, first write your answers to a question and
then check against the answers that follow all questions.
Go back to the text and review it until you feel you understand any questions
you got wrong.
If you write your answers on scrap paper, you can use the drill several
times until you get it perfect.
\begin{two-digit-list}
\item [1.] Write the Second Law of Motion.
\item [2.] In the second law, the \vect{F} stands for \linefill{1in}.
This is a (vector/scalar) quantity.
\item [3.] The \m{m} stands for \linefill{1in}.
It is a (vector/scalar) quantity.
It may take on both positive and negative values.
(T/F)
\item [4.] The \vect{a} stands for \linefill{1in}.
It is a (vector/scalar).
\item [5.] The second law is a (vector/scalar) equation.
\item [6.] Vector equations involve both \linefill{1in} and \linefill{1in}.
\item [7.] The second law says nothing about the direction of the acceleration.
(T/F)
\item [8.] The second law says that the acceleration of an object is
proportional to any force you find is being applied to it. (T/F)
\item [9.] The magnitude of the acceleration of an object can never be greater
than the magnitude of the total force applied to the object.
(T/F or inappropriate?)
\item [10.] The mass times the acceleration is equal to the \linefill{1in}
force applied to the object.
\item [11.] In adding all the forces together, you must use \linefill{1in}
addition.
This means that you do not need to worry about the directions of the
individual forces. (T/F)
\item [12.] What is the relation between the \m{x}-component of the acceleration
and the \m{y}-component of the force on the object?
\end{two-digit-list}

\xpcap{}{}{Answers}
\begin{two-digit-list}
\item [1.] \m{\sum \vect{F} = m \vect{a}}.

\item [2.] \vect{F} is a force applied to the object.
Force is a vector.

\item [3.] \m{m} is the mass of the object whose motion is being considered.
Mass is a scalar quantity and is always positive.

\item [4.] \vect{a} is the acceleration of the object.
Acceleration is a vector.

\item [5.] It is a vector equation.

\item [6.] Vectors involve both magnitude and direction.

\item [7.] False; vectors involve direction.

\item [8.] False; the acceleration is proportional to the total vector sum of
all the forces acting on the object.

\item [9.] Inappropriate.
The magnitudes of these quantities include units, so the quantities are not
comparable.
Their units are different.

\item [10.] Mass times acceleration equals total or net force applied to the
object.

\item [11.] Use vector addition.
False; vector addition involves direction.

\item [12.] None, as far as the laws of motion are concerned.
\end{two-digit-list}

\xpcap{}{}{PART~2: Solutions to Exercises}

\AsItem{1}{TX-3a}
{For the case of the single force applied to an object, the acceleration can
 be found by using \Eqnref{1}.
 Thus we have: \m{\vect{a} = \vect{F}/m}.

 The acceleration will be in the direction of the force, which in this case is
 northward.
 The magnitude of the acceleration can be found from the equation above:
 \m{|\vect{a}| = 100\unit{N}/250\unit{kg} = 0.40\unit{m/s\up{2}}}.
}

\AsItem{2}{TX-3e}
{When several forces are applied to an object, you must add the forces using
 vector addition and then divide by the mass to find the acceleration.
 The sketch below shows the situation and a suitable coordinate system.
 
 \CenteredUnframedFixedFigure{m14gr11}\newline
 Since the forces are at right angles, by picking the coordinate axes to be
 north and east you eliminate the problem of finding the components of the
 forces.
 The second law is: \m{\vect{R} = \Sigma\,\vect{F} = m\,\vect{a}}.
 The northward component of \vect{R} is then \m{R_n = (25 - 40)\unit{N} = - 15\unit{N}},
 and the northward component of \vect{a} is
 \m{a_n = R_n/m = - 15/5\unit{m/s\up{2}} = - 3\unit{m/s\up{2}}}.

 \CenteredUnframedFixedFigure{m14gr12}{The eastward component of \m{R} is \m{R_e = 15\unit{N}},
 and the eastward component of \vect{a} is \m{a_e = R_e/m = 15/5 = 3\unit{m/s\up{2}}}.
 The magnitude of \vect{a} is then:
 \m{|\vect{a}| = \left(a_n^2 + a_e^2\right)^{1/2} = 4.243\unit{m/s\up{2}}},
 and the direction of \vect{a} is given by the angle \m{\theta} defined in the
 sketch with: \m{\tan\theta = a_n/a_e = -1}; \m{\theta = 45\degrees} south of east.}
}

\newpage

\xpcap{}{}{PART~3: Solutions to Selected Problems}

To get maximum benefit from the practice problems and from this section, you
should have made a determined effort to solve the problems yourself before
you turn to this section for \Quote{special assistance.}

\AsItem{3}{PS-problem~2}
{\CenteredUnframedFixedFigure{m14gr13}{This problem is very similar to the elevator problem.
 Begin by drawing a sketch showing all the forces and other relevant data.
 Pick the \m{y} axis as being downward.
 Since all the forces and accelerations are in the \m{y} direction, you can treat
 the second law as a scalar equation.
 Watching the signs, you have:
 \begin{eqnarray*}
    \Sigma\,F & = & m\,a \\
    F_w - F_p & = & m\,a
 \end{eqnarray*}
 Since the weight of the man is given by \m{F_w = mg}, this becomes:
 \m{F_p = F_w - m a = m g - m a = m (g - a) =
 (80\unit{kg})(9.8 - 1.5)\unit{m/s\up{2}} = F_p = 664\unit{N}}.
 }
}

\AsItem{4}{PS-problem~3}
{The only problem here is to remember that you have been given the weight of
 the object and you need the mass when you use Newton's second law.
 The mass and weight are related by \m{w = m g}, and using the constant given in
 the table in Sect.\,3b, this gives: \m{m = 93.17\unit{slug}} (NOT 93.75) for the car.
}

\AsItem{5}{PS-problem~4}
{To find the force, you need to know the acceleration.
 You find the acceleration from the kinematic equations for \Quote{motion with
 constant acceleration starting (or ending) at rest}:%
 \Footnote{AS1}{See \Quote{Kinematics in One Dimension} (MISN-0-7).}
 \begin{eqnarray*}
 v_0 & = & \sqrt{2 a s} \\
  a  & = & \dfrac{v_0^2}{2s} = \dfrac{400^2}{2(16)}\unit{m/s\up{2}}
   = 5000\unit{m/s\up{2}}
 \end{eqnarray*}

 The second law then gives
 %
 \Eqn{}{F = m a = (8000)(5000)\unit{kg}\unit{m/s\up{2}} = 4.0\times10^7\unit{N}.}
 %
}

\AsItem{6}{PS-problem~5}
{Again, to find the force, you must first find the acceleration.
For constant acceleration you know: \m{v = v_0 + a\,t}.
For \m{v = 0} this gives:
%
\Eqn{}{a = - \dfrac{v_0}{t} = -\dfrac{8.8}{1.5}\unit{m/s\up{2}} = - 5.9\unit{m/s\up{2}}\,.}
%
The magnitude of the force is then given by
%
\Eqn{}{F = |m a| = 3.2\times10^2\unit{N}\,.}
%
While stopping, the bicycler goes a distance \m{s} given by:
%
\Eqn{}{s = v_0 t + \dfrac{1}{2} a t^2 = (8.8)(1.5) - (0.5)(5.8667)(1.5)^2
  = 6.6\unit{m}.}
%
The poor pedestrian has had it.
}

\AsItem{7}{PS-problem~8}
{This is a simple exercise in adding vectors through the use of unit vectors.
 If you have difficulty with this, review the unit on vector addition.%
 %
 \Footnote{AS2}{\Quote{Sums, Differences and Products of Vectors} (MISN-0-2).}
 With no further comment, here are several intermediate steps in a solution
 to  this problem:
 %
 \FiveEqns{}%
 {m\,\vect{a}   & = \Sigma\,\vect{F}}
 {             & = (3 \uvec{x} + 4 \uvec{y} - 5 \uvec{z} + 6 \uvec{x} + 6 \uvec{y} +
                    6 \uvec{z} + 21 \uvec{x} - 16 \uvec{y} + 17 \uvec{z})\unit{N}}
 {m \vect{a}    & = (30 \uvec{x} - 6 \uvec{y} + 18 \uvec{z})\unit{N}.}
 {\vect{a}      & = (5 \uvec{x} - 1 \uvec{y} + 3 \uvec{z})\unit{m/s\up{2}}}
 {|\vect{a}|    & = 5.9\unit{m/s\up{2}}.}
 %
 \furtherhelp{13}
}

\AsItem{8}{PS-problem~11}
{To solve this you need to use \m{\vect{F} = m \vect{a}} and
 \m{\vect{v}_f - \vect{v}_i = \vect{a} \Delta t}, (valid for constant \vect{a}).

 Combining these we get:
 %
 \Eqn{}{\vect{F} = \dfrac{m}{\Delta t}(v_f - v_i)}
 %
 The unit vectors simply aid in the vector addition.
 You can work out the rest of the details.
}

\AsItem{9}{PS-problem~13}
{The only trick here is to recognize that the spring scale \Quote{reads} the force
 being applied to the bag.
 In each case draw a sketch, pick a coordinate system in which to determine
 the signs of the various terms, and proceed as in the example in the text.
}

\AsItem{10}{TX-6e}
{\noindent\CenteredUnframedFixedFigure{m14gr14}

 The sketch shows all the forces on the horse and on the wagon.

 It certainly is true that the third law says the pull of the horse on the
 wagon is equal and opposite to the pull of the wagon on the horse.
 Further, the second law says the vector sum of all the forces on the wagon
 equals the mass of the wagon times its acceleration.

 However, the real question is: can the horse exert an unbalanced force on
 the wagon?
 The pull of the wagon on the horse is irrelevant because it is a force on the
 horse, not a force on the wagon.

 For further analysis, see [S-11].
}

\AsItem{11}{[S-10]}
{ The only forces acting on the wagon are:
 \begin{itemize}
 \item [1.] The pull of the horse.
 \item [2.] The friction in the wagon axles and between wagon wheels and the
 ground.
 \item [3.] The weight of the wagon (the gravitational force on it).
 \item [4.] The upward force of the ground on the wheels.
 \end{itemize}
 The horse need only pull harder than the frictional force in order to
 accelerate the wagon horizontally, to give it a non-zero horizontal velocity.
}

\AsItem{12}{PS-problem~1}
{Here are the reminders consultants say they have given students who
 were unable to work this problem correctly or who were troubled by the problem's
 wording:
 
 \begin{one-digit-list}
 \item [\m{\bullet}] kg is a metric unit of mass, not weight.

 \item [\m{\bullet}] weight = mass \m{\times} \m{g}, whether in the English system or the metric.

 \item [\m{\bullet}] An object can be described in terms of its mass, weight, color, etc.
 Thus it is just fine to speak of a \Quote{six pound baby} or a \Quote{three kilogram
 cannister} or a \Quote{red car} or, as in this problem, a \Quote{2.0\unit{kg} can of soup.}
 \end{one-digit-list}
}

\AsItem{13}{PS-problem~9}
{For how to obtain the magnitude of a vector, see module 2.
}

\AsItem{14}{PS-problem~7}
{\m{F_\text{ave} = m\,a_\text{ave}} (always)

 \m{a_\text{ave} = \dfrac{\Delta v}{\Delta t}} (always)

 Make sure you successfully work problem~4 before attempting this problem.

 Also, see [S-12].
}

}% /Sect
