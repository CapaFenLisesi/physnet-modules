\revhist{7/16/85; mpm; 8/7/91, pss; 10/3/94, pss; 3/3/95, pss; 6/21/99, pss;
         12/15/99, pss}

\Sect{}{}{\SectType{ProblemSet}}{

These problems are grouped by skill so that you can easily find the ones
corresponding to the skills for which you are responsible.
The answers are at the end of the list.
The notation \Quote{\help{n}} refers to panel \Quote{\textit{n}} in Part~3
of this module's \textit{Special Assistance Supplement}.
Problems~15 and 16 are also on this module's \textit{Model Exam}.\smallskip

\icap{Skill 1a (Problem-Solving):}
\begin{two-digit-list}
\item [1.] Forces of 65\unit{N} north and \m{5.0\times10^1\unit{N}} east are applied to a
2.0\unit{kg} can of soup.
Determine the acceleration of the can. \help{12}

\item [2.]  An \m{8.0\times10^1\unit{kg}} parachutist is \Quote{falling} with a downward
acceleration of 1.5\unit{m/s\up{2}}.
What is the magnitude of the upward force on the parachutist due to the
parachute?
\help{3}

\item [3.]  A modern small car having a weight of \m{3.0\times10^3\unit{pounds}} is
accelerating at 5.0\unit{ft/s\up{2}}.
What is the magnitude of the total force acting on it? \help{4}
\end{two-digit-list}

\icap{Skill 1b (Problem-Solving):}
\begin{two-digit-list}
\item [4.]  An \m{8.0\times10^3\unit{kg}} meteorite slashes through the sky, slams into
the earth at \m{4.0\times10^2\unit{m/s}} and comes to a stop 16\unit{m} below ground level.
Assuming the meteorite was decelerating uniformly while burrowing through the
ground, what was the magnitude of the net force on it during those times?
\help{5}

\item [5.]  An irate college student on a bicycle is bearing down on a
panic-stricken pedestrian.
At a distance of 6.0\unit{m} from the pedestrian, she slams on her brakes and
slides uniformly to a stop in 1.5\unit{seconds}.
What was magnitude of the average stopping force?
Is the pedestrian bicycle-struck?
The total mass of the bicycle and operator is 55\unit{kg}; her initial velocity
was 8.8\unit{m/s}.
\help{6}

\item [6.] A 1,950\unit{pound} Volkswagon accelerates from 0 to
\m{6.0\times10^1\unit{miles per hour}} (88\unit{ft/s}) in 45\unit{sec}.
What is the magnitude of the average force on it? \help{14}

\item [7.] A \Quote{souped-up} Volkswagon, weighing 2020\unit{lb}., reaches a speed of
45\unit{miles per hour} (66\unit{ft/s}) in a distance of 1320\unit{feet} after starting from
rest.
What is the magnitude of the average force on the car? \help{14}

\item [8.]  Three forces;

%
\eqnline{\vect{F}_1 = (3.0 \uvec{x} + 4.0 \uvec{y} - 5.0 \uvec{z})\unit{N}}
\eqnline{\vect{F}_2 = (6.0 \uvec{x} + 6.0 \uvec{y} + 6.0 \uvec{z})\unit{N}}
%
\eqnline{\vect{F}_3 = (21.0 \uvec{x} - 16.0 \uvec{y} + 17.0 \uvec{z})\unit{N}}
are applied to a 6.0\unit{kg} object.
Find its acceleration. \help{7}

\item [9.] Rod Carew hits a baseball from a practice tee.
The bat exerts a force of 55\unit{N} on the ball and the ball accelerates with
%
\Eqn{}{\vect{a} = (350 \uvec{x} + 150 \uvec{y})\unit{m/s\up{2}}.}
%
Find the mass of the ball. \help{13}

\item [10.] A large athletically inclined student applies two forces,

%
\eqnline{\vect{F}_1 = (3.0 \uvec{x} + 15.0 \uvec{y})\unit{N}}
\eqnline{\vect{F}_2 = (16.0 \uvec{x} - 25.0 \uvec{y})\unit{N}}
to a 44.0\unit{kg} (98\unit{pound}) lab instructor.
What is the magnitude of the acceleration of the instructor?
\end{two-digit-list}

\icap{Skill 1c (Problem-Solving):}

\begin{two-digit-list}
\item [11.]  An electron, mass \m{9.1 \times 10^{-31}\unit{kg}}, has an initial
velocity of:

%
\eqnline{\vect{v}_1 = (3.0\times10^6 \uvec{x} + 2.0\times10^6 \uvec{y} +
0 \uvec{z})\unit{m/s}.}
After it has undergone 5.0\unit{seconds} of constant acceleration, its velocity is:

%
\eqnline{\vect{v}_f = (0 \uvec{x} + 5.0\times10^6 \uvec{y} +
1.0\times10^7 \uvec{z})\unit{m/s}.}
Find the magnitude of the force applied to it.  \help{8}

\item [12.] A 5.0\unit{kg} mass has an initial velocity of
\m{(3.0 \uvec{x} + 6.0 \uvec{y})\unit{m/s}}.
Sixteen seconds later it has a velocity of
\m{(9.0 \uvec{x} - 9.0 \uvec{y})\unit{m/s}}.
Assuming the acceleration was constant, what were the \m{x} and \m{y} components
of the force?
\end{two-digit-list}

\icap{Skill 2 (Problem-Solving):}

\begin{two-digit-list}
\item [13.]  A bag of corn is placed on a spring scale in an elevator.
When the elevator is at rest, the scale reads \m{8.0\times10^1\unit{lb}}.
What does the scale read when:
\begin{one-digit-list}
\item [a.] the elevator accelerates upward at 8.0\unit{ft/s\up{2}}?
\item [b.] the elevator moves upward with a constant velocity of 10.0\unit{ft/s}?
\item [c.] the elevator accelerates downward at 5.0\unit{ft/s\up{2}}? \help{9}
\end{one-digit-list}

\item [14.] A 5.0\unit{kg} light fixture is hung from the ceiling of an elevator by
a cord having a breaking strength of 65\unit{N}.
What is the maximum upward acceleration the elevator can have without the
cord breaking?
\end{two-digit-list}

\icap{At Large:}

\begin{two-digit-list}
\item [15.] A 5.0\unit{kg} ham is hanging from a spring scale attached to the ceiling
of an elevator that is accelerating upward at 3.0\unit{m/s\up{2}}.
What does the scale read as the apparent weight of the ham?

\item [16.] The components of the \emph{net} force on a 25\unit{kg} dog are shown.
Sketch the direction of the acceleration of the dog and calculate the
magnitude of the acceleration.

\CenteredUnframedFixedFigure{m14gr09}

\end{two-digit-list}

\newpage

\BriefAns

\begin{two-digit-list}
\item [1.] 32.5\unit{m/s\up{2}} north, 25\unit{m/s\up{2}} east, or
           \m{|\vect{a}| = 41\unit{m/s\up{2}}} at \m{\theta = 52\degrees} north of east.
           \help{12}
\item [2.] \m{6.6\times10^2\unit{N}}.
\item [3.] \m{4.7\times10^2\unit{lb}}.
\item [4.] \m{4.0\times10^7\unit{N}}.
\item [5.] \m{3.2\times10^2\unit{N}}; The pedestrian will be struck.
\item [6.] \m{1.2\times10^2\unit{lb}}.
\item [7.] \m{1.0\times10^2\unit{lb}}. \help{14}
\item [8.] \m{\vect{a} = (5.0 \uvec{x} - 1.0 \uvec{y} + 3.0 \uvec{z})\unit{m/s\up{2}}};
           \m{|\vect{a}| = 5.9\unit{m/s\up{2}}}.
\item [9.] 144\unit{grams}.  \help{13}
\item [10.] \m{|\vect{a}| = 0.49\unit{m/s\up{2}}}.
\item [11.] \m{|\vect{F}| = 2.0\times10^{-24}\unit{N}}.
\item [12.] \m{x: 1.9\unit{N}}, \m{y: - 4.7\unit{N}}.
\item [13.] (a) 100\unit{lb}. \help{9}; (b) 80\unit{lb}. \help{9}; (c) 68\unit{lb}. \help{9}
\item [14.] 3.2\unit{m/s\up{2}}.
\item [15.] 64\unit{N}.
\item [16.] \m{|\vect{a}|} = 2.0\unit{m/s\up{2}};
            Sketch shows \m{a_x\uvec{x}}, \m{a_y\uvec{y}}, and \vect{a}.

\CenteredUnframedFixedFigure{m14gr10}
\end{two-digit-list}
}% /Sect
