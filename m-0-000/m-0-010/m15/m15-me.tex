\revhist{8/30/85, mpm; 7/28/86, pss; 12/10/89, pss; 4/24/90, pss; 8/10/91, pss; 5/9/94, pss}

\Sect{}{}{\SectType{ModelExam}}{

\begin{one-digit-list}
\item [1.] See Output Skills K1-K2 in this module's \textit{ID Sheet}.
           The actual exam may include one or both of these skills, or none.

\item [2.] At an instant when a 0.20\unit{kg} particle has position, velocity, and
acceleration given by
\m{(2.1  \uvec{x} - 4.8 \uvec{y})\unit{m}},
\m{(1.2  \uvec{x} + 1.6 \uvec{z})\unit{m/s}}, and
\m{(-2.0 \uvec{y} + 1.5 \uvec{z})\unit{m/s\up{2}}}:
\begin{one-digit-list}
\item [a.] Calculate its momentum.
\item [b.] Calculate the rate at which its momentum is changing.
\end{one-digit-list}

\item [3.] \ItemFigure{A diving competitor (\m{\text{weight } = 160\unit{lb}}) has a
            downward velocity of \m{11\unit{ft/s}} just before hitting the board.
            When contact with the board ceases, \m{0.40\unit{s}} later, the diver's
            velocity is \m{23\unit{ft/s}} at an angle of {34\degrees} with the vertical.

            Calculate the magnitude of the average resultant force on the diver
            while in contact with the board.}{m15gr21}

\item [4.] \ItemFigure{A mass \m{M} moving with a speed \m{v} collides with a
             mass \m{2 M} initially at rest.
            After the collision the two move as shown.
            Determine \m{\theta}.}{m15gr22}
\end{one-digit-list}

\BriefAns

\begin{itemize}
\item [1.] See this module's \textit{text}.

\item [2-4.] See this module's \textit{Problem Supplement}, problems~13-15.
\end{itemize}
}% /Sect
