\tcsc {1. Introduction}{}
\tcpc {a. Describing Conservation Principles}{1}
\tcpc {b. Conservation of Mass}{1}
\tcpc {c. Whether Mass Is Really Conserved}{2}
\tcpc {d. Conservation of Momentum}{2}
\tcpc {e. Newton's Laws and Conservation of Momentum}{2}
\tcsc {2. Momentum and Force}{}
\tcpc {a. Definition of Momentum}{3}
\tcpc {b. An Isolated Particle has a Constant Momentum}{3}
\tcpc {c. Force Equals Time-Rate-of-Change of Momentum}{3}
\tcpc {d. Which is More Fundamental, Force or Momentum}{4}
\tcsc {3. Collision Forces}{}
\tcpc {a. Examining the Force Exerted on a Ball by a Bat}{4}
\tcpc {b. If the Force Were Constant, Here's the Answer}{5}
\tcpc {c. But the Collision Force Isn't Constant}{7}
\tcpc {d. Why Collision Forces Are Usually So Large}{8}
\tcpc {e. The Impulse-Momentum Principle}{8}
\tcpc {f. Solution Using Components}{8}
\tcsc {4. Conservation of Momentum}{}
\tcpc {a. Dealing With Larger Systems}{10}
\tcpc {b. Derivation for Isolated Two-Particle Systems}{10}
\tcpc {c. The General Conservation of Momentum Principle}{11}
\tcpc {d. When the System Isn't Isolated}{11}
\tcsc {5. Collision Momenta}{}
\tcpc {a. Using Momentum Conservation to Analyze Collisions}{12}
\tcpc {b. Conservation of Momentum Components in Collisions}{12}
\tcpc {c. Conserved Quantities in Collisions}{13}
\tcpc {d. Example: Momentum Components in a Collision}{14}
\tcsc {Acknowledgments}{15}
\tcsc {A. Averaging a Vector Function}{15}
\tcsc {}{}
\tcsc {}{}
\tcsc {}{}
\defmodlength {40}
