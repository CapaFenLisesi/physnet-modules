\revhist{12/10/89, pss; 8/9/91, pss; 3/11/93, pss; 3/29/93, pss; 5/9/94, pss; 10/26/95, pss;
         8/24/96, pss; 11/7/97, lae; 3/22/01, pss; 3/29/01, kag; 4/23/02, pss; 5/2/02, pss;
         5/13/02, pss}
%
\Sect{1}{Introduction}{\SectType{TextMultiPara}}{
%
\pcap{1}{a}{Describing Conservation Principles}
Imagine a self-sufficient group of people living and working in an
environment somewhat isolated from the rest of society; say on a secluded
South Pacific island (see this module's cover).
The people have familiar homes, jobs, businesses, etc.
Some time ago they adopted American currency as the medium of exchange.
Lacking a formal government, they have no mints and have agreed to keep the
total number of dollars fixed, except when they acquire a surplus of goods
which are then sold to an American merchant ship that makes very infrequent
calls.
The captain of the ship always pays for the goods with American currency,
thereby increasing the total number of dollars available to the people.

Between visits of the ship the total number of dollars in the society is
fixed.
Any money added to one person's dollar possession is always balanced by a
decrease in another's.
The important point here is that if at any instant the dollar possession of
all inhabitants are totaled, the sum remains constant (until the visiting
ship adds new money).

Physicists would describe the constraint on the total money available as
a Conservation of Dollar Principle.
The number of dollars any person possesses may change fifty times per day,
but at any instant the sum of dollars summed over the societal system is
fixed, or \Quote{conserved.}
Thus a conservation principle
\Index{conservation principle}\Index{principle, conservation}
is nothing more than a statement that for a
given system (e.g., the isolated community of people) there exists some
quantity (total number of dollars) which remains constant in time provided
certain conditions are met (no dollars added by visiting ships).

\pcap{1}{b}{Conservation of Mass}
Perhaps the simplest conservation principle in classical physics is that
of mass conservation.\Index{conservation of mass}\Index{mass| conservation of}
Essentially, the principle is this.
If a system, \m{A}, is defined and then isolated from all other systems by not
permitting a flow of mass into or out of system \m{A}, the mass conservation
principle asserts that the sum of the masses of the constituent objects in
system \m{A} remains constant even though the number of objects may be changing
(two lead blocks may be melted and solidified into one, a block of cheese
may be cut into ten slices, etc.), and the mass of individual entities may
change with time (water evaporating from a glass, frictional wear of rubber
from a tire, etc.).
Classically, if the total mass of the universe at any one instant is \m{M},
this would apparently be the total mass at all times even though subsystem
masses may be changing at very rapid rates because of interaction between
the subsystems.

\pcap{1}{c}{Whether Mass Is Really Conserved}
The mass conservation principle\Index{mass| conservation of} just asserted seemed to scientists of the
nineteenth century to be an accurate statement of natural law.
In the first half of this century, careful investigation of phenomena like
relativistic physics, fission, fusion, antimatter, etc., led to the
conclusion that, in fact, the total mass of a uranium nucleus is greater than
the combined mass of the particles which result from its fission.
As a result of these new discoveries and the accompanying interpretations,
physicists concluded that mass, rather than being conserved, is just one form
of energy, a quantity which is conserved.
%
\Footnote{1}{See \Quote{Potential Energy, Conservative Forces, The Law of Conservation
of Energy} (MISN-0-21), and \Quote{Work, Kinetic Energy, Work-Energy Principle,
Power,} (MISN-0-20).
\Quote{Collisions} (MISN-0-32) illustrates the usefulness of the two conservation
principles (energy and momentum) }
%
What now?
Are you to assume (in working a problem in this course) that mass is
conserved or not?
The answer is \Quote{Yes,} but with the proviso that there are phenomena in which
mass is not conserved.
Your first experience with such phenomena will probably come in studying
either relativistic or atomic physics or chemistry.
%
\Footnote{2}{See \Quote{Relativistic Energy; Thresholds for Reactions} (MISN-0-23).}
%

\pcap{1}{d}{Conservation of Momentum}
The topic of fundamental importance in this unit is that of momentum
conservation.\Index{conservation of momentum}\Index{momentum| conservation of}
You will see that for a system of particles which are isolated from
interaction with other particles, a vector quantity called the system's
\Quote{total (linear) momentum}\Index{linear momentum} is constant in time.
This will be true even though the system's individual constituent particles
may have rapidly changing momenta.
This situation is not too different from the fixed number of dollars
available to the isolated island dwellers whose individual dollar
possessions may change rapidly as a result of interaction with other island
dwellers.

\pcap{1}{e}{Newton's Laws and Conservation of Momentum}
In Sect.\,4 we will use Newton's laws
\Index{Newton's laws, and conservation of momentum} to derive the momentum conservation
principle.
This shows that the momentum conservation principle is valid throughout
Newtonian dynamics.
While Newtonian dynamics is valid over a very wide range of natural
phenomena,
%
\Footnote{3}{See \Quote{Particle Dynamics, Mass, Force, and the Laws of Motion}
(MISN-0-14), Section 5b.}
%
radical corrections can be necessary for high speeds and/or high masses, and
for phenomena on the atomic scale.
However, the momentum conservation principle is valid for all of those
phenomena as well as for Newtonian dynamics.
There are no known data which contradict it: conservation of momentum is one
of the few universal physical principles known.
}% /Sect
%
\Sect{2}{Momentum and Force}{\SectType{TextMultiPara}}{
%
\pcap{2}{a}{Definition of Momentum}
The linear momentum\Index{linear momentum} \vect{p} associated with a particle of mass m moving with
a velocity \vect{v} is defined by:\Index{momentum| definition of}
%
\Eqn{}{\vect{p} \equiv m\,\vect{v}.}
%
As a particle moves, its velocity changes as a result of its interaction
with other particles, so we can write:
%
\Eqn{1}{\vect{p}(t) \equiv m\,\vect{v}(t)}
%

\pcap{2}{b}{An Isolated Particle has a Constant Momentum}
First, let us consider the special case of an isolated particle, one having
no measurable interaction with other particles.
This case is easy since, by Newton's first law, such a particle moves with a
constant velocity.
%
\Footnote{4}{See \Quote{Particle Dynamics: The Laws of Motion} (MISN-0-14).}
%
Since the mass is constant, it thereby moves with a constant momentum:\Index{conservation of momentum}
%
\Eqn{2}{\vect{p} = \text{constant}.\qquad \text{(Isolated Particle)}}
%

This is the purest and simplest statement of the Conservation of Momentum
Principle.\Index{momentum| conservation of}
Unfortunately, in this form it is also the most useless statement.
Do not despair, for in Sect.\,4 you will see how to enrich this principle to
make it useful.

\pcap{2}{c}{Force Equals Time-Rate-of-Change of Momentum}
Before extending the conservation discussion, let's  investigate how a
particle's momentum changes.
To do so, differentiate each side of \Eqnref{1} with respect to time:
\begin{eqnarray*}
\dfrac{d\vect{p}}{dt} & = & \dfrac{d}{dt}(m\vect{v}) \\
                     & = & m \dfrac{d\vect{v}}{dt} \qquad (m\text{ is constant}) \\
                     & = & m \vect{a}. \qquad (\vect{a} \equiv \dfrac{d\vect{v}}{dt}:\text{ Definition})
\end{eqnarray*}
Now, according to Newton's second law
\Index{Newton's second law of motion} \m{m \vect{a}} equals the resultant force
\m{\vect{F}_R} acting on the particle so:
%
\Eqn{3}{\dfrac{d\vect{p}}{dt} = \vect{F}_R.}
%
Equation (3) is another way to express Newton's second law
(\m{\vect{F}_R} is the vector sum of all forces acting on the particle whose
momentum is \vect{p}).

\pcap{2}{d}{Which is More Fundamental, Force or Momentum}
Equation (3) can be given at least two interesting interpretations.
If you prefer to think of force as being that which causes acceleration,
then you can interpret \m{\vect{F}_R = d\vect{p}/dt} as stating: a
resultant force being exerted on the object we are studying implies that the
object's momentum must be changing.
If you prefer to think of \Quote{the natural state} of an isolated particle being
one of constant momentum, then you can think of
\m{\vect{F}_R = d\vect{p}/dt} as stating:
a changing momentum\Index{change, in momentum} implies that other objects must be exerting a resultant
force on the object we are studying.\Index{momentum| change in}

Another interesting facet of momentum is that it plays a crucial role in
atomic physics, where the concept of force does not exist.

Enough of this patter about what fundamental meaning (if any) is provided
by this new way of writing Newton's second law.
The crucial information for you at this point is that the resultant force
on a particle equals the time-rate-of-change of that particle's momentum
[\Eqnref{3}].
}% /Sect
%
\Sect{3}{Collision Forces}{\SectType{TextMultiPara}}{

\pcap{3}{a}{Examining the Force Exerted on a Ball by a Bat}
Situation: Nolan Ryan (California Angel pitcher) fires his fastest pitch
(approximately 100\unit{mi/hr} or 30\unit{m/s}) to Henry Aaron (Milwaukee Brewer
designated hitter) who swings and sends a hit (at the same speed) past
Nolan's head.

Question: If the magnitude of the ball's weight is \m{W}, the average force of
the bat on the ball during contact is most nearly equal to:
\begin{center}\begin{tabular}{l l l}
(a) \m{W}      &       (c)  \m{100\,W}   &     (e)  \m{10,000\,W} \\
(b) \m{10\,W}  &       (d) \m{1000\,W}   &     (f) \m{100,000\,W} \\
\end{tabular}\end{center}

\tryit What's your best guess?

\pcap{3}{b}{If the Force Were Constant, Here's the Answer}
In order to obtain an approximate answer for the force on a ball during
collision with a bat, we assume that the force on the ball is constant
during the collision.\Index{force| collision}
This seems to be a mediocre approximation at best, since intuitively you
probably feel that the force at the times of initial and final contact is
much less than that when one of the ball's sides has \Quote{pancaked} against the
bat and is momentarily at rest with respect to it.
Nevertheless, we'll begin with the \Quote{constant force during the collision}
assumption in order to make things simple.
We start with:
%
\Eqn{}{\dfrac{d\vect{p}}{dt} = \vect{F}.}
%

We multiply each side by \m{\Delta t}, the collision time, to get:
%
\Eqn{}{\dfrac{d\vect{p}}{dt}\Delta t = \vect{F} \Delta t.}
%

\CaptionedLeftFramedFigure{1}{The momentum change during the collision.}{m15gr01}

Since \vect{F}, and so also d\m{\vect{p}/dt}, is constant in time, the quantity
\m{(d\vect{p}/dt) \Delta t} is
the change in the ball's momentum, \m{\Delta \vect{p}}, during the time of the
collision.
So now we have:
%
\Eqn{}{\Delta \vect{p} = \vect{F} \Delta t;}
%
or:
%
\Eqn{4}{\vect{F} = \dfrac{\Delta \vect{p}}{\Delta t}. \qquad (\text{Constant Force})}
%

To find the force on the ball, using \Eqnref{4}, we
must compute the ball's change in momentum, \m{\Delta\vect{p}}, during the time
interval \m{\Delta t}.
Consider \Figref{1}, showing the initial and final momenta.
Note that the momentum of the ball is reversed in direction by the collision
but is left unchanged in magnitude.
Then:
\begin{eqnarray*}
\Delta \vect{p} & = & \vect{p}_f - \vect{p}_i \\
               & = & - m v_0 \uvec{x} - (m v_0 \uvec{x}) \\
               & = & - 2 m v_0 \uvec{x},
\end{eqnarray*}
where \m{v_0} is the ball's speed (30\unit{m/s} in this case) and \m{\uvec{x}} is a unit
vector in the direction of motion before the collision.

Now what about \m{\Delta t}, the time for collision?
No doubt you have seen strobe flash pictures
%
%\Footnote{5}{See {\em Fundamentals of Physics}, Halliday and Resnick,
%Wiley (1974), \Figstoref{9}{1},p.\,154.}
%
of a ball and bat in contact which show the ball to be distorted, as
represented in \Figref{2}.
We will use what we have seen in such pictures in order to make estimates.

\CaptionedLeftFramedFigure{2}{The collision-deformed ball.}{m15gr02}

Let's estimate the maximum value of \m{x} (see \Figref{2}) to be about one-fourth
of the ball's diameter \m{D}.
This maximum compression occurs at the instant the ball is at rest with
respect to the bat (at approximately one-half \m{\Delta t}).
During this time, the first half of the collision time, the ball's speed has
gone from \m{v_0} to zero (neglecting the bat's speed).
Assuming constant deceleration, the ball's average speed is \m{v_0}/2.
Now (av.\,speed)\m{\cdot}(time)\,\m{\simeq}\,distance, so:
\begin{center}
\m{\dfrac{v_0}{2} \cdot \dfrac{\Delta t}{2} \simeq
\dfrac{D}{4};}
\end{center}
or:
\begin{center}
\m{\Delta t \simeq \dfrac{D}{v_0}.}
\end{center}
With these estimates for \m{\Delta \vect{p}} and \m{\Delta t}, we use \Eqnref{4} to
get
%
\Eqn{}{\vect{F} \simeq \dfrac{- 2 m v_0^2 \uvec{x}}{D}}
%
or
%
\Eqn{5}{F \simeq \left(\dfrac{2v_0^2}{g D}\right) m g =
\left(\dfrac{2 v_0^2}{g D}\right) W,}
%
where \m{W = m g} is the ball's weight.
Substituting approximate values for \m{v_0 \simeq 30\unit{m/s}},
\m{g \simeq 10\unit{m/s\up{2}}}, and \m{D = 0.10\unit{m}},
gives:
%
\Eqn{}{F \simeq \left(\dfrac{2 \times 900\unit{m\up{2}}/\unit{s\up{2}}}
      {10\;\unit{m/s\up{2}}\times 0.10\unit{m}}\right) W = 1800 W.}
%
So, if your answer in the original problem was (d), \m{1000\,W}, congratulations!

\tryit By the way, how many pounds or Newtons is that?

\pcap{3}{c}{But the Collision Force Isn't Constant}
Now let's see what we can do about the \Quote{constant force} swindle of the
previous paragraph.\Index{collision force, related to momentum}\Index{momentum| related to collision force}
Starting with \m{\vect{F}(t) = d\vect{p}/dt}, where we have explicitly noted the
dependence of \vect{F} on \m{t}, we integrate both sides of the equation with
respect to time, from \m{t = 0} (ball first touches bat) to \m{t = \Delta t}
(ball leaves bat):
%
\Eqn{}{\int_0^{\Delta t} \vect{F}_R(t)\,dt =
\int_0^{\Delta t} \dfrac{d\vect{p}}{dt}\,dt.}
%
But
%
\Eqn{}{\int_0^{\Delta t} \dfrac{d\vect{p}}{dt}\,dt = \vect{p}(\Delta t) - p(0) =
\Delta \vect{p}.}
%
Then:
%
\Eqn{6}{\int_0^{\Delta t} \vect{F}_R(t)\,dt = \Delta \vect{p}.}
%
In the Appendix we show that an integral of a vector function of a single
scalar variable, such as the one in \Eqnref{6}, can be written as the average
value of the function \m{\vect{F}_{R,av}} multiplied by the averaging interval
\m{\Delta t}.
So:
%
\Eqn{}{\vect{F}_{R,av} \Delta t = \Delta \vect{p};}
%
or:
%
\Eqn{7}{\vect{F}_{R,av} = \dfrac{\Delta \vect{p}}{\Delta t}.}
%
Compare Eqs.\,(4) and (7).
The time-average resultant force over a time interval \m{\Delta t} is exactly
equal to the constant force which would cause that same momentum change
during that same time interval.
Consequently, our previous estimate of approximately 1800~times the ball's
weight is exactly correct for the average resultant force on the ball.

\pcap{3}{d}{Why Collision Forces Are Usually So Large}

\CaptionedFullFramedFigure{3}{A collision in which a particle's
momentum changes from \m{\vect{p}_i} to \m{\vect{p}_f}.}{m15gr03}

While we obtained \Eqnref{7} in considering a
ball-bat collision, you can discover by rereading that its derivation was
completely general.
Thus the time-average resultant force on an object always equals its change
in momentum divided by the time interval during which the momentum change
occurred: \m{\vect{F}_{R,av} = \Delta \vect{p} / \Delta t} (see \Figref{3}).
Thus for a given momentum change, the average force increases as the
collision time decreases.
And that's why pole vaulters land on a few feet of foam rubber rather than
concrete!
In each case, their momentum change would be the same, but the increased
collision time resulting from the foam's compressibility saves the day \ldots
not to mention the pole vaulter.

\pcap{3}{e}{The Impulse-Momentum Principle}
The impulse\Index{impulse} of a force \m{\vect{F}(t)} during a time interval
\m{t_1 \leq t \leq t_2} is defined as:
%
\Eqn{}{\text{Impulse } \equiv \int_{t_1}^{t_2} \vect{F}(t)\,dt.}
%
Comparing this with \Eqnref{6}, we have the so-called Impulse-Momentum
Principle:\Index{impulse-momentum principle}

\begin{center}\begin{tabular}{|l| c |l|}\cline{1-1}\cline{3-3}
Impulse by resultant &   & Change in momentum   \\
force during time    & = & during time interval \\
interval \m{\Delta t}  &   & \m{\Delta t} \\ \cline{1-1}\cline{3-3}
\end{tabular}\end{center}
%
\pcap{3}{f}{Solution Using Components}
Here we will illustrate solving a two-dimensional average-force problem using Cartesian components
(a more elegant method is developed in the Problem Supplement).
\TextAndFigure{Problem: A car having mass \m{m}, sliding on ice at speed \m{v^i},
hits a concrete wall and bounces off, as shown, with speed \m{v^f}.
If the collision time is measured from high speed movies to be \m{\Delta t},
determine the average force (magnitude and direction) exerted on the car by the wall.}{m15gr24}

We start with \Eqnref{7}, written in terms of its Cartesian components:
%
\Eqn{15}{\vect{F}_{R,av} = \dfrac{\Delta p_x}{\Delta t}\uvec{x} + \dfrac{\Delta p_y}{\Delta t}\uvec{y}
                       + \dfrac{\Delta p_z}{\Delta t}\uvec{z}\,.}
%
We take the scalar product of \uvec{x} with each term in \Eqnref{15} and we get the \m{x}-component
of the average resultant force:
%
\Eqn{16}{F_{R,av,x} = \dfrac{\Delta p_x}{\Delta t}\,,}
%
and of course there are similar equations for the \m{y}- and \m{z}-components of \m{\vect{F}_{R,av}},
obtained by mutiplying \Eqnref{7} by \uvec{y} or \uvec{z}.
Putting in the given quantities for the \m{x}- and \m{y}-components of the car's initial momentum
\vect{p}\,\up{i} and final momentum \vect{p}\,\up{f}:
%
\Eqn{}{\Delta p_x = p^f_x - p^i_x = mv^f_x-mv^i_x = mv^f\sin\phi-mv^i\sin\theta\,,}
\Eqn{}{\Delta p_y = p^f_y - p^i_y = mv^f_y-mv^i_y = mv^f\cos\phi+mv^i\cos\theta\,.}
%
where we have taken the downward and rightward directions as positive.  Then:
%
\Eqn{}{F_{R,av,x} = \dfrac{mv^f\sin\phi-mv^i\sin\theta}{\Delta t}\,,}
%
\Eqn{}{F_{R,av,y} = \dfrac{mv^f\cos\phi+mv^i\cos\phi}{\Delta t}\,.}
%
\Eqn{}{\text{angle CCW from upward vertical } = \tan^{-1}(F_{R,av,x}/F_{R,av,y})\,.}
%
\tryit Suppose the car weighs 3200\unit{lb}, has an initial speed of 41\unit{mi/hr} (60.1\unit{ft/s}),
a final speed of 22\unit{mi/hr} (32.3\unit{ft/s}), and a collision time of 0.50\unit{s}.
Show that the average force (magnitude and direction) exerted on the car by the wall during that time
interval is \m{1.4 \times 10^4 \unit{lb}} at 32\degrees CCW from the upward direction
or 122\degrees CCW from the rightward direction.
}% /Sect

\Sect{4}{Conservation of Momentum}{\SectType{TextMultiPara}}{
%
\pcap{4}{a}{Dealing With Larger Systems}
For a single-particle system, the conservation of momentum principle is
equivalent to Newton's first law (See Sect.\,2b).\Index{conservation of momentum| for system}
We will now extend the discussion to systems with two or more particles and
discover that the momentum conservation principle acquires added
significance.

\CaptionedLeftFramedFigure{4}{An isolated two-particle system.}{m15gr04}

\pcap{4}{b}{Derivation for Isolated Two-Particle Systems}
Using Newton's second law and third laws, we can
derive conservation of momentum for any isolated system consisting of two
interacting particles.\Index{conservation of momentum| isolated system}\Index{isolated system, conservation of momentum}
This is a system in which two particles interact with each other but not
with other particles.\Index{two-particle system}
The system momentum\Index{momentum| of system of particles}\Index{system of particles, momentum for} at the time \m{t} is denoted by \vect{P}(t) and defined by
%
\Eqn{8}{\vect{P}(t) = \vect{p}_1(t) + \vect{p}_2(t),}
%
where \m{\vect{p}_n(t)} is the momentum of particle \m{n} at time \m{t} (See \Figref{4}).
Differentiating both sides of \Eqnref{8} with respect to time, and using
Newton's second law, we get:
%
\Eqn{9}{\dfrac{d\vect{P}}{dt} = \dfrac{d\vect{p}_1}{dt} +
\dfrac{d\vect{p}_2}{dt} = \vect{F}_{R1}(t) + \vect{F}_{R2}(t),}
%
where \m{\vect{F}_{Rn}(t)} is the resultant force on particle \m{n} at time \m{t}.
Since the system is assumed to be isolated, particle~1 interacts only with
particle~2 so:
%
\Eqn{}{\vect{F}_{R1}(t) = \vect{F}_{2 \text{ on } 1}(t),}
%
where \m{\vect{F}_{2 \text{ on } 1}(t)} is the force particle~2 exerts on particle~1
at time \m{t}.
Similarly,
%
\Eqn{}{\vect{F}_{R2}(t) = \vect{F}_{1 \text{ on } 2}(t).}
%
Now by Newton's third law, \m{\vect{F}_{1 \text{ on } 2}(t)} and
\m{\vect{F}_{2 \text{ on } 1}(t)}
are equal in magnitude and opposite in direction at each instant of time.
Then \Eqnref{9} becomes:
%
\Eqn{}{\dfrac{d\vect{P}}{dt} = \vect{F}_{2 \text{ on } 1}(t) +
\vect{F}_{1 \text{ on } 2}(t) = 0.}
%
Since its time rate of change is zero, the total linear momentum for an
isolated two-particle system must not change with time;\Index{system| conservation of momentum for} i.e.,
%
\Eqn{10}{\vect{P}(t) = \text{ Constant}. \qquad \text{(Isolated System)}}
%
Very interesting.
While \m{\vect{p}_1} and \m{\vect{p}_2} may be changing very rapidly because of the
two particles' mutual interaction, their sum \vect{P} remains constant.
In simple terms, \vect{P} represents the total momentum available to the two
particles (as long as they are isolated), and the two of them \Quote{swap} parts of
that momentum back and forth via their interaction.

\pcap{4}{c}{The General Conservation of Momentum Principle}
Here is the principal result of this unit:
\begin{center}\begin{tabular}{l p{0.3in} |l|} \cline{3-3}
\textit{Conservation} & & The total linear momentum for an \\
\textit{of Momentum}  & & isolated system of interacting \\
\textit{Principle}    & & particles is constant. \\ \cline{3-3}
\end{tabular}\end{center}
Thus even for \m{10^{20}} interacting particles, each of which may have an
erratically changing momentum vector, the sum of these \m{10^{20}} vectors will
remain constant provided the total system is isolated from any other system.
The proof of this principle, for Newtonian mechanics, is relegated to Problem~2 of this module's \textit{Problem Supplement}.

\pcap{4}{d}{When the System Isn't Isolated}
\Index{conservation of momentum| non-isolated system}If a system is not isolated, then the rate of change of the system's total
linear momentum equals the sum of the resultant external forces acting on all
the particles of the system.
We can easily demonstrate this for a two-particle system, starting with
\Eqnref{9}:
%
\Eqn{}{\dfrac{d\vect{P}}{dt}=\vect{F}_{R1}(t)+\vect{F}_{R2}(t).}
%
If particles~1 and 2 interact with other particles as well as each other,
then the resultant force on each is the sum of the force on it by the other
particle and the force on it by particles external to the system; i.e.,
%
\Eqn{}{\vect{F}_{R1} = \vect{F}_{2 \text{ on } 1} + \vect{F}_1^\text{ext}}
%
\Eqn{}{\vect{F}_{R2} = \vect{F}_{1 \text{ on } 2} + \vect{F}_2^\text{ext}.}
%
Substituting these into the equation for d\m{\vect{P}/dt} gives
%
\Eqn{}{\dfrac{d\vect{P}}{dt} = \vect{F}_1^\text{ext} + \vect{F}_2^\text{ext},}
%
where the internal forces cancel once again because of Newton's third law.
Now let \m{\vect{F}_R^\text{ext}} be the sum of the external force on the system;
i.e.,
%
\Eqn{}{\vect{F}_R^\text{ext}(t) = \vect{F}_1^\text{ext} + \vect{F}_2^\text{ext},}
%
so that
%
\Eqn{11}{\dfrac{d\vect{P}}{dt}=\vect{F}_R^\text{ext}.}
%
Comparing this result with the corresponding equation for a one-particle
system, \m{d\vect{p}/dt = \vect{F}_R}, you see that the momentum for the
two-particle system changes just as if the system were a single particle acted
upon by the external force \m{\vect{F}_R^\text{ext}}.
} % /Sect
%
\Sect{5}{Collision Momenta}{\SectType{TextMultiPara}}{
%
\pcap{5}{a}{Using Momentum Conservation to Analyze Collisions}
You will find momentum conservation useful in many applications, particularly
in phenomena which involve the collision of two or more particles.
In these collisions the particles usually undergo rapid momentum changes
which involve very large, but short-lived forces (remember the baseball).
Even though the particles in such a collision may not be completely isolated
from the surroundings, the collision forces\Index{collision force}\Index{force| collision}
are usually so large and the collision time so short that the system can be very
accurately approximated as being isolated for the duration of the collision.

\pcap{5}{b}{Conservation of Momentum Components in Collisions}
Each Cartesian component of momentum is separately conserved in momentum-conserving
interactions and this fact is often used in solving two-dimensional and three-dimensional
collision problems.
%
The proof of momentum component conservation starts with the statement of conservation
of momentum,
%
\Eqn{13}{\vect{p}^{\,f} = \vect{p}^{\,i}\,,}
%
where \m{\vect{p}^{\,f}} is the momentum of the system being studied at some \Quote{final} time, and
\m{\vect{p}^{\,i}} is the momentum of the system at some \Quote{initial} time.
\Equationref{13} just says that the system's momentum does not change with time; that it is \Quote{conserved.}
We now expand each momentum vector in \Eqnref{13} in its Cartesian components:
%
\Eqn{14}{p^{\,f}_x\,\uvec{x} + p^{\,f}_y\,\uvec{y} + p^{\,f}_z\,\uvec{z} =
p^{\,i}_x\,\uvec{x} + p^{\,i}_y\,\uvec{y} + p^{\,i}_z\,\uvec{z}\,.}
%
\Equationref{14} can be written more succinctly as:
%
\Eqn{20}{\sum_{n=1}^{n=3} p^{\,f}_n\,\uvec{x}_n = \sum_{n=1}^{n=3} p^{\,i}_n\,\uvec{x}_n\,,}
%
where \m{x_1 \equiv x}, \m{x_2 \equiv y}, and \m{x_3 \equiv z}.
Finally, we multiply \Eqnref{14} or \Eqnref{20} by any one of the three unit vectors and we
get conservation of momentum for that component of momentum.
For example, taking the scalar (\Quote{dot}) product of \Eqnref{14} and \uvec{x} gives conservation of the
\m{x}-component of momentum:
%
\Eqn{21}{p^{\,f}_x = p^{\,i}_x\,.}
%
To complete the proof, one need only say \Quote{and the same for the \m{y}- and \m{z}-components.}
More succinctly, taking the dot product of \Eqnref{20} and the \m{\text{n}^\text{th}} unit vector,
\m{\uvec{x}_n}, gives conservation of the \m{\text{n}^\text{th}} component of momentum:
%
\Eqn{22}{p^{\,f}_n = p^{\,i}_n; \qquad n = 1, 2, 3\,.}
%
In collisions, each momentum in \Eqnref{13} is the sum of the momenta of the individual particles
participating in the collision.
Thus one must equate the sum of the \m{x}-components of the particles in the system before the collision to
the sum of the \m{x}-components of the particles in the system after the collision.
Then one must do the same for the \m{y}-components and perhaps the \m{z}-components.
This means there will be several equations that one solves for the several unknowns in the problem.
%
\pcap{5}{c}{Conserved Quantities in Collisions}
Momentum if conserved in any isolated interaction, but other quantities are also conserved and they
place additional restrictions on what can happen during the interaction.
Whether any particular restriction applies to a specific interaction depends on what particles
are participating in the interaction.
For example, if the system is isolated from outside forces then momentum must be conserved.
As another example, if only \Quote{conservative forces} participate in the interaction then
energy must be conserved.
%
\Footnote{6}{For the definition of \textit{conservative forces}, see \textit{Potential
Energy, Conservation of Energy}, MISN-0-21.}
%
There are many such quantities that can be conserved in certain classes of interaction.
For this module the problems you are to solve will not assume energy conservation.

\pcap{5}{d}{Example: Momentum Components in a Collision}
\TextAndFigure{Problem: A mass \m{m} moving with a speed \m{u} collides with a mass \m{M} at rest.
After the collision, each moves as shown.
Assuming momentum conservation, we want to determine the speeds \m{w} and \m{V} after the collision.}{m15gr11}

Adopting the notation introduced in \Sectref{5}{b}, and adding superscripts to identify the
individual particles, the momentum \m{x}-components of the individual particles are:
%
\Eqn{}{p_x^{m,i} = m u;\quad p_x^{m,f} = m w \cos\theta;\quad p_x^{M,i} = 0;\quad p_x^{M,f} = M V \cos\phi\,,}
%
while the \m{y}-components are:
%
\Eqn{}{p_y^{m,i} = 0;\quad p_y^{m,f} = m w \sin\theta;\quad p_y^{M,i} = 0;\quad p_y^{M,f} = - M V \sin\phi\,.}
%
The system of two particles is assumed to have no external forces acting on it during the
collision, so the system momentum components are conserved:
%
\Footnote{7}{Adding a requirement of energy conservation would add another equation and that would make
it three relationships between two unknowns.
No solution exists unless we allow one of the known variables to be an unknown variable, to be fixed
by solving the three equations for the (then) three unknowns.
One usually picks one of the outgoing angles, in this case \m{\theta} or \m{\phi}, which is then
restricted by the known variables to be a single value.}
%
\Eqn{}{p_x^{m,i} + p_x^{M,i} = p_x^{m,f} + p_x^{M,f}\,,}
%
\Eqn{}{p_y^{m,i} + p_y^{M,i} = p_y^{m,f} + p_y^{M,f}\,.}
%
For our case, these equations become:
%
\TwoEqns{18}{mu & = m w \cos\theta + M V \cos\phi\,,}
            { 0 & = m w \sin\theta - M V \sin\phi\,.}
%
Note that in this particular collision the system \m{y}-component of momentum is zero
both before and after the collision.
The two \Eqnsref{18} can be solved simultaneously for the two unknowns, \m{w} and \m{V},
in terms of the given quantities (\m{m}, \m{M}, \m{u}, \m{\theta}, \m{\phi}).
It is just a little algebra.
%
\Footnote{8}{We also like to make the substitution: \m{\sin\theta\cos\phi + \cos\theta\sin\phi = \sin(\theta + \phi)}.
This makes the answer more succinct but it is not really necessary.}

\tryit{Show that: \m{w = \dfrac{u\,\sin\phi}{\sin(\theta + \phi)}\,;\qquad\text{and:}\qquad
                     V = \dfrac{m\,u\,\sin\theta}{M\,\sin(\theta + \phi)}}\,.}
%
}% /Sect
%
\Sect{}{Acknowledgments}{\SectType{Acknowledgments}}{\NsfAcknowledgment}% /Sect
%
\Sect{A}{Averaging a Vector Function}{\SectType{AppendixOnePara}}{
\noindent\vspace*{-36pt}\newline
\begin{center}(for those interested)\end{center}

\CaptionedFullFramedFigure{5}{(a) \m{\int_a^b f(\tau)\,d\tau}
equals the shaded area.
(b) \m{f_{av}} equals the height of the rectangle with area equal to
\m{\int_a^b f(\tau )\,d\tau}.}{m15gr05}

If \m{\vect{C}(\tau)} is a vector function of a single variable \m{\tau} (e.g.,
position as a function of time), the integral\Index{integral| of vector function}\Index{vector function| integral of} of \vect{C} on the interval
\m{a < \tau < b} is defined in terms of the corresponding integrals of the
components of \vect{C}; i.e.,
%
\Eqn{12}{\int_a^b \vect{C}(\tau)\,d\tau =
\uvec{x} \int_a^b C_x(\tau)\,d\tau + \uvec{y} \int_a^b C_y(\tau)\,d\tau +
\uvec{z} \int_a^b C_z(\tau)\,d\tau.}
%
Each of these three integrals are just integrals of a real function of one
variable of the form
%
\Eqn{}{\int_a^b f(\tau)\,d\tau.}
%
Remember that this integral is the area of the region inside \m{y = f(\tau)},
\m{\tau = a}, \m{\tau = b}, and the \m{x}-axis.
This area is the shaded region in \Figref{5}a.
The average value\Index{average value, of vector function}\Index{vector function| average value of} of \m{f} on this interval, denoted by
\m{f_{av}}, equals the height of the rectangle of the same area as that defined by
the integral.
Thus
%
\Eqn{}{\text{Area of Rectangle } = \int_a^b f(\tau)\,d\tau.}
%
But the area of the rectangle of height \m{f_{av}} and base \m{b - a} is
\m{f_{av}(b - a)}, so that
%
\Eqn{}{f_{av}(b - a) = \int_a^b f(\tau)\,d\tau,}
%
or:
%
\Eqn{}{f_{av} = \dfrac{1}{b - a} \int_a^b f(\tau)\,d\tau.}
%
This value is the dashed line in \Figref{5}b.
Note that it checks visually with what would expect for an \Quote{average}
value.

Using the results we just obtained for a real function of one variable, we
write for the vector function:
%
\Eqn{}{C_{x,av}(b - a) = \int_a^b C_x(\tau)\,d\tau,}
%
\Eqn{}{C_{y,av}(b - a) = \int_a^b C_y(\tau)\,d\tau,}
%
\Eqn{}{C_{z,av}(b - a) = \int_a^b C_z(\tau)\,d\tau.}
%
Substituting these into \Eqnref{12}, gives
%
\Eqn{}{\int_a^b \vect{C}(\tau)\,d\tau =
\left(C_{x,av} \uvec{x} + C_{y,av} \uvec{y} + C_{z,av} \uvec{z}\right)(b - a).}
%
We define the average of \m{\vect{C}(\tau)} on \m{a < \tau < b} as
%
\Eqn{}{\vect{C}_{av} \equiv C_{x,av} \uvec{x} + C_{y,av} \uvec{y} + C_{z,av} \uvec{z}
\qquad (\text{Definition}).}
%
Equation (13) may now be rewritten:
%
\Eqn{}{\vect{C}_{av} = \dfrac{1}{b-a} \int_a^b \vect{C}(\tau)\,d\tau.}
%
}% /Sect

