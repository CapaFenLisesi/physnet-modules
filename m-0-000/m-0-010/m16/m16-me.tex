\revhist{12/29/87, pss; 9/13/89, pss; 5/9/91, pss; 4/27/94, pss; 8/25/96, pss}

\Sect{}{}{\SectType{ModelExam}}{

\begin{one-digit-list}
\item [1.] See Output Skill K1 in this module's \textit{ID Sheet}.
\end{one-digit-list}

\noindent Treat each problem as having five separate parts:
\begin{itemize}
\item [a.] Draw a one-body diagram with the forces \Quote{as given} but with
symbols representing all quantities;
\item [b.] Draw a diagram similar to the one in part (a), but with the forces
now represented only by their components, where one axis is along the direction
of acceleration and the axes are labeled;
\item [c.] Write Newton's second law equations for the force components; and
\item [d.] Solve for the target unknown(s), still with all quantities
represented by symbols.
\item [e.] Substitute numerical values along with their units and solve for
the numerical answer(s).
Express vector answers in terms of the unit vectors for the axes shown
in part (b).
\end{itemize}

\begin{one-digit-list}
\item [2.] \ItemFigure{A block weighing 8.0\unit{lb} is projected up an incline of {20.0\degrees}
with initial velocity \m{v_0}.
For the interface between the block and the surface of the incline,
\m{\mu_{ns}} = 0.40 and \m{\mu_{sl}} = 0.30.
Determine the acceleration of the block.}{m16gr12}

\item [3.] When the block of Problem~2 stops, will it remain at rest or will it
slide back down the incline?
If it remains at rest, what frictional force will be acting on it?
\item [4.] If the block of Problem~2 is initially projected down the incline,
determine its (vector) acceleration.
\end{one-digit-list}

\newpage

\BriefAns

\begin{itemize}
\item [1.] See this module's text.

\item [2-4.] See this module's \textit{Problem Supplement}, Problems~6-8.
\end{itemize}

}% /Sect
