\revhist{12/28/87, pss; 12/29/88, pss; 9/12/89, pss; 12/9/89, pss; 5/9/91, pss; 4/27/94, pss;
         9/7/94, pss; 9/12/95, pss; 11/7/97, lae; 10/25/99, pss; 12/15/00, pss; 3/29/01, pss;
         5/17/02, pss; 6/13/02, pss; 10/4/02, pss}
%
\Sect{1}{Introduction}{\SectType{TextMultiPara}}{
%
\CaptionedLeftFramedFigure{1}{Some forces acting on a pulled block.
Reactive forces acting on the block are not shown.}{m16gr01}

\pcap{1}{a}{Friction Needs Normal and Tangential Forces}
In this module we introduce some effects of friction into some simple cases in
mechanics.
%
\Footnote{1}{See \Quote{Frictionless Applications ...} (MISN-0-10).}
%
All such cases occur when there are two objects in contact and there are two
force components that act on the objects:
(1) a \Quote{normal force} which is perpendicular to the surfaces
and whose fuction is to keep the surfaces in contact; and
(2) a \Quote{tangential or lateral force} which is parallel to the surfaces
and which would cause one surface to slide over the other if there was no friction.
These two forces are at right angles to each other: one is normal to the two
surfaces that are in contact and the other is tangential to those surfaces
(see \Figref{1}).
The normal force acting on one of the objects in contact is generally equal to,
and in the opposite direction from, the normal force acting on the other object
in contact with the first: in this way contact is
maintained between the surfaces of the two objects and there is no motion in the
direction perpendicular to the surfaces (vertically in \Figref{1}).
Similarly, the tangential or lateral force on one object is generally in the opposite
direction to the lateral force (horizontally in \Figref{1}) acting on the other object
and this means that one object would slide over the other if there was no friction.

\pcap{1}{b}{Non-Sliding and Sliding Friction}
There are many different kinds of friction, but the major ones are non-sliding fricxtion
and sliding friction.
An example of \textit{sliding} friction is the force between the wheel of a
stuck car and the icy surface that the wheel is spinning on, a force that may
cause smoke to be generated but no motion of the car.
Another example is the car sliding across an icy patch on a road, where the brakes
have stopped the tires from turning but the tires, and hence the car, are still moving across
the ice.
An example of \textit{non-sliding} friction is the force between the tires of
the same car and a bare part of the pavement or between the tires and the icy surface
after sand has been applied to the ice so the car can \Quote{get a grip} and cause the car
to be controllable again.
Two sub-categories of sliding friction are \Emph{rolling} friction and \Emph{lubricated-surface}
friction, but we will not deal with these specifically in this module.

\pcap{1}{c}{Example: Hand Rubbing}
Think of pressing your hands tightly together, exerting a strong \Quote{normal}
force (so named because it is mathematically \Quote{normal} to the plane of the
surfaces that are in contact).
If you are exerting enough normal force, then exerting a small tangential
force will not cause motion; one hand will not start rubbing over the other.
This is because the force of non-sliding friction is resisting the relative
motion.
In fact, the force of non-sliding friction is exactly equal but opposite to
your applied tangential force, resulting in zero net force on your hand and
hence zero acceleration of the hand.

If you now exert enough tangential force, it will \Quote{break} the non-sliding
friction and allow relative motion (called \Quote{rubbing} in the case of
hands).
If the normal force is strong enough and the rubbing is fast enough and
lasts long enough, you will perceive heat at the interface between the
hands.

If you make the normal force smaller by pressing your hands together
less tightly, you will find that a lower normal force results in a lower
tangential force needed to break the force of non-sliding friction and allow
motion.

}% /Sect
%
\Sect{2}{Origin and Laws}{\SectType{TextMultiPara}}{

\pcap{2}{a}{The Molecular Origin of Friction}
When normal forces hold two surfaces together and tangential forces tend to
make one surface slide over the other, the tangential force is opposed by
a \Quote{frictional}\Index{force| frictional}\Index{frictional force} force.
This frictional force is due to chains of molecules that have settled
out of the environment (dust, fingerprints, water vapor, etc.) onto each of the
surfaces before they were in contact.
Each clean surface had regular hills and valleys on the molecular level before
the \Quote{dirt} arrived but the molecular chains of dirt are somewat irregular
and can be caused to move on the surfaces.
When the two surfaces are brought together, the dirt chains \Quote{lock} the
surfaces together until enough sideways force is applied to cause a slipping from one
surface valley to the next.
%
\Footnote{22}{See \Quote{Surface Grime Explains Friction,} \textit{Physical Review Focus},
\url{http://focus.aps.org/v7/st6.html}, and references contained therein.
There is a later preprint dated 4/30/02: \textit{Statistical Mechanics of Static and
Low-velocity Kinetic Friction}, by Martin H.\,M\"user, Michael Urbakh, and
Mark O.\,Robbins, Dept.\,of Physics and Astronomy, The Johns Hopkins University.}
%
Thus if enough lateral force is applied there will be a sliding of one surface over the other
but, if the force is then removed, the motion will slow down and stop.
If an applied force is insufficient to overcome the \Quote{locking} effect of the dirt,
the surfaces will stay \Quote{stuck} and one surface will not move over the other.
The fact that there is no lateral motion in this locked case means that the lateral speed stays zero
so the lateral acceleration stays zero so the net lateral force on each object is zero.
Since we are applying a lateral force to each object, the fact that the net force on each
object is zero means that the surface friction must be resisting motion with
an exactly equal and opposite lateral resisting force on each onject.
If we apply a zero lateral force and gradually increase it, friction will produce
an exactly equal and opposite force that prevents any acceleration and hence any motion.
However, when a large enough lateral force is applied, the \Quote{locking} of the
surfaces is overcome and the surfaces start to slide over each other in a motion that appears
smooth on the macroscopic level but that is a sort of \Quote{up and down}, \Quote{slip and lock}
motion on the molecular level.

\pcap{2}{b}{Amontons's Laws}
This is the explanation of the widely-used Amontons's Laws for the force of friction between
two surfaces
%
\Footnote{24}{The description of a force as being \Quote{between two surfaces} means
that each of the objects whose surfaces are in contact experiences a force having the same
physical origin.  The two forces have equal magnitude but are opposite in direction.}
%
that are in contact.
%
\Footnote{23}{Guillaume Amontons, \textit{Histoire de l'Acad\'emie Royale des Sciences avec les
M\'emoires de Math\'ematique ey de Physique}, page 206 (1699).}
%
which state that the (lateral) force of static friction between the two surfaces is:
(1) proportional to the normal force holding the objects' surfaces together; and
(2) independent of the surface area in contact. 

\pcap{2}{c}{\Quote{Static} and \Quote{Kinetic} Friction}
In technical documents, the friction in the case where a car is static, not
moving because the wheel is slipping, is called \Quote{kinetic} friction,
\Index{force| of kinetic friction} while
the friction in the case where a car is moving without slipping, and hence
has kinetic energy, is called the \Quote{static}\Index{force| of static friction} case.
To make clear which case is appropriate in particular circumstances, we will
here use the terms \Quote{non-sliding} and \Quote{sliding} rather than \Quote{static}
and \Quote{kinetic.}
However, you must usually use \Quote{static} and \Quote{kinetic} when finding friction characteristics
in the literature.

\CaptionedLeftFramedFigure{2}{Response of a non-sliding frictional
force to an applied force.}{m16gr02}

\pcap{2}{d}{General Characteristics}
The force of non-sliding friction is tangential to two surfaces that are in
contact and it increases as necessary in order to prevent relative motion of
the two surfaces (see \Figref{2}).
It can increase in this fashion only up to some maximum that is determined:
(1) by the nature of the two surfaces; and (2) by the magnitude of the
normal force that is pressing the two surfaces together.

Note that the force of non-sliding friction itself depends only on the
applied tangential force: it is only its maximum value which depends on the
surface character and on the size of the normal force.

If an applied tangential force exceeds the capacity of non-sliding friction
to resist it, then one surface starts to move over the other and the force
of friction becomes a \Quote{sliding} one.
The magnitude of this sliding force depends on: (1) the nature of
the surfaces; (2) the magnitude of the normal force pressing the surfaces
together; and (3) the relative speed of the surfaces (the speed
of either surface with respect to the other).

Sliding friction increases only very slowly with increasing speed, while non-sliding friction
decreases with decreasing applied tangential force.

\CaptionedLeftFramedFigure{3}{A horizontal force is applied to a block.}{m16gr03}%

\CaptionedLeftFramedFigure{4}{The force diagram for \Figref{3}.}{m16gr04}

\pcap{2}{e}{Example: A Block Pulled Horizontally}
An object rests on a horizontal surface and a force \vect{P}
is applied to the block parallel to the surface (\Figref{3}).
From experience we know that if the force \vect{P} is not too large the
block will not move.
Therefore the surface must be exerting a tangential force on the block that
is equal but opposite to \vect{P}.

In the one-body diagram, \Figref{4}, the force of the surface on the block has
been decomposed into its \Quote{frictional} and \Quote{normal} components,
labeled \m{\vect{r}_{ns}}.
If the pull, \vect{P}, is increased, \vect{f} will increase as long as
its limit is not exceeded.
If \vect{P} is reversed (applied in the opposite direction) \vect{f} will
respond by reversing its direction.
In a similar fashion, \vect{N} is exactly equal but opposite to the
non-contact gravitational pull of the earth on the object, \m{m \vect{g}}.
Thus both the normal and frictional components of the force of the surface
on the block act so as to maintain the block's state of equilibrium
(keeping the resultant force zero: \m{F_{R,x} = 0, F_{R,y} = 0}).

When \m{|\vect{P}|} exceeds the maximum sustainable force of non-sliding
friction for the current value of the normal force, the block accelerates to
the right.
Now the force of sliding friction takes over and it increases slightly as speed
increases.
If \vect{P} is removed completely, while the block is in motion, the force
of sliding friction will continue to resist the motion until it brings the
block to rest.
}% /Sect
%
\Sect{3}{The Coefficients of Friction}{\SectType{TextMultiPara}}{
%
\pcap{3}{a}{The Non-Sliding Case}
\Index{static friction| coefficient of}\Index{coefficient of static friction}
It has been found experimentally that the maximum (tangential) force of
non-sliding friction, \m{\vect{f}_{ns}}(max), which any particular surface can
exert on another particular surface, is directly proportional to the normal
force \vect{N} that is pressing the two surfaces together:
%
\Eqn{}{f_{ns}(\text{max}) = \mu_{ns}\,N\,.}
%
The constant of proportionality, \m{\mu_{ns}}, is called the \Quote{coefficient
of non-sliding friction.}
%
\Footnote{2}{In scientific documents it is called the \Quote{coefficient of static
friction} and is labeled \m{\mu_\text{s}}.}
%
The coefficient of non-sliding friction, \m{\mu_{ns}}, depends on the nature
of the two surfaces involved.
It is a number without units (because it is a ratio of forces).
It is very large for rough rubber on rough rubber and very small for ice
skates on ice.

\pcap{3}{b}{The Sliding Case}
\Index{coefficient of kinetic friction}\Index{kinetic friction| coefficient of}
If one surface is moving at speed \m{v} with respect to another surface, to
which it is being pressed with normal force \m{N}, then the tangential force
of friction is found to be proportional to \m{N}:
%
\Eqn{}{f_{sl} = \mu_{sl}\,N\,.}
%
where \m{\mu_{sl}} is called the \Quote{coefficient of sliding friction.}
%
\Footnote{3}{in scientific documents it is called the \Quote{coefficient of kinetic
friction} and is labeled \m{\mu_\text{k}}.}
%
Experimentally, the coefficient of sliding friction is generally a very slowly
increasing function of velocity; we shall assume that it is a constant,
independent of velocity (for further details, see the Appendix).

\pcap{3}{c}{The Transition}
The transition from the non-sliding case to the sliding case is continuous,
as are all processes in nature.
To see this we plot, in \Figref{6}, the frictional force versus time as the
applied force \m{P} is increased at a constant rate.
Note that on the left side of the graph the frictional force is of the
non-sliding type, while on the right side it is of the sliding type.
The abrupt change in the transition has been likened to a phase change in
the material.\FnRef{22}

\CaptionedLeftFramedFigure{6}{Frictional response as \m{P} of \Figref{3}
increases linearly with time.}{m16gr06}

\pcap{3}{d}{The Speed-Independent Approximation}
Although the coefficient of sliding friction is a slowly-varying function of the relative
speed of the two surfaces, it makes life easier and it is usually a good approximation
to represent \m{\mu_{sl}} by a single value, independent of relative speed:
%
\Eqn{}{f_{sl} = \mu_{sl}\,N\,. \qquad \text{(constant)}\,.}
%
This makes the sliding force and the maximum non-sliding force seem somewhat
similar:
%
\TwoEqns{}{f_{ns}(\text{max}) & = \mu_{ns}\,N\,,}
          {f_{sl}             & = \mu_{sl}\,N\,.}
%
}% /Sect
%
\Sect{4}{A Numerical Example}{\SectType{TextMultiPara}}{
%
\pcap{4}{a}{Applied Force \m{P} on Block on Inclined Plane}
As an example of friction consider a 10\unit{lb} block placed on a {37\degrees}
incline.
A force \vect{P} is applied to the block as shown in \Figref{8}.
For the block and surface assume: \m{\mu_{ns} = 0.30}, \m{\mu_{sl} = 0.20}.
The block is initially at rest and \vect{P} is 5.0\unit{lb}.
The value of \vect{P} is then increased in steps.
Determine the frictional force acting on the block for these values of \m{P}:
(a) 5.0\unit{lb}, (b) 6.0\unit{lb}, (c) 8.0\unit{lb}, (d) 10.0\unit{lb}, (e) a subsequent drop
to 7.6\unit{lb}.

The one-body diagram is also shown in \Figref{8} where the force of the surface
on the block has been decomposed into its frictional and normal components
and the other forces have been resolved into components parallel and
perpendicular to the incline.

\noindent For all cases, \m{F_{R,y} = m a_y = 0}:
%
\Eqn{}{N - W \cos 37\degrees = 0\,; \qquad N = 10\unit{lb}\,(0.8) =
                    8\unit{lb}\,.}
%
Then:
%
\Eqn{}{f_{ns}(\text{max}) = \mu_{ns}\,N = 2.4\unit{lb}\,.}

\pcap{4}{b}{For \m{P = 5\unit{lb}}}
\m{f_{ns}} will assume a value such that \m{F_{R,x} = 0}.
Then:
%
\Eqn{}{P - W \sin 37\degrees - f_{ns} = 0\,, \qquad
      f_{ns} = 5.0\unit{lb} - (10\unit{lb} \times 0.60) = - 1.0\unit{lb}\,.}
%
Since 1.0\unit{lb} is less than \m{f_{ns}(\text{max})}, the frictional force is 1.0\unit{lb}
and directed up the incline (as indicated by the negative sign).
That is, \vect{P} and \m{\vect{f}_{ns}} together balance the component of
weight that is along the incline.

\pcap{4}{c}{For \m{P = 6.0\unit{lb}}} \m{F_{R,x} = 0}.
Then:
%
\Eqn{}{P - W \sin 37\degrees - f_{ns} = 0\,; \qquad
        f_{ns} = 6.0\unit{lb} - 6.0\unit{lb} = 0\,.}
%
No frictional force is needed to maintain equilibrium so the frictional
force is zero!
The gravitational force component along the incline is just matched by the
applied force \vect{P}.

\pcap{4}{d}{For \m{P = 8.0\unit{lb}}}
\m{F_{R,x} = 0}.
Then:
%
\Eqn{}{P - W \sin 37\degrees - f_{ns} = 0\,; \qquad
      f_{ns} = 8.0\unit{lb} - 6.0\unit{lb} = 2.0\unit{lb}\,,}
%
and \m{\vect{f}_{ns}} is directed along the incline downward.
Now \vect{P} has become so large that \m{\vect{f}_{ns}} must be in the
opposite direction so as to prevent motion.

\pcap{4}{e}{For \m{P = 10.0\unit{lb}}}
%
\Eqn{}{P - W \sin 37\degrees - f_{ns} = 0;\;\;\;
f_{ns} = 10.0\unit{lb} - 6.0\unit{lb} = 4.0\unit{lb}\,,}
%
and \m{\vect{f}_{ns}} is directed down the incline downward.
However, since 4.0\unit{lb} exceeds \m{f_{ns}(\text{max}), f_{ns}} is unable
to assume the value 4.0\unit{lb} and \vect{P} is too much for gravity plus
non-sliding friction.
There is a net force and the block will start to accelerate up the incline.
Since the block is now sliding, the force of friction is that of sliding
friction:
%
\Eqn{}{f_{sl} = \mu_{sl}N = (0.20)\,(8\unit{lb}) = 1.6\unit{lb}\,.}
%
The net force resisting the acceleration is now 7.6\unit{lb}, so the resultant
force on the block is 2.4\unit{lb}.

\CaptionedFullFramedFigure{8}{A block on an inclined plane and its
one-body force diagram.}{m16gr08}

\pcap{4}{f}{A Subsequent Drop to \m{P = 7.6\unit{lb}}}
As a result of the 10.0\unit{lb} applied force, the block is accelerating
up the incline.
Now the value of \m{P} is dropped to 7.6\unit{lb} as stated in the problem.
That value just balances the force of sliding friction and the component of
weight down the incline, so the resultant force along the incline drops to
zero.
That means the acceleration drops to zero and the block subsequently moves
up the incline with constant velocity (which is zero acceleration).
}% /Sect
%

\newpage

\Sect{5}{People, Animals, Vehicles}{\SectType{TextMultiPara}}{
%
\pcap{5}{a}{Positive Acceleration}
\Index{acceleration}If you are standing still, you must start to accelerate in order to get up
to normal walking speed.
Your acceleration is in your direction of motion, the direction of
increasing velocity, so by Newton's second law another object must be
exerting a force on you in that direction.
Your acceleration is equal to the net forward force, exerted on you by other
objects, divided by your mass.

The earth is what exerts the \Quote{forward} force on you, the force in the
direction in which you start moving.
You initiate the process by exerting a \Emph{backward} force on the earth.
In accordance with Newton's third law, the earth exerts an equal but
oppositely directed \Emph{forward} force on you and that is the forward force
which accelerates you!

The same holds true for cars, animals, and bicycles (along with their
riders).
Take a car, for example: the engine and drive-train components cause the
tires to produce a backward force against the earth.
The earth exerts the equal-but-opposite reactive force on the car.
Of course that is the force \Emph{on the car} that accelerates the car in
accordance with Newton's second law.
The accelerating force is that of non-sliding friction since there is
generally little slipping.
Sliding friction comes into play only if, say, the operator of the vehicle
applies so much force to the vehicle-earth interface that the wheels begin
to slip and spin in place.

\pcap{5}{b}{Negative and Zero Acceleration}
Car and bicycle brakes cause deceleration through the frictional force
generated by one surface in the brake sliding over another with which it is
held in tight contact.
Thus the force causing deceleration while braking in such vehicles is that
of purely sliding friction.
Note, however, that if a vehicle is being kept stationary (not moving at all) by
brake force, then that force is a non-sliding one.

\pcap{5}{c}{Jets and Rockets}
No external force accelerates a jet or rocket operating in a vacuum.
The air in the upper atmosphere is sufficiently thin so as to make this
a reasonable approximation for jet planes operating there.
With zero external force, Newton's second law says that the craft (together with its
contents) must have zero acceleration and so it cannot speed up or slow down.
However, we know from experience that such craft do speed up and slow down
while in the upper atmosphere.
The apparent discrepancy disappears if we watch the entire contents of the plane, not just
the passengers and the plane's outer shell.
During acceleration the fuel inside the plane is being burned and is thereby caused to
exit rearward at an extremely high speed.
Thus part of the \Quote{craft plus contents} moves to a higher speed forward but another part
of it acquires a high speed in the opposite direction.
This is similar to the way one can throw beanbags rearward in order to gain forward speed
while on ice skates or on an air sled.
In cases like the jet plane, where there is no external force on the total object
so its total mass must travel at constant speed, a special construct called the object's \Quote{center
of mass}
%
\Footnote{4}{For a rigorous discussion of the \Quote{center of mass} concept,
see \Quote{Tools for Static Equilibrium,} MISN-0-5, or \Quote{Derivation of the Constants of the
Motion for Central Forces,} MISN-0-58.  For a discussion of rocket propulsion, see \Quote{Mass
Changing with Time: The Vertical Rocket, etc.,} MISN-0-19}
%
is helpful for the discussion.
For the present purpose, consider an object's center-of-mass to be a mathematical point
located at the \Quote{average} position of the object's mass components (which is usually not very
far from the object's geometrical center).
The point here is that the center of mass of the totality of parts travels at constant speed
regardless of accelerations and decelerations of individual parts.
In our present case, the center of mass of the plane-plus-fuel moves along at a constant speed
regardless of accelerations and decelerations of the plane alone.
When the plane moves to a higher speed the center of mass remains at the old speed because
fuel mass was shot rearward in the process of acceleration.
The force that produces the plane's acceleration is exerted on the plane by the burning
fuel in the firing chambers of the plane's engines.
This force is exerted only on the forward parts of the chambers because the backward parts
are open to the outside vacuum.
For a deceleration, the jet of burned fuel must be sent out in the forward direction.
}% /Sect
%
\Sect{}{Acknowledgments}{\SectType{Acknowledgments}}{\NsfAcknowledgment}% /Sect
%
\Sect{A}{Velocity Dependence of Sliding Friction}{\SectType{AppendixOnePara}}{
%
\xpcap{}{}{Only for Those Interested}
For a wide variety of surface combinations, the velocity dependence of
the coefficient of sliding friction\Index{kinetic friction| coefficient of}
\Index{coefficient of kinetic friction} is a very slowly rising logarithmic function
of velocity:
%
\Eqn{}{\mu_{sl}(v) = \mu_{sl}(v_0) + a\,\ln(v/v_0) \qquad \text{(for } v \geq v_0\text{)}\,,}
%
where \m{v_0}, \m{\mu_{sl}(v_0)}, and \m{a} are constants that depend only on the nature of the surfaces.
If there is sliding friction at a particular speed and then the speed is increased or decreased,
there is a very strong but short-lived upward or downward spike in the coefficient of sliding friction,
followed by the new value of \m{\mu_{sl}(v)}.
There are interesting graphs showing this effect in the preprint: \textit{Statistical
Mechanics of Static and Low-velocity Kinetic Friction}, by Martin H.\,M\"user, Michael Urbakh, and
Mark O.\,Robbins, Dept.\,of Physics and Astronomy, The Johns Hopkins University (4/30/2002).
}% /Sect
