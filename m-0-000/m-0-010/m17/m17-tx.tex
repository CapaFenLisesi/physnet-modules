\revhist{7/18/85, mpm; 12/29/87, pss; 2/18/91, pss; 9/8/94, pss; 11/16/94, pss; 11/6/96, pss;
         3/13/97, pss; 11/7/97, pss; 3/22/01, pss; 3/29/01, kag; 10/3/02, pss}
%
\Sect{1}{Introduction}{\SectType{TextOnePara}}{
%
Here are some questions of the types we can answer from a study of
acceleration and force in uniform circular motion:
Why are highway curves banked, and what characteristics of vehicles and
terrain determine the design angle?
What happens if a car or truck does not match the vehicle characteristics that
were assumed by the highway department in designing a particular curve?
Why do drivers of mopeds, bicycles, and motorcycles lean while going around a
corner?
How much should they lean?
What happens if they don't?

In the film, {\em 2001: A Space Odyssey}, the space station rotates so as to
simulate the force of gravity\Index{gravitational force, perception of} as we know it at the earth's surface (see
\Figref{4}).
The space station is in the shape of two large wheels connected by an axle.
How does it simulate the gravitational force and at what rate must it turn?
Is it necessary to vary the rate according to the weight of each space-person?

Suppose you tie a rock on the end of a long string and then whirl it around
your head.
What governs the angle of the string?

Airplane pilots talk about \Quote{\m{g}} forces.
What are they, and why are they important to pilots?
How are they measured and what are their (obviously important) physiological
effects?
}% /Sect
%
\Sect{2}{Acceleration and Force}{\SectType{TextMultiPara}}{
%
\pcap{2}{a}{The Circle of Motion}
When an object is traveling along a circular arc, we talk about its \Quote{circle
of motion,} whether the object is traveling completely around a circle or
only around part of a circle.
If the motion is circular only on an arc, we mentally extend the arc to make
a complete circle, and that is the \Quote{circle of motion.}

\CaptionedFullFramedFigure{1}{A car traverses a
quarter-circle turn, traveling at constant speed so its acceleration
\vect{a} is purely radial.
The car's speed is such that the road exerts only a normal force on the car.
The turn is banked at angle \m{\theta}, which is also the angle by which the
road force is off the vertical.
The resultant (net, total) force on the car is labeled \m{\vect{F}_r}.}%
{m17gr01}

\pcap{2}{b}{Uniform Circular Motion}
\Index{uniform circular motion}\Index{circular motion| uniform circular motion}For any object in uniform circular motion, its acceleration is radially
inward, pointing precisely toward the center of the circle of motion, as in
the example in \Figref{1}.
The magnitude of the object's acceleration\Index{acceleration| centripetal}\Index{centripetal acceleration} is \m{a=v^2/r}, where \m{v} is the
object's speed and \m{r} is its distance from the center of the circle of
motion.
%
\Footnote{1}{See \Quote{Kinematics: Circular Motion} (MISN-0-9).}
%
If you are given that an object has many forces on it, and that it is in
uniform circular motion, then you know that the object's acceleration\Index{acceleration| in circular motion} is
toward the center of the circle.
By Newton's second law, this means the \textit{resultant} force on the object
must also be toward the center of the circle of motion.
This is illustrated in \Figref{1}.

\pcap{2}{c}{Example 1: A Car on a Turn}
Our first example of uniform circular motion\Index{uniform circular motion} is a car that is traveling at
constant speed around a highway curve (see \Figref{1}).
Geometrically, the curve is a circular arc which we can mentally extend to
make a complete imaginary circle.
While the car is on the curve, it is maintaining a constant radius from
the center of that imaginary circle.
If we were to draw the car's path on an aerial photo, using a drafting
compass, one leg of the compass would be at the center of that circle and
the other on the car's path (see \Figref{1}).
Because the car is traveling at constant speed, the acceleration \vect{a} is
exactly toward the center of the circle of motion (see \Figref{1}).
Since \m{\vect{F}_r = m \vect{a}}, the resultant force on the car must also be
toward the center of the circle of motion (see \Figref{1}).
We have drawn the force of the road on the car as being normal to the road:
this will be the case if the car is traveling at just the right speed
(more on this later).

\CaptionedFullFramedFigure{2}{A string-restrained rock is
whirled about the person's body axis at a constant speed.
The force of the string on the rock is off the vertical by the angle
\m{\theta}.}{m17gr02}

\tryit A 2000\unit{lb} car is traveling at 50\unit{mph} around the curve
illustrated in \Figref{1}, with \m{r=390\unit{ft}}.
Show that the acceleration of the car, traveling at a constant 50\unit{mph},
is \m{a=0.43\,g}.
That is, the acceleration is 43\% of the usual acceleration of gravity for
objects in free fall. \help{7}

\pcap{2}{d}{Example 2: Rock on a String}

Our second example is the case of a rock being whirled around the body at
the end of a string as in \Figref{2}.
The string will only produce a force along its physical direction, as shown
in the figure.
We assume the rock is being made to travel at constant speed so its
acceleration is \m{a=v^2/r} and is exactly toward the center of the circle of
motion.
By \m{\vect{F}_r = m \vect{a}}, the resultant force on the rock must also be
toward the center of the circle of motion (see \Figref{2}).

\pcap{2}{e}{Example 3: A Bicyclist Rounding a Corner}
Our third example is the case of a bicyclist tilting sideways while rounding
a corner, as in \Figref{3}.
We could equally well have used a person rounding a corner while doing any
of such diverse things as running, riding a moped or a motorcycle, or
skiing.
The point is that the person does not want whatever is supporting the body
to exert a sideways force, one which would tend to throw the person off the
support sideways.
Thus in \Figref{3} we have assumed the person tilts to an angle such that
the support force on the body has no sideways component.
We also assume the forward speed is constant so the acceleration is exactly
toward the center of the circle of motion.
Then, by \m{\vect{F}_r = m \vect{a}}, the resultant force on the person must also
be exactly toward the center of the circle of motion (see \Figref{3}).

\CaptionedFullFramedFigure{3}{A bicyclist tilts sideways rounding a
corner.
The force of the bicycle on the bicyclist is off the vertical by
the angle \m{\theta}.}{m17gr03}

\enlargethispage{1pc}
}% /Sect
%
\Sect{3}{The Proper Highway Banking Angle}{\SectType{TextMultiPara}}{
%
\pcap{3}{a}{A Sideways Force is Undesirable}
\Index{banking of curves}If the roadway on a highway is properly banked, as in \Figref{1}, the
\Emph{roadway} will exert no sideways force on the car as long as the car
maintains the proper speed.
This is highly desirable since a sideways force can cause the car to start
to accelerate (slide) sideways.
This will happen if the sideways force exceeds the maximum sustainable force
of non-sliding friction.
Of course the maximum sustainable non-sliding frictional force depends
drastically on the condition of the tires and the surface of the road.
The maximum sustainable frictional force can be close to zero in a snowstorm
or ice storm, or even in a sudden deluge from a summer rainstorm.

\pcap{3}{b}{Deducing the Angle}
\Index{banking angle, ideal}Our condition for the proper banking angle of a highway turn is that the
road should only exert a normal force on the car, as in \Figref{1}.
Our problem is to deduce that proper banking angle, indicated by the symbol
\m{\theta} in \Figref{1}.
Data available for the calculation include the radius of the turn and the mass
and speed of the vehicle.

\noindent Here are the five steps:\vspace*{-8pt}
\begin{itemize}
\item [1.] Combine the radius and speed to get the magnitude of the
car's acceleration.
The direction of the acceleration is known so now we know \vect{a}.
\item [2.] Use \m{\vect{F}_r = m \vect{a}} to get \m{\vect{F}_r}.
\item [3.] Require that \m{\vect{F}_r} be the result of adding the known
gravity force to a \textit{normal} road force, as shown in the one-body
diagram in \Figref{1}.
The magnitude of the road force is made a symbol since, at this point,
it is unknown.
\item [4.] Write the force equation in terms of horizontal and vertical
components.
This gives two equations in two unknowns (the magnitude of the road
force and the value of \m{\theta} used to break the road force into
components).
\item [5.] Solve for one or both of the unknowns, as desired.
\end{itemize}

\pcap{3}{c}{The Deduced Angle}
\Index{ideal banking angle}The result of carrying out the steps outlined above gives for the proper
banking angle:
%
\Eqn{1}{\theta = \tan^{-1}\left(\dfrac{v^2}{r g}\right)\,,}
%
where \m{v} is the speed of the vehicle, \m{r} is the radius of the turn, and
\m{g} is the acceleration of gravity\Index{acceleration| due to gravity} in free fall (9.8\unit{m/s\up{2}}).
Incidentally, the normal force \m{N} of the road on the car is:
%
\Eqn{2}{N = \dfrac{mg}{\cos\theta} = \dfrac{\text{weight}}{\cos\theta}\,.}
%

\tryit Carry out the five steps listed above and deduce \Eqnsref{1}
and \Eqnssref{2}.
\help{8}

\tryit For the example at the end of Sect.\,2c, show that the proper banking
angle is {23\degrees}. \help{12}

\pcap{3}{d}{Analysis of the Results}
First, note that the ideal banking angle,\Index{ideal banking angle} \m{\theta}, is independent of the
vehicle's mass!
Therefore the same banking angle can serve all vehicles provided they move
with the design speed:
%
\Eqn{}{v = \sqrt{r g \tan\theta}; \qquad \help{9}}
%
For somewhat larger or smaller speeds, the frictional force between tires
and road will keep the vehicle on the curve if tire tread and road surface
permit.

\tryit Suppose the road has become icy during a blinding snowstorm and you
head into a short-radius curve, steeply banked for 55\unit{mph}, at 25\unit{mph} (as I
once did).
What do you think will happen?
Draw the one-body diagram!
}% /Sect
%
\Sect{4}{Other Examples}{\SectType{TextMultiPara}}{
%
\pcap{4}{a}{Whirling Rock on a String}
For the whirling rock example of Sect.\,2, we follow the procedure used in
Sec.\,3 and deduce the angle of the string.
The result is:
%
\Eqn{3}{\theta = \tan^{-1}\left(\dfrac{v^2}{r g}\right)\,,}
%
where \m{v} is the speed of the rock, \m{r} is the radius of the circle, and
\m{g} is the acceleration of gravity.\Index{acceleration| due to gravity}
    
\tryit Follow the referenced steps and deduce \Eqnref{3}. \help{10}

\pcap{4}{b}{Bicyclist on a Turn}
To deduce the angle of lean, in the turning bicyclist of Sect.\,2,
we follow the procedure used above and in Sec.\,3 to find:
%
\Eqn{4}{\theta = \tan^{-1}\left(\dfrac{v^2}{r g}\right)\,,}
%
where \m{v} is the speed of the bike (and rider), \m{r} is the radius of the
turn-circle, and \m{g} is as usual.

\tryit Follow the referenced steps and deduce \Eqnref{4}. \help{11}

\pcap{4}{c}{\Quote{Weight} on a Banked Turn}
When a vehicle travels a banked turn, the driver's weight seems to
increase.
That is, the driver has the same feeling that would be experienced if the
force of gravity were suddenly increased.
Similarly, the vehicle's tires flatten more against the pavement, as though
the car was also experiencing increased gravity.\Index{gravitational force, perception of}
The driver's physical sensations are an increase of the force of the car
seat on the driver's posterior and similar sensations in the driver's
internal organs.
This apparent increase in weight, due to the increased normal force, is
given by \Eqnref{2}.

\tryit In the example at the end of Sect.\,2c, show that the effect is as
though the force of gravity had increased by about 9\% while traversing the
turn.

\pcap{4}{d}{Circular Motion and Weightlessness}
A pilot in a plane can produce a temporary feeling of \Quote{weightlessness} by
aiming the plane slightly upward and then making a tight turn downward.
The plane thus moves in a circular arc in a vertical plane.
At the peak of the plane's path, where it is instantaneously parallel to the
earth's surface, the resultant force is in the same direction as the force of
gravity.
If, at that point, the speed and radius are such that \m{a = v^2/(r) = g},
then the resultant force exactly equals the force of gravity.
Of course the resultant force on the pilot is the sum of the (downward)
force of gravity and the (upward) force of the plane's seat on the pilot's
posterior.
In our present case the resultant force equals the force of gravity alone,
so the force of the seat on the pilot must be zero and this is what accounts
for the feeling of \Quote{weightlessness.}

\pcap{4}{e}{Acceleration in \m{g}'s}
People who routinely engage in tight turns, such as fighter pilots, measure
the accelerations they experience in \m{g}'s, sometimes spelled \Quote{gees,}
which simply means they divide the acceleration in ordinary units by the
quantity \m{g}, expressed in the same units.
For example, consider the \Quote{weightless} turn described above.
If the pilot continued in the same circle at the same speed, the point at
the bottom of the circle would be a \Quote{one gee} turn. \help{13}
At this instant the pilot's weight would seem to have doubled.
Similarly, a two gee turn causes an apparent tripling of body weight.
At four to five gees, insufficient blood reaches the brain and the sitting
pilot \Quote{blacks out.}
%
\Footnote{2}{Pilots can sustain up to 10\unit{gees} by staying there no more than
seconds before returning to 2-3\unit{gees} for a short recovery period, and up to
15\unit{gees} in a reclining (rather than sitting) position.}
}% /Sect
%
\Sect{5}{Force-Words for Circular Motion}{\SectType{TextMultiPara}}{
%
\pcap{5}{a}{Centripetal Force}
The radial acceleration acting in uniform circular motion, \m{a = v^2/r}, is
frequently referred to as the \Quote{centripetal} acceleration.
The \textit{resultant} force that produces this acceleration is called the
\Quote{centripetal} force,\Index{centripetal force}\Index{force| centripetal} but note that this (resultant) force is almost always
the sum of forces produced by various agents.\Index{centripetal force, perception of}

\pcap{5}{b}{Centrifugal Force}
Newton's third law says that for every force there is an \Quote{equal but
opposite} force, and the force \Quote{equal but opposite} to the centripetal
force is called the \Quote{centrifugal} force.\Index{centrifugal force}\Index{force| centrifugal}
In the example of the airplane pilot at the bottom of a vertical circular
turn, the (upward) centripetal force is the force of the seat on the pilot.
The centrifugal force is the force of the pilot on the seat.
Similarly, for a person in a horizontal centrifuge-type ride in an amusement
park, the centripetal force is the radially inward force of the seat on the
person, producing the observed acceleration, while the centrifugal force
is the equal but opposite force of the person on the outside edge of the
seat (if the seat is not strong enough to withstand the centrifugal force,
there will be an accident).

In the case of a bicyclist making a sharp turn, the resultant force is
exerted on the bicycle and rider by the road surface as a force of
non-sliding friction.
The centrifugal force is exerted by the bicycle on the ground surface.
If the bicycle hits a loose pebble or a slippery spot, the coefficient of
friction may suddenly drop to zero with disastrous results.

In a common classroom demonstration a student sits on a rotating lab
stool, holding a weight in an outstretched arm.
The student's hand exerts the centripetal force on the weight, producing
its acceleration.
The weight exerts the reactive centrifugal force on the student's hand,
and of course this is what the student feels.

\pcap{5}{c}{The Rotating Space Station}
Gravity can be simulated in a space station by causing it to rotate at the
right speed.
Consider the rotating space station shown in \Figref{4}.
The two persons shown in the sketch are upside-down with respect
to each other, yet each exerts a centrifugal force on the floor underfoot.
If the speed of rotation \m{v} is related to \m{g} and the radius \m{r} by
\m{a = v^2 /r = g}, and if each person stands on a scale, the centrifugal
force each exerts on the local scale will be exactly equal to the person's
own weight.
%
\CaptionedFullFramedFigure{4}{Cross-sectional sketch of a possible space
station.
Rotation causes simulation of gravity for each person along the outer rim.}{m17gr04}

\pcap{5}{d}{Coriolis Force}
\Index{Coriolis force}\Index{force| Coriolis}Curving motion produces yet another force that can sometimes be important.
Called the \Quote{Coriolis} force, it arises when one observes trajectories
from a rotating frame of reference (usually the surface of the earth).
%
\Footnote{3}{See \Quote{Classical Mechanics in Rotating Frames of Reference: Effects
on the Surface of the Earth} (MISN-0-18).}
%
This is the force that causes the plane of the Foucault pendulum\Index{Foucault pendulum}\Index{pendulum, Foucault}
%
\Footnote{4}{This is the pendulum commonly seen in science museums, usually
several stories tall, consisting of a long wire and a steel ball.}
%
to appear to rotate, both demonstrating the rotation of the earth and
indicating the latitude of the pendulum.
The Coriolis force also causes the regularity in the direction of rotation
of the common extratropical cyclones seen on weather satellite photographs,
and it is involved in the prevailing directions of the major winds and ocean
currents.
It can be demonstrated on a rotating lab stool.
}% /Sect
%
\Sect{}{Acknowledgments}{\SectType{Acknowledgments}}{\NsfAcknowledgment}% /Sect
%
\Sect{}{Glossary}{\SectType{Glossary}}{
\GlossaryItem{banking angle} the angle at which the roadbed of a highway
curve is tipped from the horizontal.\Index{banking angle}

\GlossaryItem{centrifugal force} the reactive force to the centripetal force, to which it is
equal but opposite.\Index{centrifugal force}

\GlossaryItem{centripetal acceleration} the radially-inwardly component
of acceleration for an object in circular motion.\Index{acceleration| centripetal}

\GlossaryItem{centripetal force} the radially-inward component of the
resultant force on an object in circular motion.\Index{centripetal force}

\GlossaryItem{ideal banking angle} the banking angle such that, for an
object at a particular speed undergoing circular motion, there is no
\Quote{sideways} force on the object.\Index{ideal banking angle}
For a highway turn, this means that the road surface exerts only a normal
(perpendicular) force on vehicles traveling at the design speed.
The resultant of the (normal) road force and the force of gravity is the
centripetal force that causes the vehicle's velocity vector to constantly
change direction as the vehicle travels through the turn.
}% /Sect

