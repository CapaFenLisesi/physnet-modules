\revhist{9/3/86 mpm; 4/1/91, pss; 2/3/94, pss; 10/3/94, pss; 9/27/95, pss;
         8/25/96, pss; 11/13/97, pss; 2/22/99, pss}
%
\defModTitle{\ph{Rotating Frames of Reference;} \ph{Effects on the} \ph{Surface of the Earth}}
\defCtAuthor{Peter Signell, Michigan State University}
\defIdAuthor{Peter Signell, Dept.\,of Physics, Mich.\,State Univ., East Lansing,
MI}
%
\defLG{True}
\defIdItems{
    \IdVersEval{2/1/2000}{0}
    \IdHours{1}
    \begin{InputSkills}
    \item [1.]  Determine the direction and magnitude of the vector (cross) product of
    two given vectors \prrqone{0-2}.
    \item [2.]  State the vector relationships between velocity, angular velocity,
    position, and force for an object in uniform circular motion \prrqone{0-17}.
    \item [3.]  Transform velocities between different Galilean frames of reference and
    use double-subscript object-observer notation \prrqone{0-11}.
    \end{InputSkills}
    %
    \begin{KnowledgeSkills}
    \item [K1.] Derive the Coriolis and centrifugal accelerations, starting with the
    expression which relates a vector's time derivative in a rotating reference
    frame to the time derivative in a non-rotating reference frame.
    \item [K2.] Derive the rotational directions of winds near high and low pressure
    centers in the northern and southern hemispheres.
    \item [K3.] Explain the alternating prevailing east-west wind directions found as one
    moves in a north-south direction on the face of the earth.
    \item [K4.] Explain (qualitatively) the reason for the apparent rotation of the plane
    of a Foucault pendulum, and explain how observations of it prove that the
    earth rotates.
    \item [K5.] Demonstrate the Coriolis force using a rotating stool and show that, in
    all cases, the direction of the force is as given by the algebraic formula.
    \end{KnowledgeSkills}
    %
    \begin{PostOptions}
    \item [1.]  \Quote{Relativistic Gravitation I; The Equivalence Principle} (MISN-0-110).
    \end{PostOptions}
}