\revhist{9/3/86 mpm; 4/17/91, pss; 10/3/94, pss; 9/27/95, pss; 11/13/97, pss}
%
\Sect{1}{Observing Time-Dependent Vectors}{\SectType{TextMultiPara}}{
%
\CaptionedLeftFramedFigure{1}{The symbol \m{R} labels a rotating frame,
\m{NR} a non-rotating one (see text).}{m18gr01}

\pcap{1}{a}{Rotating, Non-Rotating Observers}
We are going to compare observations of a vector \vect{r}'s change with
time, as observed by a rotating (R) and a non-rotating (NR) observer.
Imagine a large horizontal turntable rotating with angular velocity
%
\Footnote{1}{The vector angular velocity of a rigidly rotating object is
defined by the direction that right-hand threads along the axis would move
the object or, mathematically, by: \m{\vect{\omega}=\vect{v}\times\vect{r}/r^2},
where \vect{v} and \vect{r} are for any point on the rigidly-rotating
object (for example, the turntable in \Figref{1}).}
%
\vect{\omega}, as shown in \Figref{1}, with a radius vector \m{\vect{r}_o} from
the center of the turntable to an object (labeled \Quote{o}) traveling with
velocity \m{\vect{v}_o}.
We will suppose that this velocity is being measured by an observer R who is
stationary in the turntable's rotating frame, and also by an observer NR who
is stationary in the non-rotating frame.
What is the relation between the two velocity vectors of the object,
the \m{\vect{v}_\text{o,R}} seen by the rotating observer and the
\m{\vect{v}_\text{o,NR}} seen by the non-rotating observer?

\pcap{1}{b}{Observing Velocities}
First, recall that the velocities of an object seen by two observers
are related by:
%
\Footnote{2}{See \Quote{Relative Linear Motion, Frames of Reference}
(MISN-0-11).}
%
\Eqn{}{\vect{v}_\text{o,A} = \vect{v}_\text{o,B} = \vect{v}_\text{B,A}\,,}
%
where \m{\vect{v}_\text{o,A}} is the velocity of the object as seen by observer
\m{A}, \m{\vect{v}_\text{o,B}} is the velocity of the object as seen by observer
B, and \m{\vect{v}_\text{B,A}} is the velocity of B as seen by A.
For our case this becomes:
%
\Eqn{}{\vect{v}_\text{o,NR} = \vect{v}_\text{o,R} + \vect{v}_\text{R,NR}\,.}

The velocity of the rotating observer's frame, \m{\vect{v}_\text{R,NR}}\,, must
be measured at the position of the object and is therefore given by
%
\Footnote{3}{See \Quote{Kinematics: Circular Motion} (MISN-0-9).}
%
\Eqn{1}{\vect{v}_\text{R,NR} = \vect{\omega}\times\vect{r}_\text{o}\,,}
%
hence the velocities of the object measured by the two observers are related
by:
%
\Eqn{2}{\vect{v}_\text{o,NR} =
                 \vect{v}_\text{o,R} + \vect{\omega}\times\vect{r}_\text{o}\,.}

\pcap{1}{c}{Observing Arbitrary Vectors}
We can use \Eqnref{2}, just derived, to find the differing time
rates-of-change seen by the two observers for arbitrary vectors.
To do this we first note that \vect{v} is equal to
\m{d\vect{r}_\text{o}/dt}, and then make use of the theorem that any general
relation valid for one vector is valid for all vectors.
%
\Footnote{4}{See \Quote{Mathematical Skills: Addition, Subtraction, and Products of
Vectors} (MISN-0-2).}
%
Substituting \m{d\vect{r}_\text{o}/dt} into \Eqnref{2}, we get:
%
\Eqn{}{\left(\dfrac{d\vect{r}_\text{o}}{dt}\right)_\text{NR} =
              \left(\dfrac{d\vect{r}_\text{o}}{dt}\right)_\text{R} +
                                          \vect{\omega}\times\vect{r}.}
%
Then, making use of the general theorem quoted above, we can replace the
radius vector to the object by an arbitrary vector which we will call
\vect{G}:
%
\Eqn{3}{\left(\dfrac{d\vect{G}_\text{o}}{dt}\right)_\text{NR} =
              \left(\dfrac{d\vect{G}_\text{o}}{dt}\right)_\text{R}
                                      + \vect{\omega}\times\vect{G}.}
%
Any vector can be substituted for \vect{G} in this equation.
}% /Sect
%
\Sect{2}{Coriolis and Centrifugal Effects}{\SectType{TextMultiPara}}{
%
\pcap{2}{a}{Acceleration}
In order to modify Newton's Second Law for use in a rotating frame of
reference, we will first determine the acceleration in that frame relative
to its value in the non-rotating frame where Newton's laws are valid.
That is easily accomplished by substituting the velocity of an object,
as seen from the non-rotating frame, into \Eqnref{3}:
%
\Eqn{}{\left(\dfrac{d\vect{v}_\text{o,NR}}{dt}\right)_\text{NR} =
            \left(\dfrac{d\vect{v}_\text{o,NR}}{dt}\right)_\text{R}
                              + \vect{\omega}\times\vect{v}_\text{o,NR}.}
%
Then substitute for \m{\vect{v}_\text{o,NR}}\,, \Eqnref{2}, on the right
side to obtain the relation between the observed accelerations:
%
\Eqn{}{\vect{a}_\text{o,NR} = \vect{a}_\text{o,R} +
                2\vect{\omega}\times\vect{v}_\text{o,R} +
        \vect{\omega}\times\left(\vect{\omega}\times\vect{r}_o\right)\,.}
%
We can now solve for the acceleration of the object as seen by the rotating
observer:
%
\Eqn{4}{\vect{a}_\text{o,R} = \vect{a}_\text{o,NR} -
                  2\vect{\omega}\times\vect{v}_\text{o,R} -
     \vect{\omega}\times\left(\vect{\omega}\times\vect{r}_\text{o}\right).}
%
It is interesting to evaluate this equation for \m{\vect{a}_\text{o,R}} when
\m{\vect{a}_\text{o,NR}} is zero, when \vect{\omega} is zero, when
\m{\vect{v}_\text{o,R}} is zero, when \m{\vect{v}_\text{o,R}} is parallel to
\vect{\omega}, etc., in order to make sure that they agree with what
one would expect.

The two \Quote{correction} terms in \Eqnref{4} are called the Coriolis and
Centrifugal terms, respectively.
Note that they both disappear as the angular velocity of the rotating
observer goes to zero.
If the object is motionless in the rotating frame then the Coriolis term
goes away.
If the object is crossing the axis of rotation the centrifugal term is zero.

\pcap{2}{b}{The Forces}
The apparent forces acting on an object, as seen from a rotating frame, can
be determined by simply multiplying \Eqnref{4} by the object's mass
\m{m_o} and applying Newton's Second Law in the non-rotating frame where it is
valid:
%
\Eqn{5}{\vect{F}_\text{o,R} = \vect{F}_{true} + \vect{F}_\text{Coriolis} +
                                            \vect{F}_\text{centrifugal}\,,}
where
%
\Eqn{6}{\vect{F}_\text{Coriolis} = -2m \vect{\omega}\times\vect{v}_\text{o,R}\,,}
%
\Eqn{7}{\vect{F}_\text{centrifugal} =
     -m \vect{\omega}\times\left(\vect{\omega}\times\vect{r}_\text{o}\right).}
%
The latter two forces are fictitious: they only seem to be there because one
is observing from a rotating frame and thinking of it as stationary.

The fictitious Coriolis and centrifugal forces are often perceived by an
observer in a rotating frame as forces which must be \Quote{canceled out} by
real restraining forces in order to make an object follow a desired path.
In reality it is only the restraining forces that are real: they produce the
required non-rotating-frame accelerations necessary to keep the object on
its prescribed path.
%
\Footnote{5}{See \Quote{Relativistic Gravitation I: The Equivalence Principle}
(MISN-0-110) for a discussion of the equivalence of gravity and
acceleration; necessary for attempts to answer the question: How does one
explain the observed Coriolis force if one believes that one can equally
well describe the earth as non-rotating and the rest of the universe as
rotating about it?}
}% /Sect
%
\Sect{3}{The Foucault Pendulum}{\SectType{TextMultiPara}}{
%
\pcap{3}{a}{Introduction}
One almost always finds a Foucault
%
\Footnote{6}{pronounced \Quote{foo c\={o}\m{\prime}} in English.}
%
pendulum hanging in a science museum, and frequently the pendulum is several
stories high.
The principal aim of showing one remains as it was with Foucault; namely, to
demonstrate that the earth rotates on its axis with a period of one day.
This is meant to be in contrast to a picture in which the earth is taken as
non-rotating and the sun is taken as following a daily circular traversal of
the earth.

The suspension system of a Foucault pendulum is constructed in such a way as
to allow it to swing as freely as possible.
As a first approximation, we can consider it to be allowed to swing freely
in any plane, even a rotating one.

\pcap{3}{b}{A Function of Latitude}
Think of transporting the pendulum to the North pole of the Earth, where it
is suspended vertically, as usual, and set to swinging back and forth over
the pole.
As seen from the inertial (non-rotating) frame of reference, the pendulum's
swings lie in a non-rotating plane.
As seen by an observer at rest on the rotating earth, the plane of swinging
would appear to rotate with an angular velocity which is just the negative
of the angular velocity of the earth:
%
\Eqn{}{\vect{\omega}_p = -\vect{\omega}_e\,,}
%
where \m{\vect{\omega}_p} is called the (apparent) precessional angular
velocity of the pendulum.

Now think of moving the same pendulum to the equator.
There it would be forced to go around with the earth while swinging, and the
precession would be absent.

It is not hard to convince oneself that the precessional rate is determined
by the component of the earth's angular velocity along the pendulum's
\Quote{vertical,} and earth radial:
%
\Eqn{8}{\omega_p = \omega_e \sin{\theta_e}\,,}
%

where \m{\theta_e} is the latitude of the suspension point ({0\degrees} at the
equator, {90\degrees} at the north pole).
\Equationref{8} can be derived exactly from the requirements:
%
\begin{itemize}
\item [(i)] \m{\omega_p = \omega_e}  at the north pole;
\item [(ii)] \m{\omega_p = 0}   at the equator;
\item [(iii)] the Coriolis force, which is causing the precession, depends
on the sine of the angle between \vect{\omega} and \vect{v}, and this
can be directly shown to be equal to the angle of latitude, \m{\theta_e}.
\end{itemize}
%
It is quite apparent that the precessional rate of a Foucault pendulum can
be combined with the latitude angle of its position to give the absolute
rotational rate of the earth.
}% /Sect
%
\Sect{4}{Air Currents near Pressure Centers}{\SectType{TextOnePara}}{
%
\CaptionedFullFramedFigure{2}{Appearance of the air currents near a low
pressure center in the northern hemisphere, as seen from a weather
satellite.}{m18gr02}

The spiral appearance of cyclonic cloud formations seen in weather satellite
photographs is due to the Coriolis force: it constitutes dramatic evidence
of the rotation of the earth.
The common spiral satellite photographs shown in texts and on TV weather
reports are of low-pressure centers.
The air currents in the vicinity of such a center move over the face of the
earth toward it and then rise from the surface and thereby produce the
lowered pressure.
From traveling over the surface these currents are warm and moisture-laden,
bringing storms and being made visible by their clouds.
These routinely-seen patterns are called extratropical cyclones.
on the other hand, high pressure centers are produced by air descending from
the upper atmosphere and then spreading out away from the center.
Such air is typically cold and clear and does not show up in satellite
photographs.

As air moves toward or away from a pressure center, the Coriolis force
produces a sideways deflection: the result is a spiral trajectory as in
\Figref{2}.
It is obviously quite easy to obtain the direction of rotation of these
spirals in each hemisphere, for each type of pressure.
}% /Sect
%
\Sect{5}{Prevailing Wind Directions}{\SectType{TextOnePara}}{
%
As one proceeds along a meridian
%
\Footnote{7}{A meridian is a north-south line on a map.}
%
on the face of the earth, the prevailing wind directions alternate.
This fact was all-important in the days of square-rigged clipper ships that
could only sail in the direction the winds were blowing.
For example, they sailed the Atlantic from America to Europe following the
prevailing west wind along {30\degrees} north latitude and then returned using
the prevailing east wind along {45\degrees} north latitude.
These prevailing-wind patterns are due to the Coriolis force, as is the
direction of the prevailing wind in our vicinity.

The alternation of prevailing wind directions is due to a combination of
differential air heating and the Coriolis force.
Begin by imagining an air mass covering a portion of the earth's surface
in the northern hemisphere.
The part nearer the equator is heated more than the rest and therefore tends
to rise.
It is easy to see that the Coriolis force deflects it to the west, resulting
in rising \Quote{east winds.}
Continuing motion results in a deflection northward and then the cooled air
moves downward and is deflected easterly, resulting in descending \Quote{west
winds.}
The whole constitutes a large air cell.
The direction of circulation of such air cells in the southern hemisphere
can be easily determined and these patterns confirm the rotation of the
earth.
}% /Sect
%
\Sect{}{Acknowledgments}{\SectType{Acknowledgments}}{\NsfAcknowledgment}% /Sect

