\revhist{12/15/91, pss; 10/3/94, pss; 1/7/02, pss}

\Sect{}{}{\SectType{ProblemSet}}{

\noindent Note: Problems~12-13 are also on this module's \textit{Model Exam}.

\begin{two-digit-list}
\item [1.] A vertical rocket weighs 13\unit{tons}, of which 9.75\unit{tons} is fuel.
           Assume a constant fuel ejection velocity of 2\unit{mi/s} and assume that
           the effect of gravity is so small that it can be neglected.
           What is the speed of the rocket when the fuel has been exhausted?

\item [2.] A rocket whose weight is 3000\unit{tons}, when fueled up on the rocket
           pad, is fired vertically upward.
           At burnout 2780\unit{tons} remains.
           Gases are exhausted at a velocity of 165,000\unit{ft/s} relative to the
           rocket; both quantities are constant while the fuel is burning.
\begin{one-digit-list}
\item [a.] What is the thrust?
\item [b.] What is the speed of the rocket at burnout?
\end{one-digit-list}

\item [3.] If the rocket in Problem~2 is fired in deep space (no external
           forces), what is its speed at burnout?

\item [4.] The 13~ton rocket in Problem~1 is remade as a 2-stage rocket.
           The first stage weighs 12\unit{tons}, of which 9\unit{tons} is fuel.
           The second stage weighs the remaining 1\unit{ton}, of which 0.75\unit{ton} is
           fuel.
           The second stage is fired after the fuel has been exhausted in the
           first stage and it has been decoupled.
           Again neglecting effects due to gravity, find the final velocity of
           the second stage.
           Note that you must apply the rocket equation twice, in succession.
           Compare to the final velocity of the one-stage rocket.

\item [5.] A boy with a pea shooter is standing on roller skates on a
           horizontal frictionless surface.
           The mass of the system (boy, skates, pea shooter, and peas) at a
           particular instant is \m{M}.
           At the same instant the boy is shooting peas of mass mp each whose
           velocity relative to the earth is \m{v_p} and the velocity of the peas
           with respect to the boy is \m{v_r}.
           All of these velocities are colinear.
           The number of peas per unit time is \m{N}.
           What is the average thrust on the boy due to the ejected peas at
           that instant?

\item [6.] Does a simple scaling-up of the fuel mass and rocket shell mass
           provide greater thrust?
           Greater velocity?

\item [7.] Given an exhaust velocity of \m{V_{ES} = 10,860\unit{mph}} and a
           fuel/fuel-container-plus-engine ratio of 90%, determine the minimum
           amount o� fuel necessary to raise a payload of 1000\unit{lb} to the
           velocity necessary to escape earth's gravity (\m{V_{ES} = 25,000\unit{mph}}).

           Note:  \m{r \equiv \dfrac{M_F}{M_{FC} + M_E} = 0.9}

           where \m{M_{FC}} is the mass of the fuel container and
           \m{M_E} is the mass of the engine.
           \m{M_F} is the mass of the fuel.

\item [8.] Show that each of the three quantities on the right hand side of
           \Eqnref{4} occurs in a reasonable position.

\item [9.] Show that the right hand side of \Eqnref{7} agrees with what one
           would expect as \m{M_F \rightarrow 0} or \m{m_S \rightarrow \infty}.

\item [10.] Show that the right hand side of \Eqnref{6} agrees with what one
            would expect if \m{m_{RF} = m_{Ri}}.

\item [11.] Use the small-\m{x} approximations
            \m{(1 \pm x)^{-1} \approx (1 \mp x)} and
            \m{\;\ell n (1 \pm x) \approx \pm x} on \Eqnref{13} to show that the
            rocket's initial acceleration is:

            \m{a = \dfrac{v_E R}{m_{Ri}} - g}.

            Show that this agrees with what one would expect from Newton's
            second law and the thrust equation applied at time zero.

\item [12.] The earth escape velocity, 25,000\unit{mph}, is the upward velocity needed
           at the surface of the earth for eventual escape from the earth's
           gravitational pull.
           If the maximum fraction of a rocket's mass that can be devoted to
           fuel is 90\%, determine the minimum exhaust velocity necessary for
           the rocket to reach escape velocity.

\item [13.] Show that the equation developed in Problem l above agrees with what
           one would expect if \m{t} is set equal to the initial time, \m{t}.
\end{two-digit-list}

\newpage

\BriefAns

\begin{two-digit-list}
\item [1.] \m{2.77\unit{mi/s} \approx 9,970\unit{mph} = \text{mach } 13.4}

\item [2.] a. \m{F = 4.67 \times 10^8\unit{lb} = 2.34 \times 10^5\unit{tons}}
\begin{one-digit-list}
\item [b.] \m{v = 7,610\unit{ft/s} = 5,190\unit{mph} = \text{mach } 7.0}
\end{one-digit-list}

\item [3.] \m{v = 12,600\unit{ft/s} = 8,590\unit{mph} = \text{mach } 11.6}

\item [4.] \m{v = v_E \; \ell n 13 = 5.13 \unit{mi/s} = 18,500\unit{mph} =
           \text{mach } 25}; 85\% greater.

\item [5.] \m{F = N v_r m_p}

\item [6.] Yes; no.

\item [7.] \m{m_F = m_p \left[ (1 - e^{- v_{ES}/v_{EX}})^{-1} -r^{-1}
           \right]^{-1} = 14 \times 10^6\unit{lb}}.

\item [12.] 10,860\unit{mph}

\item [13.] \m{v_R(t_i) = v_R(t_i)}: check!
\end{two-digit-list}

}% /Sect
