\revhist{12/15/91, pss; 10/3/94, pss; 1/7/02, pss; 10/2/02, pss; 10/3/02, pss; 10/21/02, pss}
%
\Sect{1}{Introduction}{\SectType{TextOnePara}}{
%
Ordinary locomotion on earth relies heavily on friction - the interaction of
a vehicle, be it bicycle, car, ship, or aircraft, with its environment.
For a rocket outside the earth's atmosphere there is neither earth nor air
nor water to push against; the rocket must achieve its acceleration another
way.
It is via fuel mass expulsion that such propulsion is achieved and controlled.
Here we investigate the propulsion of the vertical rocket; that is, of a
rocket propelled vertically upward against a uniform gravitational field.
The commonplace use of rockets for launching communications satellites and
for interplanetary explorations makes this subject one of considerable
technological interest.

The vertical rocket is a special case of the more general class of problems
involving systems of variable mass, due here to the expulsion of fuel.
The solution of other variable mass problems proceeds in exactly the same
manner as for the vertical rocket, except that the external force exerted on
the system is not always that of gravity.
One example is that of a jet aircraft, where the external drag force comes
mainly from air resistance.
}% /Sect
%
\Sect{2}{The Gravity-Free Rocket Equation}{\SectType{TextMultiPara}}{
%
\pcap{2}{a}{Rocket Momentum Balances Exhaust Momentum}
We will deal first with a rocket in outer space where gravitational forces
are weak and can be neglected.
How can we explain the ability of this rocket to accelerate?
Our rocket can be considered to consist of two parts; the fuel and the
shell-payload.
Expansion of the fuel during burning results in its continuous expulsion
rearward in the'form of burned-out exhaust gases.
Since there is no external force on the total system, the system's momentum
is conserved.
%
\Footnote{1}{See \Quote{Momentum: Conservation and Transfer} (MISN-0-15).}
%
This implies that any rearward momentum acquired by the fuel as exhaust must
be balanced by a newly-acquired momentum of the rocket in the forward direction.
Furthermore, the center-of-mass of the entire system is left undisturbed as
the exhaust gases and rocket move away from it in opposite directions.

Another way of describing the rocket's acceleration is in terms of Newton's
third law.
%
\Footnote{2}{See \Quote{Particle Dynamics-The Laws of Motion} (MISN-0-14).}
%
The rearward acceleration of the exhaust implies a rearward force on it by
the rocket but this in turn implies the existence of an equal but opposite
forward force-exerted on the rocket by the exhaust gases.

\CaptionedLeftFramedFigure{1}{A rocket is shown: (a) at one instant; (b) at a
time \m{dt} later.}{m19gr01}

\pcap{2}{b}{Thrust as a Function of Exhaust Velocity, Burn Rate}
We can use conservation of momentum for the gravity-free (\m{g = 0}) rocket to
derive its acceleration as a function of its design parameters.
This derivation is easiest to make if we imagine ourselves as moving along
parallel to the rocket with a constant velocity which coincides with the
rocket's at a specified instant; then during the succeeding time increment,
we observe the momentum changes in the various parts of the system.
In \Figref{1}(a) we see the rocket with mass mR at the instant when it is
apparently at rest because we are moving along with it.

In \Figref{1}(b) we see it at a time \m{dt} later when it has expelled a mass of
exhaust gas \m{dm_E} rearward with velocity \m{\vect{v}_E} relative to the rocket.
The fuel burn rate \m{R} connects \m{dm_E} to the time increment \m{dt}:
%
\Eqn{1}{dm_E = R \,dt\,.}
%
\noindent We now use conservation of momentum for our isolated system to determine the
increment of velocity, \m{d\vect{v}_R}, acquired by the rocket:
%
\Footnote{3}{The momentum form of Newton's Second Law, \m{F = d(mv)/dt}, can be
used for rocket problems only with great care.
We advise against it.}
%
\Eqn{2}{0 = d\vect{p} =
             d\vect{p}_R + d\vect{p}_E = m_R d\vect{v}_R + \vect{v}_E dm_E\,,}
%
where we have (properly) neglected a product of infinitesimals.
There are several connections we can make between the quantities in
\Eqnref{2}:
%
\Eqn{}{dm_E = - dm_R}
%
\Eqn{3}{\vect{v}_E \equiv v_E \uvec{v}_E = - v_E \uvec{v}_R\,;}
%
\Eqn{}{d\vect{v}_R = dv_R \uvec{v}_R\,.}
%
Combining \Eqnsref{1}, \Eqnssref{2}, and \Eqnssref{3}, we find the
rocket's acceleration:
%
\Eqn{4}{a_R = \dfrac{v_E R}{m_R}\,,}
%
\noindent and hence the force on it produced by the exhaust gases:
%
\Eqn{5}{F_R = v_E R\,.}
%
This force is called the rocket engine's thrust and its linear dependence on
\m{v_E} and \m{R} checks with what one would expect intuitively.
Although we have derived \Eqnsref{4} and \Eqnssref{5} for
the case of an observer moving at a certain constant velocity, the same force
and acceleration values will be found by a stationary observer.
%
\Footnote{4}{This is because accelerations add vectorially and the acceleration
of one constant-velocity observer with respect to another (of differing
velocity) is zero.
For further details see \Quote{Relative Linear Motion, Frames of Reference}
(MISN-0-11).}
%
Thus \Eqnsref{4} and \Eqnssref{5} are quite general.

\pcap{2}{c}{Velocity as a Function of Rocket Parameters}
The equation relating a rocket's velocity to its fuel expenditure is
particularly simple and obviously has practical application.
To derive it we integrate the differential form of \Eqnref{4},
%
\Eqn{}{dv_R = - \dfrac{v_E}{m_R} dm_R\,,}
%
to obtain the change in velocity:
%
\Eqn{6}{\Delta v_R = v_E \; \ell n \left( \dfrac{m_{Ri}}{m_{Rf}} \right)\,.}
%
Here \m{m_{Ri}} is the initial rocket mass (at ignition) and \m{m_{Rf}}
is its final mass (at burn-out).

Assuming that the initial rocket mass consists of fuel mass \m{m_F} and
shell-payload mass \m{m_S}, and that the final rocket mass consists only of the
shell-payload, we obtain the velocity change in terms of the fuel expenditure:
%
\Eqn{7}{\Delta v_R = v_E \; \ell n \left( 1 + \dfrac{m_F}{m_S} \right)\,.}
%
All parts of this equation check with what one would expect intuitively.

\pcap{2}{d}{Velocity as a Function of Time}
\Equationref{6}, for the change in velocity as a function of mass,
can be written as an implicit function of time:
%
\Eqn{8}{v_R(t) =
      v_R(t_i) + v_E \; \ell n \left( \dfrac{m_R(t_i)}{m_R(t)} \right) ;\; t_i \le t \le t_f\,,}
%
\noindent where \m{t_i} is the ignition time and \m{t} is any succeeding time up to the
shut-down time \m{t_f}.
The time dependence can be made explicitly if we specify \m{m_R(t)}.
For example, a constant burn rate
%
\Eqn{9}{R = -dm_R/dt = \text{ positive constant}}
%
will produce (upon integration) a linear decrease of the rocket mass
with time:
\Eqn{}{m_R(t) = m_R(t_i) - (t - t_i) R; \; t_i \le t \le t_f\,,}
%
where \m{t_f} is easily found from:
%
\Eqn{}{\Delta t =
  \dfrac{\Delta m_R}{R} ; \text{ where } \Delta t \equiv t_f - t_i, \text{ etc.}}
%
We then obtain the gravity-free rocket equation:
\Eqn{10}{v_R(t) = v_R(t_i) + v_E \; \ell n \left[ 
    1 - \dfrac{R}{m_{Ri}} (t - t_i) \right]^{-1} ; \; t_i \le t \le t_f\,.}
%
This is plotted, for a particular set of rocket parameters, as a
solid line in \Figref{2}.
}% /Sect
%
\Sect{3}{Modifications for Gravity}{\SectType{TextOnePara}}{
%
When a rocket rises vertically from the surface of the earth, against
gravity, the additional force adds to the thrust and thus
\Eqnref{4} becomes:
%
\Eqn{11}{a_R = \dfrac{v_E R}{m_R} - g\,.}
%
\CaptionedFullFramedFigure{2}{Rocket velocity as a function of time for
gravity-free space (solid line) and for a rocket rising in a constant
gravitational field (dashed line).}{m19gr02}
%
Integrating and assuming \m{v_R(t_i) = 0}, \m{t_i = 0}, we obtain a
revised \Eqnref{7}:
%
\Eqn{12}{v_R(t_f) =
            v_E \; \ell n \left( 1 + \dfrac{m_F}{m_S} \right) - g t_f\,.}
%
Finally, \Eqnref{10} becomes:
%
\Eqn{13}{v_R(t) =
    v_E \; \ell n \left( \dfrac{1}{1 - \dfrac{R t}{m_{Ri}}} \right)
                              - g t; \; 0 \le t \le t_f\,.}
%
Note that if the burn rate is high, so that \m{t_f} is small, the gravity
term in \Eqnsref{12} and \Eqnssref{13} will be small.
That is typically the case.
\Equationref{13} is plotted as a dashed line in \Figref{2}.
}% /Sect
%
\Sect{}{Acknowledgments}{\SectType{Acknowledgments}}{
M.\,Hunter and Steve Smith provided valuable feedback on an earlier version.
\NsfAcknowledgment}% /Sect

