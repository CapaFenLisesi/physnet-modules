\revhist{8/14/91, pss; 3/10/93, pss; 9/7/94, pss; 5/21/99, pss}

\Sect{}{}{\SectType{ModelExam}}{

\begin{one-digit-list}
\item [1.] See Output Skills K1-K5 in this module's \textit{ID Sheet}.
One or more, or none, of these skills may be on the actual exam.

\item [2.] \CenteredUnframedFixedFigure{m20gr13}{A constant horizontal force \vect{F} pushes an
object of mass \m{M} up a frictionless incline which makes an angle \m{\theta}
with the horizontal.}
\begin{one-digit-list}
\item [a.] Starting with the object midway \textit{up} the incline, moving the
speed of \m{v_0} directed up along the incline, use the work-kinetic energy
relation to find the velocity of the object when it's at a point a distance
\m{D} further up the incline.
\item [b.] If at this same starting point as in part (a) the object started
with the speed \m{v_0} directed \textit{down} along the incline (same forces
acting as before), use the work-kinetic energy relation to find the velocity
of the object when it's at the point a distance \m{D} further up the incline.
\item [c.] Explain the relation between your answers to parts (a) and (b).
What is the difference in the overall motion between the two cases?
\end{one-digit-list}

\item [3.] The gravitational force on an object of mass \m{m} which is at or
above the surface of the earth, say at a total distance \m{r} from the center of
the earth, has a magnitude \m{K\,m/r^2} and is directed toward the center of the
earth.
Here \m{K} is a constant equal to \m{3.99 \times 10^{14}\unit{m\up{3}/s\up{2}}}.
Hence the force is not constant but diminishes the farther the object is
from the earth's center.
\begin{one-digit-list}
\item [a.] At the surface of the earth, recall from previous knowledge that
this force is \m{mg}.
With this knowledge of the force at the surface of the earth and with \m{K}
given above, determine the radius of the earth.
\item [b.] What is the minimum amount of work that you must do (energy you must
expend) in order to move an object of mass \m{m} radially outward from the
surface of the earth to a distance \m{r} from the center of the earth
(\m{r > R_e})?
(HINT: What's the \Emph{minimum} force you must exert?
That determines the \Emph{minimum} work).
\item [c.] From this result, calculate how much energy it takes to move a
person of mass 65\,\unit{kilograms} from the surface of the earth to an infinite
distance away (disregard gravity forces other than from the earth).
\end{one-digit-list}

\item [4.] Suppose a car engine is delivering \m{1.60 \times 10^2\Emph{hp}} at a
continuous rate, keeping the car at 25\unit{m/s} (56\unit{mph}).
Determine the force the engine is overcoming.
\end{one-digit-list}
\bigskip

\BriefAns

\begin{one-digit-list}
\item [1.] See this module's \textit{Text}.

\item [2.] See this module's \textit{Problem Supplement}, problem~21.

\item [3.] See this module's \textit{Problem Supplement}, problem~22.

\item [4.] See this module's \textit{Problem Supplement}, problem~23.
\end{one-digit-list}
}% /Sect
