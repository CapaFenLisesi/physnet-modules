\revhist{5/24/91, pss; 11/24/92, pss; 12/15/92, pss; 3/9/93, pss; 9/7/94, pss;
         3/29/95, pss; 9/15/95, pss; 11/8/99, pss; 12/15/00, pss; 4/1/02, pss}

\Sect{}{}{\SectType{ProblemSet}}{

\noindent Note: If you have trouble with a problem, see the activities in the
\textit{Special Assistance Supplement}.
If there is a \textit{help} reference, it also is in the \textit{Special Assistance
Supplement}.

\noindent In all problems involving the calculation of \textit{work}, you should
start with the line-integral definition and show how it can be applied and
simplified for the particular situation.

Take \m{g = 9.8\unit{m/s\up{2}}}, 1\,HP = 550\unit{ft\,lb/sec} = 746\unit{watts}.

Problems~20, 21, and 22 also occur in this module's \textit{Model Exam}.

\begin{two-digit-list}
\item [1.] A crate whose mass is 39\unit{kg} is dragged at constant speed across the
floor for 9.0\unit{meters} by applying a force of 100.0\unit{N} at an upward angle of
35 degrees to the horizontal.
\begin{one-digit-list}
\item [a.] What is the work done by the applied force \answer{5} \help{1}
\item [b.] What is the coefficient of friction between the crate and the
floor? \answer{14} \help{1}
\end{one-digit-list}

\item [2.] A constant force
\m{\vect{F} = (5.0\uvec{x} + 2.0\uvec{y} + 2.0\uvec{z})\unit{N}}
acts on an object during the displacement
\m{\Delta\vect{r} = (10.0\uvec{x} + 5.0\uvec{y} - 6.0\uvec{z})\unit{m}}.
Determine the net work done by this force during the displacement. \answer{6}
\help{2}

\item [3.] Determine the speed of a proton that has:
\m{ \text{kinetic energy } = 8.0 \times 10^{-13}\unit{J}} and
\m{ \text{mass }          = 1.67 \times 10^{-27}\unit{kg}}. \answer{13} \help{3}

\item [4.] What is the kinetic energy of a 5.0\unit{kg} object with a velocity:
\m{\vect{v} = (3.0\,\uvec{x} + 5.0\,\uvec{y})\unit{m/s}}? \answer{3} \help{4}

\item [5.] An escalator in a department store joins two floors that are 5.0\unit{meters} apart.
\begin{one-digit-list}
\item [a.] When a 51\unit{kg} woman rides up this escalator, how much work does
the motor do to lift her? \answer{18} \help{5}
\item [b.] How much power must the motor develop in order to carry 80\,\unit{people}
between floors in one minute, if the average person's mass is 70.2\unit{kg}? \answer{23}
\help{5}
\end{one-digit-list}

\item [6.] What horsepower must be developed by the engine of a racing car when
a forward thrust of \m{2.8\times10^3\unit{N}} moves it at a constant velocity of
67.5\unit{m/s} (150\unit{mph})? \answer{2} \help{6}

\item [7.] A car with a mass of \m{1.2 \times 10^3\unit{kg}} is pulled at constant
velocity up a sloping street, inclined at a 31 degree angle from the horizontal,
by a truck having a tow cable attached at an angle of 41\degrees to the street.
The coefficient of friction between the car and the street is 0.30.
What is the minimum work the tow truck would have to do on the car to move
it 204 meters along the sloping street? \answer{12} \help{7}

\item [8.] \ItemFigure{A particle of mass \m{m} hangs from a string of
length \m{\ell} as shown.
A variable horizontal force \vect{F} starts at zero and gradually increases,
pulling the particle up very slowly (equilibrium exists at all times)
until the string makes an angle \m{\phi} with the vertical.
Calculate the work done by \vect{F} to raise \m{m} to this position. \answer{1}
\help{8}}{m20gr11}

\item [9.] A particle is moving with a velocity \m{v_0} along the \m{x}-axis at
time \m{t_0}.
It is acted upon by a constant force in the \m{x}-direction until time \m{t_1}.
Show that:
%
\Eqn{}{\int_{t_0}^{t_1} (\vect{F} \cdot \vect{v})\,dt = \dfrac{\Delta (p^2)}{2m}\,}%
%
where \m{\Delta(p^2)} is the change in the square of its momentum. \help{9}

\item [10.] A field goal kicker hits the ball at an angle of {42\degrees}
with respect to the ground.
The football, of mass 0.40\unit{kg}, sails through the air and hits the cross bar
of the goal post \m{3.0\times10^1\unit{m}} away.
If the cross bar is 3.0\,\unit{meters} above the ground, determine the work done by
the gravitational force over the flight of the ball. \answer{11} \help{10}

\item [11.] \ItemFigure{A 2.0\unit{kg} particle, currently at the origin and
            having a velocity of \m{\vect{v} = 3.0\uvec{x}\unit{m/s}}, is acted upon by the force
            shown in the graph.
\begin{one-digit-list}
\item [a.] Determine the work done, both graphically and analytically, for a
           displacement from \m{x = 0.0} to \m{x = 5.0\unit{meters}}. \answer{16} \help{11}
\end{one-digit-list}
\begin{one-digit-list}
\item [b.] What is the velocity of the particle when it is at \m{x = 5.0\unit{m}}? \answer{24} \help{11}
\end{one-digit-list}}{m20gr12}

\item [12.] A car whose mass is \m{1.8\times10^3\unit{kg}} has a safety bumper
that can withstand a collision at 5.0\unit{mph} (2.25\unit{m/s}).
Suppose the average retarding force of the energy absorbing bumper mechanism
is \m{5.0 \times 10^4\unit{N}}.
How much will the bumper be displaced if the car is going at 5.0\unit{mph} when it
hits a tree? \answer{17} \help{13}

\item [13.] An object is thrown with a velocity of 61\unit{m/s} vertically
downward from a height of 202\,\unit{meters}.
What is its velocity when it hits the ground?
Work this problem two ways--by the work-kinetic energy relation and by
the laws of linear motion. \answer{9} \help{14}

\item [14.] A bullet pierces a 4.0\unit{cm} thick piece of metal armor plate with
a velocity of 708\unit{m/s} and leaves the other side with a velocity of 310\unit{m/s}.
Determine the thickness of the metal plate that would be required to stop
the bullet completely. \answer{21} \help{15}

\item [15.] A Mercedes-Benz 450SL is moving at 41\unit{m/s} (90\unit{mph}) when the brakes become locked.
How far will the car slide on:
\begin{one-digit-list}
\item [a.] dry pavement (\m{\mu = 0.90})? \answer{22} \help{16}
\item [b.] wet pavement (\m{\mu = 0.50})? \answer{27} \help{16}
\item [c.] icy pavement (\m{\mu = 0.10})? \answer{8} \help{16}
\end{one-digit-list}

\item [16.] A girl in an archery class finds that the force required to pull
the bowstring back is directly proportional to the distance pulled.
She finds that a force of 29\unit{N} is needed to pull the bow string back a
distance of 0.10\,\unit{meter}.
If she pulls a 55\unit{gram} arrow back a distance of 0.30 meters,
\begin{one-digit-list}
\item [a.] what would be the velocity of the arrow if 81\% of the work done
was converted into kinetic energy? \answer{20} \help{17}
\item [b.] If she shoots the arrow straight up, how high will it go (neglect
the effects of air resistance)? \answer{26} \help{17}
\item [c.] If the arrow actually only goes 91\% of the height in part (b),
what is the average resistive force of the air? \answer{29} \help{17}
\end{one-digit-list}

\item [17.] A particle of mass 4.0\unit{kg} is acted on by the force:
%
\Eqn{}{\vect{F} = \left[(x + 2y)\uvec{x} + (2y + 3xy)\uvec{y}\right]\unit{N}\,,}
%
as the particle moves along a straight line path from the point (0,0) to
the point (2.0\unit{m},2.0\unit{m}), that is along \m{x = y}.
\begin{one-digit-list}
\item [a.] Find the work done by the force. \answer{19} \help{18}
\item [b.] If the particle was at rest when it was at the coordinate origin,
what is its speed when it is at \m{x=2.0\unit{m}}, \m{y=2.0\unit{m}}? \answer{7} \help{18}
\end{one-digit-list}

\item [18.] An airplane lands on an aircraft carrier, and is halted by the
arresting cable.
The force equation for the cable is \m{\vect{F} = - k x^2\uvec{x}}, where
\m{k = 1.20\times10^2\unit{N/m\up{2}}}.
For an initial plane velocity of 81\unit{m/s} and a plane mass of
\m{2.25\times10^3\unit{kg}}, determine how far the plane will travel after being
hooked. \answer{15} \help{19}

\item [19.] What horsepower engine would be required if you wish to pull
a 54\unit{kg} woman water skier up from the water in 2.0\unit{s} with a final velocity
of 13\unit{m/s}?
The mass of the boat plus driver is 206\unit{kg} and that of the engine is
45.0\unit{kg}.
During this period the average drag energy of the skier and boat is
\m{5.0\times10^4\unit{J}}. \answer{6} \help{20}

\item [20.] \ItemFigure{A constant horizontal force \vect{F} pushes an
object of mass \m{M} up a frictionless incline which makes an angle \m{\theta}
with the horizontal.}{m20gr13}

\begin{one-digit-list}
\item [a.] Starting with the object midway \textit{up} the incline, moving the
speed of \m{v_0} directed up along the incline, use the work-kinetic energy
relation to find the velocity of the object when it's at a point a distance
\m{D} further up the incline. \answer{31}) \help{21}
\item [b.] If at this same starting point as in part (a) the object started
with the speed \m{v_0} directed \textit{down} along the incline (same forces
acting as before), use the work-kinetic energy relation to find the velocity
of the object when it's at the point a distance \m{D} further up the incline.
\answer{33} \help{21}
\item [c.] Explain the relation between your answers to parts (a) and (b).
What is the difference in the overall motion between the two cases? \answer{35}
\help{21}
\end{one-digit-list}

\item [21.] The gravitational force on an object of mass \m{m} which is at or
above the surface of the earth, say at a total distance \m{r} from the center of
the earth, has a magnitude \m{K\,m/r^2} and is directed toward the center of the
earth.
Here \m{K} is a constant equal to \m{3.99 \times 10^{14}\unit{m\up{3}/s\up{2}}}.
Hence the force is not constant but diminishes the farther the object is
from the earth's center.
\begin{one-digit-list}
\item [a.] At the surface of the earth, recall from previous knowledge that
this force is \m{mg}.
With this knowledge of the force at the surface of the earth and with \m{K}
given above, determine the radius of the earth. \answer{32} \help{22}
\item [b.] What is the minimum amount of work that you must do (energy you must
expend) in order to move an object of mass \m{m} radially outward from the
surface of the earth to a distance \m{r} from the center of the earth
(\m{r > R_e})? \help{22}
(HINT: What's the \textit{minimum} force you must exert?
That determines the \textit{minimum} work). \answer{30} \help{22}
\item [c.] From this result, calculate how much energy it takes to move a
person of mass 65\,\unit{kilograms} from the surface of the earth to an infinite
distance away (disregard gravity forces other than from the earth). \answer{36}
\help{22}
\end{one-digit-list}

\item [22.] Suppose a car engine is delivering \m{1.60 \times 10^2\unit{hp}} at a
continuous rate, keeping the car at 25\unit{m/s} (56\unit{mph}).
Determine the force the engine is overcoming. \answer{34} \help{23}
\end{two-digit-list}

\newpage

\BriefAns

\begin{two-digit-list}
\item [1.] \m{m g \ell (1 - \cos\phi)}
\item [2.] 253\unit{hp}
\item [3.] 85\unit{J}
\item [4.] 48\unit{J}
\item [5.] \m{7.4\times10^2\unit{J}}
\item [6.] 51\unit{hp}.
\item [7.] 3.0\unit{m/s}
\item [8.] \m{8.6 \times 10^2\unit{m}}
\item [9.] 88\unit{m/s}
\item [11.] \m{-11.8\unit{J}}
\item [12.] \m{1.47\times10^6\unit{J}}
\item [13.] \m{3.1 \times 10^7\unit{m/s}}
\item [14.] \m{[F\cos\theta /(mg - F \sin\theta)] = 0.25}
\item [15.] 57\unit{m}
\item [16.] 25\unit{J}
\item [17.] \m{9.1 \times 10^{-2}\unit{m}}
\item [18.] \m{2.5 \times 10^3\unit{J}}
\item [19.] 18\unit{J}
\item [20.] \m{2.0 \times 10^1\unit{m/s}}
\item [21.] \m{4.9\times10^{-2}\unit{m}}
\item [22.] 95\unit{m}
\item [23.] \m{4.6 \times 10^3\unit{W} or 6.1\unit{hp}}
\item [24.] \m{5.8\,\uvec{x}\unit{m/s}}
\item [26.] \m{2.0 \times 10^1\unit{m}}
\item [27.] 0.17\unit{km}
\item [29.] \m{5.3 \times 10^{-2}\unit{N}}
\item [30.] \m{K m \left[\dfrac{1}{R_e}-\dfrac{1}{r}\right]}.
\item [31.] \m{\left[\dfrac{2D}{M}(F\cos\theta-M g \sin\theta) +
             v_0^2 \right]^{1/2}}.
\item [32.] \m{6.38 \times 10^6\unit{m}}
\item [33.] Same answer as (BM).
\item [34.] \m{1.1\times10^3\unit{lb}}
\item [35.] In the second case the object moves down, slowing, stops, turns
            around and when it gets to original point it has \m{v_0} upward
            speed, and so on up to point distance \m{D} upward.
\item [36.] \m{4.06 \times 10^9\unit{J}}.
\end{two-digit-list}

}% /Sect
