\revhist{12/20/88 ejdk; 8/29/91, pss; 3/11/93, pss; 4/20/94, pss; 11/22/96, pss; 3/3/2000, pss;
         12/18/2000, pss; 10/15/01, pss; 10/21/02, pss}

\Sect{}{}{\SectType{SpecialAssistance}}{

\AsItem{1}{TX-2f}
{Write the general definition for the work \m{W} done by force \m{F} in moving
from \m{A} to \m{B}.
\newline
*\,STOP\,* \& WRITE THE ANSWER BEFORE PROCEEDING!
\newline
\help{7}
}

\AsItem{2}{TX-2b}
{Let \m{W_{nc} \equiv \int_A^B \vect{F}_{nc} \cdot \,d\vect{s} \equiv} the work
done by a non-conservative force.
You will then have different values for \m{W_{nc}} depending on the actual
path taken between points \m{A} and \m{B}.
For example, consider moving a 50\,kg box from point \m{A} to point \m{B} on a
level floor, if \m{B} is 4\,m north and 3\,m east of \m{A}.
Assume the coefficient of friction between the box and the floor is \m{\mu = 0.40}.
If you apply a horizontal force \vect{F}, find the work which you do in
moving the box from \m{A} to \m{B}.

This problem has as many different answers as there are paths between \m{A}
and \m{B}.
The shortest path is the straight line from \m{A} and \m{B} of length 5\,m.
The only horizontal forces are \m{F} and \m{f}, i.e., the applied force and
friction.
If we assume the initial and final velocities are zero, the average
acceleration must be zero.
It follows from Newton's second law that the average resultant force must be
zero, so \m{\vect{F} = - \vect{f} = \mu\unit{N}\,\uvec{s}}.
The only vertical forces are \m{N} up and mg down, so again from Newton's
second law, \m{N = m g}.
Thus \m{\vect{F} = \mu m g \uvec{s}}.
Then we have
\begin{displaymath}
W = \int_A^B \mu m g\,\uvec{s} \cdot d\vect{s} = \mu m g \int_A^B ds =
\mu m g S,
\end{displaymath}
where \m{S} is the total distance traveled between \m{A} and \m{B}.
Clearly \m{S} could be any number \m{\geq 5}\,m depending on the path; e.g., for
the shortest path, \m{W = (0.4)(50)(9.8)5 = 980\unit{J}}.
But if you traveled 4\unit{m} north and then 3\unit{m} east, \m{S = 7\unit{m}}, and
\m{W = 1372\unit{J}}.
}


\AsItem{5}{\help{15}}
{\m{f = F = 200\unit{N}} when \m{F = \mu m g \cos\theta}.
 Here \m{\theta} is the angle the normal to the incline makes with the
 vertical.
 It is also the angle the incline makes with the horizontal.
 Now \m{\cos\theta = (L^2 - h^2/L)^{1/2}}.
 So \m{\mu = F L/\left(m g(L^2 h^2)^{1/2}\right) = 1/(8)^{1/2} = 0.35}.
}

\AsItem{7}{\help{1}}
{\begin{displaymath}
W \equiv \int_A^B  \vect{F} \cdot \,d\vect{s}
\end{displaymath}

If you did not put down exactly this result, review MISN-0-20.
}

\AsItem{14}{TX-4}
{\m{m = 60.0\unit{kg}}, \m{v_0 = 0}, \m{g = 9.8\unit{m/s\up{2}}}, \m{h = 1.0\unit{m}}, \m{v = }?

From kinematics (See \Quote{Kinematics in One Dimension,} MISN-0-7):
\m{v^2 - v_0^2 = 2as}, so in the \m{y}-direction, for this problem,
\m{v^2 = 2 g h}.
So \m{v = (2 g h)^{1/2} = 4.4\unit{m/s}}.
}

\AsItem{15}{TX-4}
{\TextAndFigure
 {\begin{tabular}[t]{r c l}
  \m{h} & = & 1.0\unit{m}; \m{m = 60.0\unit{kg}}\\
  \m{L} & = & 3.0\unit{m};\\
  \m{v} & = & 0\\
  \m{F} & = & ?\\
  \m{\Delta E_k} & = & 0.\ \ \m{\Delta E_p = 0 - m g h}.\\
  \end{tabular}
  }{m21gr05}

\m{W_{nc} = - F L}, where the negative sign comes from the fact that the force
\m{F} is in the opposite direction to the displacement.
Alternatively, one could say that the negative sign comes from the fact that the work done by the non-conservative force is work done on the environment (including the block) by the block's surface.

Thus from \Eqnref{5} we have:
\m{- F L = 0 - m g h}, from which \m{F = m g h/L = 2.0 \times 10^{2}\unit{N}}.

Most of this force will probably be supplied by friction.
Indeed, if the coefficient of friction is greater than 0.35 you will need to
push the box if you want it to go down the incline! \help{5}

Note that the work done by \m{F} is \m{- 600\unit{J}} and the work done by gravity is
\m{+ 600\unit{J}}, so the total work done on the box is zero, which equals
\m{\Delta E_k}.
}

\AsItem{16}{TX-4}
{Here are various correct equations:
 %
 \Eqn{}{E_k(t) + \Delta E_k = E_k(t + \Delta t),}
 %
 \Eqn{}{E_p(t) + \Delta E_p = E_p(t + \Delta t),}
 %
 \Eqn{}{E_k(t) + E_p(t) + \Delta E_k + \Delta E_p = E_k(t + \Delta t) + E_p(t + \Delta t),}
 %
 \Eqn{}{\Delta E_k + \Delta E_p = W_{nc,on}.}
 %
 Here \m{\Delta E_k} is the \textit{net} change in the system's kinetic energy over the time interval \m{\Delta t}, \m{\Delta E_p} is the \textit{net} change in the system's potential energy over that same time interval, and \m{W_{nc,on}} is the work done \textit{on} the system by non-conservative forces during that same interval.
}


\AsItem{22}{TX-3b}
{Consider the particles \m{A} and \m{B}.
Here are some of their properties (the \m{F}'s are different forces):
%
\begin{center}\begin{tabular}{l c c r}\hline
Property                           & Time    & \m{A}    & \m{B}               \\ \hline
kinetic energy                     & 0       & 7.0\unit{J} & 0.0\unit{J}      \\
kinetic energy                     & \m{t}     & 2.0\unit{J} & 4.0\unit{J}    \\
potential energy due to \m{F_1}      & 0       & 3.0\unit{J} & 0.0\unit{J}    \\
potential energy due to \m{F_1}      & \m{t}     & 1.0\unit{J} & 11.0\unit{J} \\
potential energy due to \m{F_2}      & 0       & 0.0\unit{J} & 5.0\unit{J}    \\
potential energy due to \m{F_2}      & \m{t}     & 0.0\unit{J} & 8.0\unit{J}  \\ \hline
\end{tabular}\end{center}
%
Problem: find the work \m{W} done on the system.

We find: \m{(7 + 0 + 3 + 0 + 0 + 5)\unit{J} + W =
                                (2 + 4 + 1 + 11 + 0 + 8)\unit{J}}.

\m{W = 11\unit{J}}.
}

\AsItem{23}{TX-4}
{Let us further develop the analogy following \Eqnref{11}.
Kinetic energy is like cash.
Potential energy is like a checking account --- one for each conservative
force acting in the problem.
Moving money from one checking account to another or to cash does not change
your total assets.
But you must not include money taken from a checking account as \Quote{money
acquired} during time \m{\Delta t}.
Similarly, the total energy in a free falling ball is constant; energy is
being taken out of the gravitational potential energy checking account and
turned into kinetic energy (cash) but the total assets are constant.
Looked at from this point of view, the work done by gravity while the ball
falls is not energy entering the system (not money acquired) since it was
already there as potential energy (i.e., in the checking account) at the
beginning of the time interval.
}

\AsItem{24}{PS-prob. 2}
{Write conservation of energy between points \m{A} and \m{B}: at point \m{A},
water is at rest at the bottom of the well; at point \m{B}, it is at height
\m{h} and moving with speed \m{v}.
Recall that power is energy (here, work) per unit time.
}

\AsItem{25}{PS-prob. 3}
{At \m{t = 0}, the diver has potential and kinetic energy.
Note that the result is independent of \m{\theta}.
}

\AsItem{26}{PS-prob. 8}
{The unknown is \m{W_m/E_{kf}} where \m{W_m} is the work done by the man and
\m{E_{kf}} is the final kinetic energy.
}

\AsItem{27}{PS-prob. 14}
{Recall that the frictional force is the coefficient of friction times
the normal force.
}

\AsItem{28}{PS-prob. 17}
{Use conservation of energy to show that \m{x = v_0 (m/k)^{1/2}}.
}

\AsItem{29}{PS-prob. 18}
{See [S-22] for a similar problem.
}

\AsItem{30}{TX-3b}
{We changed to primed symbols for the variables of integration because we
 wanted to use unprimed symbols for the upper limit of integration and
 hence for the function on the left side of the equation (our real goal).

 The term \Quote{dummy variable of integration} is standard usage.
 It indicates that the symbol used for the variable of integration is
 immaterial in a definite integral because it does not appear in the final
 answer.
 Thus
 %
 \Eqn{}{\int_0^1 x\,dx \qquad \text{and} \qquad \int_0^1 y\,dy}
 %
 give exactly the same answer.
}

}% /Sect
