\revhist{7/3/86, pss; 3/16/88, pss; 8/18/91, pss; 2/24/93, pss; 4/20/94, pss; 11/8/94, pss;
         10/28/96, pss; 11/22/96, pss; 7/15/99, abs; 3/3/2000, pss; 12/18/2000, pss; 3/22/01, pss;
         3/29/01, kag}
%
\Sect{1}{Introduction}{\SectType{TextOnePara}}{
%
\Index{energy}In this module we are primarily concerned with the concept of conservation of
energy.
We first introduce the concept of potential energies to replace those line
integrals in the work definition that produce path-independent work.
We then state the Law of Conservation of Energy, which is perhaps
the most useful unifying concept of science.

Understanding Conservation of Energy will give you insight into many ordinary
experiences from everyday life (such as why some people get fat!).
You will also be better prepared to analyze the \Quote{energy crisis} and some of
the proposed cures.
(Don't waste time with schemes that promise more energy out than you put
in!)
In addition, you will develop an extremely powerful tool for solving a large
class of physics problems with a minimum of mathematical manipulation.
In order to further restrict the mathematics needed, we shall only deal with
one-dimensional motion in this module.
}% /Sect
%
\Sect{2}{Potential Energy}{\SectType{TextMultiPara}}{
%
\pcap{2}{a}{Overview}
When work is performed on a system,
%
\Footnote{1}{See \Quote{Work, Power, Kinetic Energy} (MISN-0-20).}
%
and the energy thus transferred into the system is stored internally in such
a way that it can later be converted back to mechanical energy\Index{force| conservative, mechanical case} by reversing
the storage process, we call the stored energy \Quote{potential energy.}
That is because it has the \textit{potential} to be recovered by reversing the
storage process.
Good examples are the energy stored through raising water against the force of
gravity and the energy stored through compressing a spring in a mechanical
clock.
In the case of the elevated water the stored energy can be reclaimed to run
electrical generators while in the case of the clock spring the stored energy
is used to move the hands.

\pcap{2}{b}{Frictional Forces}
Work done against \textit{the force of friction}\Index{force| frictional} does not cause a gain in
mechanical \textit{potential} energy because the process is not mechanically
reversible.
For example, running a car at constant speed in reverse gear does not enable
you to regain the mechanical energy you previously expended running the car
at the same speed in forward gear.
In fact, running the car in reverse gear requires the same \textit{expenditure}
of energy as running it in forward gear.
While running the car at constant speed, in either gear, \textit{all} of the
energy expended goes into overcoming friction.
\help{2}

\pcap{2}{c}{Conditions for Potential Energy}
There are several {\em equivalent} ways of describing the situations in which
potential energy can be gained or lost.
Two of them are: (i) the work done is reversible; and (ii) the work integral
is independent of path.

\pcap{2}{d}{A Conservative Force}
When the above conditions are fulfilled, the storage force involved in the
energy storage is universally said by professionals to be a \Quote{conservative}
force.\Index{conservative force}
By this they mean that the energy involved is conserved, with no losses, as
\textit{mechanical} energy.
The energy-storage force is \textit{not} the force you apply as you do work
on the system; rather, it is an opposing force that is internal to the
system, a loss-less mechanical storage force.
The main two examples we will deal with here are the force of gravity
and the restoring force of springs.

Whether the force is \Quote{conservative} or not, \textit{total} energy is
\textit{always} conserved.

\pcap{2}{e}{Potential Energy Difference}
\Index{potential energy difference}The work done \textit{by} a lossless mechanical storage force whose point of
application moves from \m{A} to \m{B} is:
%
\Eqn{}{W_{A \rightarrow B} = \int_A^B \vect{F}_c \cdot d\vect{s}.}
%
where the subscript \Quote{c} explicitly shows that the force is a
\Quote{conservative} one.
The negative of this work is the work done \textit{on} the force and hence is
the change in stored energy.
We call this change the \Quote{potential energy difference,
\m{\Delta E_p}, at point \m{B} relative to point \m{A}}:
%
\Footnote{2}{Some authors use \m{U}, \m{V} or \m{PE}.
Our notation, \m{E_p}, emphasizes that this is just a particular form of
energy.}
%
\Eqn{1}{\Delta E_p \equiv - \int_A^B \vect{F}_c \cdot d\vect{s} =
- W_{c, A \rightarrow B}}
%
Note the significance of the signs of the terms: because the line integral
does not depend on path, \m{W_{A \rightarrow B} = - W_{B \rightarrow A}}:
i.e., the work done against \vect{F} in moving from \m{A} to \m{B} can be
retrieved by moving back from \m{B} to \m{A}.
Thus the net path-independent work done around any \textit{closed} path is
\textit{zero}.
Actually, this should be obvious since one possible path from a point \m{A} to
the same point \m{A} is no movement at all and that certainly involves zero
work.

\pcap{2}{f}{Path-Dependent Work}
When a non-conservative force,\Index{force| non-conservative}\Index{non-conservative force} \m{\vect{F}_{nc}}, is involved, the value
of the line integral,
%
\Eqn{}{\int_A^B \vect{F}_{nc} \cdot d\vect{s},}
%
depends on the actual path of integration. \help{1}
The integration path in such cases must follow the actual physical path taken
by the point of application of the force.
}% /Sect
%
\Sect{3}{The Potential Energy Function}{\SectType{TextMultiPara}}{
%
\pcap{3}{a}{Standard Reference Points}
We often dispense with the cumbersome {\m{A \rightarrow B}} terminology and
choose a \Quote{standard reference point,} a point which often turns out to be
the lower limit of the integral in practical problems.
We then integrate from this point as the lower limit to an arbitrary space
point (\m{x},\m{y},\m{z}) as the upper limit of the integral.
The resulting function of (\m{x},\m{y},\m{z}) is called the \Quote{potential energy
function}\Index{potential energy function} and, for an object at (\m{x},\m{y},\m{z}), \m{E_p(x,y,z)} is called the
object's \Quote{potential energy.}\Index{potential energy}\Index{energy| potential}

If you have a potential energy function and you wish to know the difference
in potential energy between a space point and the reference point, merely
evaluate the function at the point in question.
If you wish to know the difference in potential energy between the point in
question and a point that is not the reference point, merely evaluate the
function at both points and take the difference.
If you see a potential energy function and you are uncertain as to the
reference point that was used to generate it, just set the potential energy
function equal to zero and solve for (\m{x},\m{y},\m{z}).
That will be the reference point that was used.

\pcap{3}{b}{Gravitational Potential Energy}
\Index{gravitational potential energy}\Index{potential energy| gravitational}Near the surface of the earth, the gravitational force on mass \m{m} (i.e., its
weight) is essentially constant and is given by \m{m\vect{g}}.
Choosing a coordinate system with unit vectors \m{\uvec{x}}, \m{\uvec{y}}, and
\m{\uvec{z}} along east, north and up, respectively, and choosing the coordinate
origin (wherever we might have put it) as the reference point, the
gravitational potential energy at the point \m{z_0} is:
%
\Eqn{}{E_{p,\text{grav}}(z_0) =
             - \int_0^{z_0} m (-g)\,dz = m g \int_0^{z_0} \,dz = m g z_0\,.}
%
where \m{z_0} is height above the coordinate origin (which we have taken as the
zero of potential energy for this case).
We usually prefer to talk about the potential energy at some point \m{z} so we
make a change of notation in the equation above to get the usually-used form:
%
\Eqn{4}{E_{p,\text{grav}}(z) =
             - \int_0^z m (-g)\,dz' = m g \int_0^z \,dz' = m g z\,.}
%
where \m{z} is now height above the coordinate origin, our chosen
zero of potential energy.  The variable \m{z'} is called the \Quote{dummy variable of
integration.} \help{30}

Suppose an object of mass \m{m} is located at the point (\m{x_1},\m{y_1},\m{z_1}) and
is then moved to the point (\m{x_2},\m{y_2},\m{z_1 + h}).
Using \Eqnref{4} we find that its increase in gravitational potential
energy is just \m{m g h}.
Note also that this is equal to the work done by an outside force
\m{\vect{F} = m g \uvec{z}} that raises the mass \m{m} through height \m{h} against
the gravitational force that is pulling the mass and the earth toward each
other.
So \m{m g h} is called the gravitational potential energy of the mass \m{m} at
height \m{h} above some coordinate origin, the reference level we have chosen.

Thus a typical concrete block with a weight of 175\,N will have a potential
energy of 175\,N\,m relative to your toe if you are holding it at a height of
1.00 meter above your toe.

Note that the gravitational potential energy in \Eqnref{4} is \textit{not}
a property of the object of mass \m{m}: instead, it is a property of the
object-plus-earth system and that system's gravitational interaction.
The gravitational potential energy of that system increases when the distance
between the object and the earth is increased, and it decreases when that
distance is decreased.

\pcap{3}{c}{Other Illustrative Forces}
\begin{itemize}
\item [1.] Consider a conservative force of the form \m{F_c = - k x}, a
spring-type force, acting on some particle.
Using \Eqnref{1}:
%
\Eqn{5}{E_p = - \int_0^x - k x'\,dx' = k x^2/2.}
%
Just looking at the result, it is clear that the reference point is \m{x = 0}
since that is the point where \m{E_p = 0}.

\item [2.] Consider \m{F_c = k/r^2}.
Using \Eqnref{1}:
%
\Eqn{6}{E_p = - \int_{\infty}^r k/r'^2\,dr' = k/r.}
%
The reference point, the point where \m{E_p = 0}, is at infinity, so \m{E_p}
represents the work that is done by \m{F_c} on a particle as it moves from
any point at radius infinity to any point at radius \m{r}.
\end{itemize}
}% /Sect
%
\Sect{4}{Conservation of Energy}{\SectType{TextOnePara}}{
%
Many of the great turning points in the history of physics have been
characterized by the discovery of situations that appeared at first to
violate \Eqnref{1}.
But, in every case, it has been found that some form of \m{E_k} or \m{E_p} had
been overlooked and total energy was actually conserved.
We can write Conservation of Energy\Index{conservation of energy} this way:

\noindent\hspace*{\fill}\begin{tabular}{|c|}\hline
The total energy in a system at time \m{t} \\
 \m{+} \\
the energy entering the system during time \m{\Delta t} \\ \hline
\end{tabular}\hspace*{\fill}
%
\vspace*{-4pt}
\Eqn{11}{ = }
\vspace*{-4pt}
%
\noindent\hspace*{\fill}\begin{tabular}{|c|}\hline
the total energy in the system at time \m{t + \Delta t} \\
 \m{+} \\
the energy leaving the system during time \m{\Delta t}. \\ \hline
\end{tabular}\hspace*{\fill}\medskip

Note the close analogy with your financial affairs:

\noindent\hspace*{\fill}\begin{tabular}{|c|}\hline
Your total money at the beginning of some time interval \\
 \m{+} \\
money you acquired during the time interval \\ \hline
\end{tabular}\hspace*{\fill}\vspace*{-1pt}
%
\Eqn{}{=}
%
\noindent\hspace*{\fill}\begin{tabular}{|c|}\hline
money you have at the end of the time interval \\
 \m{+} \\
money you spent during the time interval. \\ \hline
\end{tabular}\hspace*{\fill}\medskip

Just as \Quote{the money you have} at a given time may be stored in various
places and in various forms, so \Quote{the energy in a system} may be stored
in many forms. \help{23}
Fortunately, we need only consider those forms that change for systems we
study during the time interval being considered.
We will only study two forms here: kinetic energy and mechanical potential
energy.\Index{force| conservative, mechanical case}

\tryit Rephrase each of the four terms in \Eqnref{11} in terms of the change in a system's
kinetic energy, \m{\Delta E_k}, the change in its potential energy, \m{\Delta E_p}, and
the work done \textit{on} the system by non-conservative forces, \m{W_{nc,on}}.
\help{16}

\tryit Suppose we have a 60.0\,kg box that we wish to unload from a truck.
Suppose this will require lowering the box through a height of 1.0\,m.
Use kinematics to find the speed of the box just before hitting the ground
if the box is simply pushed off the end of the truck. \help{14}

\tryit In the above exercise assume that the velocity of impact is
unacceptable, so we must find a gentler way of unloading the box.
It is too heavy to lift by hand, but the truck is equipped with a ramp
that is 3.0\,m long.
Use \Eqnref{11} and proper problem solving techniques to find the
average force up the incline, \m{F}, that must be applied to the box to let
it arrive at the bottom of the incline with negligible speed. \help{15}
}% /Sect
%
\Sect{5}{Examples}{\SectType{TextMultiPara}}{
%
\pcap{5}{a}{Energy in the Morning}
\begin{center}\begin{tabular}{|c|}\hline
Your total energy in the morning \\
   \m{+} \\
energy taken in during the day (from food, etc.)  \\ \hline
\multicolumn{1}{c}{\bf = } \\ \hline
your total energy at the end of the day. \\
   \m{+} \\
energy leaving during the day (work, heat, etc.) \\ \hline
\end{tabular}\end{center}

Note that significant increases in internal energy are stored as fat!

\pcap{5}{b}{Concrete Block Drops on Toe}
Consider what would happen if you let go of the concrete block mentioned near
the end of Sect.\,3b.
It started with 175\unit{J} of gravitational \m{E_p} relative to your toe.
The only significant force on the block while falling is mg, which is
conservative, so when the block first touches your toe it will have \m{E_k = 175\unit{J}}.
A short time later the block is at rest at \m{h = 0} so it has \m{E_k = 0} and
\m{E_p = 0}.
At first glance it appears that 175\unit{J} of energy was \Quote{lost.}
Actually, this energy went into breaking chemical bonds (change in chemical
potential energy) and into random kinetic energy of molecules (increase in
temperature).

\pcap{5}{c}{Box Slides Down Incline}
A box slides 5.0\,m down a {36.87\degrees} incline in a region where \m{g} is
9.8\unit{m/s\up{2}}.
The velocities of the box at the beginning and end of this interval are
4.0\unit{m/s} and 6.0\unit{m/s}, respectively, down the incline.

\tryit Find the coefficient of friction, assuming it is independent of velocity:
\newline
\hspace*{\fill}\fbox{{\setlength{\tabcolsep}{0.1pc}
 \begin{tabular}[t]{r c l}
 \m{L}      & = & 5.0\unit{m} \\
 \m{\theta} & = & {36.87\degrees} \\
 \m{v_0}    & = & 4.0\unit{m/s}\\
  \m{v}     & = & 6.0\unit{m/s}\\
  \m{g}     & = & 9.8\unit{m/s\up{2}} \\
  \m{\mu}   & = & ? \\
 \end{tabular}}
\CharacterUnframedFigure{m21gr01}}\hspace*{\fill}\newline

\noindent
From \Eqnref{11}:
%
\Eqn{}{(m g h + \dfrac{1}{2} m v_0^2) + 0 = \dfrac{1}{2 }m v^2 + f L,}
%
where \m{f} is the frictional force and \m{h} is \m{L \sin\theta}.
From Newton's second law:
%
\Eqn{}{N = m g \cos\theta.}
%
Then:
\ThreeEqns{}{\mu=\dfrac{f}{N} & = \dfrac{mg \sin\theta - m(v^2-v_0^2)/(2L)}{mg \cos\theta}}
            {                 & = \tan\theta - \dfrac{(v^2 - v_0^2)}{2 L g \cos\theta}}
            {                 & = \dfrac{3}{4}-\dfrac{(36\unit{m\up{2}}/\unit{s\up{2}}-16\unit{m\up{2}}/\unit{s\up{2}})}{2(5\unit{m})(9.8\unit{m/s\up{2}})(0.80)}=0.49\,.}

\pcap{5}{d}{Other Examples}
Potential energy curves\Index{potential energy curve}\Index{curve| potential energy} are useful in analyzing motion and can be used to
display complete information about the dynamical motion of a one-dimensional
conservative system.
%
\Footnote{3}{See \Quote{Potential Energy and Motion; Potential Curve, Turning Points}
(MISN-0-22).}
%
The application of conservation of energy to fluids is the basis of fluid
dynamics.
%
\Footnote{4}{See \Quote{Fluids, Static and Dynamic} (MISN-0-48).}
%
Conservation of energy is used to determine the motion of oscillating systems.
%
\Footnote{5}{See \Quote{Simple Harmonic Motion} (MISN-0-25).}
%
Energy concepts are applied to the analysis of the rotation of rigid
bodies.
%
\Footnote{6}{See \Quote{Translational and Rotational Motion of a Rigid Body}
(MISN-0-43).}
}% /Sect
%
\Sect{}{Acknowledgments}{\SectType{Acknowledgments}}{
I would like to thank Richard Hagan, Angelite Arnold, Richard Jones, and
Tom Hudson for their helpful reviews of this module.
\NsfAcknowledgment

\enlargethispage*{4pt}}% /Sect
%
\Sect{}{Some Symbols Used}{\SectType{AppendixOnePara}}{
%
\begin{itemize}
\item [] \m{A}, an arbitrary fixed point
\item [] \m{B}, an arbitrary fixed point
\item [] \m{d\vect{s}}, a differential displacement
\item [] \m{E_k}, kinetic energy
\item [] \m{\Delta E_k}, change in kinetic energy
\item [] \m{E_p}, potential energy
\item [] E\m{_{p,grav}}, gravitational potential energy
\item [] \m{\Delta E_p}, change in potential energy
\item [] \m{\vect{F}_c}, conservative force
\item [] \m{\vect{F}_{nc}}, non-conservative force
\item [] \vect{F}, force
\item [] \vect{f}, frictional force
\item [] \m{g}, acceleration of gravity
\item [] \m{h}, height
\item [] \m{m}, mass
\item [] \vect{r}, radius
\item [] \m{\uvec{r}}, unit vector in the radial direction (away from origin)
\item [] \m{W_{A\rightarrow B}}, work done by a force moving from \m{A} to \m{B}
\item [] \m{W_{nc,on}}, the work done on a system by non-conservative forces
\item [] \m{\uvec{x}}, a unit vector in the \m{x}-direction
\end{itemize}
}% /Sect

