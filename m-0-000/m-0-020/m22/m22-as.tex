\revhist{7/10/85, mpm; 8/30/91, pss; 2/26/93, pss; 4/20/94, pss; 11/8/94, pss}

\Sect{}{}{\SectType{SpecialAssistance}}{

\AsItem{1}{TX-Appendix A}
{Since \m{x_1 \leq x_0 \leq x_1 + \Delta x},
 %
 \Eqn{}{| x_0 - x_1 | \leq \Delta x.}
 %
 Then given any \m{\epsilon > \Delta x},
 %
 \Eqn{}{| x_0 - x_1 | < \epsilon.}
 %
}

\AsItem{2}{TX-Appendix A}
{We assumed that \m{F(x)} is continuous, so use the definition of continuity.
}

\AsItem{3}{TX-Appendix B}
{%
 \Eqn{}{a(x)\,dx = v\,dv}
 %
 \ThreeEqns{} {\int_{x_0}^x a(x')\,dx' & = \int_{v_0}^v v'\,dv' \qquad \help{5}}
              {                        & = \left. \dfrac{v'^2}{2} \right|_{v_0}^v}
              {                        & = \dfrac{1}{2}(v^2 - v_0^2)}
 %
}

\AsItem{4}{TX-4c}
{At \m{x_0} the total energy is all in the form of potential energy,
 hence the velocity of the particle at that point is zero.
 The acceleration of the particle there is also zero because:
 %
 \Eqn{}{F(x_0) = - \left.\dfrac{dE_p}{dx}\right|_{x_0} = 0.}
 %

 The left and right turning points coincide: the particle cannot move!
}

\AsItem{5}{[S-3]}
{We changed to primed symbols for the variables of integration because we
 wanted to use unprimed symbols for the upper limits of integration.
}

\AsItem{6}{TX-2a}
{The term \Quote{dummy variable of integration} is standard usage.
 It indicates that the symbol used for the variable of integration is
 immaterial in a definite integral because it does not appear in the final
 answer.
 Thus
 %
 \Eqn{}{\int_0^1 x\,dx \qquad \text{and} \qquad \int_0^1 y\,dy}
 %
 give exactly the same answer.
}

\AsItem{7}{TX-2a}
{If you are confused by the force being a function of \m{x} and the integrand
 being a function of \m{x'}, read [S-6].
 Merely write the force as a function of the variable required in the
 integrand, \m{F(x') = - k x'}, and the integration is easy.
}

}% /Sect
