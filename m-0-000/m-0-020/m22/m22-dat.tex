\revhist{4/1/91, pss; 8/30/91, pss; 11/25/92, pss; 2/26/93, pss; 4/20/94, pss;
         10/12/94, pss; 11/8/94, pss; 9/18/95, lae; 12/7/95, pss; 8/23/96, pss;
         11/13/97, pss; 2/4/99, bds; 2/22/99, pss; 11/8/99, pss; 12/18/2000, pss; 6/13/02, pss;
         10/3/02, pss}
%
\defModTitle{\ph{Energy Graphs, Motion,} \ph{Turning Points}}
\defCtAuthor{\inits{J.}\inits{S.}Kovacs, Michigan State University}
\defIdAuthor{J.\,S.\,Kovacs, Physics Dept., Mich.\,State Univ., E.\,Lansing, MI}
%
\defIdItems{
    \IdVersEval{10/3/2002}{1}
    \IdHours{1}
    \begin{InputSkills}
    \item [1.]  State the relationship between the line integral of a conservative force
    between specified limits and the potential energy difference between those
    limits \prrqone{0-21}.
    \item [2.]  Given the graph of a simple function, such as f\,=\,x\,exp(-\,a\,x), demonstrate
    the relationship between slope and derivative at any given point \prrqone{0-1}.
    \item [3.]  State the law of conservation of energy for mechanical systems, defining
    kinetic and potential energy \prrqone{0-21}.
    \end{InputSkills}
    %
    \begin{KnowledgeSkills}
    \item [K1.] Vocabulary: potential energy curve, energy diagram, turning point.
    \end{KnowledgeSkills}
    %
    \begin{ProblemSolvingSkills}
    \item [S1.] Given the potential energy of a particle as a function of position
    (in one dimension or radially with spherical symmetry) determine (for a given
    position) the force acting on that particle and its acceleration, velocity,
    and turning points (if any).
    \item [S2.] Given the graph of a one-dimensional potential energy function and the
    total energy of a particle, give a qualitative description of the motion of
    this particle and locate its turning points and regions of acceleration and
    deceleration.
    \end{ProblemSolvingSkills}
}