\revhist{12/29/87, pss; 8/29/91, pss; 11/25/92, pss; 4/20/94, pss; 12/7/95, pss;
         8/23/96, pss; 11/13/97, pss; 2/4/99 bds; 7/15/99, abs; 3/22/01, pss; 3/29/01, kag;
         6/13/02, pss; 10/3/02, pss}
%
\Sect{1}{Introduction}{\SectType{TextOnePara}}{
%
If the potential energy corresponding to a conservative force and the total
energy of a particle are known, then all dynamical quantities of the particle
can be found.
These include the force acting on the particle, the acceleration, the
velocity, and the kinetic energy, which along with the total and potential
energy, can all be determined at each point in space.
The use of potential energy functions\Index{potential energy function} and graphs for finding these
quantities is the subject of this module.
Here we will restrict ourselves to problems that involve only one coordinate,
either one-dimensional problems or three-dimensional ones that have spherical
symmetry, for which the single coordinate \m{r}, the distance of the particle
from the force center, is sufficient.
The main difference between the two cases is that the coordinate \m{x} goes
from \m{-\infty} to \m{+\infty} while the coordinate \m{r} goes from 0 to \m{\infty}.
}% /Sect
%
\Sect{2}{Force and Potential Energy}{\SectType{TextMultiPara}}{
%
\pcap{2}{a}{\m{E_p} from \m{F(x)}}
Given a conservative force acting on a particle, that particle's potential
energy function\Index{potential energy function} can be obtained from the line integral of the force function
along the path taken:
%
\Footnote{1}{See \Quote{Potential Energy, Conservative Forces, the Law of Conservation
of Energy} (MISN-0-21).}
%
\Eqn{1}{E_p(x_2) - E_p(x_1) = - \int_{x_1}^{x_2} F(x)\,dx\,,}
%
where \m{F(x)} is the force acting on the particle.
Then \m{E_p(x_2)} is the particle's potential energy function at any point \m{x_2}.
As usual with potential energies, only differences have any meaning:
%
one normally interprets \Eqnref{1} as giving a particle's potential energy at
any point \m{x_2} relative to its potential energy at some reference point \m{x_1}.

On the other hand, the same integral is often written:
%
\Eqn{}{E_p(x) - E_p(x_\text{std ref pt}) =
                  - \int_{x_\text{std ref pt}}^{x} F(x')\,dx'\,,}
%
where \m{x'} is the \Quote{dummy variable of integration.} \help{6}
In fact, this integral is usually written with the standard reference
point omitted from the left hand side:
%
\Eqn{}{E_p(x) = - \int_{x_\text{std ref pt}}^{x} F(x')\,dx'\,.}
%
The standard reference point used to generate any particular \m{E_p(x)}
can be deduced by simply seeing what value of \m{x} makes \m{E_p} zero.

For example, suppose \m{F(x) = - k x} so \Eqnref{1} produces: \help{7}
%
\Eqn{}{E_p(x_2) - E_p(x_1) = k (x_2^2 - x_1^2)/2\,.}
%
To simplify this equation we normally choose \m{x_1} = 0 and choose \m{E_p} to be
zero there.
Thus we have chosen \m{x = 0} as the reference point.
Now we replace the position symbol \m{x_2} by the equally good (and simpler)
position symbol \m{x}.
Then we quote: \m{E_p(x) = k x^2/2}.
Any other scientist can see by inspection that we have taken \m{E_p} to be zero
at \m{x = 0}.

\pcap{2}{b}{\m{F(x)} from \m{E_p(x)}}
The force acting on a particle\Index{force| found from potential energy} can be found from the particle's potential
energy function\Index{potential energy| force found from} by applying the Intermediate Value Theorem\Index{Intermediate Value Theorem} of calculus to
\Eqnref{1}.
%
\Footnote{2}{See Appendix A if interested.}
%
This produces the result:
%
\Eqn{2}{F(x) = - \dfrac{dE_p(x)}{dx}\,.}
%
Thus the force is the negative derivative of the potential energy function.
%
Graphically, \Eqnref{2} says that at any given point the force is in the
direction in which the potential energy decreases and that its magnitude is
equal to the slope of the potential energy curve at that point.
%
\Footnote{3}{For three-dimensional cases, the force points in the direction in which
the potential energy decreases most rapidly and the magnitude of the force
is the slope in the direction of most rapid decrease.}
%
For example, we can apply \Eqnref{2} to find the gravitational force near the
surface of the earth.
Taking the \m{z}-axis perpendicular to the surface of the earth and increasing
upward, the potential energy of a particle of mass m is:
%
\Footnote{4}{See \Quote{Potential Energy, Conservative Forces, the Law of Conservation of
Energy} (MISN-0-21).}
%
\Eqn{3}{E_p(z) = m g z + \text{ constant}.}
%
The force on the particle is then:
%
\Eqn{4}{F(z) = - \dfrac{dE_p}{dz} = - m g\,.}
%
The final minus sign in \Eqnref{4}, with \m{m} and \m{g} positive, indicates that the
force is in the direction in which \m{z} decreases, downward.

\CaptionedFullFramedFigure{1}{An illustrative potential energy curve (see
text).}{m22gr01}

\pcap{2}{c}{\m{F(x)} from Graph of \m{E_p(x)}}
Characteristics of the force on a particle can be deduced from a graph of
the particle's potential energy plotted as a function of position.
Such a graph is called a potential energy curve.\Index{potential energy curve}\Index{curve| potential energy}
Consider the potential energy curve in \Figref{1}.
One immediately notes that the particle with this potential energy is not
moving freely.
It feels a force which at each point is the negative of the slope of the
\m{E_p(x)} curve.
At all points for which \m{x < x_0} the slope of the curve is negative and
the particle feels a positive force, one directed to the right.
Furthermore, because the slope at \m{x_1} is steeper than at \m{x_2}, the magnitude of
the force on the particle is greater at \m{x_1} than at \m{x_2}.
At \m{x_0} the particle feels no force.
At all points for which \m{x > x_0} the particle feels a negative force, one
directed to the left.
}% /Sect
%
\Sect{3}{Deductions From Functional \m{E_p(x)}}{\SectType{TextMultiPara}}{
%
\pcap{3}{a}{\m{a(x)} from \m{E_p(x)}}
Knowing a particle's potential energy as a function of position,
one can deduce the particle's acceleration\Index{acceleration| in terms of potential energy} as a function of position.
Using Newton's second law, in the form \m{a(x) = F(x)/m}, and \Eqnref{2}, one gets:
%
\Eqn{5}{a(x) = - \dfrac{1}{m}\dfrac{dE_p}{dx}\,.}
%
One need only remember Newton's law and \Eqnref{2} and combine them any time
one wants \m{a(x)}.

\pcap{3}{b}{\m{v(x)} from Conservation of Energy}
\Index{conservation of energy}Although \m{v(x)} can be obtained from \m{E_p(x)} by integrating \m{a(x)},
%
\Footnote{5}{See Appendix B if interested.}
%
it is generally quicker and easier to use the equation for conservation of
energy,
%
\Eqn{6}{\dfrac{1}{2} m v_0^2 + E_p(x_0) = \dfrac{1}{2} m v^2(x) + E_p(x)\,.}
%
All we need do is rearrange that equation to get:
%
\Eqn{7}{v(x)^2 =
     v_0(x)^2 - \dfrac{2}{m} \left[ E_p(x)-E_p(x_0)\right]\,.}

%\enlargethispage{1cm}

\vspace*{-0.3cm}
}% /Sect
%
\Sect{4}{Deductions From Graphical \m{E_p(x)}}{\SectType{TextMultiPara}}{
%
\pcap{4}{a}{Energy Diagrams}
\Index{energy diagram}\Index{diagram| energy}If both total energy and potential energy are plotted as functions of
position on the same graph, useful information about the motion\Index{motion| determined from energy diagram} of the
particle may be determined from that diagram.
%
\Footnote{6}{How the various quantities of motion depend upon time cannot be
determined from such diagrams.}
%
The total energy, potential plus kinetic, is always a horizontal straight
line on such a graph because its value is independent of position.
%
\Footnote{7}{\Equationref{6} above, for example, illustrates that for a particle
moving in a conservative force field the kinetic plus potential energy of the
particle at any one point equals the sum at any other point.
Hence the total energy has a constant value; it is conserved.}
%
\Figref{2} shows the potential energy curve of \Figref{1} with two different
total energies superimposed on it.
For a given total energy the particle's kinetic energy\Index{kinetic energy| on energy diagram} at each point x can
be measured as the vertical distance between the potential energy at that
point and the total energy.
For example, with total energy \m{E_1}, the particle's kinetic energy, when it is
at position \m{x_A}, is the indicated vertical distance shown on the graph.
For a particle with total energy \m{E_2}, its potential energy\Index{potential energy| on energy diagram} at \m{x_A} would be
the same as in the \m{E_1} case but its kinetic energy would be less.

\pcap{4}{b}{Changes in Speed}
Once the kinetic energy \m{E_k} is obtained from the energy diagram, the speed can
be found:
%
\Eqn{}{v = (2\,E_k/m)^{1/2}\,.}
%
However, the direction of the particle's velocity is ambiguous:
the kinetic energy will match the given \m{E_k} whether the particle is moving to
the right or to the left.
However, the particle's change in kinetic energy, and hence its change in
speed, does depend on its direction of motion.
For example, if the particle in \Figref{2} is at \m{x_A} and is moving to the right,
its kinetic energy is increasing so its speed is increasing.
Such changes in speed, which depend on the particle's direction of motion,
are given by the direction of its acceleration, the negative of the slope
of its potential energy curve [\Eqnref{5}].

\CaptionedFullFramedFigure{2}{An illustrative energy diagram showing one
potential energy and two total energy curves on the same graph.
Kinetic energy is the difference between the total and potential energies.}{m22gr02}

\pcap{4}{c}{Turning Points}
\Index{turning points}There are often limits on the positions available to a particle: those
limits are called \Quote{turning points} and they depend on the particle's
potential energy function and on its total energy.
For example: for a particle with the potential energy function shown in
\Figref{2}, and with total energy \m{E_1}, the particle can go no farther to the
left than \m{x_1} and no farther to the right than \m{x_2}.
Thus \m{x_1} and \m{x_2} are the turning points for \m{E_1}.

Consider the situation in \Figref{2} with the particle at \m{x_A}, with total energy
\m{E_1}, and moving left.
As it moves left its kinetic energy decreases until the particle reaches \m{x_1}
where its potential energy equals its total energy.
Since nothing is left for its kinetic energy, its velocity at this point is
zero.
However, the particle is only instantaneously at rest.

Throughout the neighborhood of \m{x_1} the force acts on the particle to the
right, as can be seen from the negative slope of \m{E_p} in the neighborhood of
\m{x_1}.
As the particle approaches \m{x_1} from the right, going left, this force causes
it to slow down, stop as it reaches \m{x_1}, then pick up speed to the right.
Thus point \m{x_1} is truly a \Quote{turning point} of the motion.
Similarly, \m{x_2} is a turning point on the right end of the motion:
it is the point at which the particle moving to the right comes to a stop,
then starts back left.
For total energy \m{E_2} the turning points are \m{x_3} and \m{x_4}.
Now suppose the total energy line is lowered until it goes through the
minimum of the potential energy curve at point \m{x_0}: describe the particle's
motion for that case. \help{4}
}% /Sect
%
\Sect{5}{Dealing with \m{E_p(r)}}{\SectType{TextOnePara}}{
%
All of the results discussed in this module can be applied to
three-dimensional cases that have spherical symmetry.
This means cases where the force, potential energy, velocity,
etc. depend only on the particle's radius from the origin of coordinates
(which is normally the origin of the force associated with \m{E_p}).
For such cases the one-dimensional position \Quote{\m{x}} is replaced by the
radial position \Quote{\m{r}.}
That means that \m{E_p(r)} is differentiated with respect to \Quote{\m{r}} to get
\m{F(r)}, etc.
The main difference from Cartesian coordinates is that negative values of \m{r}
do not exist.
}% /Sect
%
\Sect{}{Acknowledgments}{\SectType{Acknowledgments}}{\NsfAcknowledgment}% /Sect
%
\Sect{}{Glossary}{\SectType{Glossary}}{
\GlossaryItem{energy diagram} diagram consisting of the potential energy
curve and a horizontal line representing the total energy of the particle.
\GlossaryItem{potential energy curve, potential energy graph} the graph of
the potential energy of a particle as a function of position.
\GlossaryItem{turning points} positions at which a particle changes its
direction of motion.
}% /Sect
%
\Sect{A}{Force From Potential Energy}{\SectType{AppendixOnePara}}{
%
Here we start with the definition of potential energy difference,
\Eqnref{1},
%
\Eqn{}{E_p(x_2) - E_p(x_1) = - \int_{x_1}^{x_2} F(x)\,dx\,,}
%
and invert that equation to derive \Eqnref{2}:
%
\Eqn{}{F(x) = - \dfrac{dE_p(x)}{dx}\,.}

Since the first equation holds for all values of \m{x_1} and \m{x_2} in a given
range, we let \m{x_2 = x_1 + \Delta x} and write it as:
%
\Eqn{}{E_p(x_1 + \Delta x) - E_p(x_1) =
                - \int_{x_1}^{x_1 + \Delta x} F(x)\,dx\,.}
%
If \m{F(x)} is a continuous function, then by the Intermediate Value Theorem\Index{Intermediate Value Theorem}
%
\Footnote{8}{See any calculus textbook; for example, G.\,Thomas, {\em Calculus and Analytic
Geometry}, third edition,
Addison-Wesley (1960), Sections 4-8, 4-9.}
%
there exists an \m{x_0} in the interval (\m{x_1},\m{x_1 + \Delta x}) such that
\m{F(x_0)} is an average of \m{F(x)} over that interval:
%
\Eqn{}{\dfrac{1}{\Delta x} \int_{x_1}^{x_1 + \Delta x} F(x) dx =
                                                        F(x_0)\,.}
%
Combining the above two equations we get:
%
\Eqn{}{- \dfrac{E_p(x_1 + \Delta x) - E_p(x_1)}{\Delta x} =
                                                       F(x_0)\,.}
%
Now we take the limit of both sides as \m{\Delta x \rightarrow 0}.
Then, as \m{\Delta x \rightarrow 0}, we see that \m{x_0 \rightarrow x_1}
(\help{1}) and \m{F(x_0) \rightarrow F(x_1)} (\help{2}).
Then:
%
\Eqn{}{\lim_{\Delta x \rightarrow 0} F(x) = \lim_{\Delta x \rightarrow 0}
\left[ - \dfrac{E_p(x_1+\Delta x)-E_p(x_1)}{\Delta x} \right]\,,}
%
so:
%
\Eqn{}{F(x_1) = - \dfrac{dE_p(x_1)}{dx_1}\,.}
%
The equation just above holds for any point \m{x_1} in the given range, so
the subscripts \Quote{1} can be removed and the proof is finished.
}% /Sect
%
\Sect{B}{\m{v(x)} and \m{x(t)} from \m{a(x)}}{\SectType{AppendixOnePara}}{
%
This appendix deals with how to obtain \m{v(x)} and \m{x(t)} from \m{a(x)}.
Notice the difference from the more straightforward case where all three
of those quantities are functions of time.
In that case we obtain \m{v(t)} and \m{x(t)} from \m{a(t)} by integrating with
respect to time:
%
\Eqn{}{v(t_2) - v(t_1) = \int_{t_1}^{t_2} \, a(t)\,dt\,,}
%
\Eqn{}{x(t_2) - x(t_1) = \int_{t_1}^{t_2} \, v(t)\,dt\,.}

To obtain \m{v(x)} and \m{x(t)} from \m{a(x)}, we first relate \m{v} to \m{a(x)}
through the chain rule
%
\Footnote{9}{See \Quote{Review of Mathematical Skills-Calculus: Differentiation and
Integration} (MISN-0-1).}
%
to obtain:
%
\Eqn{}{a(x) = \dfrac{dv}{dt} = \dfrac{dv}{dx} \dfrac{dx}{dt} =
                                                     \dfrac{dv}{dx} v\,.}
%
Integration of that equation produces the velocity at any point \m{x_2},
providing one also knows the particle's speed at some one reference point
\m{x_1} (\help{3}):
%
\Eqn{}{\int_{x_1}^{x_2} a(x)\,dx =
                \dfrac{1}{2} \left[ v(x_2)^2 - v(x_1)^2 \right]\,.}
%
Although we will not be concerned with it in this module, another
integration can be made using \m{v = dx/dt} in the form
%
\Eqn{}{dt = \dfrac{dx}{v(x)}\,,}
%
producing \m{t} as a function of \m{x}, \m{t(x)}, and of course a constant of
integration will be introduced.
The resulting expression can be inverted to give \m{x(t)}.
}% /Sect

