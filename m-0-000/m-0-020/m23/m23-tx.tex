\revhist{5/29/85 mpm; 12/18/91, pss; 6/22/94, pss; 4/15/02, pss}
%
\Sect{1}{Readings Alternatives: Choose One}{\SectType{TextMultiPara}}{
%
\pcap{1}{a}{Alternative 1: AF}
Read in AF
%
\Footnote{1}{M.\,Alonso and E.\,J.\,Finn, {\em Physics}, Addison-Wesley (1970)
(see this module's Local Guide for details on obtaining this reference).}
%
at least Sections 12.1, 12.4, 12.6, and pages 242-3, paying attention
to those parts pertaining to the Ouput Skills.
We will not use the concept of \Quote{relativistic mass} in this unit.
The CM frame of reference, called the \Quote{C-frame} in AF, is the regular
center-of-mass frame; meaning the zero-total-momentum frame.
For two particles the CM frame of reference is thus the one in which the
particles have equal and opposite momenta.
If the two particles have equal mass then they have equal but opposite
velocities as viewed from this frame.
For the purposes of this unit we ignore the \Quote{L-frame} material in AF.
Do these AF Chapter 12 problems: 3-6, 8, 10.

\noindent {Brief Answers}:\vspace{-6pt}
\begin{itemize}
\item [3.] \m{8.20 \times 10^{-14}\unit{J}}, \m{1.51 \times 10^{-10}\unit{J}}, 0.51\unit{MeV},
       940\unit{MeV}.
\item [4.] \m{v_p = 0.99954\unit{c}},  \m{p = 32.9576\unit{m\,c}}.
\item [5.] b. \m{v_p = 0.027\unit{c} = 8.1 \times 10^6\unit{m/s}}.
\item [6.] a. \m{4.03 \times 10^{-5}}.
\item [  ] b. 0.0480.
\item [  ] c. 0.749.
\item [8.] electron:
\begin{itemize}
\item [a.] 0.0789\unit{MeV}
\item [b.] 0.581\unit{MeV}
\item [c.] 0.463\unit{MeV}
\item [d.] 1.98\unit{MeV}
\end{itemize}
\item [  ] proton:
\begin{itemize}
\item [a.] 27.8\unit{MeV}
\item [b.] 107\unit{MeV}
\item [c.] 852\unit{MeV}
\item [d.] 3650\unit{MeV}
\end{itemize}
\item [10.] \m{0.115\unit{c} = 3.45 \times 10^7\unit{m/s}}
\end{itemize}

\pcap{1}{b}{Alternative 2: RR}
Read in RR
%
\Footnote{2}{R.\,Resnick, \textit{Basic Concepts in Relativity and Early
Quantum Theory}, Wiley(1972)
(see this module's Local Guide for details on obtaining this reference).}
%
at least Sections 3.3, 3.4, and 3.5, including examples.
Do problems~2, 3, 4, 7, 26, and 34 on p.\,106-109.

\noindent Brief Answers:\vspace{-4pt}
\begin{two-digit-list}
\item [2.] \m{0.866\unit{c}}, No
\item [3.] \NullItem
\begin{one-digit-list}
\item [a.] \m{0.942\unit{c} = 2.83 \times 10^8\unit{m/sec}}
\item [b.] \m{621\,M_e}
\item [c.] 212\unit{MeV}, \m{1.6 \times 10^{-19}\unit{kg\,m/sec}}
\end{one-digit-list}
\item [4.] \NullItem
\begin{one-digit-list}
\item [a.] \m{1962\,M_e}
\item [b.] \m{0.99999987\unit{c}}
\item [c.] \m{2.961\,M_e,\; 0.940\unit{c}}
\end{one-digit-list}
\item [7.] Derivation
\item [26.] 92.1\unit{MeV}
\item [34.] 29.6\unit{MeV} (neutrino), 4.15\,\m{M_e} (muon)
\end{two-digit-list}

\pcap{1}{c}{Alternative 3: WSM}
Read in WSM
%
\Footnote{3}{R.\,Weidner and R.\,Sells, {\em Elementary Modern Physics}, 3rd
Edition, Allyn and Bacon (1980)
(see this module's Local Guide for details on obtaining this reference).}
%
at least Sections 3-1, 3-2, and 3-3, including examples.
For the purposes of this module, it is not necessary to learn the derivations
given in Section 3-1; only the results.

\noindent Do problems~3-2, 3-4, 3-10, p.\,83.

\noindent {Brief Answers}:

\begin{tabular}{l l}
3-2.   & Show it. \\
3-4.   & 0.14\unit{c}. \\
3-10   & 0.999998\unit{c}. \\
\end{tabular}
}% /Sect
%
\Sect{2}{Collisions}{\SectType{TextOnePara}}{
%
Consider the problem of particle \m{A} colliding with particle \m{B}, producing
particles \m{C} and \m{D}: \m{A + B \rightarrow C + D}.
If the total mass after an interaction is greater than before, then the
mass-energy has increased during the interaction.
Since overall energy must be conserved, some other form of energy must have
decreased during the interaction.
Usually this is kinetic energy.
For such cases, conservation of energy becomes:
%
\Eqn{}{E_A + E\,_B = E_C + E_D\,,}
%
where \m{E_A} is the total energy (mass energy plus kinetic energy) of particle
\m{A}, etc.
The minimum initial energy necessary for this reaction will be that which
barely allows the final two particles to be created: the kinetic energy of
that initial state is referred to as the threshold kinetic energy necessary
for the reaction.
The CM kinetic energy of the final state will be zero when the initial energy
is at the threshold.

\tryit Problem A: Use Conservation of Energy to determine the CM-frame threshold
kinetic energy, in Joules, for the process:
%
\Footnote{5}{If this module is part of a book, conversion factors should be in an appendix.
Otherwise, see the conversion factors at the beginning of this module's \textit{Problem Supplement}.}
%
\Eqn{}{\bn\up{0} + \mpi\up{0} \rightarrow \bp\up{+} + \mpi\up{-} \,,}
%
\begin{eqnarray*}
m_{\bn\up{0}} & = & 939.550\unit{MeV/c\up{2}} \\
%
m_{\mpi\up{0}} & = & 134.975\unit{MeV/c\up{2}} \\
%
m_{\mpi\up{-}} & = & 139.578\unit{MeV/c\up{2}} \\
%
m_{\bp\up{+}} & = & 938.256\unit{MeV/c\up{2}}. \\
\end{eqnarray*}
}% /Sect
%
\Sect{3}{Binding Energy}{\SectType{TextOnePara}}{
%
Consider the deuteron nucleus used in the example on p.\,243 of AF, or
Example 6 on p.\,101 of RR, or Example 3-3 on p.\,85 of WSM.
Suppose we bombard such nuclei with a beam of {\mgamma} particles which have
the characteristic that they can be completely converted to energy upon
absorption.
Our {\mgamma}'s have been produced in a manner such that each one in the beam
has the same energy.
We now gradually increase the beam energy, the energy of each {\mgamma} in the
beam, and look for the \Quote{photodisintegration} reaction,
% 
\Eqn{}{\mgamma + d \rightarrow n + p\,,}
%
in which a {\mgamma} is annihilated and its energy is absorbed by
the deuteron's constituents, the neutron and proton.
We find experimentally that this reaction is not observed to take place
until the energy of each {\mgamma} reaches 2.225\unit{MeV}.
This disintegration threshold energy is called the \Quote{binding energy} of the
conglomerate deuteron.
At threshold the {\bn} and {\bp} are produced essentially at rest.
If the beam energy of the {\mgamma}'s is increased further, the excess energy
goes into kinetic energy of the final-state {\bn} and {\bp}.

\noindent
NOTE: the binding energy is actually defined as the disintegration threshold
energy as measured in the CM frame so that the {\bn} and {\bp} can be produced
completely at rest.
However, the \m{\gamma} carries so little momentum that the deuteron-at-rest
frame is very close to being a zero-momentum frame.

\tryit Problem B: The binding energy of the nitrogen monoxide molecule (NO)
has been measured to be 5.296\unit{eV}.
Use Conservation of Energy to calculate the difference between the mass
of the molecule and the total mass of its constituents, in amu.
Note that \m{m_N = 14.0067\unit{amu}}, and \m{m_0 = 15.9994\unit{amu}}.

\noindent Brief Answers:\vspace{-6pt}
\begin{itemize}
\item [A.] \m{\underbrace{E_\text{mass}\text{(initial)}}_%
             {\ds (m_{\mpi\up{0}} + m_\bn ) c^2} +
            \underbrace{E_k\text{(initial)}}_{\ds E\text{(thresh.)}} =
            \underbrace{E_\text{mass}\text{(final)}}_%
             {\ds (m_{\mpi\up{-}} + m_\bp) c^2} +
            \underbrace{E_k\text{(final)}}_{\ds 0\text{(thresh.)}}}

\item [  ] \m{E_\text{thresh.} = 5.30 \times 10^{-13}}\,J.

\item [B.] \m{\underbrace{E_\text{mass}\text{(initial)}}_%
             {\ds (m_{NO}) c^2} +
            \underbrace{E_k\text{(initial)}}_{\ds E_B} =
            \underbrace{E_\text{mass}\text{(final)}}_%
             {\ds (m_N + m_O) c^2} +
            \underbrace{E_k\text{(final)}}_{\ds 0\text{(thresh.)}}}

\item [  ] \m{\Delta m \equiv (m_N + m_O) - (m_{NO}) = E_B/c^2 =
            5.69 \times 10^{-9}\unit{amu}}
\end{itemize}
}% /Sect
%
\Sect{4}{Reduction to Non-Relativistic Form}{\SectType{TextOnePara}}{
%
Expand the function
%
\Footnote{4}{See \Quote{Taylor's Series for the Expansion of a Function About
a Point} (MISN-0-4).}
%
\Eqn{}{k(x) = (1 - x)^{-1/2}}
%
in a power series about the origin, where \m{x \equiv v^2/c^2} and \m{k \equiv
E / (m_0 c^2)}.
We find:
%
\Eqn{}{E = m_0 c^2 + (1/2)m_0 v^2 + (3/8) m_0 (v^4/c^2) + \ldots}
%
Then for \m{v^2 \ll c^2}, the third and higher terms can be neglected compared to
the first and second.
The first term is not kinetic since it is not a function of velocity.
The leading kinetic term is:
%
\Eqn{}{E_k \approx (1/2) m_0 v^2 \;;\; v^2 \ll c^2\,.}
%
NOTE WELL: The reduction given in AF is insufficient.
}% /Sect
%
\Sect{}{Acknowledgments}{\SectType{Acknowledgments}}{\NsfAcknowledgment}% /Sect

