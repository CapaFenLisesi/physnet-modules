\revhist{5/29/85, mpm; 12/19/91, pss; 10/11/94, pss; 1/8/02, pss}

\Sect{}{}{\SectType{ProblemSet}}{

\noindent Note: Problems 1, 3, and 4 also occur on this module's \textit{Model Exam}.

\begin{one-digit-list}
\item [1.] The rest mass of a {\bSigma\up{+}} (\Quote{sigma-plus}) particle is
           1189.37\unit{MeV/c\up{2}}.
           If it is traveling at speed 0.8000\unit{c}, compute its energy in \unit{MeV}
           and its momentum in \unit{MeV/c}.

\item [2.] Reduce the expression for relativistic momentum to its
           non-relativistic form, using the general expression for Taylor's
           Series for the expansion of a function about a point.

\item [3.] In the decay of a \bSigma\up{+} particle at rest, to a proton and a
           \mpi\up{0} particle, calculate the momentum of the proton in \unit{MeV/c}.
           \begin{eqnarray*}
           m_\bp             & = & 938.2592\unit{MeV/c\up{2}} \\
           m_{\mpi\up{0}}    & = & 134.9645\unit{MeV/c\up{2}} \\
           m_{\bSigma\up{+}} & = & 1189.37 \unit{MeV/c\up{2}} \\
           \end{eqnarray*}
           Compare your answer to the (189\unit{MeV/c}) listed in \Quote{Review of Particle
           Properties,} \textit{Physics Letters}, Vol.\,239, April 12, 1990.

\item [4.] Show that \m{\vect{F} = m \vect{a}} is generally valid only for \m{v^2 \ll c^2}.
\end{one-digit-list}

\newpage

\BriefAns

\begin{one-digit-list}
\item [1.] 1982\unit{MeV}, 1586\unit{MeV/c}.

\item [2.] Taylor's Series:

           \m{f(x)=f(0) + \dfrac{f'(0)}{1!} x +
                                 \dfrac{f}{2!} x^2 + \ldots}

           \m{k(x) = (1 - x)^{-1/2}} where \m{x \equiv v^2/c^2} for our case

           \m{k'(x) = (1/2)(1 - x)^{-3/2}}

           \m{k''(x) = (3/4)(1 - x)^{-5/2}}

           \m{k(x) = 1 + (1/2)x + (3/8)x^2 + \ldots}

           \m{k(v^2) = 1 + (1/2)(v^2/c^2) + (3/8)(v^4/c^4) + \ldots}

           \m{\text{mom.} = k v m_0 = m_0 v + (1/2) m_0 (v^3 / c^2) + \ldots}

           Then if \m{v^2 \ll c^2} we can neglect the second term and get:
           \Eqn{}{\text{mom.} = m_0 v, \qquad \text{for } v^2 \ll c^2.}

\item [3.] As in Problem B, derive:
		   %
           \Eqn{}{k_p = \dfrac{m_\bSigma^2 - m_{\mpi}^2 + m_\bp^2}{2 m_{\bSigma} m_bp}}
           %
           and find the numerical value of the momentum.

\item [4.] Newton's Second Law, verified relativistically, is:
           %
           \Eqn{}{\vect{F} = \dfrac{d\vect{p}}{dt},}
           %
           but then:
           %
           \ThreeEqns{}{\vect{F} & = \dfrac{d(m\vect{v})}{dt}}
                       {        & = m \vect{a} + \vect{v} \dfrac{dm}{dt}, \qquad \text{where } m \equiv k m_0}
                       {        & \neq m \vect{a}, \text{ unless } v^2 \ll c^2 \text{ so } dm/dt = 0\,.}
           %
\end{one-digit-list}
}% /Sect
