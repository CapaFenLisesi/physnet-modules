\revhist{5/29/85 mpm; 4/30/87, pss; 12/18/91, pss; 10/11/94, pss; 3/27/97, pss}
%
\Sect{1}{Readings Alternatives: Choose One}{\SectType{TextMultiPara}}{
%
\pcap{1}{a}{Alternative 1: AF}
Read in AF:
%
\Footnote{1}{M.\,Alonso and E.\,J.\,Finn, {\em Physics}, Addison-Wesley (1970)
(see this module's Local Guide for details on obtaining this reference).}
%
Sections 12.4, 12.5, and 12.6.
Optional Chapter 12 problems: 2, 15, 22, 28.

\pcap{1}{b}{Brief Answers}
\begin{two-digit-list}
\item [2.] When \m{v = \dfrac{1}{\sqrt 2}c}

\item [  ] \m{E_\text{total} = \sqrt{2} m_0 c^2}

\item [  ] \m{E_\text{kin} = (\sqrt{2} - 1) m_0 c^2}

\item [15.] Electron: 0.788\unit{MeV}, 0.761\unit{c}

\item [   ] Proton: 938.3\unit{MeV}, \m{1.92 \times 10^5\unit{m/s} = 6.39 \times 10^{-4}\unit{c}}

\item [22.] \m{\vect{p}_1\hspace{-1pt}\prime = \vect{p}}

\item [   ] \m{\vect{p}_2\hspace{-1pt}\prime = - \vect{p}}

\item [   ] where

\item [   ] \m{p = \dfrac{c}{2m_0} \left( m_0^4 + m_1^4 + m_2^4 -
                   2 m_0^2 m_1^2 - 2 m_0^2 m_2^2 \right)^{1/2}}
\end{two-digit-list}

\pcap{1}{c}{Alternative 2: RR}
Read in RR:
%
\Footnote{2}{R.\,Resnick, \textit{Basic Concepts in Relativity and Early
Quantum Theory}, Wiley(1972)
(see this module's Local Guide for details on obtaining this reference).}
%
Sections 3.2, 3.3, and 3.4 with the exception of those areas in
3.4 pertaining to electric and magnetic fields.
Optional Problems: 20, 24, and 29 on pp.\,106-109.

\pcap{1}{d}{Brief Answers}
\begin{two-digit-list}
\item [20.] \m{4.42 \times 10^{-36}\unit{kg}}; \m{2.208 \times 10^{-32}\unit{kg}}.

\item [24.] b. 0.511\unit{MeV}

\item [   ] c. 938\unit{MeV}.

\item [29.] a. (7/12)\unit{M\dn{0}c}

\item [   ] b. (1/5)\unit{M\dn{0}c}

\item [   ] c. (32/12)\unit{M\dn{0}c\up{2}}

\item [   ] d. (34.29/12)\unit{M\dn{0}}

\item [   ] e. (0.71/12)\unit{M\dn{0}c\up{2}}
\end{two-digit-list}

\pcap{1}{e}{Alternative 3: WSM}
Read in WSM:
%
\Footnote{3}{R.\,Weidner and R.\,Sells, {\em Elementary Modern Physics}, 3rd
Edition, Allyn and Bacon (1980)
(see this module's Local Guide for details on obtaining this reference).}
%
Sections 3-1, 3-2, and 3-3, including the examples.
For the purposes of this module it is not necessary to learn the derivations
given in Section 3-1; only the results.

\noindent Optional Chapter 3 problems: 1, 8, 25, pp.\,83-4.

\pcap{1}{f}{Brief Answers}
\begin{two-digit-list}
\item [1.] 0.87\unit{c}

\item [8.] \m{8.3 \times 10^4}

\item [25.] \m{5.6 \times 10^{19}\unit{J}} for spaceship alone
\end{two-digit-list}
}% /Sect
%
\Sect{2}{Work These Problems}{\SectType{TextMultiPara}}{
%
\pcap{2}{a}{Problem A}
The rest mass of a \mpi\up{+} (\Quote{pi-plus}) meson is about
140\unit{MeV/c\up{2}}.
If a \mpi\up{+} is traveling at 0.8\unit{c}, compute its energy in \unit{MeV} and momentum
in \unit{MeV/c}.

\pcap{2}{b}{Problem B}
In the decay of a {\bLambda\up{0}} (\Quote{lambda-zero})
particle, at rest, to a neutron (\bn) and a {\mpi\up{0}} (\Quote{pi-zero}), use
conservation of energy and momentum to show that the momentum of the neutron is
104\unit{MeV/c}.
%
\begin{center}\begin{tabular}{l c l}
mass of \bLambda\up{0} & = & 1115.63\unit{MeV/c\up{2}} \\
mass of \bn            & = & 939.5656\unit{MeV/c\up{2}} \\
mass of \mpi\up{0}     & = & 134.9739\unit{MeV/c\up{2}} \\
\end{tabular}\end{center}
%
\begin{itemize}
\item [Note:] \m{\text{(momentum)}^2 = k^2 v^2 m_0^2 = (k^2 - 1) m_0^2 c^2} is a useful
identity that you can easily prove.

\item [Note:] For further decay problems see, \Quote{Review of Particle
Properties,} \textit{Physics Letters}, Vol.\,239, April 12, 1990.
\end{itemize}

You might also like to try Problem~12.28, AF.
}% /Sect
%
\Sect{}{Acknowledgments}{\SectType{Acknowledgments}}{
%
We wish to thank Ben Oh, J.\,Whitmore and G.\,A.\,Smith of Michigan State
University for the cover photograph showing two 2-body decays as well as
a 5-body one.
\NsfAcknowledgment}% /Sect
%
\Sect{}{Answers to Problems}{\SectType{AppendixOnePara}}{
%
\begin{one-digit-list}
\item [A.] 233\unit{MeV}, 187\unit{MeV/c}.

\item [B.] Consv. of \m{E}:  \m{m_\bLambda = k_\bn m_\bn + k_\mpi m_\mpi}

           Consv. of \vect{p}:
                \m{0 = k_\bn \vect{v}_\bn m_\bn + k_\mpi \vect{v}_\mpi m_\mpi}

           Squaring the second equation and using the identity in the Note,

           (a) \m{(k_\bn^2 - 1) m_\bn^2 = (k_\mpi^2 - 1) m_\mpi^2}.

           Squaring the first equation after solving it for \m{k_\mpi m_\mpi}:

           (b) \m{k_\mpi^2 m_\mpi^2 =
                   m_\bLambda^2 - 2 m_\bLambda k_\bn m_\bn + k_\bn^2 m_\bn^2}.

           Then putting (b) into (a) to eliminate \m{k_\mpi} gives:

           \Eqn{}{k_\bn = \dfrac{m_{\bLambda}^2 - m_{\mpi}^2 + m_\bn^2}
                          {2 m_{\bLambda} m_\bn}\,.}

           Again using the identity we find:

           \Eqn{}{\text{(momentum)}_\bn = \left( k_\bn^2 - 1 \right)^{1/2} m_\bn c =
                                      103.9\unit{MeV/c}\,.}
\end{one-digit-list}
}% /Sect

