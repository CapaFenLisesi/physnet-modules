\revhist{9/12/90, pss; 10/30/90, pss; 11/17/93, pss}

\Sect{}{}{\SectType{SpecialAssistance}}{

\AsItem{1}{TX-2b}
{
\m{x = A \cos(\omega\,t)} and \m{\cos(0) = 1}
}

\AsItem{2}{TX-2c}
{
\m{\omega = 4 \pi\unit{rad/s}}; \m{A = 5.0\unit{cm}}

\m{x = A \cos(\omega\,t)}

\ \ \,= 5.0\unit{cm}\ \m{\cos (4\pi\unit{rad\,s\up{-1}}}\,t)

at \m{t = 1.0\unit{s}} : \m{x = 5.0\unit{cm} \cos (4 \pi) = 5.0\unit{cm}}

Note: \m{\cos n 2 \pi} = 1 when \m{n} is an integer.
}

\AsItem{3}{TX-2d}
{
\m{\omega = 4\pi\unit{rad/s}}

\m{T = 2\pi/\omega = 2\pi\unit{rad}/(4\pi\unit{rad\,s\up{-1}}) = 0.5\unit{s}}

\m{\nu = 1/T = 1/0.50\unit{s} = 2.0\unit{cycles/s} = 2.0\unit{Hz}}
}

\AsItem{4}{TX-3c and [S-11]}
{
\Quote{As the oscillator passes the origin, the energy is all kinetic; the
potential energy is zero and the kinetic energy is at its maximum.
As the displacement increases positively, the potential energy increases
so the kinetic energy decreases in order to keep the total energy constant.
When the kinetic energy reaches zero, the displacement is at its maximum value
and the energy is all potential.
As the displacement starts decreasing, \m{\ldots} }
}

\AsItem{5}{TX-3a}
{
\Quote{As time advances, the point moves along the line, from upper left to
lower right and back again.
As it passes the origin, where the displacement and acceleration are
zero, the point is moving at maximum speed.
As it approaches an end, where both the position and acceleration are at
their maximally positive or negative values (extrema), the point slows down,
stops, and reverses its direction.}
}

\AsItem{6}{TX-2e}
{
Note that the cosine function is at its maximum at \m{t = 0} and
thereafter decreases, crossing the axis one-fourth of a period later and
reaching its maximally negative value one-fourth of a period after that.
Use these characteristics to sketch in the curve in those regions.
Continue drawing the curve forward in time, making sure the curve crosses
the axis each half period, as a cosine curve always does.
For the other curves, use their equations and the same procedure.

If you don't know what sine and cosine curves look like, see your math
textbooks from past math courses in college or high school.
}

\AsItem{7}{TX-3a}
{
This is just simple algebra!
}

\AsItem{8}{TX-2f}
{
The point is claimed to follow a circle with radius \m{r = A} and a
time-changing angle: \m{\theta (t) = - (\omega t)}.
This means that Eqs.\,(2) and (5) can be written this way in polar coordinates:

\m{x(t) = r \cos \theta (t);\ \ \ \ y(t) = r \sin \theta (t)} .

These are the equations for a point on a circle, so the claim is
proved.
Now note that, as time increases, \m{\theta} increases negatively.
Since positive polar angles are measured counterclockwise from the x-axis,
and negative angles clockwise, the point moves clockwise as time increases.
}

\AsItem{9}{TX-2f}
{
First, see [S-8].
The angular velocity of the plotted point is:

\m{d\theta/dt = d/dt (-\omega t) = -\omega} .

This shows that the magnitude of the angular velocity of the plotted point is
just the \m{\omega} of the oscillation, but the minus sign means that the
direction with increasing time is clockwise.
That is, the plotted angle increases negatively with time.
}

\AsItem{10}{TX-3a}
{
The force constant, \m{k}, is the negative of the slope of this line:
\CenteredUnframedFixedFigure{m25gr07}
}

\AsItem{11}{TX-3c}
{
See [S-4].
}

\AsItem{12}{PS-2c}
{
If you are having trouble with this problem, do Problem~4 first
and then redo the \Quote{try-it} in Sect.\,2d.
}

\AsItem{13}{PS-3}
{
The period of the minute hand is 60\unit{minutes} because it takes one hour
for it to go through a complete cycle (once around the clock face).
}

\AsItem{14}{TX-2b}
{
As its argument increases from zero, a sine function increases positively
from zero.
Thus as \m{t} increases from zero, \m{x} increases from zero.
Increasing values of \m{x} correspond to the direction {\em upward}.
}

\AsItem{15}{TX-2c}
{
The units of \m{\nu} are oscillations per unit time while the units of
\m{\omega} are angular interval per unit time.
As commonly used \m{\nu} is in cycles/sec while \m{\omega} is in radians/sec.
This is because the \Quote{2 \m{\pi}} you use to convert one to the other is
really \Quote{2 \m{\pi} radians (the angular interval around a complete circle)
per complete oscillation.}
}

}% /Sect
