\revhist{12/8/89, pss; 1/15/90, pss; 10/30/90, pss; 1/19/91, pss; 11/16/93, pss;
         10/13/94, pss; 4/5/95, pss; 4/7/95, pss; 9/27/95, pss; 4/5/96, pss;
         11/6/96, pss; 2/22/99, pss; 1/5/01, pss; 4/16/02, pss; 4/23/02, pss; 6/14/02, pss}
%
\defModTitle{\ph{Simple Harmonic Motion}}
\defCtAuthor{Kirby Morgan, Charlotte, MI}
\defIdAuthor{Kirby Morgan, Handi Computing, Charlotte, MI}
%
\defIdItems{
    \IdVersEval{6/14/2002}{0}
    \IdHours{1}
    \begin{InputSkills}
    \item [1.] Vocabulary: angular frequency, frequency, uniform circular
    motion \prrqone{0-9}, kinetic energy (0-20), potential energy, total energy (0-21).
    %
    \item [2.] Write down the coordinates, as a function of time, of a particle in
    uniform circular motion \prrqone{0-9}.
    %
    \item [3.] Relate the one-dimensional position of a particle, as a function of
    time, to the particle's velocity and acceleration, and the force acting on it
    \prrqone{0-15}.
    %
    \item [4.] Given the total energy of a particle, plus a graph of its potential
    energy versus its one-dimensional coordinate position, describe its motion
    and the force on it as time progresses \prrqone{0-22}.
    \end{InputSkills}
    %
    \begin{KnowledgeSkills}
    \item [K1.] Vocabulary: oscillatory motion, simple harmonic oscillator, simple
    harmonic motion, displacement, initial time, amplitude, phase, scaled phase
    space, frequency, period.
    %
    \item [K2.] Write down a general equation for SHM displacement as a function of
    time, assuming zero initial phase and maximum initial displacement, and
    identify the amplitude, angular frequency, phase, and displacement.
    Derive the corresponding equations for velocity and acceleration.
    \end{KnowledgeSkills}
    %
    %\begin{id-subitems}
    %and \m{v} vs. \m{x} as \m{t} increases (see this module's Demonstration Supplement).
    %
    \begin{ProblemSolvingSkills}
    \item [S1.] For a specific SHO, use given items in this list to produce
    others, as requested: displacement, time, frequency, period, phase, velocity,
    angular frequency, acceleration, force, kinetic energy, potential energy, total
    energy, and word and graphical descriptions of the motion in real space and
    scaled phase space.
    \end{ProblemSolvingSkills}
}