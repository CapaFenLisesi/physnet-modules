\revhist{9/13/88, pss; 9/11/90, pss; 10/29/90, pss; 11/17/93, pss; 10/13/94, pss;
         4/7/95, pss; 4/5/96, pss; 11/6/96, pss; 11/14/96, pss; 1/5/01, pss}

\Sect{}{}{\SectType{ProblemSet}}{

Note:  Make sure your calculator is set for the correct angular units among
the choices available: radians, degrees, etc.
Answers are in coded order, given by bracketed letters.

Note: Problem~6 also occurs in this module's \textit{Model Exam}.

\begin{one-digit-list}
\item [1.] Fill in the chart below for a particle in SHM, writing \Quote{min,} \Quote{max,}
or \Quote{0} in each space. \answer{5}

\renewcommand{\arraystretch}{1.5}
\tabcolsep 0.2in
\begin{tabular}{| c | c | c | c | c | c |}\hline
Phase\,\m{\rightarrow}  & 5\m{\pi}/2 & 3\m{\pi} & 7\m{\pi}/2 & 4\m{\pi} & 9\m{\pi}/2 \\ \hline
   x     & & & & & \\ \hline
   v     & & & & & \\ \hline
   a     & & & & & \\ \hline
   F     & & & & & \\ \hline
   E\m{_k} & & & & & \\ \hline
   E\m{_p} & & & & & \\ \hline
\end{tabular}

\item [2.] An ant sits on the end of the minute hand (15.0\unit{cm} long) of a clock.
Write the equation for its displacement along an axis that goes through 3:00,
9:00, and the center of the dial. \answer{3}

\item [3.] The displacement of a 1.0\unit{kg} mass attached to the end of a vibrating
spring is given by the equation:
%
\Eqn{}{x = 0.040\unit{m}\,\cos(\pi\unit{s}^{-1}\,t)\,.}
%
\begin{one-digit-list}
\item [a.] Determine the amplitude, phase, angular frequency, frequency, and
period of this motion. \answer{1}
\item [b.] Determine the force, potential energy, kinetic energy and total
energy when \m{t = 0.10\unit{s}}. \answer{1}
\item [c.] Sketch the kinetic energy and potential energy, along with the
displacement, as functions of time. \answer{1}
\end{one-digit-list}

\item [4.] The maximum force acting on a certain 2.0\unit{kg} mass, which oscillates
with SHM, is found to be 10.0\unit{N}.
At that time its displacement from equilibrium is 0.10\unit{m}.
Determine the angular frequency, period and frequency of its motion. \answer{4}

\item [5.] A particle is moving with simple harmonic motion.
Its displacement at \m{t = 0.167\unit{sec}} is 0.0050\unit{m} and its period is
 1.00\unit{sec}.
Determine the displacement, velocity and acceleration as functions of time.
Evaluate \m{x}, \m{v}, \m{a}, and \m{\delta} at \m{t = 0.50\unit{s}}. \answer{2}

\item [6.] An object of mass 1.0\unit{kg} is moving with SHM, with an amplitude of
0.010\unit{m}, an angular frequency of \m{4\pi\unit{rad/s}}, and a maximum displacement
at time zero.
\begin{one-digit-list}
\item [a.] Write down the kinematical expression for the displacement as a
function of time.
\item [b.] Find the displacement, velocity, potential energy and kinetic energy
at a time 3/8 of a period past a time when \m{x = 0.010\unit{m}}.
\item [c.] Sketch the displacement and potential energy versus time on a graph.
\item [d.] Use the above sketch to describe how the velocity changes as
position changes through a cycle.
\end{one-digit-list}
\item [  ]  \answer{6}

\end{one-digit-list}

\BriefAns

\begin{one-digit-list}
\item [1.] \NullItem
\begin{one-digit-list}
\item [a.] \m{x = 0.040\unit{m} \cos(\pi\unit{s\up{-1}}\,t)}
\item [  ] \m{A = 0.040\unit{m}}; \m{\omega\,t = \pi\,t}; \m{\omega = \pi\unit{rad/s}}
\item [  ] \m{\nu = \omega / (2 \pi) = (1/2) \unit{cycle/s}}; \m{T = 1/\nu = 2.0\unit{s}}.
\item [b.] \m{F = - m\,\omega^2\,x}; \m{m = 1.0\unit{kg}}; \m{t = 0.10\unit{s}}
\item [  ] \m{F = - (1.0\unit{kg})(\pi\unit{s\up{-1}})^2\,[0.040\unit{m} \cos(0.10\,\pi)] = -0.38\unit{N}}
\item [  ] \m{E_p = (1/2) k x^2 = (1/2) m \omega^2 x^2}
\item [  ] \m{\phantom{E_p} = (1/2) \,(1.0\unit{kg})(\pi\unit{s\up{-1}})^2
                      (0.038\unit{m})^2 = 0.0071\unit{J}}
\item [  ] \m{E_k = (1/2) (m\,\omega^2)(A^2 - x^2)}
\item [  ] \m{\phantom{E_k} = (1/2) [(1.0\unit{kg})(\pi\unit{s\up{-1}})^2]
                      [(0.040\unit{m})^2 - (0.038\unit{m})^2] =
                      0.0008\unit{J}}
\item [  ] \m{E = (1/2) k A^2 = (1/2) m \omega^2 A^2}
\item [  ] \m{\phantom{E} = (1/2) (1.0\unit{kg})(\pi/\unit{s})^2
                    (0.040\unit{m})^2 = 0.0079\unit{J}}
\item [c.] \CenteredUnframedFixedFigure{m25gr06}
\item [  ] \hfill \help{12}
\end{one-digit-list}

\item [2.] \m{\omega = 2 \pi}\,rad/s

\item [  ] \m{x(t) = A \cos(\omega t)}

\item [  ] \m{x(0.167\unit{s}) = 0.0050\unit{m} = A \cos(2 \pi\unit{rad/sec} \times 0.167\unit{s})};
           solve for the amplitude.
\item [  ] \m{A = 0.0050\unit{m} / (0.50) = 0.0100\unit{m}}

\item [  ] At all times:
\begin{one-digit-list}
\item [  ] \m{x = 0.0100\unit{m} \cos(2\pi\unit{s\up{-1}}\,t)}
\item [  ] \m{v = dx/dt =
           -\,0.0200\,\pi\unit{m/s}\,\sin(2\pi\unit{s\up{-1}}\,t)}
\item [  ] \m{a = d^2x/dt^2 =
           -\,0.0400\,\pi^2\unit{m/s\up{2}}\,\cos(2\pi\unit{s\up{-1}}\,t)
           = - (2\pi/\unit{sec})^2\,x}
\end{one-digit-list}

\item [  ] At t = 0.50\,s:
\begin{one-digit-list}
\item [  ] \m{x = 0.0100\unit{m} \cos(\pi) = - 0.0100\unit{m}}
\item [  ] \m{v = 0.000\unit{m/s}}
\item [  ] \m{a = 0.395\unit{m/s\up{2}}}
\item [  ] \m{\delta = \omega\,t = 1.8 \times 10^2\unit{degrees}}
\end{one-digit-list}

\item [3.] \m{x = A \cos\,\omega t}

\item [  ] \m{\omega = 2 \pi/T = 2 \pi/60\unit{min} = 0.10/\unit{min}} \help{13}

\item [  ] \m{x = 15.0\unit{cm} \cos(0.10\unit{min\up{-1}}\,t)}

\item [4.] \m{F = -\,k\,x}

\item [  ] \m{k = - F / x = - (-10.0\unit{N}) / 0.10\unit{m} = 1.0 \times 10^2\unit{N/m}}

\item [  ] \m{\omega = (k/m)^{1/2} = (100\unit{N\,m\up{-1}}/2.0\unit{kg})^{1/2} = 7.1\unit{rad/s}}

\item [  ] \m{T = 2 \pi / \omega = 2 \pi /(7.1/\unit{s}) = 0.88\unit{s}}

\item [  ] \m{\nu = 1/T = 1 / 0.88\unit{s} = 1.1\unit{cycles/s} = 1.1\unit{Hz}}.

\item [5.]
\begin{tabular}[t]{| c | c | c | c | c | c |}\hline
Phase\,\m{\rightarrow} & 5\m{\pi}/2 & 3\m{\pi} & 7\m{\pi}/2 & 4\m{\pi} & 9\m{\pi}/2 \\ \hline
   \m{x}               &    0     &   min  &   0      &  max   &   0      \\ \hline
   \m{v}               &   min    &    0   &  max     &   0    &  min     \\ \hline
   \m{a}               &    0     &   max  &   0      &  min   &   0      \\ \hline
   \m{F}               &    0     &   max  &   0      &  min   &   0      \\ \hline
   \m{E_k}             &   max    &  0=min &  max     & 0=min  &  max     \\ \hline
   \m{E_p}             &  0=min   &   max  & 0=min    &  max   & 0=min    \\ \hline
\end{tabular}

\item [6.] \NullItem
\begin{one-digit-list}
\item [a.] \m{x = 0.010\unit{m} \cos (4 \pi \unit{rad/s} t)}
\item [b.] \m{T = (1/2)\unit{s}}.
\item [  ] at \m{t = (3/16)\unit{s}}:
\begin{one-digit-list}
\item [  ] \m{x = -0.0071\unit{m}} (see graph below)
\item [  ] \m{v = -0.089\unit{m/s}} (see slope of graph below)
\item [  ] \m{E_p = 0.0040\unit{J}} (see graph below)
\item [  ] \m{E_k = 0.0040\unit{J}}
\end{one-digit-list}
\item [c.] \NullItem

\CenteredUnframedFixedFigure{m25gr08}
%
\item [d.] See text.
\end{one-digit-list}

\end{one-digit-list}

}% /Sect
