\revhist{9/10/90, pss; 11/16/93, pss; 4/5/95, pss; 11/7/97, lae; 3/22/01, pss; 3/29/01, kag;
         4/16/02, pss; 4/23/02, pss; 6/13/02, pss}
%
\Sect{1}{Oscillatory Motion and SHM}{\SectType{TextMultiPara}}{
%
\pcap{1}{a}{Oscillatory Motion}
One of the most important regular motions encountered in science and
technology is oscillatory motion.\Index{motion| oscillatory}\Index{oscillatory motion}
Oscillatory (or vibrational)\Index{vibrational motion}\Index{motion| vibrational} motion is any motion that repeats itself
periodically, i.e. goes back and forth over the same path, making each complete
trip or cycle in an equal interval of time.
Some examples include a simple pendulum swinging back and forth and
a mass moving up and down when suspended from the end of a spring (see \Figref{1}).
Other examples are a vibrating guitar string, air molecules in a sound wave,
ionic centers in solids, and many kinds of machines.

\pcap{1}{b}{Simple Harmonic Motion}
The basic kind of oscillatory motion is Simple Harmonic Motion;\Index{simple harmonic motion}\Index{motion| simple harmonic} it is used
to analyze most other oscillatory motions for purposes of understanding
and design.
A motion is said to be Simple Harmonic if the oscillating object's position,
\m{x(t)}, can be represented mathematically by a \Quote{harmonic}\Index{motion| harmonic}\Index{harmonic motion} function of time;
that is, by a sine or cosine function.
The word \Quote{simple} refers to a need for only one sine or cosine term
to represent the position of the oscillating object.
Any object that undergoes simple harmonic motion (abbreviated SHM)\Index{SHM (Simple Harmonic Motion)} is called a
simple harmonic oscillator\Index{simple harmonic oscillator}\Index{harmonic oscillator, simple} (here abbreviated SHO).\Index{SHO (Simple Harmonic Oscillator)}

\pcap{1}{c}{Uses of SHM}
The mathematical techniques used in the study of simple harmonic motion form
the basis for understanding many phenomena: the interactions of elementary
particles, atoms, and molecules; the sounds of various musical
instruments, radio and television broadcasting; high-fidelity sound
reproduction, and the nature of light and color.
They are used to study the vibrations in car engines, aircraft wings,
and shock absorbers, and to study brain waves.

\CaptionedLeftFramedFigure{1}{A weight on a spring.}{m25gr01}
}% /Sect
%
\Sect{2}{The Kinematics of SHM}{\SectType{TextMultiPara}}{
%
\pcap{2}{a}{The Displacement Equation}
By definition, a particle is said to be in simple harmonic motion if its
displacement\Index{displacement| for SHM} \m{x} from the center point of the oscillations can be expressed
this way:\Index{displacement equation, for SHM}
%
\Footnote{1}{For definitions of the symbols used in this equation, see
\Quote{Uniform Circular Motion,} MISN-0-9.
A more general form of Eq.\,1, useful when you do not want to restrict
the initial time, is presented in \Quote{SHM: Boundary Conditions,} MISN-0-26.}
%
\Eqn{1}{x(t) = A \cos (2 \pi \nu t)\ ,}
%
where \m{\nu} is the frequency of the oscillation and \m{t} is the elapsed time
since a time when the displacement \m{x} was equal to \m{A}.

\tryit Show that \m{x(0) = A}, regardless of the value of \m{\nu}. \help{1}

\pcap{2}{b}{Example: Mass on Vertical Spring}
We illustrate Eq.\,1 with the example of an object with mass oscillating
up and down at the end of a vertical spring, as in \Figref{1}.
The displacement \m{x} is then the height of the object, measured from the
center point of the oscillations.
This height needs to be at its maximum value at time zero since Eq.\,1
produces \m{x(0) = A}, which is the maximum value the displacement can have.
We can assure that this is true by measuring \m{t} on a stopwatch which
we start at a time when the object is precisely at the top point of its motion.
Alternatively, we can grab the mass and move it up to \m{x = A}, then let
it go at the exact time we start the stopwatch.

\tryit Show that if we had used a sine function instead of a cosine in Eq.\,1,
we would have had to start the stopwatch at a time when the object was passing
through \m{x = 0} headed upward. \help{14}

\newpage

\pcap{2}{c}{Displacement Equation Parameters}
\Index{displacement| for SHM}The cosine function varies between \m{-1} and \m{+1} so the value of the
displacement from center, \m{x(t)}, varies between \m{-A} and \m{+A}.
The maximum displacement, \m{A}, is called the \Quote{amplitude}\Index{amplitude, for SHM} of the motion.
Typical units are {\em meters}.

Typical units for the frequency\Index{units| of frequency} are {\em cycles per second}, also called
{\em Hertz}, abbreviated Hz\Index{hertz}\Index{SI units| of frequency}.

The quantity \m{2 \pi \nu t}, the {\em argument} of the cosine, is called
the motion's \Quote{phase.}\Index{phase, for SHM}
Typical units are {\em radians} and {\em degrees}.
Although the phase has the units of an angle, it does not
usually correspond to a space angle in the problem (look at the system in
\Figref{1} where obviously there is no angle involved).

The time-derivative of the phase is called the \Quote{angular frequency}\Index{angular frequency, for SHM} and is
denoted by the symbol \m{\omega}:
%
\Eqn{}{\omega = 2 \pi \nu\,.}
%
This enables us to write \Eqnref{1} in a more succinct form:
%
\Eqn{2}{x = A \cos (\omega t)\,.}
%
Typical units for \m{\omega} are {\em radians per second}.

\tryit Contrast the units of \m{\omega} with those of \m{\nu}. \help{15}

\tryit How would you write the displacement equation for a particle in
simple harmonic motion with an angular frequency of 4\m{\pi}\,rad/s and an
amplitude of 5\,cm; and what is the displacement of the particle at
\m{t\,=\,1.0}\,s? \help{2}

\pcap{2}{d}{The Oscillator's Period}
The \Quote{period}\Index{period| for SHM} of the oscillation is defined as the amount of time
it takes for the oscillator to go through one complete oscillation or \Quote{cycle.}
Since the cosine function repeats itself whenever \m{\omega}t is increased by
2\m{\pi}, it repeats itself whenever t is increased by 2\m{\pi/\omega}.
This, then, is the period, \m{T}, of a SHO:
%
\Eqn{3}{T = \dfrac{2 \pi}{\omega}\ .}
%
The period is the inverse of the frequency; that is,
%
\Eqn{4}{T = \dfrac{1}{\nu} = \dfrac{2 \pi}{\omega}\ .}
%

\tryit Find the period and frequency for an angular frequency of 4\m{\pi}\unit{rad/s}.
\help{3}

\pcap{2}{e}{Velocity and Acceleration in SHM}
The velocity\Index{velocity| for SHM} and acceleration\Index{acceleration| for SHM} of an SHO can be easily found
by differentiating the displacement equation, \Eqnref{2}.
The velocity is
%
\Eqn{5}{v = \dfrac{dx}{dt} = - A \omega \sin (\omega t)}
%
and the acceleration is
%
\Eqn{6}{a = \dfrac{dv}{dt} = - A \omega^2 \cos (\omega t)}
%
so:
%
\Eqn{7}{a = -\omega^2 x\ .}
%
Equation (7) shows that an SHO's acceleration is proportional and opposite
to its displacement.
We have plotted \m{x}, \m{v/\omega}, and \m{a/\omega^2} as functions of time in
\Figref{2}.

\tryit Make sure that you, yourself, can construct and interpret \Figref{2}!
\help{6}

%\tryit Use a real SHO to demonstrate \Figref{2} (see [D-1] in this module's
%Demonstration Supplement).

\CaptionedFullFramedFigure{2}{Displacement, velocity and acceleration
versus time for an SHO.
The velocity has been scaled by (1/\m{\omega}) and the acceleration by
(1/\m{\omega^2}).
The quantity T is the SHO's period.}{m25gr02}

\newpage

\CaptionedLeftFramedFigure{3}{\m{v/\omega} vs. \m{x} for the SHO in \Figref{2}.}{m25gr03}

\newpage

\pcap{2}{f}{SHM in a Scaled Phase Space}
In developing an understanding of the time-development of SHM, it is useful
to look at a plot of the oscillator's displacement versus its velocity.
At any specific time, displacement and velocity each have a specific value
and so determine a point on the plot of displacement vs. velocity.
As time goes on, the SHO's displacement and velocity change so the
corresponding point on the plot moves accordingly.
Since the SHO's displacement and velocity are cyclical, the point on the plot
traverses the same complete closed path once every cycle.

In order to simplify the SHO's displacement vs. velocity trajectory, we scale
the velocity-axis by a factor of 1/\m{\omega} (see \Figref{3}).
Then any SHO's trajectory will be a circle of radius \m{A} (as shown in \Figref{3}).
In this space, the SHO's point is always at the \Quote{phase} angle, \m{\delta},
marked off clockwise from the positive \m{x}-axis (see \Figref{3}).

As time increases the point representing the SHO moves around the circle
of radius \m{A} with constant angular velocity \m{-\,\omega} (the minus sign
merely means the motion is clockwise).
Its angular position at any particular time is the phase angle at that time.

\tryit Show that the circle in \Figref{3} and the clockwise motion around it
follow from Eqs.\,(2) and (5). \help{8}

\tryit Show that (\m{- \omega}) is the angular velocity of the point that
simultaneously represents the oscillator's position and velocity in
\Figref{3}. \help{9}

\tryit For a real SHO, demonstrate \m{v/\omega} vs. \m{x} as \m{t}
increases, as in \Figref{3} (see [D-2] in this module's Demonstration Supplement).
}% /Sect
%
\newpage

\Sect{3}{The Dynamics of SHM}{\SectType{TextMultiPara}}{
%
\pcap{3}{a}{Force is Proportional and Opposite to Displacement}
\Index{dynamics| of SHM}Using Newton's second law, \m{F\,=\,ma}, and \Eqnref{7}, \m{a\,=\,-\omega^2\,x}, it
is easy to find the force\Index{force| for SHM} necessary for a particle of mass \m{m} to oscillate
with simple harmonic motion:
%
\Eqn{8}{F = - m \omega^2 x\,.}
%
Note that the force on an SHO is linearly proportional to its displacement but
has the opposite sign.
For a positive displacement the force is negative, pointing back toward the
origin.
For a negative displacement, the force is positive, again pointing back
toward the origin.
A force which is linear and always points back to the place where F\,=\,0
is called a \Quote{linear restoring force.}
For simplicity we write \Eqnref{8} in the form:
%
\Eqn{9}{F = - k x\,.}
%
where \m{k \equiv m \omega^2} is called the \Quote{force constant} (or \Quote{spring constant}
or \Quote{spring stiffness}) for the particular oscillator being observed.

\tryit Write down \m{\omega}, \m{T}, and \m{\nu} in terms of \m{k} and \m{m} for SHM.
For a weight-on-spring SHO, show that as its mass is increased its spring must
be stiffened in order that its amplitude and frequency remain unchanged.
\help{7}

\tryit Sketch a plot of \m{F} versus \m{x} for an SHO.
On your plot, identify \m{k}. \help{10}

\tryit Describe the motion of an SHO's point on a plot of \m{F} vs. \m{x}.
Contrast the motion of the point on this plot with that of the similar point
in \Figref{3}. \help{5}

\tryit For a real SHO, demonstrate \m{F} vs. \m{x} as \m{t} increases (see [D-3] in
this module's Demonstration Supplement).

\pcap{3}{b}{Potential and Kinetic Energy for SHM}
Knowing the force acting on an SHO, we can calculate its kinetic\Index{kinetic energy| for SHM} and
potential energy.\Index{potential energy| for SHM}

First, to obtain the potential energy we use \m{F\,=\,- k x} and the definition
of potential energy:
%
\Footnote{2}{See \Quote{Potential Energy, Conservative Forces, The Law of Conservation of
Energy} (MISN-0-21).}
%
\Eqn{10}{E_p = - \int F dx = \int k x dx = \dfrac{1}{2} k x^2\,.}
%
Thus the potential energy has its minimum value, zero, at \m{x}\,=\,0.

The kinetic energy of the SHO is:
%
\begin{center}\begin{eqnarray}\setcounter{equation}{11}
E_k & = & \dfrac{1}{2} m v^2 = \dfrac{1}{2} m \omega^2 A^2 \sin^2 (\omega t)
= \dfrac{1}{2} m \omega^2 [A^2 -A^2 \cos^2 (\omega t)] \nonumber \\
  & = & \dfrac{1}{2} k (A^2 - x^2)\,.
\end{eqnarray}\end{center}
%
The kinetic energy is a maximum at \m{x}\,=\,0 and is zero at the extremes of the
motion (\m{x\,=\,\pm \,A}), as shown in \Figref{4}.

\CaptionedFullFramedFigure{4}{Kinetic, potential, and total energy and
displacement vs. time for the SHO of \Figref{2}.}{m25gr04}

\pcap{3}{c}{Total Energy for SHM}
\Index{point of stable equilibrium}\Index{stable equilibrium, point of}The total energy of a simple harmonic oscillator is:
%
\Eqn{12}{E = E_k + E_p = \dfrac{1}{2} k A^2\,,}
%
which is a constant quantity (independent of time).
Thus during an oscillation, as the potential energy increases and decreases,
the kinetic energy decreases and increases so the total energy remains constant.

\CaptionedLeftFramedFigure{5}{Potential and total energy vs.
displacement for an SHO.}{m25gr05}

The potential energy curve, \m{E_p\,=\,k\,x^2/2} (which is a parabola), and the
total energy curve \m{E\,=\,k\,A^2/2} (which is a horizontal line), are shown
as functions of displacement in \Figref{5}.
The points where the line and the curve intersect (\m{x\,=\,\pm \,A}) are the
limits of the motion.

\tryit Use \Figref{5} to describe the major events in the values of the potential,
kinetic, and total energies, and the displacement, as time advances
during a complete SHM cycle. \help{4}

\tryit Use words alone (no graph) to describe the major events in the values
of the potential, kinetic, and total energies, and the displacement,
as time advances during a complete SHM cycle. \help{11}
}% /Sect
%
\Sect{}{Acknowledgments}{\SectType{Acknowledgments}}{\NsfAcknowledgment}% /Sect
%
\Sect{}{Glossary}{\SectType{Glossary}}{
\GlossaryItem{amplitude} maximum value of displacement.

\GlossaryItem{angular frequency} time rate of change of the phase.

\GlossaryItem{angular velocity in SHM} the angular velocity of the SHO's point in
scaled phase space.
Its value is the negative of the SHO's angular frequency \m{\omega}.

\GlossaryItem{displacement} position relative to the center-point of the SHM.

\GlossaryItem{frequency} number of complete cycles per unit time.

\GlossaryItem{harmonic function} a sine or cosine function.

\GlossaryItem{oscillatory motion} motion that exactly repeats itself periodically.

\GlossaryItem{period} the time for one complete cycle.

\GlossaryItem{phase} the argument of the harmonic function describing the SHM.
Here we have chosen the initial time to be when the displacement is at
a maximum, so the harmonic function is a cosine and its phase angle
\m{\delta} is \m{\omega t}.

\GlossaryItem{scaled phase space} a space in which the two axes are the SHO's
displacement \m{x} and \m{v/\omega}.
The current state of an SHO is a point in this space.
The point continually traverses a circle of radius \m{A} with constant angular
velocity \m{- \omega}.

\GlossaryItem{simple harmonic motion\,\m{\equiv}\,SHM} any motion whose time-dependence
can be described by a single harmonic function.

\GlossaryItem{simple harmonic oscillator\,\m{\equiv}\,SHO} any object that is undergoing
simple harmonic motion.
}% /Sect
%
\Sect{}{Equations}{\SectType{AppendixOnePara}}{
%
\begin{center}\begin{tabular}{l l}
\m{x = A \cos (\omega t)   }       & \ \ \ \ \ \ \m{F = - m \omega^2 x = - k x}       \\
\\
\m{T = \dfrac{2 \pi}{\omega}}      & \ \ \ \ \ \ \m{E_p = \dfrac{1}{2} k x^2}         \\
\\
\m{\nu = \dfrac{1}{T}}             & \ \ \ \ \ \ \m{E_k = \dfrac{1}{2} k (A^2 - x^2)} \\
\\
\m{\delta = \omega\,t} \\
\end{tabular}\end{center}
}% /Sect

