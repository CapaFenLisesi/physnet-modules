\revhist{9/8/91, pss; 11/18/93, pss; 11/14/96, pss; 11/7/97, lae; 3/22/01, pss; 3/29/01, kag}
%
\Sect{1}{Dynamics of Harmonic Motion}{\SectType{TextMultiPara}}{
%
\pcap{1}{a}{Force Varies in Magnitude and Direction}
The force associated with simple harmonic motion%
\Index{simple harmonic motion}\Index{SHM (Simple Harmonic Motion)}
%
\Footnote{1}{See \Quote{Simple Harmonic Motion: Kinematics and Dynamics} (MISN-0-25).}
%
is probably a more complex kind of force than any you have encountered so
far in dynamics.
First you met forces of constant magnitude and direction.
Then there was the centripetal force
%
\Footnote{2}{See \Quote{Centripetal and \m{g} Forces in Circular Motion} (MISN-0-17).}
%
which varied in direction but had a constant magnitude.
Still another was the impulsive force,%
%
\Footnote{3}{See \Quote{Momentum, Force and Conservation of Energy} (MISN-0-15).}
%
whose direction was constant but whose magnitude varied.
Here, the force associated with harmonic motion varies both in magnitude and
direction.

\pcap{1}{b}{Potential Energy For a Restoring Force}
The potential energy function\Index{potential energy function, for SHM} \m{E_p} for a particle undergoing harmonic motion
can be shown to have a minimum at its equilibrium position.\Index{potential energy minimum}
Figure~1 shows the potential energy curve for a particle oscillating between
the limits \m{x_1} and \m{x_2}, with \m{x_0} being the equilibrium position.\Index{equilibrium position}
The force acting on the particle is related to the potential energy by
%
\Eqn{1}{F=-\dfrac{dE_p}{dx}}
%
and thus is represented at any point by the negative value of the slope at
that point.
The slope is zero where the potential energy is minimum\Index{potential energy minimum} (at \m{x = x_0}),
therefore the force is zero at the equilibrium position \m{x_0}.
Moreover, since the slope is positive (negative) to the right (left) of
\m{x_0}, the force always points towards \m{x_0} \help{1},
%
\Footnote{4}{For help, see sequence [S-1] in this module's \textit{Special Assistance
Supplement}.}
%
making it a point of stable equilibrium.\FnRef{1}
This force is called a restoring force\Index{force| restoring} because it always acts to accelerate
the particle back toward the equilibrium position.

\CaptionedFullFramedFigure{1}{Potential Energy for a Restoring Force.}{m26gr01}

\CaptionedLeftFramedFigure{2}{Potential energy for SHM.}{m26gr02}

\pcap{1}{c}{Potential Energy for SHM}
The potential energy of a particle undergoing simple harmonic motion about
the point \m{x_0} is:
%
\Eqn{2}{E_p = \dfrac{1}{2} k (x - x_0)^2.}
%
Given that the potential energy for SHM is \m{E_p = k x^2/2}
for displacement about the point \m{x = 0},\FnRef{1} \Eqnref{2} is the more general
expression for motion about \m{x = x_0}.
The potential energy curve for SHM (see \Figref{2}) has a minimum at the
equilibrium position and is symmetrical about that point.
Thus the limits of the motion, the points where the potential energy equals
the total energy, are equally spaced about the equilibrium position.
The force is found from the potential energy:\Index{potential energy| for SHM}
%
\Eqn{}{F = - dE_p/dx}
%
\Eqn{3}{F = - k (x - x_0).}
%
which is the general form of the SHM force.\Index{SHM force}\Index{force| for SHM}
}% /Sect
%
\Sect{2}{The Force}{\SectType{TextMultiPara}}{
%
\pcap{2}{a}{Hooke's Law}
\Index{Hooke's law}\Index{force| Hooke's law}The relation \m{F = - k (x - x_0)} was first found empirically by Robert Hooke in
the 17th century and so is known as Hooke's law.
Hooke found that, for springs and other elastic bodies, the restoring force\Index{linear restoring force}
was proportional to the displacement from equilibrium provided the elastic
solid was not deformed beyond a certain point called the elastic limit.\Index{elastic limit}
Past this point the elastic body will not return to its original shape when
the applied force is removed.
It turns out that Hooke's law holds for many common materials and can be
applied to almost any situation where the displacement is small enough.
%
\Footnote{5}{See \Quote{Small Oscillations} (MISN-0-28).}
%
Since by Hooke's law the force is linearly proportional to the displacement,
it is called a linear restoring force or, equivalently, a \Quote{Hooke's law
force.}\Index{Hooke's law force}
The proportionality constant \m{k} is known as the \Quote{force constant.}\Index{force constant}
%
\CaptionedLeftFramedFigure{3}{Mass hung on a spring.  Note that the
position of stable equilibrium is at \m{x_0}, not at the origin.}{m26gr03}
%

\pcap{2}{b}{Example: Mass Hung on a Spring}
\Index{spring, mass hung on}Consider now a mass \m{m} attached to a spring of
negligible mass and force constant \m{k}.
The spring is hung vertically as shown in \Figref{3}.
The force acting to stretch the spring is just the weight of the mass, i.e.
\m{m g}.
When the mass is hung from the spring it causes the spring to stretch from
its original length until the magnitude of the restoring force equals the
weight:
%
\Eqn{4}{m g = | F | = k x_0}
%
Here we have chosen \m{x_0} as the distance the mass stretches the spring.
The mass-spring-gravity system thus has an equilibrium point at \m{x = x_0} and
the resultant force is zero there.
Now if the mass is displaced from \m{x_0}, Hooke's law states that it will
experience a force \m{F = - k (x - x_0)} acting so as to return the mass to
\m{x_0} (so long as the elastic limit of the spring is not exceeded).
}% /Sect
%
\Sect{3}{Equation of Motion}{\SectType{TextMultiPara}}{
%
\pcap{3}{a}{Displacement Equation from Hooke's Law}
Hooke's law, \m{F = - k (x - x_0)}, and Newton's second law, \m{F = m a}, can be
combined to find the displacement equation for SHM.
Writing the acceleration as \m{d^2x/dt^2}, the equation of motion is:
%
\Eqn{5}{m\dfrac{d^2x}{dt^2} = - k (x - x_0),}
%
or:
%
\Eqn{6}{\dfrac{d^2x}{dt^2} + \dfrac{k}{m} (x - x_0)=0.}
%
Letting \m{\omega^2 \equiv k/m} and \m{X \equiv x - x_0}, \Eqnref{6} can be
rewritten in terms of \m{X}, the displacement from equilibrium:
%
\Eqn{7}{\dfrac{d^2X}{dt^2} + \omega^2 X = 0}
%
which has the general solution
%
\Footnote{6}{If you have not encountered this before, here is all you need to
know about general solutions of equations like \Eqnref{7} for this module:
If the highest derivative in the equation is a second derivative, and
if all terms in the equation contain the unknown function on the right
side of each term, then any solution that satisfies the equation is in fact
the unique solution to the equation if the solution contains two independent
constants undetermined by the equation.
The two undetermined constants in \Eqnref{8} are \m{a} and \m{b}.
The statement is very powerful, since it says that you need to know nothing
about differential equations: just find a function with two constants
that satisfies the equation (you must merely take derivatives) and you are
guaranteed that you have the only solution!
\Equationref{8} obviously satisfies the criterion.
Note: If the highest derivative in such an equation is a first derivative,
then only one undetermined constant is needed.}
%
\Eqn{8}{X = a \cos\omega t + b \sin\omega t}
%
or the totally equivalent general solution: \help{2}
%
\Eqn{9}{X = x(t) - x_0 = A \cos(\omega t + \delta_0).}
%
This is just the displacement equation%
\Index{displacement equation, for SHM}\Index{equation of motion, for SHM}
for a simple harmonic oscillator,
where \m{X = x - x_0} is the displacement, \m{A} is the amplitude,\Index{amplitude, for SHM} \m{\omega} is
the angular frequency, and \m{\delta_0} is the \Quote{initial phase.}\Index{initial phase, for SHM}

\pcap{3}{b}{\m{\omega}, \m{\nu} and \m{T} in Terms of Force Constant \m{k}.}
The angular frequency,\Index{angular frequency, for SHM}
\m{\omega}, frequency,\Index{frequency| for SHM} \m{\nu}, and period,\Index{period| for SHM} \m{T},
can all be expressed in terms of the force constant \m{k} and mass \m{m}.
Since \m{\omega^2 = k/m}, the angular frequency is
%
\Eqn{10}{\omega = (k/m)^{1/2}.}
%
The period is found to be
%
\Eqn{11}{T = \dfrac{2\pi}{\omega} = 2 \pi (m/k)^{1/2},}
%
while the frequency is
%
\Eqn{12}{\nu = \dfrac{1}{T} = \dfrac{1}{2\pi} (k/m)^{1/2}.}
%
Observe that \m{\omega}, \m{\nu}, and \m{T} all depend only on \m{k} and \m{m} and are
independent of the amplitude of the oscillations.

\tryit What would \m{\omega}, \m{\nu}, and \m{T} be when \m{m = 3\unit{kg}} and
\m{k = 48\unit{N/m}}? \help{3}

\pcap{3}{c}{Mass Hung on a Spring}\Index{spring, mass hung on}
Suppose a spring of negligible mass and of force constant 9.8\unit{N/m} hangs
vertically and has a mass of 0.10\unit{kg} attached to its free end.
We define the direction of positive displacement \m{x} as being \textit{downward}.
The equilibrium point is found from \Eqnref{4} to be
%
\Eqn{}{x_0 = \dfrac{mg}{k} = \dfrac{(0.10\unit{kg})(9.8\unit{m/s\up{2}})}{9.8\unit{N\,m}} = 0.10\unit{m}}
%
and Hooke's law can be written:
%
\Eqn{}{F = - (9.8\unit{N/m})\,(x - 0.10\unit{m}).}
%
Suppose the mass is initially pulled up a distance 0.2\unit{m} above its
equilibrium position and then released.
It will undergo SHM with its equation for displacement being
%
\Eqn{}{X(t) \equiv x(t) - x_0 = 0.20\unit{m} \cos\left(\dfrac{9.9}{\unit{s}}\,t + \pi\right)\,}
%
where the angular frequency, 9.9\unit{radians/s}, was found from \Eqnref{10}.\help{4}
%
\Footnote{7}{We normally omit the word \Quote{radians} in stating units; so when you
see \m{\omega} given in inverse seconds you should always take it to mean
\Quote{radians per second.}}
%
Note that \m{\delta_0 = \pi\unit{radians} = 180\degrees} because the mass was
released from the position of maximum negative displacement (hence zero
velocity).
\help{5}

\CaptionedLeftFramedFigure{4}{The oscillator point's position at
time zero defines the oscillator's initial phase \m{\delta_0}.}{m26gr04}
}% /Sect
%
\Sect{4}{Scaled Phase Space}{\SectType{TextOnePara}}{
%
It is often useful to make a pictorial representation of the simultaneous
values of an oscillator's position and velocity.
If we scale the velocity axis by a factor of (1/\m{\omega}), the point
representing the values of the two variables traverses a circle as time
progresses (see \Figref{4}).
In fact, it is easy to see that the point goes around the circle clockwise
with the same frequency as the oscillator.
Thus in the time it takes the oscillator to execute one complete cycle of
its straight-line motion, the point representing it in the scaled phase
space goes around its circle once.
Then the \Quote{angular velocity} \m{\omega} in the equation for the oscillator's
linear displacement is just the angular velocity of the oscillator's point
as it goes around its circle in the scaled phase space.
Finally, on the scaled phase space diagram we can easily plot the
oscillator's phase angle at time zero, its
\Quote{initial phase,}\Index{initial phase, for SHM} \m{\delta_0}(see \Figref{4}).
}% /Sect
%
\Sect{5}{Usefulness of Hooke's Law}{\SectType{TextOnePara}}{
%
Hooke's law,\Index{Hooke's law} an empirical relation describing the elasticity of solids, has
proved to be very helpful in the quantitative description of the SHM of such
solids.
Moreover the derivation of the equation of motion gives valuable insight
into more complicated systems such as those where frictional forces are
present.
%
\Footnote{8}{See \Quote{Damped Mechanical Oscillations} (MISN-0-29).}
}% /Sect
%
\Sect{}{Acknowledgments}{\SectType{Acknowledgments}}{\NsfAcknowledgment}% /Sect
%
\Sect{}{Glossary}{\SectType{Glossary}}{
\GlossaryItem{elastic limit}\Index{elastic limit} the point beyond which an elastic body does not
obey Hooke's law.

\GlossaryItem{equilibrium position}\Index{equilibrium position} the position where the force is zero so
a particle put there would not move.

\GlossaryItem{force constant}\Index{force constant} the magnitude of the proportionality constant between
force and displacement in Hooke's law.

\GlossaryItem{Hooke's law force}\Index{Hooke's law force} a restoring force that is linearly
proportional to displacement from equilibrium (a \Quote{linear restoring force}).

\GlossaryItem{initial phase}\Index{initial phase, for SHM} the phase of an oscillation at time zero.

\GlossaryItem{linear restoring force}\Index{linear restoring force} a restoring force that is linearly
proportional to displacement from equilibrium.

\GlossaryItem{restoring force}\Index{restoring force} a force that always acts toward an equilibrium position.
}% /Sect

