\revhist{2/26/85 mpm; 4/9/91, pss; 10/11/94, pss}
%
\Sect{1}{Study Procedures}{\SectType{TextMultiPara}}{
%
\pcap{1}{a}{Readings}
Read Section 9.2 in AF
%
\Footnote{1}{M.\,Alonso and E.\,J.\,Finn, {\em Physics}, Addison-Wesley (1970),
on reserve for you in the Physics Library as \Quote{Readings for Unit 27.}}
%
and examine \Figref{1}.
This figure shows the phase relationships among various quantities associated
with a harmonic oscillator whose displacement is given by
%
\Footnote{2}{The sine function could be used just as well; that substitution
would have exactly the same effect as subtracting a constant ninety degrees
from the cosine phase. \help{1}}
%
\m{x(t)\,=\,A\,\cos{(\omega t\,+\,\alpha)}}, where \m{\alpha} is the \Quote{initial
phase.}
The rotating vectors are called \Quote{phasors} and do not, as vectors, correspond
to any real quantity.
That is why they are marked with asterisks.
The \m{y*}-axis is an artificial construct and is not to be confused with an
axis in a physical space.
Here the \m{x}-axis is the real line of motion of the oscillator.

\CaptionedFullFramedFigure{1}{Phasor diagram showing the displacement,
velocity, acceleration, and force phasors for a harmonic oscillator at time
zero.
Each phasor has a different horizontal and vertical scale (length units for
\m{\vect{x}*}, length-per-unit-time units for \m{\vect{v}*}, etc.).
Note that the lengths of the phasors are written on them.}{m27gr01}

\pcap{1}{b}{Answer these questions}
From the readings, determine:
\begin{itemize}
\item [1.] Which are real, the phasors or their projections along the \m{x}-axis?
\item [2.] What do the lengths of the phasors signify?
\item [3.] What do the directions of the phasors signify?
\item [4.] Which way do the phasors rotate, clockwise or counter-clockwise?
\item [5.] Why is the velocity phasor drawn ninety degrees \Quote{ahead} of the
displacement phasor rather than ninety degrees \Quote{behind} it?
\item [6.] Are the phasors all locked together, or do some rotate faster than others?
\item [7.] At what angular velocity do the phasors rotate?
\item [8.] How can you tell from the form of \m{x(t)} that the phase angle should
be measured counter-clockwise, from the positive \m{x}-axis to the
\m{\vect{x}*}-phasor?
\end{itemize}

\pcap{1}{c}{Work problems}
Work the problems in the Problem Supplement.
}% /Sect

