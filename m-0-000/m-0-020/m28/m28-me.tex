 % 9/13/88, pss; 12/20/88, ejdk; 1/12/90, pss; 5/29/91, pss; 10/11/94, pss

\Sect{}{}{\SectType{ModelExam}}{

\Eqn{}{\text{reduced mass: }\mu =  \dfrac{m_1 m_2}{m_1 + m_2}}

\begin{one-digit-list}
\item [1.] Define...(see Output Skill K1 in this module's \textit{ID Sheet}).
\item [2.] The present eccentric motion of the earth can be graphically
illustrated by plotting the current orbit parameters on a potential energy
curve.
Including both the centrifugal and gravitational potentials, we find for the
earth-sun potential energy function:
%
\Eqn{}{E_p(r) = a (r^{-2} - b r^{-1})}
%
where:
%
\TwoEqns{}{a & = 2.65649 \times 10^{33}\unit{J}\unit{(AU)\up{2}}}
          {b & = 2\unit{(AU)\up{-1}}}
%
\Eqn{}{1\unit{AU} \equiv \text{one \Quote{Astronomical Unit of distance}} =
1.49647 \times 10^{11}\unit{m}}
%
\Eqn{}{\text{earth mass} = 1.345\,\times\,10^{32}\unit{J}\unit{(AU)\up{-2}}\unit{yr\up{2}}}
%
and \m{r} is the earth-sun separation.

Determine the frequency of small oscillations of the earth about its
equilibrium distance from the sun, through explicit use of these
steps, labeled this way in your answers:
(a) symbolic \m{F}; (b) symbolic \m{r_0}; (c) numerical \m{r_0}; (d) graphical check;
(e) symbolic \m{F'}; (f) numerical \m{k}; (g) appropriate mass; and
(h) numerical \m{\nu_0}.

\item[] As always, show all of your reasoning.
\end{one-digit-list}

\BriefAns

\begin{one-digit-list}
\item [1.] See this module's text.

\item [2.] See this module's \textit{Problem Supplement}, Problem~2.
\end{one-digit-list}

}% /Sect
