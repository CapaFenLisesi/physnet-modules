\revhist{11/29/89, pss; 1/11/89, pss; 3/18/92, pss; 10/11/94, pss; 9/15/95, pss
         8/25/97, pss; 11/7/97, lae; 11/3/01, pss; 4/16/02, pss; 10/21/02, pss}

\Sect{}{}{\SectType{ProblemSet}}{

\xpcap{}{}{HOW TO WORK THESE PROBLEMS}
Each problem involves the exercise of eight skills.
We suggest you practice the first skill across the first few problems,
one after the other, until you get the hang of it.
Then come back to the first problem and move on to the next skill.
The answers are arranged to match that procedure.
When you have mastered all eight skills, try working the rest of the problems
straight through.
Here are the skills:
\begin{one-digit-list}
\item[a.] Differentiate \m{E_p} and use that derivative to get the force
function \m{F}.
\item[b.] Solve for the symbolic \m{r_0}, a zero of \m{F}.
\item[c.] Substitute numbers to find the numerical value of \m{r_0}.
\item[d.] Plot \m{F} around \m{r_0}; check value of \m{r_0} and check descending linearity.
\item[e.] Differentiate \m{F} to get the symbolic \m{F'(r_0)}.
\item[f.] Substitute numbers to get \m{k = - F'(r_0)}.
\item[g.] Determine the appropriate mass for the frequency formula.
\item[h.] Calculate \m{\nu _0}, the \Quote{frequency of small oscillations.}
\end{one-digit-list}

\xpcap{}{}{NOTE} The introductory paragraphs and the units that are used
need not be understood to work these problems!
Just plunge in and apply each of the eight skills, one after the other.
Work problems~1-6 for practice in using the formal method for skills b-c.
Work problems~7-9 for practice in using the numerical-graphical method for
skills b-c.

\begin{one-digit-list}
\item [1.] {\bf The Gravity-Body Segment Interaction and Stepping Frequency
[General Interest, Biomechanics, Simple Pendula]}

Let us see if the stepping frequency in normal human walking can be related
to the pendulum-swing frequencies of the arm and the lower leg.
First, we can calculate the arm and leg swing frequencies using body-segment
mass distribution data.
The computed frequencies can then be directly compared to the normal stepping
frequency, or can be combined with observed distance-covered-per-step data
to compare to normal walking speed.

The model outlined above treats the arm or lower leg as a pendulum that
swings freely from its point of suspension.
Of course each of them is, in reality, driven to some extent, but they should
swing fairly freely in relaxed walking where the individual does not wish to
exert much effort.
Thus our results should be most valid for relaxed walking.
Nevertheless, we will also be able to gain some insight into the way arm
positions and motions change for running.

The arm and the lower leg are not simple pendula, but may be reasonably
treated as such in a first approximation.
We will denote the pendulum parameters by the symbols shown in \Figref{2}
in the text.

In terms of the pendulum mass's arc-displacement s and the effective
pendulum length \m{\ell}, its potential energy function can be written (see
\Figref{2}):  \help{2}
%
\Eqn{}{E_p = mg\ell \,[1 - cos(s/\ell )] = 2 mg \ell \,sin^2(s/2 \ell)\,.}
%

To treat the arm-plus-hand system as a simple pendulum, we take data from
Hanavan's computer model of the adult male and Fischer's estimate of
radius-of-gyration ratios for body segments.
%
\Footnote{PS1}{See D.\,I.\,Miller and R.\,C.\,Nelson, {\em Biomechanics of Sport:
A Research Approach}, Lea \& Febiger, Philadelphia (1973), esp. p.\,85, 98.
For other details of stepping see \Quote{The Control of Walking,} Keir Pearson,
{\em Scientific American 235}, p.\,72 (1976).}
%
The arm-plus-hand numbers we get for a 5\unit{ft},\,10\unit{in}, 150\unit{lb} person are:
\m{\ell = 0.77\unit{ft}}, \m{\text{wt. } = 9.0\unit{lb}}.
Thus: \m{m = 9.0\unit{lb}/g} where \m{g = 32\unit{ft/s\up{2}}}.

Note that \Figstoref{1}{4} in the text illustrate this arm-gravity problem.

Carry out the eight steps to find the frequency of small oscillations of the
arm-hand system about the equilibrium position.

{\bf After you have completed the eight skills, carry out these steps
to see how this problem relates to real life}:
\begin{itemize}
\item Check the arm-swing frequency you just computed against your
own stepping frequency (the frequency at which you take steps when you allow
your arms to swing like pendulums).
\item Use the arm-swing frequency you just calculated and 5 or 6\unit{feet} as the
distance covered per step, or your own measured stepping distance,
to compute your walking speed in \unit{ft/s} or \unit{mi/hr}.
\item Demonstrate how runners shorten their arm-plus-hand value of \m{\ell} as they
pick up speed, and explain why they must shorten \m{\ell}.
\item Use the dependence of \m{\nu} on \m{\ell} to explain why a tall person has to exert
a significant effort to keep in synchronous step with a short person who is
walking normally.
\end{itemize}

\item [2.] {\bf The Earth-Sun Interaction, Orbital Eccentricity
[General Interest, Astrophysics, Meteorology, Geology].}
The cycle of the earth-orbit's eccentricity, or \Quote{out of roundness,} governs
the time between the earth's short bursts of warm temperature between
glacial periods, while the two shorter Milankovic cycles participate in
determining the height and width of these pulses.
%
\Footnote{PS2}{J.\,D.\,Hays, John Imbrie, and N.\,J.\,Shackelton, \Quote{Variations in the Earth's
Orbit:  Pacemaker of the Ice Ages,}  {\em Science}, 194, 1121 (Dec.\,10, 1976),
esp. Figure 9B, center, and {\em Science News}, 110, 356 (Dec.\,4, 1975).
For the 25,800 yr. cycle see \Quote{Vector Angular Momentum and Torque: Precession
of the Equinoxes} (MISN-0-77).
Also: Nigel Calder, {\em The Weather Machine}, Viking Press, NY
(1976, Viking Compass Edition).}
%
The earth-orbit eccentricity oscillates on a (roughly) 93,000 year cycle, so
it will be about that amount of time before the earth has another \Quote{high}
like the just-ending one during which our present civilization developed.
The period of ice and drought now starting may well provide an interesting
test of mankind's ability to fashion artificial climates.%
%
\Footnote{PS3}{See \Quote{Small Oscillations Revisited,} MISN-0-32.}
%

The present eccentric motion of the earth can be graphically illustrated by
plotting the current orbit parameters on a potential energy curve.
Including both the centrifugal and gravitational potentials, we find for the
earth-sun potential energy function:
%
\Eqn{}{E_p(r) = a (r^{-2} - b r^{-1})}
%
where:
%
\Eqn{}{a = 2.65649 \times 10^{33}\unit{J}\unit{(AU)\up{2}}}
%
\Eqn{}{b = 2\unit{(AU)\up{-1}}}
%
\Eqn{}{\unit{AU} \equiv \text{one \Quote{Astronomical Unit} of distance} =
1.49647\,\times 10^{11}\unit{m}}
%
\Eqn{}{\text{earth mass} = 1.345\,\times 10^{32}\unit{J}\unit{(AU)\up{2}}\unit{yr\up{2}}}
%
and \m{r} is the earth-sun separation.

Carry out the eight steps to find the frequency of small oscillations of the
earth about its equilibrium distance from the sun.

\item [3.] {\bf The 8n-8p Interaction in \up{16}O  [Nuclear Physics].}
The giant resonances are considered some of the most spectacular phenomena in
nuclear physics, for several reasons.
First, their coherent group-motions of nuclear particles are very different
from the individual-particle motions implied by the hallowed shell model.
Second, they are spectacular simply for their sheer size, sometimes
comprising 90\% or more of all the observed scattering for a given projectile
and target excitation.
Finally, they convey information about the nuclear force: for example,
neutron excitation of the giant isovector dipole state could not occur if
the two-body nuclear force was isospin independent.

The giant dipole state in \up{16}O is due to a separation of the center-of-mass
(CM) of the eight protons (8p) from that of the eight neutrons (8n).
If we denote the distance between the two CM's by \m{x}, the separation
potential energy can be written
%
\Footnote{PS4}{G.\,Bertsch and K.\,Stricker, \Quote{On the Macroscopic Theory of the Giant Dipole
State,} {\em Phys. Rev. C} {\bf 13}, 1312 (1976) and private communication from
G.\,Bertsch.}
%
\Eqn{}{E_p(x) = -\,(a + bx^2 + cx^4)\eexp{-dx^2}}
%
\Eqn{}{a = 277.21\unit{MeV}}
%
\Eqn{}{b = 17.329\,MeV\unit{fm\up{-2}}}
%
\Eqn{}{c = 0.7825\,MeV\unit{fm\up{-4}}}
%
\Eqn{}{d = 0.16215\unit{fm\up{-2}}.}
%
\Eqn{}{m_\text{8p} = 8.34\,\times 10^{-44}\unit{MeV\,s\up{2}fm\up{-2}}}
%
\Eqn{}{m_\text{8n} = 8.35\,\times 10^{-44}\unit{MeV\,s\up{2}fm\up{-2}}}

Carry out the eight steps to find the frequency of small oscillations of the
(8p) CM and the (8n) CM about their positions of equilibrium separation.
%
\item [4.] {\bf The Argon-Argon \Quote{6-12} Interaction
[Solid State, Low Temperature Physics; Materials Science].}
Solid argon has recently been studied as a model solid, one whose properties
are theoretically tractable.
It is a simple solid to study because each atom's electrons are tightly
bound to it, resulting in each lattice site being occupied by an identical
neutral argon atom.
In contrast to ionic crystals, such a neutral-atom solid has no inter-atomic
Coulombic interactions.
The remaining interaction is thus weak and of short range, with the result
that only nearest neighbors can interact significantly.
This produces a tremendous simplification in deducing the properties of the
solid from the pair-wise interatomic forces.

The interaction potential energy for two argon (Ar) atoms a distance r apart
can be well-represented by the Lennard-Jones function:
%
\Footnote{PS5}{G.\,L.\,Pollack, \Quote{The Solid State of Rare Gases,} {\em Rev. Mod. Phys.}
{\bf 36}, 748 (1964).}
%
\Eqn{}{E_p(r) = -ar^{-6} + br^{-12}}
%
\Eqn{}{a = 17.33\unit{eV\,\AA\up{6}}}
%
\Eqn{}{b = 2.923\,\times 10^4\unit{eV\,\AA\up{12}}.}
%
\Eqn{}{m_{Ar} = 4.135\,\times 10^{-27}\unit{eV\,\AA\up{-2}}\unit{s\up{2}}}
%
Carry out the eight steps to find the frequency of small oscillations of
two Ar atoms their positions of equilibrium separation.
%
\item [5.] \textbf{The He-He \Quote{6-12} Interaction
[Solid State, Liquid State, Quantum Crystal Physics].}
Helium crystals have aroused interest recently because they are truly
quantum crystals; the helium atoms of which the crystals are composed cannot
be treated as classical particles.
In fact, direct application of classical dynamics produces an erroneous
unstable equilibrium point exactly at the observed lattice site of an atom,
along with erroneous stable equilibrium points elsewhere.
Thus the classical lattice treatment says that the observed stable lattice
should be unstable, and it erroneously says that other lattice arrangements
should be stable.
In the correct quantum treatment, one begins with a potential energy
function for pairs of atoms, treated as classical point-particle force
centers, then obtains the potential energy function at any particular
lattice site by summing over nearby quantum densities.

The two-body potential energy function for two helium atoms a distance r
apart is well approximated by
%
\Footnote{PS6}{L.\,H.\,Nosanow, \Quote{Theory of Quantum Crystals,} {\em Phys. Rev.} {\bf 146}, 120
(1966).}
%
\Eqn{}{E_p(r) = a\, \left[ \left( \dfrac{\sigma}{r} \right)^{12} - \left( \dfrac{\sigma}{r} \right)^6 \right]}
%
where
%
\Eqn{}{a = 40.88\unit{K}\,k_\text{Boltzmann}}
%
\Eqn{}{\sigma = 2.556\unit{\AA}.}
%
\Eqn{}{m_{He} = 4.808 \times 10^{-24}\,k_\text{B}\unit{K\,\AA\up{-2}}\unit{s\up{2}}}

Carry out the eight steps to find the frequency of small oscillations of
two He atoms about their positions of equilibrium separation.

\item [6.] {\bf The Electron-Proton Interaction (\m{\ell=1})
[Atomic Physics].}
The hydrogen atom is the simplest of all atoms and the only one for which
the correct quantum equation can be solved formally.
The input for the quantum Schr\"{o}dinger equation is the classical
electron-proton potential energy function.

For any \m{\ell = 1} state of atomic hydrogen, the electron(e)-proton(p) potential
energy function is:
%
\Eqn{}{E_p(r) = ar^{-2} - br^{-1}}
%
where:
%
\Eqn{}{a = 0.0763\unit{eV}\unit{nm\up{2}}}
%
\Eqn{}{b = 1.440\unit{eV}\unit{nm}}
%
\Eqn{}{m_p = 1.043\,\times 10^{-26}\unit{eV\,nm\up{-2}s\up{2}}}
%
\Eqn{}{m_e = 5.678\,\times 10^{-30}\unit{eV\,nm\up{-2}s\up{2}}}
%
The first term is the \Quote{centrifugal repulsion} and the
second term is the Coulombic e-p attraction.

Carry out the eight steps to find the frequency of small oscillations of the
electron and the proton in the \m{\ell = 1} state of the hydrogen atom.

\item [7.] {\bf The Reid Nucleon-Nucleon Potential [Nuclear Physics].}
The \Quote{Reid soft-core potential} is an interaction sometimes used in nuclear
calculations to describe the force between each pair of nucleons in a nucleus.
Here is Reid's potential for the \m{\up{1}\text{S}_0} part of the force between a
neutron (n) and a proton (p):
%
\Footnote{PS7}{R.\,V.\,Reid,\,Jr., \Quote{Local Phenomenological Nucleon-Nucleon Potentials,}
{\em Annals of Physics} (NY) {\bf 50}, 411 (1968).}
%
\Eqn{}{E_p(r) = (-he^{-x} - Ae^{-4x} + Be^{-7x})/x}
%
where
%
\Eqn{}{x = 0.7\unit{fm\up{-1}}\,r}
%
\Eqn{}{A = 1650.6\unit{MeV}}
%
\Eqn{}{B = 6484.2\unit{MeV}}
%
\Eqn{}{h = 10.463\unit{MeV}.}
%
\Eqn{}{m_p = 1.043\,\times 10^{-44}\unit{MeV\,fm\up{-2}s\up{2}}}
%
\Eqn{}{m_n = 1.044\,\times 10^{-44}\unit{MeV\,fm\up{-2}s\up{2}}}

Carry out the eight steps to find the frequency of small oscillations of a
neutron and a proton about their positions of equilibrium separation.

\item [8.] {\bf Ba-Ti-O \Quote{1-6-9} Interactions (Solid State Physics,
Materials Science).}
Barium titanate crystals (BaTiO\m{_3}) are of considerable commercial interest
because they exhibit a region of ferroelectricity, with a relative dielectric
constant of the order of ten thousand!

Our picture of normal barium titanate is that of an ionic crystal in which
the Ba, Ti, and O ions are regarded as point force centers.
For any two of the Ba, Ti, or O ions, with charges e\m{_1} and e\m{_2} and a
distance \m{r} apart, the interaction produces a potential energy represented by
%
\Footnote{PS8}{A.\,F.\,Devonshire, \Quote{Theory of Barium Titanate}, {\em Phil. Mag.}
{\bf 40}, 1040 (1949).}
%
\Eqn{}{E_p(r)\,=\,e_1\,e_2\,r^{-1} - \mu\,r^{-6} + \lambda\,r^{-9}.}
%
The first term is the Coulombic interaction, the second is the Van der Waals
attraction, and the third is a short range repulsion due mainly to the
exclusion principle.
The charges on the Ba, Ti, and O ions are \m{2e}, \m{4e}, and \m{ - 2e} respectively,
with \m{-e} being the charge on the electron.

For an interacting oxygen-ion-barium-ion pair,
%
\Eqn{}{\lambda_\text{O\,-\,Ba} = 6180\unit{eV\,\AA\up{9}}}
%
\Eqn{}{\mu_\text{O\,-\,Ba} = 101\unit{eV\,\AA\up{6}}}
%
\Eqn{}{e_1\,e_2 = - 4 e^2 = - 57.6\unit{eV\,\AA}}
%
\Eqn{}{m_\text{O}  = 1.656\times 10^{-27}\unit{eV\,\AA\up{-2}s\up{2}}}
%
\Eqn{}{m_\text{Ba} = 1.421\times 10^{-26}\unit{eV\,\AA\up{-2}s\up{2}}}

Carry out the eight steps to find the frequency of small oscillations of a
barium ion and an oxygen ion about their equilibrium separation in BaTiO\m{_3}.

\item [9.] {\bf The Quark-Antiquark \Quote{Gluon-Bag} Interaction (\m{\ell = 1})
[Elementary Particle, High Energy Physics].}
Perhaps the most exciting experimental discovery in recent elementary
particle physics is that of the \m{\psi} particles, each apparently
consisting of a charmed quark (\m{q}) and a charmed antiquark (\m{\bar{q}}) held
together in their bag by gluons.
(The charm of the quark and antiquark add to zero, so the \m{\psi} has
no charm.)

A typical P-state potential energy function assumed for the \m{q} and
\m{\bar{q}} pair, separated by a distance \m{r} is
%
\Footnote{PS9}{E.\,Eichten, K.\,Gottfried, T.\,Kinoshita, J.\,Kogut, K.\,D.\,Lane, and
T.-M.\,Yan, {\em Phys. Rev. Lett.} {\bf 34}, 369 (1975).}
%
\Eqn{}{E_p(r) = a\,r^{-2} - b\,r^{-1} + c\,r}
%
where
%
\Eqn{}{a = 0.097\unit{GeV\,fm\up{2}}}
%
\Eqn{}{b = 0.20\unit{GeV\,fm}}
%
\Eqn{}{c = 5.0 \unit{GeV\,fm\up{-1}}}
%
and:
%
\Eqn{}{m_q = m_{\bar{q}} = 1.78\times 10^{-47}\unit{GeV\,fm\up{-2}s\up{2}}.}

Carry out the eight steps to find the frequency of small oscillations of the
\m{q\bar{q}} pair about their positions of equilibrium separation.
\end{one-digit-list}

\BriefAns

\begin{itemize}
\item[1a.] \m{F=-(W)\sin(s/\ell)}, where \m{W} is the weight
\item[2a.] \m{F=2ar^{-3}-abr^{-2}}
\item[3a.] \m{F=[x(-2ad+2b)+x^3(4c-2bd)+x^5(-2cd)]e^{-dx^2}}
\item[4a.] \m{F=-6ar^{-7}+12br^{-13}}
\item[5a.] \m{F=a(12\sigma^{12}r^{-13}-6\sigma^6r^{-7})}
\item[6a.] \m{F=2ar^{-3}-br^{-2}}
\item[7a.] \m{F=-(1/r)(he^{-ar}+4Ae^{-4ar}-7Be^{-7ar})}
\item[\ ]  \m{\,\ \ \ \ \ \ -(1/ar^2)(he^{-ar}+Ae^{-4ar}-Be^{-7ar})};
           \m{a \equiv x/r}
\item[8a.] \m{F=e_1e_2r^{-2}-6\mu r^{-7}+9\lambda r^{-10}}
\item[9a.] \m{F=2ar^{-3}-br^{-2}-c}
\item[1b.] \m{s_0=0}
\item[2b.] \m{r_0=2/b}
\item[3b.] \m{x_0=0}
\item[4b.] \m{r_0=(2b/a)^{1/6}}
\item[5b.] \m{r_0=\sigma (2)^{1/6}}
\item[6b.] \m{r_0=2a/b}
\item[7b.] graphical or numerical
\item[8b.] graphical or numerical
\item[9b.] graphical or numerical or solve cubic equation to get:
\item[\ ] \m{r_0=\dfrac{(-3)2^{2/3}b+[54ac^{1/2}+(2916a^2c+108b^3)^{1/2}]^{2/3}}%
                  {(3)2^{1/3}c^{1/2}[54ac^{1/2}+(2916a^2c+108b^3)^{1/2}]^{1/3}}}
%\item[\ ] \m{r_0=\dfrac{\textstyle (-3)2^{2/3}b+[54ac^{1/2}+(2916a^2c+108b^3)^{1/2}]^{2/3}}%
%{\textstyle (3)2^{1/3}c^{1/2}[54ac^{1/2}+(2916a^2c+108b^3)^{1/2}]^{1/3}}}
\item[1c.] \m{s_0=0}
\item[2c.] \m{r_0=1\unit{AU}}
\item[3c.] \m{x_0=0}
\item[4c.] \m{r_0=3.872666\unit{\AA}  \,\approx\,3.87\unit{\AA}}
\item[5c.] \m{r_0=2.86901\unit{\AA}   \,\approx\,2.87\unit{\AA}}
\item[6c.] \m{r_0=0.10597222\unit{nm} \,\approx\,0.106\unit{nm}}
\item[7c.] \m{r_0=0.844818\unit{fm}   \,\approx\,0.845\unit{fm}}
\item[8c.] \m{r_0=2.31836\unit{\AA}   \,\approx\,2.32\unit{\AA}}
\item[9c.] \m{r_0=0.299353627\unit{fm}\,\approx\,0.299\unit{fm}}
\item [1d.] \CharacterUnframedFigure{m28gr07} 2d. \CharacterUnframedFigure{m28gr08}
\item [3d.] \CharacterUnframedFigure{m28gr09} 4d. \CharacterUnframedFigure{m28gr10}
\item [5d.] \CharacterUnframedFigure{m28gr11} 6d. \CharacterUnframedFigure{m28gr12}
\item [7d.] \CharacterUnframedFigure{m28gr13} 8d. \CharacterUnframedFigure{m28gr14}
\item [9d.] \CharacterUnframedFigure{m28gr15}
\item[1e.] \m{F'(s_0)=-W/\ell }
\item[2e.] \m{F'(r_0)=-ab^4/8}
\item[3e.] \m{F'(x_0)=2(b-ad)}
\item[4e.] \m{F'(r_0)=-18(a^2/b)(a/2b)^{1/3}}
\item[5e.] \m{F'(r_0)=-(a/\sigma^2)(18)2^{-1/3}}
\item[6e.] \m{F'(r_0)=-b^4/(8a^3)}
\item[7e.] advice: do it numerically or graphically
\item[8e.] \m{F'(r_0)=-2e_1e_2r_0^{-3}+42\mu r_0^{-8}-90\lambda r_0^{-11}}
\item[9e.] \m{F'(r_0)=-6ar_0^{-4}+2br_0{-3}}
\item[1f.] \m{k=11.688\unit{lb/ft}}
\item[2f.] \m{k=5.31\times 10^{33}\unit{J/(AU)\up{2}}}
\item[3f.] \m{k=55.2\unit{MeV/fm\up{2}}}
\item[4f.] \m{k=0.01233\unit{eV/\AA\up{2}}}
\item[5f.] \m{k=89.4\,k_\text{B}\unit{K/\AA\up{2}}}
\item[6f.] \m{k=1210\unit{eV/nm\up{2}}}
\item[7f.] \m{k=2130\unit{MeV/fm\up{2}}}
\item[8f.] \m{k=39.1\unit{eV/\AA\up{2}}}
\item[9f.] \m{k=57.6\unit{GeV/fm\up{2}}}
\item[1g.] \m{m=0.28125\unit{lb\,ft\up{-1}s\up{2}}}
\item[2g.] \m{m=1.339\times 10^{47}\unit{J\,(AU)\up{-2}s\up{2}}}
\item[3g.] \m{m=\text{ reduced mass }= 4.172\times 10^{-44}\unit{MeV\,fm\up{-2}s\up{2}}}
\item[4g.] \m{m=\text{ reduced mass }=20.625\times 10^{-28}\unit{eV\,\AA\up{-2}s\up{2}}}
\item[5g.] \m{m=\text{ reduced mass }= 2.404\times 10^{-24}\,k_\text{B}\unit{\AA\up{-2}s\up{2}}}
\item[6g.] \m{m=\text{ reduced mass }= 5.675\times 10^{-32}\unit{eV\,\AA\up{-2}s\up{2}}}
\item[7g.] \m{m=\text{ reduced mass }=52.17 \times 10^{-46}\unit{MeV\,fm\up{-2}s\up{2}}}
\item[8g.] \m{m=\text{ reduced mass }=14.83 \times 10^{-28}\unit{eV\,\AA\up{-2}s\up{2}}}
\item[9g.] \m{m=\text{ reduced mass }= 8.90 \times 10^{-48}\unit{GeV\,fm\up{-2}s\up{2}}}
\item[1h.] \m{\nu_0=1.03\unit{Hz}}
\item[2h.] \m{\nu_0=3.17\times 10^{-8}\unit{Hz}}
\item[3h.] \m{\nu_0=5.79\times 10^{21}\unit{Hz}}
\item[4h.] \m{\nu_0=3.89\times 10^{11}\unit{Hz}}
\item[5h.] \m{\nu_0=9.71\times 10^{11}\unit{Hz}}
\item[6h.] \m{\nu_0=2.32\times 10^{15}\unit{Hz}}
\item[7h.] \m{\nu_0=1.02\times 10^{23}\unit{Hz}}
\item[8h.] \m{\nu_0=2.58\times 10^{13}\unit{Hz}}
\item[9h.] \m{\nu_0=4.05\times 10^{23}\unit{Hz}}
\end{itemize}

}% /Sect