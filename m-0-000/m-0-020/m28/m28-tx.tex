\revhist{12/29/87, pss; 7/10/88, pss; 9/8/88, pss; 12/8/89, pss; 10/11/94, pss; 8/23/96, pss;
         11/7/97, lae; 3/22/01, pss; 3/29/01, kag}
%
\Sect{1}{Introduction}{\SectType{TextMultiPara}}{
%
\pcap{1}{a}{Approximating Real-Force Oscillations}
The purpose of this lesson is to help you develop skill in using Simple
Harmonic Motion\Index{simple harmonic motion} (SHM)\Index{SHM (Simple Harmonic Motion)} as an approximation to general oscillatory motions.
Strictly speaking, SHM applies only to oscillations produced by a linear
restoring force\Index{linear restoring force}: \m{F = -k(x-x_0)}.
Other forces are non-linear functions of position \m{x} but we often
approximate them with the SHM force because: (1) the SHM solutions can be
easily obtained in terms of well-known mathematical functions; (2) most
oscillations are well-approximated by SHM when their amplitudes, driving
forces and damping are small; (3) the SHM approximation often gives us
useful insights; and (4) we often need some approximation with which to check
the computer program used to obtain the actual motion.
%
\Footnote{1}{For descriptions of the terms \Quote{driving forces} and \Quote{damping}
see {\em Damped Driven Oscillations and Mechanical Resonances} (MISN-0-31).}
%

\pcap{1}{b}{Most Forces Look Like SHM Over Small Displacements}
The reason SHM  is a good approximation for \Quote{small} oscillations\Index{oscillations, small}\Index{small oscillations}
is because almost any force function looks rather like the linear SHM force
if you examine it in a small enough region around the center point of its
oscillations\Index{center point, of small oscillations} (see \Figref{1}).
Thus if the amplitude of the oscillation is small enough, for the particular
force you have at hand, then SHM will be a good enough approximation to the
actual motion and no further work need be done.
You must decide whether the SHM solution is good enough for your particular
application.

\pcap{1}{c}{Checking Computer Codes}
If computer solutions are needed, the computer code should be
checked by comparing small-amplitude numerical solutions, produced by the
code, to the corresponding SHM solutions.
One does this by running the program with the input constants
temporarily set at a succession of \Quote{small oscillation} amplitudes.
One plots the resulting frequencies versus the log of the amplitude
and extrapolates the curve to zero amplitude.
This \Quote{zero amplitude} frequency should agree exactly with the SHM
frequency.\Index{frequency| for SHM}
If they do agree, one then puts the amplitude back at its physical value
and runs the computer program again to get the physical answer.

\pcap{1}{d}{Real Force In, Approximate Motion Out}
In this module we show you how to start with: (1) some particular restoring
force: and (2) the mass of the oscillator driven by that force, and from that
input produce the particular SHM solution which the real oscillation
approaches as the amplitude approaches zero.
The \Quote{solution} consists of the SHM approximation to: (1) the oscillator's
frequency (hence also its period); and (2) the center point of its oscillations.\Index{center point, of small oscillations}
Using these, you can immediately write down the oscillator's position as a
function of time.

\CaptionedFullFramedFigure{1}{A non-linear function usually looks rather
straight over a small enough region.}{m28gr00}

\enlargethispage{0.4cm}
}% /Sect
%
\Sect{2}{Example: a Pendulum}{\SectType{TextMultiPara}}{
%
\pcap{2}{a}{The Physical Oscillator}
Our illustrative example is the simple pendulum.
This consists of a rod which is pivoted at its upper end and which has a
weight, called a \Quote{bob,} attached at the rod's lower end (see \Figref{2}).
The rod is assumed to be weightless and of length \m{\ell}.
The bob is assumed to be the size of a mathematical point and is assumed to
have mass \m{m}.
Our pendulum is assumed to be swinging back and forth, under the influence of
gravity, without friction.
We can think of it as a slightly idealized version of the pendulum in a
Grandmother's or Grandfather's clock.

\TwoCaptionedFramedFigures{2}{A simple pendulum.}{m28gr02}%
                          {3}{Position vs. time for the pendulum of \Figref{2}.}{m28gr03}

\pcap{2}{b}{The Position Oscillates}
The position of the bob oscillates as the bob travels back and forth
along its circular-arc trajectory.
This position is denoted \m{s} in \Figref{2}, and it is taken as zero at the
lowest point on the trajectory.
The position \m{s(t)} oscillates: as time increases it goes smoothly from
positive values to negative ones to positive ones to negative ones, etc.
(see \Figref{3}).

\pcap{2}{c}{Finding the Appropriate SHM}
We produce the approximating SHM solution, the one which the real force's
oscillations approach as amplitude becomes small, by carrying out these three
steps:
\begin{itemize}
\item[1.] Obtain the net force acting on the oscillator.
For our pendulum, we start by noting that two different objects
exert forces on the bob, which is the mass that oscillates: the earth through
gravity, and the rod through its point of contact with the bob.
Show that the resultant force is tangential to the trajectory and its
component in the direction of increasing \m{s} is: \m{F(s) = -mg\,sin(s/\ell)}
(this is the only component there is). \help{2}\,
%
\Footnote{2}{For help, see sequence S-2 in this module's Special Assistance
 Supplement.}
%
\item[2.] Construct a linear force whose value and slope agree with those of
the true force at the point where the true force descends through zero (the
point of zero displacement for the oscillations).
This means that we must first find the place where the true force descends
through zero, then find the slope of the true force there.
For our pendulum, the point where the true force (the force on the bob)
descends through zero is seen by inspection to be \m{s = 0}.
The slope of the force there is: \m{dF/ds(0) = -mg/\ell}.
The linear force which has that value and slope at that point is (check this
in your head): \m{F(s) = -mgs/\ell}.
For sufficiently small displacements around the origin, the linear force and 
the true force appear indistinguishable to the eye (see \Figref{1}).
\item[3.] Calculate the SHM frequency for the linear approximating force.
For our pendulum, the force constant\Index{force constant} of the approximating SHM is:
\m{k = mg/\ell}.
This gives the SHM frequency:
\m{\nu_0 = (1/2\pi) \omega} where \m{\omega = \sqrt{g/\ell}}.
If the pendulum amplitude is small, its frequency is close to this SHM
frequency.
\end{itemize}

\CaptionedLeftFramedFigure{4}{Exact and SHM frequencies, model as in
\Figref{2}.}{m28gr04}

\pcap{2}{d}{Comparing the Exact and SHM Values}
In \Figref{4} we compare the SHM frequency to the true frequency as a function
of the amplitude of the oscillations.
The SHM frequency is independent of the SHM oscillator's amplitude so it is
plotted as a constant.
However, the pendulum's frequency varies with the pendulum's amplitude.
The figure shows that if the pendulum swings to {60\degrees}, the SHM prediction
for the frequency is off by about 6\%, but if the swing is to {30\degrees}
then SHM is off by only about 1.5\%!
%
\Footnote{3}{See \Quote{Small Oscillations Revisited,} MISN-0-32.}

In \Figref{5} we compare the positions as functions of time for various
amplitudes.
Note that the angle of swing must get up to {125\degrees} (picture that
swing!) before the curve no longer looks like an SHM sine
curve.
Note also that the frequency decreases (so the period increases in \Figref{5})
as the amplitude increases.
\CaptionedFullFramedFigure{5}{Position vs. time for various amplitudes,
model as in \Figref{2}.}{m28gr05}
}% /Sect
%
\Sect{3}{Variations}{\SectType{TextMultiPara}}{
%
\pcap{3}{a}{Summary}
We must often deal with these variations: (1) the center point of the
oscillations is not at the coordinate origin; (2) the potential energy is
specified instead of the force; and (3) rather than a fixed force acting on a
single mass, we have two masses exerting mutual forces on each other as they
each oscillate.

\pcap{3}{b}{Oscillation Not About the Origin}
It is often inconvenient to put the origin of the coordinate system at the
place where the force descends through zero.
The coordinate position where the force descends through zero is called the
\Quote{Point of Stable Equilibrium}\Index{point of stable equilibrium} and it is denoted by adding a subscript
\Quote{zero} to the coordinate symbol.
We will usually refer to the Point of Stable Equilibrium as the \Quote{PSE.}\Index{PSE (Point of Stable Equilibrium)}
%
\Footnote{4}{For a discussion of classes of equilibrium points, see \Quote{Static
Equilibrium} (MISN-0-6).
The abbreviation \Quote{PSE,} used here, is not in common use: there is no commonly
used abbreviation.}
%
For example, if the position of the oscillator is labeled \m{x} then
its PSE is labeled \m{x_0}.
If the position of the oscillator is labeled \m{s} then its PSE is labeled \m{s_0}.

\pcap{3}{c}{The Potential Energy is Specified}
Often a potential energy function is specified rather than a force function.
%
\Footnote{5}{For a discussion of how to find the force function corresponding to a
given potential energy function see \Quote{Potential Energy Curves, Force, and
Motion} (MISN-0-22).}
%
Since \m{F(x) = -dE_p(x)/dx}, \m{x_0} is the point where the slope of \m{E_p},
\m{E_p'}, is ascending through zero.
The force constant is then the second derivative: \m{k = E_p''(x_0)}.
When the potential energy function \m{E_p(x)} is plotted as a function of \m{x},
it shows a minimum at the PSE (see \Figref{6}).

\CaptionedLeftFramedFigure{6}{Potential energy vs. displacement, model as
in \Figref{2}.}{m28gr06}

\pcap{3}{d}{Two Mutually-Interacting Bodies}
\Index{small oscillations| of two-body system}If two masses are oscillating due to equal but opposite forces each exerts
on each other, the parameters we have been using must be reinterpreted.
Now \m{x} is the (variable) distance between the two objects, \m{F(x)} is
the force each body exerts on the other, and \m{x_0} is the equilibrium
separation of the two objects, the separation at which each exerts zero
force on the other.
When \m{x} is larger than \m{x_0}, the two masses are attracted toward each other;
when \m{x} is smaller than \m{x_0}, they are repelled from each other.
The force constant for the system is defined as before: \m{k=-dF/dx(x_0)}.
The two objects' oscillations will be identical but {180\degrees} out of
phase (they both head away from each other, then they both
turn around and head toward each other, then they turn around and head
away from each other, etc., each oscillating about its own end of \m{x_0}).
%
\Footnote{6}{Note that the total momentum of the two-body system is always zero.}
%
The SHM frequency is then: \m{\nu = (1/2\pi)\sqrt{k/\mu}}, where
the system's \Quote{reduced mass}\Index{mass| reduced}\Index{reduced mass} \m{\mu} is given by:
\m{\mu \equiv m_1 m_2/(m_1 + m_2)}.
%
\Footnote{7}{This is derived in: \Quote{Two-Body Kinematics and Dynamics}
(MISN-0-45).}
}% /Sect
%
\Sect{4}{Solution Steps}{\SectType{TextOnePara}}{
%
\begin{itemize}
\item[1.] If \m{E_p} is given, find \m{F}.
\item[2.] Find the PSE.
\item[3.] Find \m{k}.
\item[4.] Determine the appropriate mass: take \m{m} or calculate \m{\mu}.
\item[5.] Find \m{\nu}.
\end{itemize}
}% /Sect
%
\Sect{5}{The Point of Stable Equilibrium}{\SectType{TextMultiPara}}{
%
\pcap{5}{a}{Formal Solution}
One way for finding the zeros of a function is formally, which means
setting the function equal to zero and then solving that equation formally.
The solutions are the function's zeros.
Then you must find which of those zeros are PSE's by examining the sign of
\m{F'} at each of them.
If there is more than one PSE, your particular application must specify
which PSE to focus on.

For a linear force, find the PSE without writing anything down.
For example, in the function \m{F(x) = -a(x - b)}, the point \m{x = b} is
obviously a PSE.
Here is another linear form for you to try: \m{F(x) = -a^2 x + b^2}. \help{4}

For a quadratic force, use the quadratic root equation.
For example, for \m{F = 3 - 16x + 5x^2}, use the quadratic root
formula to show that: (1) there is a PSE at \m{x_0 = 0.2} \help{10};
and (2) there is another root at \m{x = 3} but this one is not a PSE because
the force is ascending, not descending, through zero there.
This latter point is called a \Quote{point of unstable equilibrium.}
The word \Quote{equilibrium} is used because an object placed at rest at this
point will not move, and the word \Quote{unstable} is used because any small
displacement of the object from the point will cause the object to be
accelerated away from that point.

There are now good computer programs, employing artificial
intelligence techniques, that can produce most of the known formal
solutions for roots.
The most used programs at this time are MACSYMA, REDUCE, and MATHEMATICA.

\pcap{5}{b}{Numerical Techniques}
\Index{derivative| numerical method for}\Index{numerical derivative method}When a formal solution cannot be found, or sometimes when it cannot be
found easily, we use numerical techniques.

Sometimes we use computer programs or calculators (like the HP - 41) that
incorporate a whole bag of numerical zero-finding tricks.
If you have access to a zero-finding calculator or computer, try it on the
quadratic force given above.

Sometimes we use numerical interpolation, mainly on a hand calculator:
this always works, albeit slowly.
Try numerical interpolation on your own calculator, on the force
\m{F(x) = -\cos\,x}.
Calculate values of \m{F(x)} for various values of \m{x}, finding a region where
\m{F} changes from positive to negative.
Single out the adjacent pair of points between which \m{F} changes sign.
Use linear interpolation on this pair of points to predict the point where
\m{F} is zero.
Calculate the actual value of \m{F} at that point and repeat the interpolation
process as many times as you wish. \help{15}

\pcap{5}{c}{Graphical Method}
For the graphical method, use your hand calculator and graph 
paper, or a computer, to plot the force function.
Note the point where the force descends through zero.
Increase your accuracy by \Quote{blowing up} the region around that zero on
a new plot.
Repeat this enlarging process until you have located the PSE to the desired
accuracy.
You can try this on the quadratic function given above.
}% /Sect
%
\Sect{6}{The Force Constant}{\SectType{TextMultiPara}}{
%
\pcap{6}{a}{Formal Method}
Once the PSE has been found, by whatever means, the force constant\Index{force constant, calculating the} \m{k} can
usually be found formally; that is, by taking the first derivative of the
force and then evaluating it at the PSE.
For example,
%
\Eqn{}{F = (3 - 16\,x\unit{m\up{-1}} + 5\,x^2\unit{m\up{-2}})\unit{N}\,,}
%
can be differentiated at its PSE, \m{x_0 = 0.2\unit{m}}, to obtain:
\m{F' = -14}\,N/m.
Then the force constant can be seen by inspection: \m{k = 14}\,N/m.
If you don't know a particular function's derivatives, look them up in a
table of derivatives
%
\Footnote{8}{See the Appendix or, for example, {\em A Short Table of
Integrals}, B.\,D.\,Peirce, Ginn and Co., Boston (1929).}
%
or use an appropriate formal-math computer program.

\pcap{6}{b}{Finite Difference and Graphical Methods}
\Index{Finite Difference Method}If the derivative cannot easily be taken formally, it can be measured as
the slope of the force on a graph, or it can be evaluated using finite
difference equations.
The graphical method is usually not practical for taking second derivatives,
so finite difference equations are better when the potential energy function
is given instead of the force function.

The relevant finite difference equations are:
%
\Footnote{9}{See \Quote{Taylor's Series for the Expansion of a Function About a Point}
(MISN-0-4).}
%
\Eqn{}{f'(x)\simeq\dfrac{f(x+\Delta)-f(x-\Delta)}{2\Delta}\,;}
%
\Eqn{}{f''(x)\simeq\dfrac{f(x+\Delta)-2f(x)+f(x-\Delta)}{\Delta^2}\,.}
%
In the above formulas \m{x} is the PSE and \m{\Delta} is a small distance.
The distance \m{\Delta} should be made sufficiently small so that making it
smaller results in no significant improvement in the value of \m{k} (for our
present purposes, 2-3 significant digits).
Do not make \m{\Delta} so small that \m{k}'s first 2-3 digits are affected by
calculator error.
You can practice on: \m{F = -\,5\,x^{-4} + 20\,x^{-6}}.  \help{3}

Note that use of the graphical or finite difference methods as a check
provides a virtually fail-safe method for locating all errors inadvertently
made while taking formal derivatives.

Here is a function on which you can easily practice all three
methods (formal, graphical, and numerical): \m{F = -\,\cos\,x}. \help{1}
}% /Sect
%
\Sect{}{Acknowledgments}{\SectType{Acknowledgments}}{
For their generous assistance with this module, I would like to thank: Jack
Hetherington, George Bertsch, Tom Kaplan, Gerry Pollack, Jon Pumplin, Jules
Kovacs, Robert Spence, and Jerry Cowen.
\NsfAcknowledgment
}% /Sect
%
\Sect{}{Glossary}{\SectType{Glossary}}{
\GlossaryItem{Point of Stable Equilibrium (PSE)}\Index{PSE (Point of Stable Equilibrium)} for a particular force, a space-point
where any small displacement produces a return force (Sections 3b and 5).

\GlossaryItem{Small Oscillation}\Index{small oscillations}\Index{oscillations, small} an oscillation whose motion is simple harmonic within
tolerable limits.

\GlossaryItem{Linear Approximation\Index{linear approximation} (to a particular mathematical function in
the neighborhood of a specific point)}
the straight line which has the same value and slope as the designated
function at the designated point.
}% /Sect
%
\Sect{}{Table of Derivatives}{\SectType{AppendixMultiPara}}{
%
\xpcap{}{}{Combinations of functions}
\begin{itemize}
\item[1.] \m{\dfrac{d}{dx}\,\left(f(u)\right) = \dfrac{df(u)}{du}\,\,\,\cdot\,\dfrac{du}{dx}}
\item[2.] \m{\dfrac{d}{dx}\,\left(f\,\pm\,g\right) = \dfrac{df}{dx}\,\pm\,\dfrac{dg}{dx}}
\item[3.] \m{\dfrac{d}{dx}\,\left(fg\right) = f\,\dfrac{dg}{dx} + g\,\dfrac{df}{dx}}
\item[4.] \m{\dfrac{d}{dx}\,\left(f/g\right) = \left(g\,\dfrac{df}{dx} - f\,\dfrac{dg}{dx}\right)\,/\,g^2}
\end{itemize}

\xpcap{}{}{Specific Functions}
\begin{itemize}
\item[5.] \m{\dfrac{d}{dx}\,\left(x^n\right) = n\,x^{n-1}}
\item[6.] \m{\dfrac{d}{dx}\,\left(e^x\right) = \eexp{x}}
\item[7.] \m{\dfrac{d}{dx}\,\left(\ln\,x\right) = 1\,/\,x}
\item[8.] \m{\dfrac{d}{dx}\,\left(\sin\,x\right) = \cos\,x}
\item[9.] \m{\dfrac{d}{dx}\,\left(\cos\,x\right) = -\,\sin\,x}
\end{itemize}

\xpcap{}{}{When the argument contains a constant}
For example, in
%
\Eqn{}{\dfrac{d}{dx}\,\left(\sin\,(ax)\right),}
%
first use formula \#1 (\m{u = ax} in this example), then use the formula
appropriate to the function (\#8 in this example).
}% /Sect
