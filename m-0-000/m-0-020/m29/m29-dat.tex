\revhist{3/30/78, pss; 4/1/91, pss; 12/16/91, pss; 10/11/94, pss; 9/12/95, pss; 2/5/99 bds;
         2/22/99, pss; 10/3/02, pss; 10/21/02, pss}
%
\defModTitle{\ph{Damped} \ph{Mechanical Oscillations}}
\defCtAuthor{Peter Signell, Michigan State University}
\defIdAuthor{Peter Signell, Michigan State University}
%
\defIdItems{
    \IdVersEval{10/21/2002}{0}
    \IdHours{1}
    \begin{InputSkills}
    \item [1.] Work with Simple Harmonic Motion equations, especially the differential
    equation of motion \prrqone{0-26}.
    \item [2.] Use and understand complex algebra and complex functions,
    especially \m{\exp(x+iy)} type functions \prrqone{0-59}.
    \end{InputSkills}
    %
    \begin{KnowledgeSkills}
    \item [K1.] Vocabulary: underdamped oscillator, critically damped oscillator,
    overdamped oscillator, oscillation envelope.
    \item [K2.] Write down the displacement equation solutions for underdamped,
    critically damped, and overdamped oscillators.
    Show that each is, in fact, a solution to Newton's Second Law for the force
    \m{F\,=\,-\,kx\,-\,\lambda v}, for the proper relationships of \m{\gamma} and
    \m{\omega} to \m{m}, \m{k}, and \m{\lambda}.
    \item [K3.] Sketch a graph of the displacement, \m{x(t)} for the
    underdamped oscillator, labeling the envelope of the oscillations
    and indicating the initial phase used (not zero).
    \item [K4.] Sketch graphs of displacement vs. time showing the transition from
    small to large damping, including the critically damped case.
    \end{KnowledgeSkills}
    %
    \begin{ProblemSolvingSkills}
    \item [S1.] Given the decrease of an amplitude with time, find the damping
    constant and vice versa.
    \end{ProblemSolvingSkills}
}